% !TeX root = ./wangshuo_phdthesis.tex

% 论文基本信息配置

\bnusetup{
  %******************************
  % 注意:
  %   1. 配置里面不要出现空行
  %   2. 不需要的配置信息可以删除
  %   3. 建议先阅读文档中所有关于选项的说明
  %******************************
  %
  % 输出格式
  %   选择打印版(print)或用于提交的电子版(electronic),前者会插入空白页以便直接双面打印
  %
  output = electronic,
  %
  % 标题
  %   可使用“\\”命令手动控制换行
  %
  title  = {大气污染复杂系统的数据驱动建模:\\预测、推断与模拟},
  title* = {Data-Driven Modeling of Air Pollution as a Complex System:\\Prediction, Inference, and Simulation},
  %
  % 学科门类
  %   1. 学术型
  %      - 中文
  %        需注明所属的学科门类,例如:
  %        哲学、经济学、法学、教育学、文学、历史学、理学、工学、农学、医学、
  %        军事学、管理学、艺术学
  %      - 英文
  %        博士:Doctor of Philosophy
  %        硕士:
  %          哲学、文学、历史学、法学、教育学、艺术学门类,公共管理学科
  %          填写“Master of Arts“,其它填写“Master of Science”
  %   2. 专业型
  %      直接填写专业学位的名称,例如:
  %      教育博士、工程硕士等
  %      Doctor of Education, Master of Engineering
  %   3. 本科生不需要填写
  %
  % degree-category  = {教育硕士},
  % degree-category* = {Master of Education},
  %
  % 培养单位
  %   填写所属院系的全名
  %
  department = {系统科学学院},
  %
  % 学科
  %   1. 研究生学术型学位,获得一级学科授权的学科填写一级学科名称,其他填写二级学科名称
  %   2. 本科生填写专业名称,第二学位论文需标注“(第二学位)”
  %
  discipline  = {系统分析与集成},
  discipline* = {System Analysis and Integration},
  %
  % 专业领域
  %   1. 设置专业领域的专业学位类别,填写相应专业领域名称
  %   2. 2019 级及之前工程硕士学位论文,在 `engineering-field` 填写相应工程领域名称
  %   3. 其他专业学位类别的学位论文无需此信息
  %
  % professional-field  = {学科教学(数学)},
  % professional-field* = {Subject Teaching(Mathematics)},
  %
  % 姓名
  %
  author  = {王硕},
  author* = {Wang Shuo},
  student-id = {202031250018},
  %
  % 导师
  %   中文姓名和职称之间以英文逗号“,”分开,下同
  %
  supervisor  = {张江, 教授},
  supervisor* = {Professor Zhang Jiang},
  supervisor-department = {系统科学学院},
  %
  % 副导师
  %
  % associate-supervisor  = {陈文光, 教授},
  % associate-supervisor* = {Professor Chen Wenguang},
  %
  % 联合导师
  %
  % co-supervisor  = {某某某, 教授},
  % co-supervisor* = {Professor Mou Moumou},
  %
  % 日期
  %   使用 ISO 格式;默认为当前时间
  %
  date = {2026-02-01},
  %
  % 是否在中文封面后的空白页生成书脊(默认 false),仅博士需要。
  %
  include-spine = true,
  spine-school = {北京师范大学},
  %
  % 密级和年限
  %   秘密, 机密, 绝密
  %
  % secret-level = {秘密},
  % secret-year  = {10},
  %
}

% 载入所需的宏包

% 定理类环境宏包
\usepackage{amsthm}
% 也可以使用 ntheorem
% \usepackage[amsmath,thmmarks,hyperref]{ntheorem}

\bnusetup{
  %
  % math-style = GB,  % GB | ISO | TeX
  % 数学公式字体,如果认为xits字体不满意的话,可尝试更改为newcm.
  math-font  = xits,  % stix | xits | libertinus | newcm 
  % 图表编号时是否带有章节序号,默认为不带有false
  figurestables-chapternumber=false,  %false / true
}

% 可以使用 nomencl 生成符号和缩略语说明
% \usepackage{nomencl}
% \makenomenclature

% 表格加脚注
\usepackage{threeparttable}

% 表格中支持跨行
\usepackage{multirow}

% 固定宽度的表格。
% \usepackage{tabularx}

% 跨页表格
\usepackage{longtable}

% 算法
\usepackage{algorithm}
\usepackage{algorithmic}

% 量和单位
\usepackage{siunitx}

\usepackage{xspace}
\usepackage{graphicx}
\usepackage{subcaption}
\usepackage{amsmath}
\usepackage{makecell}
\usepackage{booktabs}   % 用于三线表
\usepackage{multirow}   % 用于多行单元格
\usepackage{pifont}     % 用于显示对勾和叉号
\usepackage{comment}
\usepackage{csquotes}   % 中文引号

% 参考文献使用 BibTeX + natbib 宏包
% 顺序编码制
\usepackage[sort]{natbib}
\bibliographystyle{bnuthesis-numeric}

% 著者-出版年制
% \usepackage{natbib}
% \bibliographystyle{bnuthesis-author-year}

% 本科生参考文献的著录格式
% \usepackage[sort]{natbib}
% \bibliographystyle{bnuthesis-bachelor}

% 参考文献使用 BibLaTeX 宏包
% \usepackage[style=bnuthesis-numeric]{biblatex}
% \usepackage[style=bnuthesis-author-year]{biblatex}
% \usepackage[style=gb7714-2015]{biblatex}
% \usepackage[style=apa]{biblatex}
% \usepackage[style=mla-new]{biblatex}
% 声明 BibLaTeX 的数据库
% \addbibresource{ref/refs.bib}

% 定义所有的图片文件在 figures 子目录下
\graphicspath{{figures/}}

% 数学命令
\makeatletter
\newcommand\dif{%  % 微分符号
  \mathop{}\!%
  \ifbnu@math@style@TeX
    d%
  \else
    \mathrm{d}%
  \fi
}
\makeatother

% 处理中文引号问题
\newcommand{\cqt}[1]{“#1”}  % Chinese Quote,使用正确的中文引号对

% =========================================================
% 核心命令定义(直接嵌入以确保 Overleaf 兼容性)
% =========================================================
% 污染物
\def\PM{\ensuremath{\mathrm{PM}_{2.5}}\xspace}
\def\PMten{\ensuremath{\mathrm{PM}_{10}}\xspace}
\def\ozone{\ensuremath{\mathrm{O}_{3}}\xspace}
\def\NOx{\ensuremath{\mathrm{NO}_{x}}\xspace}
\def\NOtwo{\ensuremath{\mathrm{NO}_{2}}\xspace}
\def\VOCs{\ensuremath{\mathrm{VOCs}}\xspace}
\def\CO{\ensuremath{\mathrm{CO}}\xspace}
\def\SOtwo{\ensuremath{\mathrm{SO}_{2}}\xspace}
\def\AOD{\ensuremath{\mathrm{AOD}}\xspace}

% 模型名称
\def\ModelPred{\ensuremath{\mathrm{PM}_{2.5}}\text{-GNN}\xspace}
\def\ModelSurr{PCDCNet\xspace}
\def\ModelInfer{SPIN\xspace}
\def\ModelSim{IGNN\xspace}
\def\SystemName{KnowAir\xspace}

% 损失函数
\def\LossAOD{\ensuremath{\mathcal{L}_{\mathrm{AOD}}}\xspace}
\def\LossInfer{\ensuremath{\mathcal{L}_{\mathrm{infer}}}\xspace}
\def\LossInit{\ensuremath{\mathcal{L}_{\mathrm{init}}}\xspace}
\def\LossDIC{\ensuremath{\mathcal{L}_{\mathrm{DIC}}}\xspace}

% 评价指标
\def\RMSE{\ensuremath{\mathrm{RMSE}}\xspace}
\def\MAE{\ensuremath{\mathrm{MAE}}\xspace}
\def\Rsquare{\ensuremath{R^{2}}\xspace}

% 数据源
\def\MEIC{MEIC\xspace}
\def\ERAFive{ERA5\xspace}
\def\CMAQ{CMAQ\xspace}
\def\GNN{GNN\xspace}
\def\GRU{GRU\xspace}
\def\LSTM{LSTM\xspace}

% hyperref 宏包在最后调用
\usepackage{hyperref}

% 只在"生成 PDF 书签"时生效的替换规则
\pdfstringdefDisableCommands{%
  % 你前面已经可以在这里放 \mathrm / \symcal 之类
  % \def\mathrm#1{#1}%
  % \def\symcal#1{#1}%

  % 下面是关键:告诉 hyperref,书签里这些宏长这样
  \def\PM{PM2.5}%
  \def\ozone{O3}%
  \def\ModelPred{PM2.5-GNN}%
  \def\LossAOD{Loss-AOD}%
  \def\LossInfer{Loss-Infer}%
  \def\LossInit{Loss-Init}%
}
