% ============================================================
% 第二章 物理启发的时空图神经网络建模范式
% 基于复杂系统数据驱动建模的大气污染研究
% ============================================================

\chapter{物理启发的时空图神经网络建模范式}
\label{chap:methodology}

大气污染作为典型的开放复杂巨系统,其时空演化过程涉及排放、传输、化学反应与沉降等多个物理化学子过程的耦合作用。本章系统阐述物理启发的时空图神经网络建模范式,为后续预测(第\ref{chap:prediction}章)、推断(第\ref{chap:inference}章)和模拟(第\ref{chap:simulation}章)三类核心任务提供统一的理论框架与方法论基础。

% ------------------------------------------------------------
% 2.1 图神经网络基础
% ------------------------------------------------------------
\section{图神经网络基础}
\label{sec:gnn_basics}

% 从系统科学角度引入
从系统科学的角度来看,图神经网络(Graph Neural Network, GNN)可被视为一种学习系统动力学的通用框架\citep{battaglia2018relational}。复杂系统由相互作用的实体构成,其宏观行为涌现于微观个体间的局部交互。GNN的核心思想是:首先根据问题特点构建图结构(将实体建模为节点、交互关系建模为边),然后在图上运行神经网络,通过迭代执行消息传递与聚合操作学习实体间的交互规律。神经网络提供的可学习参数(如卷积核权重、注意力系数等)使模型能够从数据中自动拟合复杂的非线性映射关系,从而逼近系统的演化动力学。这种\cqt{从局部交互到全局涌现}的建模思想与大气污染系统的物理本质高度契合——污染物浓度的时空演化正是由无数局部的排放、传输、反应与沉降过程共同决定的。

\subsection{图的基本概念}
\label{subsec:graph_concept}

图(Graph)是描述实体及其关系的数学结构,形式化定义为$\mathcal{G} = (\mathcal{V}, \mathcal{E}, \mathbf{A})$,其中$\mathcal{V} = \{v_1, v_2, \ldots, v_N\}$为节点集合,$\mathcal{E} \subseteq \mathcal{V} \times \mathcal{V}$为边集合,$\mathbf{A} \in \mathbb{R}^{N \times N}$为邻接矩阵。每个节点$v_i$附带特征向量$\mathbf{x}_i \in \mathbb{R}^{F}$,边$e_{ij}$附带边特征$\mathbf{e}_{ij}$。在大气污染问题中,节点通常对应监测站点或城市,节点特征包含污染物浓度与气象变量,边特征可编码节点间的空间距离或风场信息。具体的图构建方法将在各应用章节(第\ref{chap:prediction}--\ref{chap:simulation}章)中根据任务特点详细阐述。

\subsection{图神经网络的两大范式}
\label{subsec:gnn_paradigms}

图神经网络的发展形成了两大技术路线:基于谱的方法和基于消息传递的方法。两者各有特点,本文的模型设计综合采用了这两种范式。

\subsubsection{谱方法:图卷积网络}

谱方法从图信号处理的角度出发,在图的频谱域定义卷积操作\citep{bruna2014spectral}。其核心思想是利用图拉普拉斯矩阵$\mathbf{L} = \mathbf{D} - \mathbf{A}$(式中$\mathbf{D}$为度矩阵,$\mathbf{A}$为邻接矩阵)的特征分解进行滤波。下面通过公式\eqref{eq:spectral_conv}--\eqref{eq:gcn}展示谱方法从理论形式到实用形式的演化过程。

对于图信号$\mathbf{x} \in \mathbb{R}^N$,谱卷积的理论形式定义为:
\begin{equation}
\mathbf{y} = \mathbf{U} g_\theta(\boldsymbol{\Lambda}) \mathbf{U}^\top \mathbf{x}
\label{eq:spectral_conv}
\end{equation}
\noindent 式中,$\mathbf{y}$为输出信号,$\mathbf{x}$为输入图信号,$\mathbf{L} = \mathbf{U} \boldsymbol{\Lambda} \mathbf{U}^\top$为拉普拉斯矩阵的特征分解($\mathbf{U}$为特征向量矩阵,$\boldsymbol{\Lambda}$为特征值对角矩阵),$g_\theta(\cdot)$为可学习的频谱滤波器。

然而,公式\eqref{eq:spectral_conv}中的全特征分解计算复杂度为$O(N^3)$,难以应用于大规模图。为此,ChebNet\citep{defferrard2016convolutional}采用切比雪夫多项式近似滤波器$g_\theta(\boldsymbol{\Lambda})$,将复杂度降至$O(|\mathcal{E}|)$($|\mathcal{E}|$为边数):
\begin{equation}
\mathbf{y} = \sum_{k=0}^{K_c-1} \theta_k T_k(\tilde{\mathbf{L}}) \mathbf{x}
\label{eq:chebynet}
\end{equation}
\noindent 式中$\theta_k$为可学习的多项式系数,$T_k(\cdot)$为$k$阶切比雪夫多项式,$\tilde{\mathbf{L}} = 2\mathbf{L}/\lambda_{\max} - \mathbf{I}$为归一化拉普拉斯矩阵,$K_c$控制滤波器的阶数(即感受野大小)。

图卷积网络(Graph Convolutional Network, GCN)\citep{kipf2017semi}进一步将ChebNet简化为一阶形式(取$K_c=1$),并引入归一化技巧以稳定训练:
\begin{equation}
\mathbf{H}^{(l+1)} = \sigma\left(\tilde{\mathbf{D}}^{-\frac{1}{2}} \tilde{\mathbf{A}} \tilde{\mathbf{D}}^{-\frac{1}{2}} \mathbf{H}^{(l)} \mathbf{W}^{(l)}\right)
\label{eq:gcn}
\end{equation}
\noindent 式中,$\mathbf{H}^{(l)}$为第$l$层的节点特征矩阵,$\mathbf{H}^{(l+1)}$为第$l+1$层的输出,$\tilde{\mathbf{A}} = \mathbf{A} + \mathbf{I}$为添加自环的邻接矩阵,$\tilde{\mathbf{D}}$为对应的度矩阵,$\mathbf{W}^{(l)}$为可学习参数矩阵,$\sigma(\cdot)$为非线性激活函数(如ReLU)。经过$L$层图卷积后,最终的节点表示$\mathbf{H}^{(L)}$可通过读出层(如线性映射$\mathbf{Y} = \mathbf{H}^{(L)}\mathbf{W}_{\text{out}}$)生成预测输出。

谱方法的优势在于具有坚实的数学基础,卷积操作在频谱域具有明确的物理意义(低频对应平滑信号,高频对应突变信号);其局限性在于基于邻接矩阵或拉普拉斯矩阵进行运算,难以直接输入边属性(Edge Attributes),但近年来边条件卷积等扩展方法已逐步克服这一限制。

\subsubsection{消息传递神经网络}

与谱方法从频谱域定义卷积不同,消息传递神经网络(MPNN)\citep{gilmer2017neural}从空间域出发,通过邻居间的显式信息交互更新节点表示:
\begin{equation}
\mathbf{h}_i^{(l+1)} = \phi\left(\mathbf{h}_i^{(l)}, \bigoplus_{j \in \mathcal{N}(i)} \psi\left(\mathbf{h}_i^{(l)}, \mathbf{h}_j^{(l)}, \mathbf{e}_{ji}\right)\right)
\label{eq:message_passing}
\end{equation}
\noindent 式中$\mathbf{h}_i^{(l)}$为节点$i$在第$l$层的隐藏表示,$\mathcal{N}(i)$为节点$i$的邻居集合,$\psi(\cdot)$为消息函数,$\bigoplus$为聚合操作(如求和、均值),$\phi(\cdot)$为更新函数。该范式的物理直觉是:节点从邻居收集\cqt{消息}并聚合,再结合自身状态生成新的表示。这与大气污染物的传输机制一致——站点的污染水平受本地排放与周边传输的共同作用。

消息传递方法的优势在于灵活性强,可自然处理有向图、动态图和边特征;局限性在于缺乏全局视野,多跳依赖需堆叠多层网络。

\subsubsection{两种范式的统一视角}

从数学角度,谱方法与消息传递方法可以统一为图滤波框架:两者本质上都是对邻居信息的加权聚合,区别在于权重的定义方式——谱方法通过频谱滤波器隐式定义,消息传递方法通过显式的消息函数定义。

从计算实现角度,两种范式具有不同的适用场景。谱方法基于邻接矩阵或拉普拉斯矩阵的稠密/稀疏矩阵运算,通过矩阵乘法一次性完成所有节点的信息聚合,计算效率高,适用于节点规模较小的图(如本文的城市级网络,$N \approx 10^2$)。消息传递方法则采用边索引(edge index)表示图结构,以$[2, |\mathcal{E}|]$的张量形式存储所有边的源节点-目标节点对,逐条边计算消息并聚合。这种实现方式对稀疏图的适应性更强:当图规模增大时(如$N \approx 10^5 \sim 10^6$),稠密邻接矩阵$\mathbf{A} \in \mathbb{R}^{N \times N}$可能无法装入显存,而边索引表示的空间复杂度仅为$O(|\mathcal{E}|)$,对于稀疏图($|\mathcal{E}| \ll N^2$)具有显著优势。

本文根据不同任务的特点灵活选择两种范式:第\ref{chap:prediction}章PM$_{2.5}$-GNN采用消息传递范式,利用其对边特征的灵活建模能力将风场编码为有向边权重;第\ref{chap:prediction}章PCDCNet采用基于拉普拉斯矩阵的一阶谱卷积,第\ref{chap:simulation}章IGNN采用ChebNet的多阶滤波特性捕捉多尺度空间依赖;第\ref{chap:inference}章SPIN则结合两种范式,分别建模扩散过程(谱卷积)和平流过程(有向消息传递)。

\subsection{时空图神经网络}
\label{subsec:stgnn}

时空图神经网络(Spatiotemporal Graph Neural Network, STGNN)在空间维度的图卷积基础上,进一步引入时间维度的建模能力,是处理时空序列预测问题的有效工具\citep{yu2018spatio,li2018diffusion}。

典型的STGNN架构包含交替堆叠的空间模块与时间模块。空间模块采用上述GCN(公式\eqref{eq:gcn})或MPNN(公式\eqref{eq:message_passing})捕捉节点间的空间依赖;时间模块沿时间轴捕捉序列依赖,常用膨胀卷积(Dilated Convolution):
\begin{equation}
\mathbf{Y}_t = \sum_{s=0}^{K_c-1} \mathbf{W}_s \star \mathbf{X}_{t-d \cdot s}
\label{eq:tcn}
\end{equation}
\noindent 式中$\mathbf{Y}_t$为时刻$t$的输出,$\mathbf{X}_{t-d \cdot s}$为时刻$t-d \cdot s$的输入,$\mathbf{W}_s$为卷积核权重,$d$为膨胀因子。膨胀因子$d$通常按指数递增设置,即$d \in \{1, 2, 4, \ldots, 2^{L-1}\}$,使得$L$层TCN以线性参数量获得$\mathcal{O}(2^L)$的感受野。

综上,完整的STGNN通过空间模块(公式\eqref{eq:gcn}或\eqref{eq:message_passing})与时间模块(公式\eqref{eq:tcn})的交替堆叠,实现时空联合建模。这与大气污染系统的物理特性高度吻合——空间上相邻节点通过传输相互影响,时间上当前状态依赖历史演化。

\subsection{图神经网络的训练}
\label{subsec:gnn_training}

图神经网络的训练遵循深度学习的标准优化范式——基于梯度的反向传播算法\citep{goodfellow2016deep}。训练目标是最小化损失函数$\mathcal{L}(\Theta) = \frac{1}{M} \sum_{i=1}^{M} \ell(f_\Theta(\mathcal{G}_i), \mathbf{y}_i)$,其中$M$为训练样本数,$f_\Theta$为参数$\Theta$的GNN模型,$\mathcal{G}_i$为第$i$个输入图,$\mathbf{y}_i$为对应的标签,$\ell(\cdot)$为任务相关的损失函数。反向传播算法通过链式法则高效计算梯度,并采用Adam等优化器\citep{kingma2015adam}更新参数$\Theta^{(t+1)} = \Theta^{(t)} - \eta \nabla_\Theta \mathcal{L}$,其中$\eta$为学习率,$\nabla_\Theta \mathcal{L}$为损失函数关于参数的梯度。从系统科学角度,经过充分训练后的图神经网络,其学习到的消息传递与聚合机制可视为对真实系统中数据传播动力学的模拟——节点间的信息流动对应物理空间中物质或信号的传输过程,而训练所得的边权重与聚合函数则编码了系统的传播规律。因此,一个训练良好的GNN本质上构成了目标系统动力学过程的数据驱动模拟器,能够在给定初始状态与边界条件下,复现系统的时空演化行为。


% ------------------------------------------------------------
% 2.2 图神经网络在地球科学中的应用
% ------------------------------------------------------------
\section{图神经网络在地球科学中的应用}
\label{sec:gnn_earth_science}

近年来,图神经网络在地球科学领域取得了突破性进展,涵盖气象预报与大气污染预测两大方向。本节综述该领域的代表性工作,提炼这类方法的共性特点,为本文的模型设计提供方法论基础。

\subsection{气象预报领域}

在气象预报领域,Google DeepMind的GraphCast\citep{lam2023learning}首次实现数据驱动模型在中期天气预报精度上超越欧洲中期天气预报中心(European Centre for Medium-Range Weather Forecasts, ECMWF)业务系统;NVIDIA的FourCastNet\citep{pathak2022fourcastnet}基于傅里叶神经算子在频谱域建模全球动力学。GraphCast的核心创新在于多尺度图结构——采用\cqt{编码器--处理器--解码器}架构,在细网格(ERA5再分析数据格点,约100万节点)与粗网格(准均匀球面网格,约4万节点)间建立跨尺度连接,实现计算效率(10天全球预报仅需1分钟)与长程依赖建模的平衡。这类基于规则网格构建的稠密图结构,节点按经纬度均匀分布,适用于全域场的连续建模。本文第\ref{chap:inference}章的网格推断任务采用了类似的稠密网格图结构,在规则格点上进行空间插值与推断。

\subsection{大气污染预测领域}

在大气污染预测领域,时空图神经网络(STGNN)已成为主流技术架构\citep{zhou2020graph,wu2020comprehensive}。与气象预报的规则网格不同,空气质量监测站点在空间上呈稀疏且不规则分布——站点数量有限(通常为数百至数千个)、站点位置受城市规划与地形约束,形成天然的稀疏图结构。如第\ref{chap:introduction}章所述,这类模型通过图结构(监测站点网络)引入归纳偏置,使用图卷积刻画站点间的空间关联,配合时序卷积或长短期记忆网络(Long Short-Term Memory, LSTM)建模浓度的时间演变。代表性工作包括:STGCN\citep{yu2018spatio}建立了时空图建模的基础框架;DCRNN\citep{li2018diffusion}将信息传播建模为有向图上的扩散过程,与大气污染物的扩散传输机制天然对应;Graph WaveNet\citep{wu2019graph}引入自适应邻接矩阵学习机制,无需预定义图结构即可端到端学习节点间的隐式依赖。本文除第\ref{chap:inference}章的网格推断任务外,第\ref{chap:prediction}章(预测)和第\ref{chap:simulation}章(模拟)均采用这种基于监测站点或城市的稀疏图结构。

在物理启发融合方向,AirPhyNet\citep{hettigeairphynet}将质量守恒原理嵌入网络结构,在稀疏数据和突变场景下展现出更强的鲁棒性;Air-DualODE\citep{tianair}采用双分支架构,物理分支求解边界感知的扩散-平流方程,数据驱动分支学习额外的依赖关系,实现物理可解释性与数据拟合能力的平衡。

\subsection{方法共性与统一框架}

上述工作尽管在具体实现上各有侧重,但可抽象为统一的时空图神经网络框架:
\begin{equation}
\mathbf{H}^{(l+1)} = f_{\text{temporal}}\left(f_{\text{spatial}}\left(\mathbf{H}^{(l)}, \mathcal{G}\right)\right)
\label{eq:stgnn_general}
\end{equation}
\noindent 式中,$\mathbf{H}^{(l)} \in \mathbb{R}^{N \times T \times F}$为第$l$层的节点时空表示($N$为节点数,$T$为时间步数,$F$为特征维度),$f_{\text{spatial}}(\cdot)$为空间模块(GCN/ChebNet/MPNN),$f_{\text{temporal}}(\cdot)$为时间模块,如时间卷积网络(Temporal Convolutional Network, TCN)、门控循环单元(Gated Recurrent Unit, GRU)或Transformer,$\mathcal{G}$为图结构。

这类方法具有以下共性特点:

(1)图结构编码空间依赖。通过邻接矩阵$\mathbf{A}$编码节点间的空间关系,可基于地理距离、风场、相关性等多种先验构建。图结构的设计直接影响模型捕捉空间传输模式的能力。

(2)时空交替建模。空间模块与时间模块交替堆叠,分别捕捉空间依赖与时序演化,最终实现时空联合建模。典型架构为$L$层时空块的串联:$[\text{S-Conv} \rightarrow \text{T-Conv}]^L$。

(3)编码-处理-解码架构。遵循\cqt{编码器--处理器--解码器}的通用范式:编码器将原始输入映射到隐空间,处理器在隐空间中执行时空演化,解码器将隐表示映射回目标空间。

(4)数据驱动的算子学习。核心思想是学习从输入场到输出场的映射算子,而非显式求解偏微分方程。这种范式避免了数值离散化的稳定性限制,同时通过大规模数据驱动实现高精度预测。

基于上述共性认识,本文在后续章节中将针对大气污染问题的特点进行定制化设计:通过风场驱动的有向图建模平流传输方向性(第\ref{chap:prediction}章),通过扩散--平流双图融合两类物理传输过程(第\ref{chap:inference}章),通过非自回归映射消除长期模拟的误差累积(第\ref{chap:simulation}章)。


% ------------------------------------------------------------
% 2.3 大气污染的物理基础
% ------------------------------------------------------------
\section{大气污染的物理基础}
\label{sec:atmospheric_dynamics}

纯数据驱动的图神经网络虽然具备强大的表示学习能力,但在地球科学领域面临两个核心挑战:一是数据稀疏性导致的过拟合风险,二是预测结果可能违背物理规律(如质量守恒)。将领域知识融入深度学习模型——即物理启发的机器学习(Physics-Informed Machine Learning)——是应对这些挑战的有效途径\citep{karniadakis2021physics,reichstein2019deep}。

本节的核心问题是:大气污染建模的物理基础是什么?如何将这些物理知识转化为神经网络的归纳偏置(Inductive Bias,即学习算法从有限数据泛化时所依赖的先验假设\citep{battaglia2018relational})?

在大气污染物理领域,存在两个核心的约束方程:\textbf{对流-扩散方程}(公式\eqref{eq:advection_diffusion})描述污染物浓度场的时空演化规律,\textbf{质量守恒方程}(公式\eqref{eq:mass_conservation})作为其积分形式,约束封闭系统内污染物总量的收支平衡。化学传输模式(如CMAQ、WRF-Chem)通过数值方法直接求解这些方程来模拟污染物演化。本文的策略有所不同:不直接求解方程,而是将这两个核心约束\textbf{隐式地融入图神经网络的架构与训练过程}中——对流-扩散的动力学通过隐空间的消息传递机制体现,质量守恒约束则在解码到物理空间后,以损失函数的形式与数据拟合项并列作用。

\subsection{对流-扩散方程}
\label{subsec:advection_diffusion}

大气污染物浓度场的时空演化遵循对流-扩散方程,这是化学传输模式求解的基本控制方程:
\begin{equation}
\frac{\partial C}{\partial t} = \underbrace{-\mathbf{u} \cdot \nabla C}_{\text{平流项}} + \underbrace{\nabla \cdot (K \nabla C)}_{\text{扩散项}} + \underbrace{R}_{\text{化学反应}} + \underbrace{S}_{\text{排放源}} - \underbrace{D}_{\text{沉降汇}}
\label{eq:advection_diffusion}
\end{equation}
\noindent 式中$C$为污染物浓度,$\mathbf{u}$为风速矢量,$K$为扩散系数,$R$、$S$、$D$分别为化学反应项、排放源项和沉降汇项。该方程的各项在本文的GNN设计中均有对应实现(见表\ref{tab:pde_gnn_mapping}):

\begin{table}[htbp]
    \centering
    \caption{对流-扩散方程各项与GNN设计的对应关系}
    \caption*{对流-扩散方程的各物理项在本文提出的图神经网络模型中均有对应的结构设计,实现了物理机理与数据驱动的深度融合。}
    \label{tab:pde_gnn_mapping}
    \begin{tabular}{@{}lll@{}}
        \toprule
        方程项 & 物理含义 & GNN实现(本文章节) \\
        \midrule
        平流项$-\mathbf{u} \cdot \nabla C$ & 风场驱动的定向迁移 & 有向图消息传递(第\ref{chap:prediction}、\ref{chap:inference}章) \\
        扩散项$\nabla \cdot (K \nabla C)$ & 湍流混合的各向同性扩散 & 对称邻接矩阵/扩散核(第\ref{chap:inference}章) \\
        化学反应项$R$ & 光化学反应生成与消耗 & 节点级非线性变换(第\ref{chap:prediction}章LID模块) \\
        排放源项$S$ & 污染物排放速率 & 显式输入特征(第\ref{chap:prediction}--\ref{chap:simulation}章) \\
        沉降汇项$D$ & 干湿沉降移除过程 & 时间维度衰减(第\ref{chap:prediction}章TAD模块) \\
        \bottomrule
    \end{tabular}
\end{table}

\subsection{质量守恒约束}
\label{subsec:physics_constraints}

对流-扩散方程\eqref{eq:advection_diffusion}的积分形式即为质量守恒定律。在封闭系统内,污染物总量的变化应等于源汇之差:
\begin{equation}
\frac{\partial}{\partial t} \int_{\Omega} C \, d\Omega = \int_{\Omega} (S - D + R) \, d\Omega - \oint_{\partial\Omega} \mathbf{F} \cdot \mathbf{n} \, dS
\label{eq:mass_conservation}
\end{equation}
\noindent 式中$\Omega$为控制区域,$\partial\Omega$为区域边界,$\mathbf{F}$为通量,$\mathbf{n}$为边界法向量。

综上,公式\eqref{eq:advection_diffusion}(对流-扩散方程)与公式\eqref{eq:mass_conservation}(质量守恒方程)是大气污染物理的两个核心约束。在本文的图神经网络建模中,这两个约束被隐式地融入网络的不同层次:

对流-扩散动力学$\rightarrow$隐空间消息传递。公式\eqref{eq:advection_diffusion}描述的平流与扩散过程,通过GNN在隐空间(Latent Space)中的消息传递机制体现。具体而言,编码器(Encoder)将观测数据从物理空间映射到隐空间表示;在隐空间中,图卷积的消息传递模拟污染物在节点间的传输——平流项对应有向图上的定向传递,扩散项对应无向图上的各向同性扩散。

质量守恒$\rightarrow$物理空间损失约束。公式\eqref{eq:mass_conservation}的守恒约束则在物理空间中施加。解码器(Decoder)将隐空间表示映射回物理空间的浓度预测后,质量守恒以损失函数的形式约束模型输出——该物理约束项与数据拟合损失并列,共同引导模型学习符合物理规律的时空表示。这种设计使得守恒约束直接作用于可观测的物理量,而非抽象的隐层表示,确保约束的物理可解释性。

下一节将具体阐述物理知识融入GNN的三种实现途径。

\subsection{物理知识融入图神经网络的途径}
\label{subsec:physics_guidance}

基于上述物理方程,本文将物理知识融入GNN的途径归纳为以下三个层面:

(1)图网络设计。通过邻接矩阵$\mathbf{A}$将物理传输机制编码为图的拓扑结构。例如,基于风场构建有向边以模拟平流项的方向性(第\ref{chap:prediction}章PM$_{2.5}$-GNN),设计扩散核与平流核分别模拟公式\eqref{eq:advection_diffusion}中的扩散项与平流项(第\ref{chap:inference}章SPIN)。

(2)网络架构设计。将对流-扩散方程的物理过程解耦为独立的网络模块。例如,第\ref{chap:prediction}章PCDCNet的LID--STD--TAD三模块分别对应化学反应与排放项、平流与扩散项、沉降与累积项。

(3)损失函数设计。通过物理约束项软约束模型输出满足物理规律:
\begin{equation}
\mathcal{L} = \mathcal{L}_{\text{data}} + \lambda \mathcal{L}_{\text{physics}}
\label{eq:physics_loss}
\end{equation}
\noindent 式中$\mathcal{L}_{\text{data}}$为数据拟合项(如均方误差),衡量模型预测与观测数据之间的偏差;$\mathcal{L}_{\text{physics}}$为物理约束项,将领域物理知识以软约束的形式嵌入优化目标;$\lambda$为平衡系数,控制数据拟合与物理一致性之间的权衡。物理约束项$\mathcal{L}_{\text{physics}}$的具体形式因任务而异:第\ref{chap:prediction}章设计了领域知识约束(DIC)损失,基于公式\eqref{eq:mass_conservation}的质量守恒原理,约束相邻时间步之间空间传输贡献的时间连续性与空间一致性;第\ref{chap:inference}章则利用卫星遥感反演的AOD(气溶胶光学厚度)数据构建梯度约束,引导模型学习与遥感观测一致的空间分布模式。


% ------------------------------------------------------------
% 2.4 本文研究范式
% ------------------------------------------------------------
\section{本文研究范式}
\label{sec:our_paradigm}

前述各节介绍了图神经网络的基本原理(第\ref{sec:gnn_basics}节)、其在地球科学中的前沿应用(第\ref{sec:gnn_earth_science}节)以及大气污染动力学的物理基础(第\ref{sec:atmospheric_dynamics}节)。本节在此基础上,阐述\textbf{本文提出的物理启发时空图神经网络研究范式},为后续三个应用章节奠定方法论框架。

\subsection{核心思想:物理启发的时空图神经网络}
\label{subsec:core_idea}

本文提出的研究范式以时空图神经网络为核心建模工具,通过在其架构中嵌入大气科学的领域知识,实现\cqt{数据拟合}与\cqt{物理一致}的双重目标。选择GNN源于其与大气污染系统的天然契合——监测站点构成图的节点,站点间的空间关联构成图的边,污染物的时空演化可建模为图上的信号传播过程。根据融入方式的不同,物理知识可在数据、模型、损失三个环节嵌入\citep{willard2022integrating}:数据环节通过图结构编码物理先验(如风场有向图),模型环节通过网络架构嵌入物理结构(如平流/扩散解耦),损失环节通过物理惩罚项约束模型输出(如质量守恒)。

\subsection{统一建模框架}
\label{subsec:unified_framework}

基于上述理论基础,本文构建了面向预测、推断与模拟三类核心任务的统一建模框架,如图\ref{fig:unified_framework}所示。该框架将物理启发的时空图神经网络抽象为参数化映射$\mathcal{F}_\Theta$,通过端到端的梯度优化从数据中学习模型参数$\Theta$。尽管三类任务在具体形式上有所差异,但均可在统一的\cqt{编码$\rightarrow$隐空间动力学$\rightarrow$解码}架构下表达。

\begin{figure}[htbp]
    \centering
    \includegraphics[width=0.95\textwidth]{figures/chap02_unified_framework.pdf}
    \caption{物理启发的时空图神经网络统一框架}
    \caption*{蓝色模块为深度学习流水线(Pipeline),橙色标注为本文的核心创新——在编码层、隐空间动力学、解码层三个层次融入先验知识。模型通过损失函数$\mathcal{L}$计算预测结果$\hat{\mathbf{X}}$与观测数据的差异,并通过反向传播更新参数$\Theta$,实现数据驱动的端到端优化。}
    \label{fig:unified_framework}
\end{figure}

如图\ref{fig:unified_framework}所示,该框架包含三个核心模块,每个模块均可融入物理启发的设计:

编码层(Encoder)——将多源异构输入映射到高维隐空间。输入包括污染物历史浓度$\mathbf{X}$、气象变量$\mathbf{M}$(风速、风向、温度、湿度等)以及排放数据$\mathbf{E}$。编码层负责特征提取、异构数据融合与图结构嵌入。如图中橙色标注所示,物理启发可通过图网络设计实现——例如基于风场信息构建有向图,将大气传输的方向性先验编码到图的拓扑结构中。

隐空间动力学(Latent Dynamics)——在隐空间中学习污染物的时空演化规律,是模型的核心计算模块。如图所示,该模块包含空间模块与时间模块的交互建模:空间模块采用消息传递机制(Message Passing)或图卷积网络(GCN/ChebNet)捕获站点间的空间依赖;时间模块采用循环网络(LSTM/GRU)或时序卷积(TCN)建模时间演化。物理启发可通过图算子设计实现——例如设计扩散核(对称,模拟湍流扩散)与平流核(非对称,模拟风场输送),使网络的信息传播机制与物理传输过程相对应。

解码层(Decoder)——将隐空间表示映射回目标空间,输出预测、推断或模拟结果$\hat{\mathbf{X}}$。物理约束可通过损失函数嵌入——在数据拟合损失之外添加物理惩罚项(如基于质量守恒的DIC约束、基于遥感反演的AOD梯度约束),引导模型学习符合物理规律的时空表示。

图\ref{fig:unified_framework}右侧展示了模型的训练机制:损失函数$\mathcal{L}$综合数据拟合项与物理约束项,通过比较模型输出$\hat{\mathbf{X}}$与观测数据(Ground Truth)计算误差,并通过反向传播算法计算梯度、更新模型参数$\Theta$。这种端到端的梯度优化机制是本文方法区别于数值模式的关键——数值模式通过求解偏微分方程获得解析解,而本文方法通过数据驱动的参数学习自适应拟合观测数据,同时通过物理约束保证结果的物理一致性。

基于该统一框架,本文针对三类任务设计了具体模型:

预测任务(第\ref{chap:prediction}章):给定历史观测、未来气象和排放数据,预测未来污染物浓度。本文提出PM$_{2.5}$-GNN与PCDCNet模型,通过DIC损失显式约束时空连续性。

推断任务(第\ref{chap:inference}章):给定稀疏监测站点观测和辅助信息(AOD、气象),推断全域连续浓度场。本文提出SPIN模型,通过基于物理启发的扩散核与平流核设计增强空间插值能力。

模拟任务(第\ref{chap:simulation}章):给定未来气象情景和假设排放情景,模拟对应的浓度响应。本文提出IGNN模型,通过大规模数据驱动隐式学习排放-浓度响应关系。

三类任务共享相同的\cqt{编码$\rightarrow$隐空间$\rightarrow$解码}框架(图\ref{fig:unified_framework}蓝色模块),但物理启发的嵌入方式因任务特点而异(图\ref{fig:unified_framework}橙色标注)。这种灵活的物理启发机制使得统一框架能够适应不同任务的特定需求,同时保持方法论上的一致性。

\subsection{与现有方法的对比}
\label{subsec:comparison}

表\ref{tab:paradigm_comparison}对比了本文研究范式与现有方法的主要区别。

\begin{table}[htbp]
    \centering
    \caption{本文研究范式与现有方法的对比}
    \caption*{本文提出的融合物理启发的图神经网络建模范式兼具数值模式的物理基础和数据驱动方法的计算效率,在泛化能力和可解释性之间实现平衡。}
    \label{tab:paradigm_comparison}
    \begin{tabular}{@{}lp{2.8cm}p{2.8cm}p{3cm}@{}}
        \toprule
        \textbf{维度} & \textbf{数值模式} & \textbf{纯数据驱动} & \textbf{本文范式} \\
        \midrule
        物理基础 & 显式PDE求解 & 无 & 隐式物理启发 \\
        计算效率 & 低(小时级) & 高(秒级) & 高(秒级) \\
        数据需求 & 初边值条件 & 大量历史数据 & 中等+物理先验 \\
        泛化能力 & 强(物理外推) & 弱(分布内) & 中等(物理启发) \\
        可解释性 & 强 & 弱 & 中等 \\
        \bottomrule
    \end{tabular}
\end{table}

数值模式(如CMAQ、WRF-Chem)基于第一性原理求解偏微分方程,具有完备的物理机理,但计算代价高昂;纯数据驱动方法(如标准LSTM)计算高效,但缺乏物理启发,泛化能力有限。本文提出的融合物理启发的图神经网络建模范式取两者之长——保持深度学习的计算效率,同时通过多层次物理启发增强模型的物理一致性与泛化能力。


% ------------------------------------------------------------
% 2.5 本章小结
% ------------------------------------------------------------
\section{本章小结}
\label{sec:method_summary}

本章系统阐述了物理启发的时空图神经网络建模范式,为后续三个应用章节奠定理论基础。首先介绍了图神经网络的基础理论(谱方法与消息传递范式)及其在地球科学中的前沿应用;然后阐述了对流-扩散方程的物理含义与质量守恒约束,并从图结构、网络架构、损失函数和输入特征四个层面说明了物理知识融入GNN的具体途径;最后提出了\cqt{融合物理启发的图神经网络建模}研究范式,构建了预测、推断、模拟三类任务的统一框架(图\ref{fig:unified_framework})。本章所建立的方法论框架为后续章节的具体模型设计提供了统一的理论视角。
