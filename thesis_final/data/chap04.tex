% ============================================================
% 第四章 基于多源数据融合的大气污染空间推断
% 基于复杂系统数据驱动建模的大气污染研究
% ============================================================

\chapter{基于多源数据融合的大气污染空间推断}
\label{chap:inference}

上一章解决了已知监测站点的时间预测问题,本章则聚焦于更具挑战性的空间推断问题——如何基于稀疏的地面监测和多源遥感数据,重建全域高分辨率的污染物浓度场。本章提出SPIN模型,通过物理启发的归纳式图神经网络实现对无监测区域的精确推断。

% ------------------------------------------------------------
% 4.1 引言
% ------------------------------------------------------------
\section{引言}
\label{sec:infer_intro}

正如第\ref{chap:introduction}章所述,大气污染对公众健康构成严重威胁,而精准的健康风险评估依赖于高时空分辨率的污染物分布数据。在环境流行病学研究中,暴露评估的准确性直接决定了健康效应估计的可靠性;在区域联防联控决策中,精细化的污染分布图是识别重点管控区域的基础依据;在公众健康服务中,无缝覆盖的空气质量信息能够帮助敏感人群规避高暴露风险区域。

然而,地面监测站点的覆盖范围有限,难以满足上述精细化空间分析的需求。以中国为例,尽管已建成全球规模最大的空气质量监测网络,但监测站点主要分布在城市建成区,广大农村地区、山区和边远地区仍存在大面积监测盲区。这种空间覆盖的不均衡性导致了\cqt{城市-农村}污染暴露评估的系统性偏差,构成当前环境健康研究的主要瓶颈。

空间推断任务的目标是:基于稀疏、不规则分布的地面监测站点数据,结合多源异构信息(气象场、排放清单、卫星遥感),对空间中任意未观测位置的污染物浓度进行精确推断。与第\ref{chap:prediction}章的预测任务不同,推断任务解决的是\cqt{未知位置的当前状态估计}问题,两者共享相同的物理基础——对流--扩散方程,但推断任务需要解决空间泛化问题。

核心挑战与现有方法局限。实现高精度的空间推断面临以下三方面挑战:

(1)监测网络稀疏与空间泛化。现有监测网络(约1618个国控站点)虽然规模庞大,但超过80\%的人口仍生活在缺乏直接监测的区域\citep{southerland2022global,wei2023first}。监测站点主要集中在城市建成区,广大农村地区、山区和边远地区存在大面积监测盲区。传统空间插值方法(如克里金、反距离加权)假设空间平稳性,难以处理站点分布不均匀、地形与气象条件差异显著的复杂场景。如何基于稀疏的已知观测,对空间中任意未观测位置进行精确推断,并具备归纳式泛化能力——即推广至训练时未见的位置——是首要挑战。

(2)遥感约束数据的大面积缺失。卫星遥感的气溶胶光学厚度(AOD)与\PM 具有明确的物理关联(柱积分气溶胶光学特性),可为无观测区域提供空间结构线索。然而,卫星AOD受云层遮挡影响严重,在重污染高发的冬季缺失率可超过70\%。现有方法多将AOD作为模型的强制输入特征,在AOD缺失时推断质量急剧下降。如何有效利用AOD的空间分布信息作为约束,同时在AOD不可用时自动回退至物理驱动模式以实现全天候推断,是数据融合的核心挑战。

(3)大气动力过程的显式建模。空间推断并非静态的空间插值——污染物浓度场受对流--扩散方程(公式\ref{eq:advection_diffusion})支配,呈现显著的时间演化特征。风场驱动的平流传输使得上风向站点的历史观测对下风向无观测区域具有重要参考价值。因此,推断模型需要以历史时间窗口(而非单一时刻快照)作为输入,通过时序特征提取捕捉动力过程的时空耦合效应。如何将物理传输机制显式嵌入推断框架,是保证推断结果物理合理性的关键挑战。

针对上述挑战,本章提出SPIN(Spatiotemporal Physics-Guided Inference Network)模型,通过四阶段推断流程系统性地加以解决:动态节点掩码赋予模型归纳式泛化能力(应对挑战一);时间卷积网络从历史时间窗口编码本地物理特征(应对挑战三);物理启发的扩散--平流双核空间传播将观测信息传递至无观测区域(应对挑战一与三);掩码AOD梯度约束将遥感数据作为空间结构约束而非强制输入,实现全天候推断(应对挑战二)。


% ------------------------------------------------------------
% 4.2 问题定义
% ------------------------------------------------------------
\section{问题定义}
\label{sec:infer_problem}

本章研究的核心问题可以形式化定义为归纳式时空克里金(Inductive Spatiotemporal Kriging)——利用稀疏的监测数据和辅助物理场,对空间中任意未观测位置的\PM 浓度进行推断。本章以\PM 为研究对象,原因在于:(1)\PM 是当前中国空气质量达标的主要制约因子,具有最迫切的精细化空间评估需求;(2)\PM 与卫星AOD具有明确的物理关联(柱积分气溶胶),使得遥感约束机制在物理上自洽。图\ref{fig:inference_problem}展示了该问题的整体框架。

\begin{figure}[htbp]
    \centering
    \includegraphics[width=\textwidth]{figures/chap04_inference_problem.pdf}
    \caption{大气污染空间推断问题示意图}
    \caption*{输入包括观测站点的历史时序、全域气象场、排放场和卫星AOD数据;输出对应两个子任务——站点推断(图$\mathcal{G}^{\text{s}}$,蓝色曲线)和网格推断(图$\mathcal{G}^{\text{g}}$,红色热力图)。两个子任务采用相同的SPIN架构但各自独立训练。图中黑色实心圆点为观测站点,空心圆点为待推断的目标站点,斜线阴影区域表示AOD缺失区域,虚线圆圈示意图神经网络的多跳感受野。详细说明见正文。}
    \label{fig:inference_problem}
\end{figure}

如图所示,空间推断问题包含\textbf{两个子任务},它们共享相同的输入但产生不同粒度的输出。\textbf{输入}(两个子任务共享)包括四类数据:观测站点的历史时序数据$\mathbf{X}^{\text{obs}}_{t-T'+1:t}$(图中黑色圆点表示的观测节点)、全域气象场$\mathbf{M}^{F}_{t-T'+1:t}$(包括风速风向、温度、湿度、边界层高度等)、排放场$\mathbf{E}^{F}_{t-T'+1:t}$以及卫星遥感AOD数据$\mathbf{X}^{\text{AOD}}_{t-T'+1:t}$(图中斜线区域表示缺失)。\textbf{输出}对应两个不同的子任务:(1)\textbf{站点推断}——在站点图$\mathcal{G}^{\text{s}}$上推断目标站点$\mathcal{V}_{\text{target}}$的\PM 浓度时序$\hat{\mathbf{X}}^{\text{target}}_{t-T'+1:t}$(图中蓝色曲线),解决离散站点间的空间插值问题;(2)\textbf{网格推断}——在网格图$\mathcal{G}^{\text{g}}$上推断全域所有$N^{\text{g}}$个$0.25^{\circ}$网格像素的\PM 浓度分布$\hat{\mathbf{X}}^{F}_t \in \mathbb{R}^{N^{\text{g}}}$(图中红色热力图),将离散站点观测扩展为覆盖整个研究区域的连续浓度场。两个子任务虽然输出粒度不同(离散站点 vs.\,全域网格),但通过统一的SPIN架构加以解决。

与第\ref{chap:prediction}章的\textbf{时间维度预测}不同,本章聚焦于\textbf{空间维度推断}。类似于第\ref{chap:prediction}章,模型同样采用历史时间窗口作为输入:给定任意时刻$t$之前$T'$个时间步($t-T'+1, \ldots, t$)观测站点的历史数据,推断该时刻$t$未观测位置的浓度值。时间窗口的引入使模型能够捕捉污染物的时序演化特征,从而提升空间推断的准确性。这种设计使得模型能够同时服务于历史重建(用于暴露评估)和实时监测(用于填补监测盲区)两类应用场景。

该问题包含两个层次化的子任务,分别对应两套不同粒度的图结构:

(1)\textbf{站点推断(Station Inference)}。在站点图$\mathcal{G}^{\text{s}} = (\mathcal{V}^{\text{s}}, \mathcal{E}^{\text{s}})$上进行,其中$\mathcal{V}^{\text{s}}$为监测站点节点集合($|\mathcal{V}^{\text{s}}| = N^{\text{s}} = 152$,如图~\ref{fig:study_area}(b)所示),$\mathcal{E}^{\text{s}}$为站点间的边集合。推断保留测试集(Held-out)中未观测站点的时序数据,用于评估模型的插值泛化能力。这一任务模拟的场景是:当新建一个监测站点时,如何基于周边已有站点的观测数据估算其污染水平。

(2)\textbf{网格推断(Grid Inference)}。在网格图$\mathcal{G}^{\text{g}} = (\mathcal{V}^{\text{g}}, \mathcal{E}^{\text{g}})$上进行,其中$\mathcal{V}^{\text{g}}$为$0.25^{\circ}$网格像素节点集合($|\mathcal{V}^{\text{g}}| = N^{\text{g}}$,如图~\ref{fig:study_area}(d)所示),$\mathcal{E}^{\text{g}}$为网格像素间的边集合。对区域内所有网格点进行推断,生成连续的高分辨率污染场$\hat{\mathbf{X}}^{F}_t$,以消除监测盲区。这一任务的应用场景包括:生成空间连续的污染暴露评估图、识别缺乏监测的污染热点、支撑精细化的区域联防联控决策。

需要强调的是,虽然两个子任务使用不同粒度的图$\mathcal{G}^{\text{s}}$和$\mathcal{G}^{\text{g}}$,但它们采用\textbf{完全相同的模型架构}和\textbf{图核构建方法}(扩散核公式\ref{eq:diffusion_adjacency}--\ref{eq:diffusion_normalized}与平流核公式\ref{eq:advection_adjacency}--\ref{eq:advection_normalized}),仅在输入节点集合、训练数据与损失函数上有所区别(网格推断额外引入AOD空间约束损失),且各自独立训练。由于节点集合不同,两套图上所得的邻接矩阵维度不同:在$\mathcal{G}^{\text{s}}$上$\tilde{A}^{\mathcal{D}}, \tilde{A}^{\mathcal{A}} \in \mathbb{R}^{N^{\text{s}} \times N^{\text{s}}}$,在$\mathcal{G}^{\text{g}}$上$\tilde{A}^{\mathcal{D}}, \tilde{A}^{\mathcal{A}} \in \mathbb{R}^{N^{\text{g}} \times N^{\text{g}}}$。这种设计体现了SPIN的核心思想:同一物理传输架构可在不同空间尺度上复用。

表\ref{tab:task_comparison}对比了两个子任务的主要特征与差异。

\begin{table}[htbp]
    \centering
    \caption{站点推断与网格推断任务对比}
    \caption*{两个子任务采用相同的SPIN架构和图核构建方法,但在节点集合、输出粒度和损失函数上有所区别。}
    \label{tab:task_comparison}
    \begin{tabular}{@{}lll@{}}
        \toprule
        特征 & 站点推断 & 网格推断 \\
        \midrule
        图结构 & $\mathcal{G}^{\text{s}}$ & $\mathcal{G}^{\text{g}}$ \\
        节点数 & $N^{\text{s}}=152$ & $N^{\text{g}}$(全域网格) \\
        输出类型 & 时序浓度$\hat{\mathbf{X}}^{\text{target}}_{t-T'+1:t}$ & 空间场$\hat{\mathbf{X}}^{F}_t$ \\
        应用场景 & 离散站点插值 & 连续空间制图 \\
        AOD约束 & 无 & 有($\mathcal{L}_{\mathrm{AOD}}$) \\
        参数 & $\Theta^{\text{s}}$ & $\Theta^{\text{g}}$ \\
        \bottomrule
    \end{tabular}
\end{table}

形式化地,以$\mathcal{G} = (\mathcal{V}, \mathcal{E})$统一表示两类图结构($\mathcal{G} \in \{\mathcal{G}^{\text{s}}, \mathcal{G}^{\text{g}}\}$),其中$\mathcal{V}$为节点集合($|\mathcal{V}| = N$),$\mathcal{E}$为边集合。我们将节点动态划分为观测节点集$\mathcal{V}_{\text{obs}}$和目标推断节点集$\mathcal{V}_{\text{target}}$。给定观测节点过去$T'$个时间步的历史污染数据$\mathbf{X}^{\text{obs}}_{t-T'+1:t}$,以及全域气象场$\mathbf{M}^{F}_{t-T'+1:t}$和排放场$\mathbf{E}^{F}_{t-T'+1:t}$,两个子任务分别学习映射$\mathcal{F}^{\text{s}}_{\Theta^{\text{s}}}$与$\mathcal{F}^{\text{g}}_{\Theta^{\text{g}}}$:
\begin{align}
    \text{站点推断:} \quad \hat{\mathbf{X}}^{\text{target}}_{t-T'+1:t} &= \mathcal{F}^{\text{s}}_{\Theta^{\text{s}}}(\mathbf{X}^{\text{obs}}_{t-T'+1:t}, \mathbf{M}^{\text{s}}_{t-T'+1:t}, \mathbf{E}^{\text{s}}_{t-T'+1:t}, \mathcal{G}^{\text{s}}) \label{eq:infer_problem} \\
    \text{网格推断:} \quad \hat{\mathbf{X}}^{F}_t &= \mathcal{F}^{\text{g}}_{\Theta^{\text{g}}}(\mathbf{X}^{\text{obs}}_{t-T'+1:t}, \mathbf{M}^{F}_{t-T'+1:t}, \mathbf{E}^{F}_{t-T'+1:t}, \mathcal{G}^{\text{g}}) \label{eq:infer_problem_grid}
\end{align}
\noindent 其中站点级气象与排放数据$\mathbf{M}^{\text{s}}, \mathbf{E}^{\text{s}}$通过对全域网格场$\mathbf{M}^{F}, \mathbf{E}^{F}$按各站点经纬度进行空间索引(Spatial Indexing)得到,网格推断则直接使用完整的全域场$\mathbf{M}^{F}, \mathbf{E}^{F}$。
\noindent 式中$\mathcal{F}^{\text{s}}_{\Theta^{\text{s}}}$与$\mathcal{F}^{\text{g}}_{\Theta^{\text{g}}}$具有相同的网络架构(即四阶段流水线),但各自独立训练,$\Theta^{\text{s}}$与$\Theta^{\text{g}}$分别为两套独立的模型参数。两者的核心差异在于:网格推断额外引入了AOD空间约束损失$\mathcal{L}_{\mathrm{AOD}}$以保证无观测区域推断结果的物理合理性。如图~\ref{fig:inference_problem}所示:站点推断(公式\ref{eq:infer_problem})在站点图$\mathcal{G}^{\text{s}}$上推断目标站点$\mathcal{V}_{\text{target}}$在$T'$个时间步上的浓度时序$\hat{\mathbf{X}}^{\text{target}}_{t-T'+1:t}$(图中蓝色曲线);网格推断(公式\ref{eq:infer_problem_grid})在网格图$\mathcal{G}^{\text{g}}$上推断全域所有$N^{\text{g}}$个网格像素在时刻$t$的浓度分布$\hat{\mathbf{X}}^{F}_t \in \mathbb{R}^{N^{\text{g}}}$(图中红色热力图)。

与第\ref{chap:prediction}章的预测问题(公式\eqref{eq:pred_problem})相比,本章的关键区别在于:目标节点$\mathcal{V}_{\text{target}}$本身没有任何历史观测数据,模型必须完全依靠空间传播和环境上下文来进行推断。这一区别决定了本章需要采用归纳式学习范式,而非第\ref{chap:prediction}章的直推式范式。

在网格推断任务中,一个关键挑战是如何保证广阔无观测区域内推断结果的物理合理性。虽然卫星AOD能提供空间指导,但存在严重的缺失问题。因此,本框架将AOD数据$\mathbf{X}^{\text{AOD}}$作为训练时的空间结构约束(Spatial Structural Constraint),而非直接输入,从根本上解决了传统方法对卫星数据的过度依赖问题。


% ------------------------------------------------------------
% 4.3 研究区域与多源数据
% ------------------------------------------------------------
\section{研究区域与多源数据}
\label{sec:infer_data}

高质量、多源异构数据的融合是实现精准时空推断的基础。本节将详细阐述研究区域的选取依据,并系统介绍构建\cqt{地面--气象--排放--遥感}多模态数据集的过程。

\subsection{研究区域概况}
\label{subsec:study_area}

本研究选取京津冀及周边地区作为核心研究区域,共包含152个国控监测站点(区域定义详见第\ref{subsec:aq_data}节)。该区域覆盖了\cqt{2+26}城市大气污染传输通道\citep{lu2021estimation},总面积约43万平方公里,是我国大气污染防治的重点区域,如图~\ref{fig:study_area}所示。

选择该区域作为研究试验台(Testbed)主要基于以下考量:

(1)地形的高度异质性。该区域西倚太行山脉,北枕燕山山脉,东临渤海,呈半封闭的盆地状地形。正如第\ref{subsec:complex_system}节所述,这种复杂的地貌特征导致了独特的中尺度气象模式,如山前阻挡效应和海陆风环流,极易造成污染物的局地累积与回流\citep{zhong2018feedback}。太行山脉对西部气流的阻挡作用尤为显著,导致山前地带(石家庄、邯郸等城市)成为污染累积的高发区\citep{quan2020regional}。

(2)污染梯度的显著性。区域内包含了北京、天津等超大城市,同时也分布着大量重工业基地(如唐山、邯郸的钢铁产业)和广阔的农业农村地区。这种城乡二元结构和密集的工业布局,使得该区域内的污染物浓度在空间上呈现出极陡峭的梯度变化(Sharp Gradients)\citep{chen2020influence}。研究表明,相邻城市之间的\PM 浓度差异可达50 $\mu$g/m$^3$以上,对推断模型的空间解析能力提出了极大挑战。

(3)传输通道的典型性。\cqt{南风输送型}重污染过程在该区域频繁发生:污染气团从河北中南部沿太行山东麓北上,经石家庄、保定至北京,形成跨越数百公里的\cqt{污染传输带}\citep{yin2025regional}。这种定向传输特征为验证平流核的有效性提供了理想的实验条件。

\begin{figure}[htbp]
    \centering
    \includegraphics[width=0.90\textwidth]{figures/chap04_study_area.pdf}
    \caption{京津冀及周边地区研究区域与任务设置}
    \caption*{(a) 区域内的\cqt{2+26}城市分布;(b) 站点图$\mathcal{G}^{\text{s}}$($N^{\text{s}} = 152$个节点):基于地理邻近性($<$200 km)构建的站点级空间图结构(邻接矩阵$A^{\text{s}} \in \mathbb{R}^{N^{\text{s}} \times N^{\text{s}}}$),节点为监测站点,连边表示地理距离阈值内的空间邻接关系,用于问题一(站点推断);(c) $N^{\text{s}}$个地面监测站点的稀疏分布;(d) 网格图$\mathcal{G}^{\text{g}}$($N^{\text{g}}$个节点):用于网格推断的$0.25^{\circ}$标准网格(邻接矩阵$A^{\text{g}} \in \mathbb{R}^{N^{\text{g}} \times N^{\text{g}}}$),图构建规则与(b)相同,但节点为网格像素,用于问题二(网格推断)。两套图$\mathcal{G}^{\text{s}}$与$\mathcal{G}^{\text{g}}$采用相同的扩散核$\tilde{A}^{\mathcal{D}}$和平流核$\tilde{A}^{\mathcal{A}}$的构建方法,但各自独立训练(参数不共享),仅节点集合与矩阵维度不同。}
    \label{fig:study_area}
\end{figure}


\subsection{多源异构数据集构建}
\label{subsec:multimodal_data}

为了全面捕捉\PM 的时空演变规律,我们构建了一个包含地面观测、气象再分析、排放清单和卫星遥感的多源异构数据集。所有数据的时间跨度为2018年1月1日至2023年12月31日。为了支持网格推断任务,所有数据均经过时空对齐处理,统一为小时级时间分辨率,并插值到$0.25^{\circ} \times 0.25^{\circ}$的标准网格上。各类数据的详细规格如表~\ref{table:data_summary}所示。

\begin{table}[htbp]
\centering
\caption{京津冀及周边地区多源数据集汇总}
\caption*{数据集涵盖地面空气质量监测(CNEMC)、气象再分析(ERA5)、排放清单(MEIC)和卫星遥感(Himawari-8)四类数据源,时间跨度为2018--2023年。}
\label{table:data_summary}
\small
\begin{tabular}{@{}llllcc@{}}
\toprule
\textbf{类型} & \textbf{变量} & \textbf{符号} & \textbf{单位} & \textbf{分辨率} & \textbf{来源} \\
\midrule
\textbf{目标($\mathbf{X}$)} & \PM & $X_{\text{PM}_{2.5}}$ & $\mu$g/m$^3$ & 1h & CNEMC \\
\midrule
\multirow{5}{*}{\textbf{气象($\mathbf{M}$)}}
 & 气温/露点 & $M_{\text{t2m}}, M_{\text{d2m}}$ & K & 1h & \multirow{5}{*}{ERA5} \\
 & 湿度/气压 & $M_{\text{rh}}, M_{\text{sp}}$ & \%, Pa & 1h & \\
 & 降水/边界层 & $M_{\text{tp}}, M_{\text{blh}}$ & m & 1h & \\
 & 风速 & $M_{u}, M_{v}$ & m/s & 1h & \\
 & 短波辐射 & $M_{\text{rad}}$ & W/m$^2$ & 1h & \\
\midrule
\multirow{2}{*}{\textbf{排放($\mathbf{E}$)}}
 & 气态前驱物 & $E_{\text{NO}_x}$等 & ton & 月$\to$1h & \multirow{2}{*}{MEIC} \\
 & 颗粒物 & $E_{\text{PM}_{2.5}}$等 & ton & 月$\to$1h & \\
\midrule
\textbf{辅助} & AOD & $\mathbf{X}^{\text{AOD}}$ & - & 1h & Himawari-8 \\
\bottomrule
\end{tabular}
\end{table}


\subsubsection{地面空气质量监测数据($\mathbf{X}$)}

地面观测数据(记为$\mathbf{X} \in \mathbb{R}^{N \times 1}$,本章以\PM 为研究对象)是模型训练和验证的\cqt{真值(Ground Truth)}。我们从中国环境监测总站(CNEMC)获取了该区域内152个国家级空气质量自动监测站点的实时小时浓度数据。为了保证数据质量,执行了严格的质量控制流程:剔除因设备故障导致的连续缺失值超过24小时的记录,去除极值异常点(如$\PM > 1000$ $\mu$g/m$^3$),并对短期缺失($<$6小时)采用线性插值填补。最终保留的数据集涵盖了城市中心、近郊及部分背景站点,构成了稀疏但高精度的观测网络。

如图~\ref{fig:study_area}(c)所示,监测站点的空间分布呈现明显的城市偏向性——站点密集于北京、天津等大城市,而广大农村地区几乎没有覆盖。这种分布不均匀性正是本章研究的出发点:如何利用城市区域的稠密观测来推断农村地区的污染水平。


\subsubsection{气象驱动数据($\mathbf{M}$)}

气象条件(记为$\mathbf{M} \in \mathbb{R}^{N \times D_M}$)是决定污染物扩散、输送和沉降的关键动力学因素。本研究采用欧洲中期天气预报中心(ECMWF)发布的ERA5再分析资料\citep{hersbach2020era5}。相比于传统的地面气象站点观测,ERA5提供了全覆盖的格点化数据,能够更好地描述区域气象场的空间连续性。

选取的气象变量与第\ref{chap:prediction}章一致,涵盖了影响\PM 演化的关键物理过程:风场向量($u_{100}, v_{100}$)直接决定污染物的水平平流输送方向与强度;边界层高度(BLH)决定污染物垂直扩散的混合层体积;降水(TP)与相对湿度(RH)反映湿沉降清除机制和吸湿性增长过程\citep{tai2010correlations}。

值得注意的是,对于推断任务,气象数据在所有节点(包括目标节点)上都是可用的——这与目标节点缺乏历史污染观测形成鲜明对比。这一特点为模型提供了重要的物理上下文信息,使得即使在完全无观测的位置,模型也能基于当地的气象条件进行合理推断。


\subsubsection{排放清单数据($\mathbf{E}$)}

人为排放(记为$\mathbf{E} \in \mathbb{R}^{N \times D_E}$)是大气污染的物质基础。本研究使用了清华大学开发的中国多尺度排放清单模型(MEIC)\citep{zheng2018trends,li2017anthropogenic}。该清单提供了包括\NOx($E_{\text{NO}_x}$)、VOC($E_{\text{VOC}}$)、\SOtwo($E_{\text{SO}_2}$)、\NHthree($E_{\text{NH}_3}$)以及一次\PM($E_{\text{PM}_{2.5}}$)等主要污染物的月度网格化排放数据。

由于月度数据的时效性不足以支撑小时级推断,采用与第\ref{chap:prediction}章相同的时间分配方法\citep{inventory},将月度排放总量降尺度为小时分辨率(详见第\ref{subsec:emis_data}节)。

排放数据在推断任务中的角色与预测任务类似:通过提供污染物的\cqt{源强}信息,帮助模型学习\cqt{高排放区域污染水平通常较高}的关联规律。这一信息对于推断缺乏监测的工业区域尤为重要。


\subsubsection{卫星遥感AOD数据($\mathbf{X}^{\text{AOD}}$)}
\label{subsubsec:aod_data}

卫星遥感数据(记为$\mathbf{X}^{\text{AOD}} \in \mathbb{R}^{N}$,附带有效性掩码$\boldsymbol{\omega}^{\text{AOD}} \in \{0,1\}^{N}$)是弥补地面监测空间盲区的关键信息源。本研究采用日本气象厅(JMA)发射的新一代地球静止气象卫星葵花8号(Himawari-8)搭载的高级葵花成像仪(AHI)所反演的气溶胶光学厚度产品\citep{bessho2016introduction}。

相比于MODIS等极轨卫星每天仅能提供1--2次观测,作为静止卫星的Himawari-8具有极高的时间分辨率,能够对同一区域进行连续观测(全盘扫描周期为10分钟)\citep{wei2021himawari}。本研究使用的是L2级气溶胶产品,其能够反映整层大气柱内的气溶胶消光能力,与近地面颗粒物浓度具有较强的正相关性,能够清晰地展示污染气团的空间形态和传输路径。

然而,如第\ref{sec:infer_intro}节所述,AOD数据存在以下局限性:(1)云层遮挡导致非随机缺失——在AHI的反演算法中,被判定为有云的像素将被直接标记为无效值,夜间也完全不可用;(2)地表反射率干扰——在冬季的北方地区,地表积雪的高反射率会与大气信号混淆;(3)垂直分布解耦——当存在高空传输层时,高AOD值并不一定代表近地面高污染。

鉴于上述特性,若将AOD作为模型的输入特征,会导致模型在云覆盖区域因特征缺失而瘫痪。因此,本研究采取了全新的策略:不将AOD作为输入,而是将其作为训练时的空间结构约束。我们对原始AOD数据进行了严格的质量控制,剔除置信度较低的像素,并生成对应的有效性掩码矩阵$\boldsymbol{\omega}^{\text{AOD}}$,使得损失函数能够自动忽略无效区域。

\subsection{图结构构建($\mathcal{G}^{\text{s}}$与$\mathcal{G}^{\text{g}}$)}
\label{subsec:spin_graph}

在进入推断流水线之前,需要为两个子任务分别构建空间传播的基础图结构。两套图采用相同的连边规则,仅节点集合不同。

\textbf{站点图$\mathcal{G}^{\text{s}} = (\mathcal{V}^{\text{s}}, \mathcal{E}^{\text{s}})$}。如图~\ref{fig:study_area}(b)所示,节点集$\mathcal{V}^{\text{s}}$为$N^{\text{s}} = 152$个国控监测站点,连边基于站点间的测地距离构建(邻接矩阵$A^{\text{s}} \in \mathbb{R}^{N^{\text{s}} \times N^{\text{s}}}$),权重反映站点间的空间关联强度。

\textbf{网格图$\mathcal{G}^{\text{g}} = (\mathcal{V}^{\text{g}}, \mathcal{E}^{\text{g}})$}。节点集$\mathcal{V}^{\text{g}}$为覆盖研究区域的$N^{\text{g}}$个$0.25^{\circ}$网格像素,连边规则与站点图完全相同(邻接矩阵$A^{\text{g}} \in \mathbb{R}^{N^{\text{g}} \times N^{\text{g}}}$)。为了在网格图上提供训练监督信号,需要将图~\ref{fig:study_area}(c)中的站点观测映射到$0.25^{\circ}$网格上:将每个站点的坐标匹配到最近的网格像素,若同一像素内落入多个站点则取算术平均。如图~\ref{fig:study_area}(d)所示,蓝色区域为映射后拥有观测值的网格像素,其余大面积区域为无观测像素,直观展示了地面监测网络在网格尺度上的极端稀疏性——这正是网格推断任务需要克服的核心困难。

两套图的邻接矩阵均基于地理学第一定律\cqt{相近事物更相关}(Tobler's First Law of Geography),按节点间的测地距离定义:
\begin{equation}
    A_{ij} = \begin{cases}
        \exp\left(-\frac{d_{ij}^2}{2\sigma_d^2}\right), & \text{if } d_{ij} < d_{\text{th}} \\
        0, & \text{otherwise}
    \end{cases}
\label{eq:diffusion_adjacency}
\end{equation}
\noindent 式中$d_{ij}$为节点$i$和$j$之间的测地距离,$\sigma_d$为距离衰减系数(默认50 km),$d_{\text{th}}$为截断阈值(默认200 km)。公式\eqref{eq:diffusion_adjacency}确定了图的拓扑结构,即哪些节点对$(i,j)$之间存在边($A_{ij}>0$)。在此基础拓扑上,第\ref{sec:spin_model}节的阶段三将为每一条边赋予两类物理含义不同的传播权重——扩散核$\tilde{A}^{\mathcal{D}}_{ij}$对$A_{ij}$作对称归一化,建模各向同性的浓度梯度扩散;平流核$\tilde{A}^{\mathcal{A}}_{ij}$根据实时风场重新计算边权,建模各向异性的定向平流输送(第\ref{subsec:stage3_propagation}节)。两类核共享同一拓扑连接($d_{ij} < d_{\text{th}}$),但对边权施加不同的物理变换,在图传播时每个节点同时沿两条通道聚合邻居信息。


% ------------------------------------------------------------
% 4.4 数据融合与推断流程
% ------------------------------------------------------------
\section{数据融合与推断流程}
\label{sec:spin_model}

针对第\ref{sec:infer_intro}节提出的两个核心问题,本节详细阐述SPIN(Spatiotemporal Physics-Guided Inference Network)模型的设计思路与数据融合流程。该模型的关键不在于单一网络架构的创新,而在于如何将多源异构数据通过物理合理的方式融合到统一的推断框架中。具体而言,SPIN需要融合四类信息:(1)地面观测数据——提供稀疏但精确的\cqt{锚点}浓度值;(2)气象与排放数据——为每个空间位置(包括无观测区域)提供本地物理环境背景;(3)图结构与物理传播机制——将扩散与平流的大气传输规律编码为节点间的信息传递路径,实现从有观测区域向无观测区域的物理合理外推;(4)卫星遥感AOD数据——提供覆盖面广但存在非随机缺失的空间结构约束。这四类数据在精度、覆盖率和可靠性上各有优劣,SPIN的核心设计思想是让它们在推断流程中各司其职、互为补充。

该模型遵循第\ref{subsec:paradigm}节提出的\cqt{物理启发的数据驱动建模}范式,与第\ref{chap:prediction}章的PCDCNet形成互补关系:PCDCNet通过LID--STD--TAD三模块解耦时间维度的化学反应、空间传输和累积过程,用于预测已知站点的未来状态;而SPIN通过扩散--平流双核解耦空间维度的各向同性与各向异性传输过程,用于推断未知位置的当前状态。

基于第\ref{subsec:spin_graph}节构建的站点图$\mathcal{G}^{\text{s}}$与网格图$\mathcal{G}^{\text{g}}$,SPIN的推断流程由四阶段流水线组成。两个子任务$\mathcal{F}^{\text{s}}_{\Theta^{\text{s}}}$与$\mathcal{F}^{\text{g}}_{\Theta^{\text{g}}}$采用完全相同的四阶段架构,但各自独立训练,仅在输入节点集合、训练数据与损失函数上有所区别。

如图~\ref{fig:model_architecture}所示,四个阶段依次为:阶段一,通过动态节点掩码模拟数据缺失场景,赋予模型归纳式泛化能力(第\ref{subsec:stage1_masking}节);阶段二,利用时间卷积网络编码本地物理特征,为所有节点提供初始浓度估计(第\ref{subsec:stage2_encoding}节);阶段三,在基础图上构建扩散核$\tilde{A}^{\mathcal{D}}$与平流核$\tilde{A}^{\mathcal{A}}$,通过物理启发的双流传播机制实现空间信息交互(第\ref{subsec:stage3_propagation}节);阶段四,施加多层级损失约束,其中网格推断$\mathcal{F}^{\text{g}}_{\Theta^{\text{g}}}$额外引入AOD梯度约束(第\ref{subsec:stage4_constraints}节)。

\begin{figure}[htbp]
    \centering
    \includegraphics[width=1.0\linewidth]{figures/chap04_spin_architecture.pdf}
    \caption{SPIN模型总体架构图}
    \caption*{上半部分为前向推断流程(Input $\to$ Stage 1 $\to$ Stage 2 $\to$ Stage 3 $\to$ Output),下半部分为Stage 4的多层级损失约束(橙色虚线表示梯度回传路径)。\textbf{输入}(左侧)包含气象场$\mathbf{M}^{F}_{t-T'+1:t}$、排放场$\mathbf{E}^{F}_{t-T'+1:t}$与观测站点历史数据$\mathbf{X}^{\text{obs}}_{t-T'+1:t}$三类数据源。\textbf{阶段一}(动态节点掩码):在图$\mathcal{G}$上对站点进行随机划分,生成观测节点集$\mathcal{V}_{\text{obs}}$与目标节点集$\mathcal{V}_{\text{target}}$。\textbf{阶段二}(本地物理特征编码):将多源输入经嵌入层送入时间卷积网络(TCN),为每个节点生成初始隐表示$\mathbf{H}^{(0)}$;同时通过Readout分支输出初始猜测$\hat{\mathbf{X}}^{\text{init}}$,用于$\mathcal{L}_{\mathrm{init}}$约束。\textbf{阶段三}(物理启发的双图空间传播):在时空图神经网络(ST-GNN)中,利用扩散核$\tilde{A}^{\mathcal{D}}$(各向同性,基于距离)与平流核$\tilde{A}^{\mathcal{A}}$(各向异性,基于风场)双通道聚合邻居信息,经门控融合更新节点表示。\textbf{输出}(右侧):站点推断产生目标站点时序$\hat{\mathbf{X}}^{\text{target}}_{t-T'+1:t}$,网格推断产生全域浓度场$\hat{\mathbf{X}}^{F}_t$。\textbf{阶段四}(多层级约束与AOD融合,下半部分):$\mathcal{L}_{\mathrm{init}}$约束初始猜测与网格真值$\mathbf{X}^{F}_t$的一致性;$\mathcal{L}_{\mathrm{infer}}$约束推断输出与站点/网格真值的匹配;$\mathcal{L}_{\mathrm{AOD}}$利用卫星遥感AOD数据对网格推断施加空间梯度约束。站点推断$\mathcal{F}^{\text{s}}_{\Theta^{\text{s}}}$与网格推断$\mathcal{F}^{\text{g}}_{\Theta^{\text{g}}}$采用完全相同的四阶段架构,但各自独立训练。}
    \label{fig:model_architecture}
\end{figure}


\subsection{阶段一:归纳式训练策略——动态节点掩码}
\label{subsec:stage1_masking}

本阶段为站点推断和网格推断提供统一的训练基础。为了使模型具备对任意未观测节点(无论是监测站点还是网格像素)的推断能力,采用了动态节点掩码(Dynamic Node Masking)策略\citep{wu2021inductive}。与传统的直推式(Transductive)方法不同,本策略不依赖固定的图结构,而是通过模拟数据缺失场景,强迫模型学习通用的时空插值规律。

记所有拥有真实观测数据的监测站点全集为$\mathcal{V}_{\text{stations}}$(即图~\ref{fig:study_area}(a)(c)中展示的152个国家级监测站点)。归纳式训练的核心思想是:在每个训练批次$(b)$中,将$\mathcal{V}_{\text{stations}}$随机划分为两个互补子集——观测节点集$\mathcal{V}_{\text{obs}}^{(b)}$和目标节点集$\mathcal{V}_{\text{target}}^{(b)}$:
\begin{equation}
\mathcal{V}_{\text{obs}}^{(b)} = \text{RandomSample}(\mathcal{V}_{\text{stations}}, 1 - r_{\text{mask}}), \quad \mathcal{V}_{\text{target}}^{(b)} = \mathcal{V}_{\text{stations}} \setminus \mathcal{V}_{\text{obs}}^{(b)}
\label{eq:inductive_sampling}
\end{equation}
\noindent 式中$r_{\text{mask}}$为目标节点比例(如30\%、50\%),$(b)$表示第$b$个批次。$\mathcal{V}_{\text{obs}}^{(b)}$中的节点扮演问题定义(公式\ref{eq:infer_problem})中$\mathcal{V}_{\text{obs}}$的角色(图~\ref{fig:inference_problem}中黑色实心圆点),模型可获取其完整的历史污染数据;$\mathcal{V}_{\text{target}}^{(b)}$中的节点扮演$\mathcal{V}_{\text{target}}$的角色(图~\ref{fig:inference_problem}中空心圆点),其历史污染数据被遮蔽,模型需要对其进行推断。该划分过程对应算法\ref{alg:spin}第3--4行。

在训练过程中,$\mathcal{V}_{\text{target}}^{(b)}$中的节点仅能获取:(1)本地的气象与排放特征;(2)$\mathcal{V}_{\text{obs}}^{(b)}$节点经图传播传递过来的时空上下文信息。这种机制模拟了真实的推断场景——即推断无历史数据的盲点。更重要的是,由于每个批次的划分是随机生成的,模型在训练过程中会\cqt{见到}同一站点在不同批次中分别扮演观测节点和目标节点的角色。这种随机性有效防止了模型过拟合于特定站点的ID特征,迫使模型学习\cqt{基于环境背景和邻域关系进行推断}的通用物理规律。


\subsection{阶段二:本地物理特征编码}
\label{subsec:stage2_encoding}

本阶段为站点推断和网格推断提供统一的特征编码,为所有节点(包括阶段一中被掩码的目标节点,无论是站点还是网格像素)提供基于本地物理条件的初始浓度估计。

% 时序特征编码器
在大气系统中,某一点的污染物浓度不仅受周边传输影响,更取决于本地的排放强度和气象条件(如湿度影响吸湿性增长,边界层高度影响垂直稀释)\citep{tai2010correlations}。为了捕捉这些本地过程的时间依赖性,我们采用时间卷积网络(Temporal Convolutional Network, TCN)作为特征提取器。

令$\mathbf{x}_i^{in} = [\mathbf{M}_i, \mathbf{E}_i] \in \mathbb{R}^{T' \times F_{in}}$表示节点$i$在过去$T'$个时间步内的输入特征序列,其中包含了气象变量$\mathbf{M}$和排放变量$\mathbf{E}$。TCN将该序列映射为高维潜在表示$\mathbf{h}_i^{(0)}$:
\begin{equation}
    \mathbf{h}_i^{(0)} = \text{TCN}(\mathbf{x}_i^{in})
\label{eq:tcn_encoding}
\end{equation}

相较于循环神经网络(RNNs,如LSTM/GRU),TCN在本任务中具有显著优势:通过膨胀卷积(Dilated Convolution),TCN能够以指数级扩大的感受野捕捉长周期的累积效应(例如过去24小时的排放累积);同时支持并行处理整个时间序列,极大提升了大规模网格推断时的计算效率\citep{bai2018empirical}。

TCN由多层膨胀卷积组成。第$l$层的输出为:
\begin{equation}
\mathbf{h}^{(l)} = \text{ReLU}\left(\text{Conv1D}(\mathbf{h}^{(l-1)}, \mathbf{W}^{(l)}, \text{dilation}=2^l) + \mathbf{b}^{(l)}\right)
\label{eq:tcn_layer}
\end{equation}
\noindent 式中膨胀因子$2^l$使得$L$层TCN的感受野达到$2^L$个时间步。本研究采用4层膨胀卷积(膨胀因子$[1, 2, 4, 8]$),感受野覆盖16小时,足以捕捉日变化特征。

TCN的输出$\mathbf{h}_i^{(0)}$实际上为每个节点提供了一个基于本地物理条件的\cqt{初始猜测(Initial Guess)}。这一设计是实现归纳式推断的关键:目标节点虽然缺乏历史污染观测,但其气象和排放数据是完整可用的。

对于观测节点$v \in \mathcal{V}_{\text{obs}}$,将观测值$X_v$与时序表征拼接:
\begin{equation}
\mathbf{h}_v^{(0)} = [\mathbf{h}_v^{\text{temp}}, X_v] \in \mathbb{R}^{d_h+1}
\label{eq:obs_init}
\end{equation}
\noindent 对于目标节点$v \in \mathcal{V}_{\text{target}}$,仅使用时序表征作为初始化:
\begin{equation}
\mathbf{h}_v^{(0)} = [\mathbf{h}_v^{\text{temp}}, 0] \in \mathbb{R}^{d_h+1}
\label{eq:target_init}
\end{equation}

这种差异化的初始化策略体现了归纳式学习的核心思想:观测节点携带真实的污染信息,而目标节点仅携带环境背景信息,后者需要通过图传播从前者\cqt{借用}污染信息。


\subsection{阶段三:物理启发的双图空间传播}
\label{subsec:stage3_propagation}

经过阶段二的编码后,观测节点携带真实的污染信息,而目标节点(站点或网格像素)仅携带环境背景信息。本阶段在第\ref{subsec:spin_graph}节构建的基础图结构上,进一步定义两类具有不同物理意义的传播核——扩散核与平流核,分别对应大气污染物传输的两种基本机制(各向同性扩散与各向异性平流),从而在图神经网络中显式建模对流--扩散方程(公式\ref{eq:advection_diffusion})的物理过程。两个子任务$\mathcal{F}^{\text{s}}_{\Theta^{\text{s}}}$与$\mathcal{F}^{\text{g}}_{\Theta^{\text{g}}}$采用相同的核\textbf{构建方法}和\textbf{网络层设计}(包括传播权重$\mathbf{W}^{\mathcal{D}}$、$\mathbf{W}^{\mathcal{A}}$和门控参数$\mathbf{W}_g$的结构),但各自独立训练,参数不共享;由于节点集合不同,所得核矩阵维度分别为$N^{\text{s}} \times N^{\text{s}}$和$N^{\text{g}} \times N^{\text{g}}$。


\subsubsection{扩散核}

扩散核建模污染物在浓度梯度驱动下产生的各向同性随机运动。基于第\ref{subsec:spin_graph}节的距离邻接矩阵$A$(公式\ref{eq:diffusion_adjacency}),对其进行对称归一化,构造扩散算子$\tilde{A}^{\mathcal{D}}$:
\begin{equation}
    \tilde{A}^{\mathcal{D}} = \mathbf{D}^{-\frac{1}{2}} A \mathbf{D}^{-\frac{1}{2}}
\label{eq:diffusion_normalized}
\end{equation}
\noindent 式中$\mathbf{D}$是$A$的度矩阵。该对称归一化对应于图拉普拉斯算子的平滑作用\citep{kipf2017semi},使得节点能够聚合来自所有方向邻居的信息,模拟了公式(\ref{eq:advection_diffusion})中扩散项$K_h\left(\frac{\partial^2 C}{\partial x^2} + \frac{\partial^2 C}{\partial y^2}\right)$所描述的大气湍流导致的浓度均一化过程。扩散核为静态核,在模型整个生命周期内保持不变。


\subsubsection{平流核}

传统的空间图往往只考虑距离,隐含了各向同性假设。然而在大气环境中,风场主导的平流输送具有强烈的各向异性——上游节点对下游节点的影响远大于反向影响,对应公式(\ref{eq:advection_diffusion})中的平流项$-\left(u\frac{\partial C}{\partial x} + v\frac{\partial C}{\partial y}\right)$。

为此,本研究构建了平流核,这是一个随风场实时更新的动态有向传播机制。如图~\ref{fig:advection_kernel}所示,从节点$j$到节点$i$的平流权重定义为:
\begin{equation}
    A_{ij}^{\mathcal{A}} = \begin{cases}
        \max\left( \frac{|\mathbf{u}_j|}{d_{ij}} \cdot \cos(\alpha_{ji}), 0 \right) \cdot \exp\left(-\frac{d_{ij}}{L_s}\right), & \text{if } d_{ij} < d_{\text{th}} \\
        0, & \text{otherwise}
    \end{cases}
\label{eq:advection_adjacency}
\end{equation}
\noindent 式中$|\mathbf{u}_j|$是源节点$j$处的风速大小,$\alpha_{ji}$是风向向量$\mathbf{u}_j$与边向量$\mathbf{e}_{ji}$(从$j$指向$i$)之间的夹角,$L_s$为特征传输长度尺度(默认100 km)。$\max(\cdot, 0)$函数确保了权重的非负性和方向性:只有当风从$j$吹向$i$(即$\cos(\alpha_{ji}) > 0$)时,边才存在。这一设计与第\ref{chap:prediction}章PM$_{2.5}$-GNN中的平流系数(公式\eqref{eq:advection_coeff})异曲同工,都显式编码了\cqt{上游影响下游}的定向传输机制。

归一化采用行归一化(出度归一化):
\begin{equation}
\tilde{A}^{\mathcal{A}} = \mathbf{D}_{\text{out}}^{-1} A^{\mathcal{A}}
\label{eq:advection_normalized}
\end{equation}
\noindent 式中$D_{\text{out},ii} = \sum_j A_{ij}^{\mathcal{A}}$。行归一化确保每个节点的出流总和为1,符合质量守恒的物理约束——这与第\ref{chap:prediction}章\ref{subsec:pcdcnet_training}节讨论的DIC约束在理念上一致。

\begin{figure}[htbp]
    \centering
    \includegraphics[width=0.5\textwidth]{figures/chap04_advection_kernel.pdf}
    \caption{平流核(Advection Kernel)构建示意图}
    \caption*{以节点$i$为中心展示平流权重$A^{\mathcal{A}}_{ij}$的计算过程。棕色箭头为节点$i$处的风速矢量(风速$|\mathbf{v}|$、风向角$\beta$),蓝色箭头为从节点$j$(节点1)到节点$i$的边向量(方位角$\gamma$、距离$d$),$\alpha = |\beta - \gamma|$为风向与边方向的夹角。红色实心箭头表示节点1因$\cos\alpha > 0$(位于上风方向)而获得较大的平流权重$A^{\mathcal{A}}_{ij}$;红色虚线箭头表示节点2、3因与风向夹角较大而权重较小或为零。该机制显式编码了\cqt{上游影响下游}的定向传输规律。}
    \label{fig:advection_kernel}
\end{figure}


\subsubsection{双流传播层}

基于上述扩散核$\tilde{A}^{\mathcal{D}}$与平流核$\tilde{A}^{\mathcal{A}}$,我们设计了双流时空图神经网络层。对于节点$v$的特征表示$\mathbf{h}_v$,双图传播的更新规则为:

扩散传播(各向同性,无方向性):
\begin{equation}
\mathbf{m}_v^{\mathcal{D}} = \sum_{u \in \mathcal{N}(v)} \tilde{A}_{uv}^{\mathcal{D}} \cdot \mathbf{W}^{\mathcal{D}} \mathbf{h}_u^{(l)}
\label{eq:diffusion_message}
\end{equation}

平流传播(各向异性,沿风向):
\begin{equation}
\mathbf{m}_v^{\mathcal{A}} = \sum_{u \in \mathcal{N}(v)} \tilde{A}_{uv}^{\mathcal{A}} \cdot \mathbf{W}^{\mathcal{A}} \mathbf{h}_u^{(l)}
\label{eq:advection_message}
\end{equation}
\noindent 式中$\mathbf{W}^{\mathcal{D}}, \mathbf{W}^{\mathcal{A}} \in \mathbb{R}^{d_h \times d_h}$为可学习的权重矩阵,$l$为层索引。

两类消息通过门控机制融合,允许模型自适应平衡扩散和平流的相对贡献:
\begin{equation}
\mathbf{g}_v = \sigma\left(\mathbf{W}_g [\mathbf{m}_v^{\mathcal{D}}, \mathbf{m}_v^{\mathcal{A}}] + \mathbf{b}_g\right)
\label{eq:gate_weight}
\end{equation}
\begin{equation}
\mathbf{m}_v = \mathbf{g}_v \odot \mathbf{m}_v^{\mathcal{D}} + (1 - \mathbf{g}_v) \odot \mathbf{m}_v^{\mathcal{A}}
\label{eq:gate_fusion}
\end{equation}
\noindent 式中$\sigma$为Sigmoid函数,$\odot$为逐元素乘法,$\mathbf{W}_g \in \mathbb{R}^{d_h \times 2d_h}$为门控权重。具体而言,$\mathbf{g}_v \in \mathbb{R}^{d_h}$为门控权重向量(对应算法\ref{alg:spin}第15行的$\mathbf{G}$),其每个分量取值于$[0,1]$,逐维度控制扩散消息与平流消息的混合比例;$\mathbf{m}_v \in \mathbb{R}^{d_h}$为融合后的空间消息(对应算法\ref{alg:spin}第16行的$\Delta\mathbf{H}$),综合了来自两类图结构的邻域信息,随后通过残差连接注入节点表示。

门控机制的物理意义在于:在静风条件下,扩散过程主导污染物的空间分布,模型应侧重扩散核($\mathbf{g}_v \to 1$);在强风条件下,平流输送成为主导机制,模型应侧重平流核($\mathbf{g}_v \to 0$)。这种自适应的权重分配使得模型能够根据局地气象条件动态调整传播策略。

融合消息$\mathbf{m}_v$通过残差连接更新节点特征(对应算法\ref{alg:spin}第17行):
\begin{equation}
\mathbf{h}_v^{(l+1)} = \text{ReLU}\left(\mathbf{h}_v^{(l)} + \mathbf{m}_v\right)
\label{eq:residual_update}
\end{equation}

\noindent 残差连接确保了信息的有效传递和梯度的稳定流动,同时允许模型在必要时\cqt{跳过}某些传播层,直接利用初始特征。经过$L$层传播后,每个节点的最终表示$\mathbf{h}_v^{(L)}$已融合了TCN提取的本地物理特征(阶段二)与GNN从邻域聚合的外部污染信息(阶段三),随后送入阶段四的读出层生成浓度推断。




\subsection{阶段四:多层级约束与AOD融合}
\label{subsec:stage4_constraints}

本阶段是站点推断(图$\mathcal{G}^{\text{s}}$)与网格推断(图$\mathcal{G}^{\text{g}}$)的主要差异所在。经过阶段三的空间传播后,每个节点的潜在表示$\mathbf{H}_i^{(L)}$已深度融合了本地物理特征和邻域污染信息。两个子任务均采用相同设计的读出层、推断损失$\mathcal{L}_{\mathrm{infer}}$与初始化损失$\mathcal{L}_{\mathrm{init}}$,但由于独立训练,各自拥有独立的读出层参数。核心差异在于:AOD梯度损失$\mathcal{L}_{\mathrm{AOD}}$仅在网格推断$\mathcal{F}^{\text{g}}_{\Theta^{\text{g}}}$中启用,通过将卫星数据作为空间结构约束而非强制输入,解决非随机缺失遥感数据的有效融合问题。

\subsubsection{读出层}

经过$L$层物理启发的空间传播后,由于目标节点本身没有任何历史观测数据,无法采用自回归方式。因此设计了一个节点共享的读出层(Readout Layer),由多层感知机构成,直接将高维潜在特征映射为标量的\PM 浓度推断值(对应算法\ref{alg:spin}第21行及图\ref{fig:model_architecture}右侧):
\begin{equation}
    \hat{X}_v = \mathbf{W}_{\text{out}} \mathbf{h}_v^{(L)} + b_{\text{out}}
\label{eq:decoder_stage4}
\end{equation}

\noindent 式中$\mathbf{h}_v^{(L)} \in \mathbb{R}^{d_h}$为节点$v$经过阶段三传播后的最终潜在表示,$\mathbf{W}_{\text{out}} \in \mathbb{R}^{1 \times d_h}$和$b_{\text{out}} \in \mathbb{R}$为共享的读出参数,$\hat{X}_v \in \mathbb{R}$为节点$v$的\PM 浓度推断标量值。将所有节点的推断值堆叠即得到全局推断向量$\hat{\mathbf{X}} = [\hat{X}_1, \hat{X}_2, \ldots, \hat{X}_N]^\top \in \mathbb{R}^{N}$,对应问题定义(公式\ref{eq:infer_problem}--\ref{eq:infer_problem_grid})中的$\hat{\mathbf{X}}^{\text{target}}$或$\hat{\mathbf{X}}^{F}$。该输出直接用于后续三部分损失函数的计算:对目标节点计算推断损失$\mathcal{L}_{\mathrm{infer}}$(公式\ref{eq:infer_loss}),对观测节点计算初始化损失$\mathcal{L}_{\mathrm{init}}$(公式\ref{eq:init_loss}),在网格推断中还通过相邻节点的$\hat{X}_v$差值计算AOD梯度损失$\mathcal{L}_{\mathrm{AOD}}$(公式\ref{eq:aod_loss})。

共享参数的设计使得模型学习到的是一个通用的\cqt{特征到浓度}映射函数,而非针对特定站点的过拟合表示,确保了模型能够完全基于物理环境特征来填补数据盲区,实现真正的归纳式推断。

\subsubsection{复合损失函数设计}

为了在数据驱动的基础上引入物理约束,特别是解决卫星AOD数据\cqt{虽有空间结构但存在缺失与偏差}的难题,设计了一个包含三部分的复合损失函数:
\begin{equation}
    \mathcal{L} = \mathcal{L}_{\mathrm{infer}} + \lambda_1 \mathcal{L}_{\mathrm{init}} + \lambda_2 \mathcal{L}_{\mathrm{AOD}}
\label{eq:total_loss_chap3}
\end{equation}
\noindent 式中$\lambda_1$和$\lambda_2$为平衡系数。三项损失各司其职:$\mathcal{L}_{\mathrm{infer}}$为主损失,约束经图传播后目标节点的最终推断精度;$\mathcal{L}_{\mathrm{init}}$为辅助损失,约束TCN编码器在图传播之前即能给出合理的本地初始估计;$\mathcal{L}_{\mathrm{AOD}}$为空间结构约束,仅在网格推断中启用,利用卫星AOD的空间梯度引导预测场的形态。下面分别介绍各项的技术设计。


\subsubsection{推断损失(\LossInfer)}

这是模型的主要优化目标。计算经过图网络传播后的最终预测值$\hat{\mathbf{X}}$与被掩码目标节点的真实观测值之间的L1误差:
\begin{equation}
    \mathcal{L}_{\mathrm{infer}} = \frac{1}{|\mathcal{V}_{\text{target}}|} \sum_{v \in \mathcal{V}_{\text{target}}} |\hat{X}_v - X_v|
\label{eq:infer_loss}
\end{equation}

该损失函数直接驱动模型学习如何综合利用本地环境特征和邻居传播信息来填补数据盲区。采用$L_1$损失(MAE)而非$L_2$损失(MSE),以降低对异常值的敏感性——这在大气污染数据中尤为重要,因为极端污染事件虽然是重要的预测目标,但不应过度主导训练过程。


\subsubsection{初始化损失(\LossInit)}

为了确保时序特征编码器(TCN)提取的特征$\mathbf{H}^{(0)}$具有明确的物理意义,引入了辅助监督信号。要求TCN的直接输出(在进入图网络传播之前)也能对观测节点给出一个合理的\cqt{初始提案(Initial Proposal)}:
\begin{equation}
    \mathcal{L}_{\mathrm{init}} = \frac{1}{|\mathcal{V}_{\text{obs}}|} \sum_{v \in \mathcal{V}_{\text{obs}}} |\hat{X}_v^{\text{init}} - X_v|
\label{eq:init_loss}
\end{equation}
\noindent 式中$\hat{X}_v^{\text{init}} = \text{MLP}(\mathbf{h}_v^{(0)})$为节点$v$在阶段二(第\ref{subsec:stage2_encoding}节)中由TCN编码得到的潜在表示$\mathbf{h}_v^{(0)}$经一个独立的MLP映射后产生的标量浓度估计值,代表了\textbf{仅基于本地气象与排放信息、尚未经过任何图传播}的初始浓度判断。换言之,$\hat{X}_v^{\text{init}}$是阶段二的直接输出,而$\hat{X}_v$(公式\ref{eq:decoder_stage4})是经阶段三图传播后的最终输出,两者分别由$\mathcal{L}_{\mathrm{init}}$和$\mathcal{L}_{\mathrm{infer}}$约束。

该损失的设计动机在于:在推断阶段,目标节点$\mathcal{V}_{\text{target}}$没有历史观测数据,其初始表示$\mathbf{h}_v^{(0)}$完全来自本地气象与排放特征(公式\ref{eq:target_init})。如果TCN的编码缺乏物理意义,目标节点的初始估计将毫无信息量,后续的图传播将难以有效修正。因此,$\mathcal{L}_{\mathrm{init}}$通过在观测节点上施加监督,迫使TCN在没有任何邻居信息的情况下,仅凭本地条件就能给出一个尽可能接近真实值的基准估计。这样,目标节点即使缺乏观测数据,其基于相同TCN编码的初始估计也具有物理合理性,为后续的图传播提供了稳健的起始状态。

从物理角度理解,$\mathcal{L}_{\mathrm{init}}$迫使TCN学习\cqt{局地生成(Local Production)}的映射关系——即在排放强度高、边界层低、静风的条件下,本地污染水平应该较高。这一关系是公式(\ref{eq:advection_diffusion})中源项$S$和化学反应项$R$的数据驱动近似。


\subsubsection{掩码AOD空间梯度损失(\LossAOD)}

这是本研究解决AOD数据缺陷的核心创新。针对AOD数据存在的非随机缺失(云遮挡)和数值偏差(反演误差),不再将其作为输入,而是将其作为空间结构约束。

具体而言,约束模型预测场$\hat{\mathbf{X}}$的空间梯度(Spatial Gradient)与卫星AOD场$\mathbf{X}^{\text{AOD}}$的空间梯度保持一致。定义边$(i, j)$上的预测梯度和AOD梯度:
\begin{equation}
\nabla_{ij}(\hat{\mathbf{X}}) = \hat{X}_j - \hat{X}_i, \quad \nabla_{ij}(\mathbf{X}^{\text{AOD}}) = X_j^{\text{AOD}} - X_i^{\text{AOD}}
\label{eq:gradient_definition}
\end{equation}

为了处理缺失值,利用节点级有效性掩码$\boldsymbol{\omega}^{\text{AOD}} \in \{0,1\}^{N}$(第\ref{sec:infer_data}节),其中$\omega_i^{\text{AOD}} = 1$表示节点$i$处有有效AOD观测,$\omega_i^{\text{AOD}} = 0$表示该节点因云遮挡而缺失。在此基础上定义边级掩码$\Omega_{ij}^{\text{AOD}} = \omega_i^{\text{AOD}} \cdot \omega_j^{\text{AOD}}$,当且仅当边$(i,j)$两端节点均有有效AOD数据时$\Omega_{ij}^{\text{AOD}} = 1$:
\begin{equation}
    \mathcal{L}_{\mathrm{AOD}} = \sum_{t=1}^{T'} \sum_{(i,j) \in \mathcal{E}} \left| \nabla_{ij}(\hat{\mathbf{X}}_t) - \nabla_{ij}(\mathbf{X}^{\text{AOD}}_t) \right| \cdot \Omega_{ij}^{\text{AOD}}
\label{eq:aod_loss}
\end{equation}
\noindent 式中边级掩码由节点级掩码的乘积导出:
\begin{equation}
\Omega_{ij}^{\text{AOD}} = \omega_i^{\text{AOD}} \cdot \omega_j^{\text{AOD}}
\label{eq:aod_mask}
\end{equation}

\begin{figure}[htbp]
    \centering
    \includegraphics[width=0.75\textwidth]{figures/chap04_aod_mask_loss.pdf}
    \caption{网格图上的掩码AOD梯度约束示意图}
    \caption*{左图为模型预测浓度场在网格图上沿边$(i,j)$的空间梯度$\nabla_{ij}(\hat{\mathbf{X}}) = \hat{X}_j - \hat{X}_i$;右图为对应的卫星AOD场梯度$\nabla_{ij}(\mathbf{X}^{\text{AOD}})$,其中斜线阴影区域($\boldsymbol{\omega}^{\text{AOD}} = 0$)表示AOD因云层遮挡而缺失的网格节点。损失函数$\mathcal{L}_{\mathrm{AOD}}$(公式\ref{eq:aod_loss})仅在两端点均有有效AOD观测($\Omega^{\text{AOD}}_{ij}=1$)的边上计算,从而避免对缺失区域的错误约束。}
    \label{fig:aod_mask_loss}
\end{figure}

如图~\ref{fig:aod_mask_loss}所示,这一设计的物理直觉在于:

(1)关注\cqt{形}而非\cqt{值}。通过约束梯度,模型学习的是污染物的空间分布形态(如羽流的形状、扩散的边界),而不受AOD绝对数值偏差的影响。这实现了\cqt{数值解耦}——即使AOD系统性地高估或低估了地面浓度,只要其空间相对分布是正确的,模型仍能从中受益。

(2)鲁棒性。掩码机制$\boldsymbol{\omega}^{\text{AOD}}$使得损失函数自动忽略云层遮挡区域,避免了对缺失值的错误强行拟合。模型不会因为AOD在某些位置缺失而产生异常梯度。

(3)全天候适用。在夜间或云覆盖时,$\mathcal{L}_{\mathrm{AOD}}$约束自动失效(因为$\boldsymbol{\omega}^{\text{AOD}}$全为零),模型平滑回退到由物理核(扩散--平流)驱动的推断模式,仅凭气象场和地面站点生成了物理上合理的平滑场。这种\cqt{自适应回退}特性是本方法相比传统AOD输入方法的核心优势。

算法\ref{alg:spin}给出了SPIN的完整训练与推理流程。

\begin{algorithm}[htbp]
\caption{SPIN 模型训练与推理流程}
\caption*{\textbf{SPIN}:算法对应第\ref{sec:spin_model}节所述的四阶段数据融合与推断流程——阶段一:动态节点掩码(第\ref{subsec:stage1_masking}节),阶段二:TCN时序编码(第\ref{subsec:stage2_encoding}节),阶段三:双图传播(第\ref{subsec:stage3_propagation}节),阶段四:复合损失优化(第\ref{subsec:stage4_constraints}节)。}
\label{alg:spin}
\begin{algorithmic}[1]
    \REQUIRE 训练数据集$\mathcal{D}$,初始化参数$\Theta$,学习率$\eta$,掩码比例$r_{\text{mask}}$
    \ENSURE 优化后的参数$\Theta$(训练),推断结果$\{\hat{\mathbf{X}}\}$(推理)

    \FOR{每个样本$D_k = (\mathbf{X}, \mathbf{M}, \mathbf{E}, \mathbf{X}^{\text{AOD}}) \in \mathcal{D}$}
        \STATE \textbf{阶段1:动态节点掩码(第\ref{subsec:stage1_masking}节)}
        \STATE $\mathcal{V}_{\text{target}}^{(b)} \leftarrow \text{RandomSample}(\mathcal{V}_{\text{stations}},\; r_{\text{mask}})$ \COMMENT{随机采样目标节点}
        \STATE $\mathcal{V}_{\text{obs}}^{(b)} \leftarrow \mathcal{V}_{\text{stations}} \setminus \mathcal{V}_{\text{target}}^{(b)}$ \COMMENT{剩余为观测节点}

        \STATE \textbf{阶段2:时序特征编码(第\ref{subsec:stage2_encoding}节)}
        \FOR{每个节点$v \in \mathcal{V}$}
            \STATE $\mathbf{h}_v^{\text{temp}} \leftarrow \text{TCN}([\mathbf{M}_v, \mathbf{E}_v])$ \COMMENT{本地时序编码}
        \ENDFOR
        \STATE $\mathbf{h}_v^{(0)} \leftarrow [\mathbf{h}_v^{\text{temp}}, X_v]$($v \in \mathcal{V}_{\text{obs}}^{(b)}$);$\mathbf{h}_v^{(0)} \leftarrow [\mathbf{h}_v^{\text{temp}}, 0]$($v \in \mathcal{V}_{\text{target}}^{(b)}$) \COMMENT{差异化初始化}

        \STATE \textbf{阶段3:物理启发的图传播(第\ref{subsec:stage3_propagation}节)}
        \STATE 根据当前时刻风场更新平流图$\tilde{\mathbf{A}}^{\mathcal{A}}$
        \FOR{$l \leftarrow 1$ to $L$}
            \STATE $\mathbf{H}^{\mathcal{D}} \leftarrow \tilde{\mathbf{A}}^{\mathcal{D}} \mathbf{W}^{\mathcal{D}} \mathbf{H}^{(l-1)}$ \COMMENT{扩散消息(公式\ref{eq:diffusion_message})}
            \STATE $\mathbf{H}^{\mathcal{A}} \leftarrow \tilde{\mathbf{A}}^{\mathcal{A}} \mathbf{W}^{\mathcal{A}} \mathbf{H}^{(l-1)}$ \COMMENT{平流消息(公式\ref{eq:advection_message})}
            \STATE $\mathbf{G} \leftarrow \sigma(\mathbf{W}_g [\mathbf{H}^{\mathcal{D}}, \mathbf{H}^{\mathcal{A}}])$ \COMMENT{门控权重$g_v$(公式\ref{eq:gate_weight})}
            \STATE $\Delta\mathbf{H} \leftarrow \mathbf{G} \odot \mathbf{H}^{\mathcal{D}} + (1-\mathbf{G}) \odot \mathbf{H}^{\mathcal{A}}$ \COMMENT{融合消息$m_v$(公式\ref{eq:gate_fusion})}
            \STATE $\mathbf{H}^{(l)} \leftarrow \text{ReLU}(\mathbf{H}^{(l-1)} + \Delta\mathbf{H})$ \COMMENT{残差更新(公式\ref{eq:residual_update})}
        \ENDFOR

        \STATE \textbf{阶段4:读出与损失计算(第\ref{subsec:stage4_constraints}节)}
        \STATE $\hat{\mathbf{X}} \leftarrow \mathbf{W}_{\text{out}} \mathbf{H}^{(L)} + \mathbf{b}_{\text{out}}$ \COMMENT{读出层$\hat{X}_v$(公式\ref{eq:decoder_stage4})}
        \STATE $\hat{\mathbf{X}}^{\text{init}} \leftarrow \text{MLP}(\mathbf{H}^{(0)})$ \COMMENT{初始估计$\hat{X}_v^{\text{init}}$}
        \STATE $\mathcal{L}_{\mathrm{infer}} \leftarrow \frac{1}{|\mathcal{V}_{\text{target}}^{(b)}|}\sum_{v \in \mathcal{V}_{\text{target}}^{(b)}} |\hat{X}_v - X_v|$ \COMMENT{公式\ref{eq:infer_loss}}
        \STATE $\mathcal{L}_{\mathrm{init}} \leftarrow \frac{1}{|\mathcal{V}_{\text{obs}}^{(b)}|}\sum_{v \in \mathcal{V}_{\text{obs}}^{(b)}} |\hat{X}_v^{\text{init}} - X_v|$ \COMMENT{公式\ref{eq:init_loss}}
        \STATE $\mathcal{L}_{\mathrm{AOD}} \leftarrow \sum_{(i,j) \in \mathcal{E}} |\nabla_{ij}(\hat{\mathbf{X}}) - \nabla_{ij}(\mathbf{X}^{\text{AOD}})| \cdot \Omega_{ij}^{\text{AOD}}$ \COMMENT{公式\ref{eq:aod_loss}}
        \STATE $\mathcal{L} \leftarrow \mathcal{L}_{\mathrm{infer}} + \lambda_1 \mathcal{L}_{\mathrm{init}} + \lambda_2 \mathcal{L}_{\mathrm{AOD}}$ \COMMENT{公式\ref{eq:total_loss_chap3}}
    \ENDFOR

    \STATE $\Theta \leftarrow \Theta - \eta \nabla_{\Theta} \mathcal{L}$
    \RETURN $\Theta$(训练)\textbf{或} $\{\hat{\mathbf{X}}\}$(推理)
\end{algorithmic}
\end{algorithm}


% ------------------------------------------------------------
% 4.6 实验与结果
% ------------------------------------------------------------
\section{实验与结果}
\label{sec:infer_experiment}

为了全面验证SPIN的归纳泛化能力和物理一致性,实验评估围绕第\ref{sec:infer_problem}节定义的两个子任务展开:(1)\textbf{站点推断}——在保留测试站点上建立模型的定量优势(第\ref{subsec:station_inference}节),并通过极端数据稀缺(50\%站点缺失)条件下的案例分析深入剖析模型行为(第\ref{subsec:robustness_analysis}节);(2)\textbf{网格推断}——展示物理核与AOD约束如何协同生成连续、高保真的污染地图(第\ref{subsec:grid_inference}节)。


\subsection{实验设置}
\label{subsec:infer_exp_setup}

(1)数据集划分。为了严格防止时间维度的数据泄露,本研究采用按年份划分数据集:2018--2021年为训练集,2022年为验证集,2023年为测试集。此外,为了评估模型对季节性污染特征的适应能力,定义了两个季节性子集:

\begin{itemize}
    \item 冬季(供暖季):11月至次年3月,特征为高排放、静稳天气多发,\PM 污染严重;
    \item 夏季:5月至9月,特征为强对流天气频繁、大气扩散条件较好,\PM 浓度相对较低。
\end{itemize}

对于站点推断任务,将152个监测站点按7:3的比例划分为训练集(106个站点)和完全未观测的测试集(46个站点)。测试集中的站点在整个训练过程中从未被模型\cqt{看到}。

(2)评价指标。采用以下指标评估模型性能:

\begin{itemize}
    \item MAE(平均绝对误差):$\text{MAE} = \frac{1}{n}\sum_{i=1}^{n}|y_i - \hat{y}_i|$,反映平均推断偏差;
    \item RMSE(均方根误差):$\text{RMSE} = \sqrt{\frac{1}{n}\sum_{i=1}^{n}(y_i - \hat{y}_i)^2}$,对大误差敏感;
    \item $R^2$(决定系数):反映推断值与真值的相关性,$R^2 = 1 - \frac{\sum(y_i-\hat{y}_i)^2}{\sum(y_i-\bar{y})^2}$。
\end{itemize}

对于网格推断任务,额外计算空间相关系数(Spatial Correlation)以评估场的结构一致性。

(3)基线方法。将SPIN与三类主流方法进行了对比:

\begin{itemize}
    \item 特征工程模型:XGBoost\citep{chen2016xgboost}、MLP,代表传统机器学习方法;
    \item 时序深度学习模型:LSTM\citep{hochreiter1997long}、GRU\citep{cho2014learning},代表纯时序建模方法;
    \item 时空图神经网络模型:STGCN\citep{yu2018spatio}、IGNNK\citep{wu2021inductive},代表当前最先进的空间插值方法。
\end{itemize}

(4)实现细节。TCN采用4层膨胀卷积,膨胀因子为$[1, 2, 4, 8]$,卷积核大小为3。双图GNN采用2层消息传递。隐藏维度$d_h = 64$。优化器为Adam,初始学习率$10^{-3}$,采用余弦退火调度。批大小32,训练200个epoch,早停patience为20。损失权重$\lambda_1 = 0.5$,$\lambda_2 = 0.1$(基于验证集调优)。实验在NVIDIA A100 GPU上进行,单次训练约4小时。


\subsection{推断精度评估}
\label{subsec:station_inference}

针对子任务(1)站点推断,首先在保留的30\%测试站点(完全未参与训练的\cqt{盲点})上评估模型的推断精度。表~\ref{table:performance_comparison}展示了各模型在不同季节下的平均绝对误差(MAE)。

\begin{table}[htbp]
\centering
\caption{未观测站点(30\% Hold-out)的推断性能对比}
\caption*{评价指标为MAE($\mu$g/m$^3$),对比方法涵盖特征工程类(XGBoost、MLP)、时序类(LSTM、GRU)、时空图网络类(STGCN、IGNNK)以及本文提出的物理启发类方法SPIN。}
\label{table:performance_comparison}
\resizebox{0.9\textwidth}{!}{%
\begin{tabular}{@{}ll|c|c|c@{}}
\toprule
\multirow{2}{*}{\textbf{模型类别}} & \multirow{2}{*}{\textbf{方法}} & \textbf{全年(All Year)} & \textbf{冬季(Winter)} & \textbf{夏季(Summer)} \\
 &  & MAE$\downarrow$ & MAE$\downarrow$ & MAE$\downarrow$ \\ \midrule
\multirow{2}{*}{特征工程类} & XGBoost & 24.21 & 35.97 & 15.49 \\
 & MLP & 25.71 & 38.67 & 15.30 \\ \midrule
\multirow{2}{*}{时序类} & LSTM & 24.65 & 34.96 & 14.49 \\
 & GRU & 22.79 & 34.17 & 15.00 \\ \midrule
\multirow{2}{*}{时空图网络类} & STGCN & 17.84 & 23.43 & 11.23 \\
 & IGNNK & 12.73 & 16.29 & 8.53 \\ \midrule
\textbf{物理启发类} & \textbf{SPIN}(本文) & \textbf{9.52} & \textbf{15.09} & \textbf{7.65} \\ \bottomrule
\end{tabular}%
}
\end{table}

实验结果表明:

(1)总体性能优势。SPIN取得了全场最优的MAE(9.52 $\mu$g/m$^3$),相比目前最先进的归纳式基线IGNNK(12.73 $\mu$g/m$^3$),误差降低了25.2\%。这有力地证明了显式建模物理传输过程比纯数据驱动的统计插值更具优势。

(2)冬季鲁棒性。在污染最严重、气象条件最复杂的冬季,SPIN仍保持最优性能(MAE 15.09),相比IGNNK(16.29)降低7.4\%,相比STGCN(23.43)降低35.6\%。冬季的绝对误差高于夏季,反映了该季节污染过程本身的高变异性。STGCN等传统模型依赖固定的静态图结构,隐含了各向同性假设,难以适应冬季季风带来的强烈定向输送;而SPIN的动态平流核能够根据实时风场调整权重,从而精准捕捉了跨区域的污染传输——这与第\ref{chap:prediction}章PM$_{2.5}$-GNN的发现一致。

(3)基线缺陷分析。特征类和时序类模型由于完全忽略了空间相关性,误差普遍较高($>$20 $\mu$g/m$^3$)。IGNNK虽然引入了K-近邻图进行归纳推理,但其基于欧氏距离的构图缺乏物理方向性(各向同性假设),无法区分上游和下游的影响。SPIN通过扩散--平流双核的设计,显式解耦了这两种物理过程,实现了更准确的空间依赖建模。


\subsection{鲁棒性分析}
\label{subsec:robustness_analysis}

继续围绕子任务(1)站点推断,通过具体站点的案例分析深入剖析模型行为。为了测试模型在数据极度稀缺条件下的边界能力,将站点掩码比例提升至50\%,即仅利用一半的站点来推断另一半。

(1)区域泛化能力。图~\ref{fig:bj_group}展示了京津冀核心城市群(北京、天津、石家庄)中三个代表性未观测站点的推断结果。即便缺失了一半的监测网络,模型依然能够高精度地($R^2 > 0.85$)复现污染物的日变化和综观趋势。特别是对于2020年1月底的区域性重污染过程,模型准确捕捉到了污染的累积(Accumulation)和消散(Dissipation)阶段,证明物理核成功学到了区域传输的宏观规律。

\begin{figure}[htbp]
    \centering
    \includegraphics[width=\textwidth]{figures/chap04_station_beijing.png}\\[0.35em]
    \includegraphics[width=\textwidth]{figures/chap04_station_tianjin.png}\\[0.35em]
    \includegraphics[width=\textwidth]{figures/chap04_station_shijiazhuang.png}
    \caption{京津冀核心城市群的推断表现}
    \caption*{从上至下分别为北京定陵站、天津前进道站、石家庄人民会堂站。即便在50\%站点不可见的情况下,模型依然保持了极高的相关性($R^2 > 0.85$),准确捕捉了冬季污染过程的动态演变。}
    \label{fig:bj_group}
\end{figure}

(2)推断性能的上下界分析。图~\ref{fig:rmse_extremes}展示了推断性能的两个极端案例。上图(商丘粮食局站)展示了在密集邻居支持下的完美重建($R^2=0.96$)——密集的连接使得扩散核能够有效聚合梯度信息,实现高保真重建;下图(许昌监测站)展示了在复杂局部动态下的较大误差($R^2=0.644$),这可能是由于微尺度排放突变被邻居平均所平滑。然而,模型仍正确把握了污染的起止时间和峰值位置,证明平流核成功捕捉了宏观传输模式。

\begin{figure}[htbp]
    \centering
    \includegraphics[width=\textwidth]{figures/chap04_station_shangqiu.png}\\[0.35em]
    \includegraphics[width=\textwidth]{figures/chap04_station_xuchang.png}
    \caption{推断性能的上下界分析}
    \caption*{上图(商丘)展示了在密集邻居支持下的完美重建($R^2=0.96$);下图(许昌)展示了在复杂局部动态下的误差,但模型仍正确把握了污染起止时间和趋势方向。}
    \label{fig:rmse_extremes}
\end{figure}

(3)极端稀疏与物理保真。图~\ref{fig:sparse_neighbors}聚焦于一个极端的边缘站点——张家口1057A站。该站点位于西北山区边缘,上游几乎没有传感器支持。在此极端条件下,定量指标有所下降($R^2=0.132$),这是由于高频局部波动的丢失。然而,深入分析发现一个重要的物理特性:模型并未产生幻觉(Hallucination)或输出零值,而是成功恢复了正确的低频背景趋势(20--50 $\mu$g/m$^3$)。

这验证了扩散核在数据真空中起到了\cqt{保守低通滤波器}的作用——在缺乏信息时,模型倾向于输出物理上合理的背景值,而非随机噪声。这种\cqt{宁可平滑,绝不谬误}的特性对于环境决策支持系统至关重要:即使在最不利的数据条件下,模型也能提供一个合理的基线估计。

\begin{figure}[htbp]
    \centering
    \includegraphics[width=\textwidth]{figures/chap04_station_edge.png}\\[0.35em]
    \includegraphics[width=\textwidth]{figures/chap04_station_zhangjiakou.png}
    \caption{边缘稀疏区域的物理保真性}
    \caption*{上图展示了该站点在监测网络中的边缘位置(上游几乎无传感器覆盖);下图(张家口人民公园站)展示了在极度缺乏上游信息时,物理启发机制确保了模型输出合理的背景趋势,而非随机噪声或幻觉值。}
    \label{fig:sparse_neighbors}
\end{figure}


\subsection{网格推断分析}
\label{subsec:grid_inference}

针对子任务(2)网格推断,本节评估模型生成连续高分辨率污染场的能力。如第\ref{sec:infer_problem}节所定义,网格推断将研究区域离散为$0.25^{\circ} \times 0.25^{\circ}$的规则网格(如图~\ref{fig:study_area}(d)所示),所有网格像素构成目标节点集$\mathcal{V}_{\text{target}}$,所有152个监测站点构成观测节点集$\mathcal{V}_{\text{obs}}$。网格推断与站点推断采用相同的模型架构设计与物理图核构建方法(阶段一至三,第\ref{sec:spin_model}节),其归纳式训练策略使模型能够直接泛化至训练中从未出现过的网格节点——每个网格节点的初始表示由本地气象与排放特征经TCN编码生成(阶段二),随后通过扩散图与平流图从周围观测站点聚合污染信息(阶段三)。两类任务的主要差异在于阶段四:网格推断在训练时额外启用AOD梯度损失$\mathcal{L}_{\mathrm{AOD}}$(公式\ref{eq:aod_loss}),利用卫星观测的空间梯度约束预测场的形态。需要强调的是,AOD仅作为训练时的空间结构约束,而非推理时的直接输入特征,因此模型在推理阶段不依赖卫星数据的可用性。

图~\ref{fig:grid_inference}通过四个典型案例,深入剖析了上述机制的实际效果——SPIN如何利用物理核与AOD梯度损失来应对卫星数据的各类缺陷。图中的AOD面板作为外部参考,用于验证模型是否成功内化了污染的空间结构模式。

\begin{figure}[htbp]
    \centering
    \begin{subfigure}{\textwidth}
        \centering
        \includegraphics[width=0.9\textwidth]{figures/chap04_grid_ideal.png}
        \caption{案例(a):理想一致性——结构迁移}
    \end{subfigure}
    \begin{subfigure}{\textwidth}
        \centering
        \includegraphics[width=0.9\textwidth]{figures/chap04_grid_missing.png}
        \caption{案例(b):完全缺失——物理回退}
    \end{subfigure}
    \begin{subfigure}{\textwidth}
        \centering
        \includegraphics[width=0.9\textwidth]{figures/chap04_grid_conflict.png}
        \caption{案例(c):结构冲突——鲁棒性}
    \end{subfigure}
    \begin{subfigure}{\textwidth}
        \centering
        \includegraphics[width=0.9\textwidth]{figures/chap04_grid_calibration.png}
        \caption{案例(d):幅度偏差——自动校准}
    \end{subfigure}
    \caption{网格推断与AOD机制分析}
    \caption*{左列:模型推断的\PM 场;中列:地面站点真值;右列:卫星AOD参考(仅作训练约束)。(a) 理想情况下的结构迁移;(b) AOD缺失时的物理回退;(c) 抗干扰鲁棒性;(d) 幅度偏差的自动校准。}
    \label{fig:grid_inference}
\end{figure}

案例(a) 理想一致性(结构迁移)。当AOD数据完整且与地面观测一致时,模型通过梯度约束,成功将AOD的羽流细节\cqt{迁移}到了\PM 推断场中,实现了平滑且精细的插值。图中清晰地展示了沿太行山脉分布的高浓度污染带,与地形阻挡效应高度吻合。这验证了$\mathcal{L}_{\mathrm{AOD}}$约束的有效性:当卫星数据可用且可靠时,模型能够充分利用其空间结构信息。

案例(b) 完全缺失(物理回退)。在夜间或全云覆盖导致AOD完全缺失的场景下,AOD损失项自动失效($\boldsymbol{\omega}^{\text{AOD}}$全为零)。此时,模型并未瘫痪,而是稳健地回退(Fallback)到由物理核(平流与扩散)驱动的推断模式,仅凭气象场和地面站点生成了物理上合理的平滑场。这证明了模型具备全天候运行能力——这是相比传统AOD输入方法的核心优势。

案例(c) 结构冲突(鲁棒性)。当AOD出现虚假高值或纹理(可能由地表反射率误差引起)时,模型优先遵循地面观测的强约束($\mathcal{L}_{\mathrm{infer}}$),未被卫星伪影误导,避免了虚假热点的产生。这种\cqt{地面为主、卫星为辅}的层级约束结构确保了推断结果的可靠性。

案例(d) 幅度偏差(自动校准)。当AOD整体高估(颜色过深)时,模型仅提取了其\cqt{形状(梯度)}信息,而利用地面站点校准了\cqt{数值(幅度)}。这证明了基于梯度的损失函数成功实现了形状与数值的解耦——即使AOD存在系统性偏差,模型仍能从中提取有价值的空间结构信息。


% ------------------------------------------------------------
% 4.7 本章小结
% ------------------------------------------------------------
\section{本章小结}
\label{sec:infer_summary}

针对区域大气污染监测中的两个核心问题——稀疏监测网络下的空间泛化推断(问题一)和非随机缺失遥感数据的有效融合(问题二),本章提出了基于物理启发的归纳式图推断网络(SPIN),通过四阶段数据融合与推断流程(第\ref{sec:spin_model}节)系统性地加以解决。站点推断$\mathcal{F}^{\text{s}}_{\Theta^{\text{s}}}$与网格推断$\mathcal{F}^{\text{g}}_{\Theta^{\text{g}}}$采用相同的四阶段架构与图核构建方法,但各自独立训练、参数不共享,仅在输入节点集合与损失函数上有所区别(网格推断额外引入AOD梯度约束)。实验评估围绕第\ref{sec:infer_problem}节定义的两个子任务展开:站点推断实验验证了模型的空间泛化精度与鲁棒性,网格推断实验展示了物理核与AOD约束协同生成全域污染地图的能力。主要贡献总结如下:

(1)\textbf{构建了物理嵌入的双流图网络架构。}通过显式设计扩散核与平流核,模型成功解耦了浓度梯度驱动的各向同性扩散(公式\eqref{eq:diffusion_normalized})与风场驱动的各向异性平流输送(公式\eqref{eq:advection_adjacency})两类物理过程,分别对应公式(\ref{eq:advection_diffusion})中的扩散项与平流项。平流核根据实时风场动态调整边权重,显式编码\cqt{上游影响下游}的定向传输规律;扩散核模拟浓度梯度驱动的均一化过程,在数据稀疏区域起到\cqt{保守滤波}作用。这种设计与第\ref{chap:prediction}章PM$_{2.5}$-GNN和PCDCNet的物理启发思想一脉相承。

(2)\textbf{实现了归纳式时空推断能力。}通过动态掩码训练策略(公式\eqref{eq:inductive_sampling}),模型摆脱了对固定图结构的依赖,具备了对任意空间位置进行\cqt{虚拟传感}的能力。站点推断实验证明,该模型在未观测站点上的推断精度显著优于现有最优方法(MAE降低25.2\%),特别是在冬季复杂气象条件下表现出极强的鲁棒性。

(3)\textbf{提出了全天候制图的AOD融合范式。}创新性地提出\cqt{掩码AOD梯度损失}(公式\eqref{eq:aod_loss}),实现了从\cqt{依赖卫星输入}到\cqt{利用卫星约束}的范式转换。网格推断实验表明,这一机制不仅有效利用了晴空下的卫星空间信息,更在卫星数据缺失(云/夜)时,保证了系统能够平滑回退到物理驱动模式,从而生成了时空连续、物理一致的高分辨率污染地图。

从方法论角度,本章将第\ref{subsec:paradigm}节提出的\cqt{物理启发数据驱动建模}范式从时间维度(第\ref{chap:prediction}章的预测)扩展到空间维度(推断),通过在图神经网络中显式嵌入扩散--平流双核物理机制并采用归纳式学习策略,使模型从\cqt{监测什么推断什么}的直推式范式跨越到\cqt{学习物理规律、推断任意位置}的归纳式范式。SPIN有效地充当了\cqt{虚拟传感器}的角色:在站点推断中为未受监控的郊区和农村地区提供可靠的逐小时浓度估计(MAE降低25.2\%),在网格推断中生成$0.25^{\circ}$分辨率的全天候污染场,可用于识别监测盲区中的污染热点与评估环境暴露公平性。本章构建的高分辨率污染场也为后续第\ref{chap:simulation}章开展精细化的情景模拟奠定了空间数据基础。