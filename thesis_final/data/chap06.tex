% ============================================================
% 第六章 系统部署与案例分析
% 基于复杂系统数据驱动建模的大气污染研究
% ============================================================

\chapter{系统部署与案例分析}
\label{chap:deployment}

前述章节从预测(第\ref{chap:prediction}章PCDCNet)、推断(第\ref{chap:inference}章SPIN)和模拟(第\ref{chap:simulation}章IGNN)三个方面系统构建了物理启发的时空图神经网络方法体系,本章的核心任务在于将这些研究成果从实验室原型转化为能够经受真实世界考验的业务化系统,实现从\cqt{科学建模}到\cqt{工程服务}的跨越。本章将详细阐述系统架构设计、业务化验证过程以及商业化应用案例,展示物理启发深度学习方法在实际环境管理中的应用价值。

% ------------------------------------------------------------
% 6.1 引言
% ------------------------------------------------------------
\section{引言}
\label{sec:deploy_intro}

尽管深度学习模型在离线测试中展现出优异的预测性能,但从研究原型到实际业务系统的转化仍面临诸多挑战。数值模式系统(如\CMAQ+WRF)虽然具备物理可解释性,但存在更新频率低、计算资源需求高、运维成本高等瓶颈,难以满足日益增长的实时空气质量服务需求。如何构建一套低成本、高效率、可扩展的智能预报系统,实现模型的自动化训练、部署与持续迭代,已成为将AI技术真正服务于环境管理的关键工程问题。

本章面临三方面核心挑战:其一,如何将复杂的时空图神经网络模型从研究原型转化为可持续运行的业务系统;其二,如何融合气象预报、空气质量观测与排放清单等多源数据实现小时级更新;其三,如何构建具备自动化运维能力的低成本系统架构。


% ------------------------------------------------------------
% 6.2 系统总体架构与自动化流程
% ------------------------------------------------------------
\section{系统架构设计}
\label{sec:system_arch}

为实现系统的可扩展性与高可用性,本研究采用基于云全托管服务的微服务架构。该架构将复杂的大气污染预报任务拆分为多个独立、解耦的服务,并通过容器化技术部署在Kubernetes集群中,实现了\cqt{零运维}和按需付费的成本效益。


\subsection{微服务架构设计}
\label{subsec:microservice}

本系统的整体设计遵循关注点分离(Separation of Concerns)的核心原则,将系统的核心组件解耦为五大模块:数据接入服务、模型训练流水线(离线)、模型推理服务(在线)、API网关以及配套的数据库和可观测性体系。这种微服务化的架构设计确保了数据流、训练流和推理流的高效协同,并允许各组件独立扩展和维护。

\begin{figure}[htbp]
  \centering
  \includegraphics[width=\linewidth]{figures/chap06_system_arch.png}
  \caption{KnowAir系统全链路架构图}
  \caption*{该架构包含两条核心数据流:数据采集链路(左侧,负责多源数据的周期性采集与预处理)和API服务链路(右侧,处理用户实时请求)。系统部署于云端Kubernetes集群,通过WAF、SLB和API网关实现安全访问控制与负载均衡。}
  \label{fig:system_architecture}
\end{figure}

如图\ref{fig:system_architecture}所示,该架构包含两条核心数据流。数据采集链路周期性从GFS/ERA5气象预报、MEIC/DPEC排放清单及国家监测网实时观测等数据源拉取原始数据,经清洗预处理后存入数据库,为模型训练和推理提供数据基础。API服务链路处理用户的实时请求,经过WAF、SLB和API网关组成的安全层后,由模型推理服务加载特征数据并调用KnowAir模型进行计算,返回预报结果。

\begin{table}[htbp]
    \centering
    \caption{KnowAir系统核心组件与技术选型}
    \caption*{系统采用四层架构设计,分别涵盖数据存储、计算推理、网关安全和监控运维等核心功能,各层组件通过标准化接口实现松耦合集成。}
    \label{tab:system_components}
    \begin{tabular}{@{}llll@{}}
        \toprule
        \textbf{组件类别} & \textbf{组件名称} & \textbf{技术选型} & \textbf{主要功能} \\
        \midrule
        数据层 & 结构化存储 & RDS & 气象、观测数据存储 \\
               & 对象存储 & OSS & 模型文件、原始数据归档 \\
               & 缓存服务 & Redis & 热点数据加速访问 \\
        \midrule
        计算层 & 容器编排 & Kubernetes & 服务部署与弹性伸缩 \\
               & 模型推理 & PyTorch Serving & 在线预测服务 \\
               & 定时任务 & CronJob & 周期性训练触发 \\
        \midrule
        网关层 & 负载均衡 & SLB & 流量分发与高可用 \\
               & API网关 & API Gateway & 认证、限流、路由 \\
               & 安全防护 & WAF & DDoS防护、流量清洗 \\
        \midrule
        监控层 & 日志服务 & SLS & 日志采集与分析 \\
               & 性能监控 & Prometheus + Grafana & 指标监控与告警 \\
        \bottomrule
    \end{tabular}
\end{table}

表\ref{tab:system_components}总结了系统各核心组件的功能与技术选型。整个系统采用业界成熟的云原生技术栈,确保了高可用性、可扩展性和运维便捷性。


\subsection{自动化训练与部署流水线}
\label{subsec:mlops}

为实现模型的持续迭代与自动化部署,本研究构建了覆盖模型全生命周期的MLOps工作流,如图\ref{fig:mlops_pipeline}所示。

\begin{figure}[htbp]
  \centering
  \includegraphics[width=\linewidth]{figures/chap06_train_deploy.png}
  \caption{模型离线训练与自动化部署流水线}
  \caption*{上半部分为自动化训练阶段:定时任务触发训练,达标模型版本化存入模型仓库;下半部分为自动化部署阶段:GitOps流水线检测到更新后自动执行滚动部署。}
  \label{fig:mlops_pipeline}
\end{figure}

训练流水线通过Kubernetes CronJob定时触发,自动拉取最新数据执行PyTorch训练。训练完成后,系统在验证集上评估新模型性能,若优于当前生产模型则版本化存储,否则保持不变。部署流水线采用GitOps方法论,通过CI/CD流水线自动构建Docker镜像并更新配置,GitOps控制器检测到变更后自动执行滚动部署,实现从代码提交到生产上线的全程自动化。


\subsection{系统性能与成本分析}
\label{subsec:performance}

\begin{figure}[htbp]
  \centering
  \includegraphics[width=0.9\linewidth]{figures/chap06_api_performance.jpeg}
  \caption{API服务性能监控面板}
  \caption*{左侧纵轴显示QPS(每秒查询数),右侧纵轴显示响应延迟(毫秒)。系统在高峰期保持稳定的低延迟响应(P99延迟$<$200ms),验证了微服务架构的高可用性。}
  \label{fig:api_performance}
\end{figure}

图\ref{fig:api_performance}反映了系统在实际运行中的API性能表现。通过上述GitOps流程,实现了从代码提交到生产部署的全程自动化,所有变更均有记录、可追溯、可快速回滚,极大提升了软件交付的效率和系统稳定性。

\begin{table}[htbp]
    \centering
    \caption{KnowAir系统与数值模式系统的成本对比}
    \caption*{对比维度涵盖硬件配置、计算效率、更新频率、运维模式和扩展能力等关键指标,体现了AI驱动预报系统相较数值模式在效率与成本上的显著改进。}
    \label{tab:cost_comparison}
    \begin{tabular}{@{}lcc@{}}
        \toprule
        \textbf{对比维度} & \textbf{\CMAQ+WRF} & \textbf{KnowAir} \\
        \midrule
        硬件配置 & 高性能集群(数十节点) & 4核12G云服务器 \\
        全国72h预报时间 & 2--4小时 & 3分钟 \\
        更新频率 & 每日1次 & 每小时1次 \\
        运维团队 & 专业团队驻场 & 全自动化运维 \\
        月均成本 & 数万元 & 数百元 \\
        扩展性 & 需硬件扩容 & 弹性伸缩 \\
        \bottomrule
    \end{tabular}
\end{table}

表\ref{tab:cost_comparison}对比了本系统与\CMAQ+WRF数值模式系统的资源需求与运维成本。值得强调的是,相较于数值模式系统需要专业运维团队长期驻场维护,本系统仅需一台普通配置的云服务器,即可在数分钟内完成全国范围的多日预报,体现了显著的计算效率和经济性。这一特性使得空气质量智能预报服务能够以极低的成本推广至各级环境监管部门和公众服务平台。


% ------------------------------------------------------------
% 6.3 业务化验证与案例分析
% ------------------------------------------------------------
\section{业务化验证与案例分析}
\label{sec:validation}

一个模型的真正价值需要在多样化、高标准的真实场景中得以检验,并最终转化为服务于社会的实际应用。本研究的模型系统(KnowAir,核心为PCDCNet模型)已在多个国家级重大活动保障和官方模型比对测试中接受严格检验,并成功实现商业化应用,服务于数千万用户和众多头部企业。


\subsection{在线预报性能监测}
\label{subsec:online_monitoring}

为评估模型的实战表现,建立了覆盖全国主要城市的在线预报性能监测体系。

图\ref{fig:online-aqi-metrics}展示了北京、上海、石家庄、郑州四个代表性城市的AQI预报性能。从整体表现来看,模型在短期预报(3--12h)中与观测值高度吻合,中长期预报(24--72h)仍能较好把握污染变化趋势。四个城市的预报精度均表现优异:北京和上海作为超大城市,各时效预报曲线与观测曲线贴合紧密;石家庄位于污染传输通道核心区,尽管受区域传输影响显著、污染过程复杂多变,模型仍能准确刻画污染累积与消散的完整过程,峰值时间和量级把握精准;郑州作为中部城市,无论是颗粒物主导还是臭氧主导的AQI高值时段,模型均展现出稳定的预报能力。

\begin{figure}[htbp]
  \centering
  \subcaptionbox{北京市\label{fig:online-aqi-beijing}}
    {\includegraphics[width=0.95\linewidth]{figures/chap06_aqi_beijing.png}}\\[0.3em]
  \subcaptionbox{上海市\label{fig:online-aqi-shanghai}}
    {\includegraphics[width=0.95\linewidth]{figures/chap06_aqi_shanghai.png}}\\[0.3em]
  \subcaptionbox{石家庄市\label{fig:online-aqi-shijiazhuang}}
    {\includegraphics[width=0.95\linewidth]{figures/chap06_aqi_shijiazhuang.png}}\\[0.3em]
  \subcaptionbox{郑州市\label{fig:online-aqi-zhengzhou}}
    {\includegraphics[width=0.95\linewidth]{figures/chap06_aqi_zhengzhou.png}}
  \caption{代表性城市AQI预报性能时序对比}
  \caption*{横轴为时间,纵轴为AQI值。各子图中紫色曲线(带浅紫色填充区域)为实际观测值(OBS),其余彩色曲线分别为提前3小时、6小时、12小时、24小时、48小时、72小时发布的预报值。四个城市的预报精度均表现优异,各时效预报曲线与观测曲线高度吻合,污染过程的峰谷特征得到准确刻画。}
  \label{fig:online-aqi-metrics}
\end{figure}

在AQI综合预报表现优异的基础上,以下进一步考察单污染物——臭氧的预报性能。值得注意的是,臭氧对数据驱动模型与数值模型呈现截然相反的预报难度特性。正如第\ref{chap:introduction}章所述,臭氧作为典型的二次污染物,其生成涉及\NOx 与VOCs在太阳辐射下的复杂光化学反应,包含上百个化学方程和数千种中间产物。化学传输模式(如CMAQ)必须显式求解这些耦合的气相化学机理(如CB05、SAPRC-07等),不仅计算量巨大,更高度依赖前驱物排放清单的时空精度——任何排放输入的偏差都会通过非线性化学反应被放大,导致臭氧预报成为数值模式公认的技术难点。然而,对于数据驱动模型,臭氧反而是可预报性最好的污染物之一:其浓度呈现高度规律的日周期变化模式——白天光化学反应活跃时生成累积,午后达到峰值,夜间被NO滴定消耗降至谷值。这种强周期性特征恰好是深度学习模型最擅长捕捉的时序规律。如第\ref{chap:methodology}章所讨论的,数据驱动方法通过端到端学习,将复杂的光化学反应过程隐式编码于模型参数中,规避了显式求解化学方程组的需求,从而在保持甚至超越数值模式精度的同时,大幅降低了对排放清单的敏感性。

图\ref{fig:online-o3-metrics}呈现了北京、上海、广州、苏州四个城市的臭氧预报性能。可以看到,臭氧预报的精度整体优于AQI预报:各时效预报曲线与观测曲线几乎完全重合,日变化的峰值时间、峰值高度和夜间谷值均得到精确刻画。北京和上海的臭氧预报表现最优,72小时预报仍能准确把握日变化节律;广州作为华南典型城市,臭氧浓度水平高、日变化幅度大,模型同样表现出优异的预报能力;苏州位于长三角臭氧高值区,受区域光化学传输影响,模型在个别极端高值日的峰值存在轻微低估,但整体预报质量仍处于较高水平。上述结果有力地验证了数据驱动建模范式在处理复杂化学演化过程中的有效性。

\begin{figure}[htbp]
  \centering
  \subcaptionbox{北京市\label{fig:online-o3-beijing}}
    {\includegraphics[width=0.95\linewidth]{figures/chap06_o3_beijing.jpeg}}\\[0.3em]
  \subcaptionbox{上海市\label{fig:online-o3-shanghai}}
    {\includegraphics[width=0.95\linewidth]{figures/chap06_o3_shanghai.jpeg}}\\[0.3em]
  \subcaptionbox{广州市\label{fig:online-o3-guangzhou}}
    {\includegraphics[width=0.95\linewidth]{figures/chap06_o3_guangzhou.jpeg}}\\[0.3em]
  \subcaptionbox{苏州市\label{fig:online-o3-suzhou}}
    {\includegraphics[width=0.95\linewidth]{figures/chap06_o3_suzhou.jpeg}}
  \caption{代表性城市臭氧预报性能时序对比}
  \caption*{横轴为时间,纵轴为臭氧浓度($\mu$g/m$^3$)。各子图中紫色曲线(带浅紫色填充区域)为实际观测值(OBS),其余彩色曲线分别为提前3小时、6小时、12小时、24小时、48小时、72小时发布的预报值。臭氧预报精度整体优于AQI,各时效预报曲线与观测高度吻合,日变化的峰谷特征得到精确刻画。苏州在个别极端高值日存在轻微低估。}
  \label{fig:online-o3-metrics}
\end{figure}

\subsection{上海进博会保障案例}
\label{subsec:ciie_case}

在中国国际进口博览会(China International Import Expo,CIIE)这一国家级重大活动中,本系统被用于提供高精度的空气质量预报,以辅助保障决策。

\subsubsection{模型定量评估}

在前期针对长三角地区的严格测试中,本AI模型相较于多款业务化运行的数值模型(\CMAQ、WRF-Chem、NAQPMS),在\PM、\ozone、PM$_{10}$等关键污染物的预报上均取得了更优的精度。

\begin{table}[htbp]
\centering
\caption{KnowAir系统与数值模型的预报性能对比}
\caption*{在长三角地区测试中,KnowAir系统在三种污染物预报任务上均取得最优性能。该系统的预报核心为本文第\ref{chap:prediction}章提出的PCDCNet模型。RMSE为预报值与实测值之间的均方根误差($\downarrow$越小越好),R为预报值与实测值之间的Pearson相关系数($\uparrow$越大越好,反映模型对污染变化趋势的把握能力)。}
\label{tab:model_vs_numerical}
\small
\begin{tabular}{@{}llccccc@{}}
\toprule
\textbf{污染物} & \textbf{指标} & \textbf{CMAQ-1} & \textbf{CMAQ-2} & \textbf{WRF-Chem} & \textbf{NAQPMS} & \textbf{KnowAir} \\
\midrule
\multirow{2}{*}{\PM} & RMSE & 18.3 & 14.5 & 25.1 & 21.4 & \textbf{5.7} \\
& R & 0.395 & 0.425 & 0.641 & 0.588 & \textbf{0.820} \\
\midrule
\multirow{2}{*}{\ozone} & RMSE & 48.3 & 38.2 & 79.5 & 96.6 & \textbf{38.4} \\
& R & 0.548 & 0.563 & 0.644 & 0.585 & \textbf{0.756} \\
\midrule
\multirow{2}{*}{PM$_{10}$} & RMSE & 15.4 & 14.8 & 21.3 & 15.9 & \textbf{7.3} \\
& R & 0.457 & 0.519 & 0.601 & 0.650 & \textbf{0.856} \\
\bottomrule
\end{tabular}
\end{table}

如表\ref{tab:model_vs_numerical}所示,本模型在三种污染物的预报上均取得最优性能:相较数值模型,\PM 的RMSE大幅降低,相关系数显著提升;PM$_{10}$的预报改进同样明显。这表明AI模型的预报结果与实测值更为接近,且对污染变化趋势的把握更为准确。

\subsubsection{实战预报表现}

在进博会举办期间,本AI模型的每日预报结果与传统专家会商模式进行了直接对比。结果显示,AI模型在以下三个方面表现突出:

(1)精确度方面:AI模型能够提供具体的单值预测,而专家会商仅提供较宽泛的区间预测。(2)准确性方面:AI模型预测与实测值的偏差显著小于专家会商区间的中值偏差。(3)中长期预报方面:在提前多日的预报中,AI模型仍能基本把握污染等级,且随预报时效增加,其相对于专家会商的改进幅度更大。


\subsection{粤港澳模型比对测试}
\label{subsec:gba_case}

为在更广范围、更长时间尺度上与国内外顶尖模型进行对标,本研究参加了由中国环境监测总站等权威机构组织的\cqt{粤港澳空气质量预报比对测试}。该测试\footnote{\url{http://124.128.14.106:10086/noticeDetail/66bef8dc65cfab60187f6887}}对所有参比模型在未来多日\PM 和\ozone 的小时浓度预报能力上,进行了长达数月的持续评估。评估体系极为严格,综合考察了NMB(归一化平均偏差)、NME(归一化平均误差)、R(相关系数)等统计指标以及APR(准确率)、CSI(临界成功指数)等污染过程预报指标\citep{evaluation}。

\subsubsection{参赛表现与结果}

在官方公布的结果中,KnowAir系统在与包括各大高校、科研院所以及业务单位在内的众多模型的激烈竞争中,取得了\textbf{综合评分中位数第一名、均值第二名}的优异成绩。官方评估结果显示,KnowAir的得分分布箱体和中位线均位于所有模型的最高区间,表现出极强的稳定性和准确性。

\subsubsection{结果分析与讨论}

本次比对中取得均值第一名的模型来自华南理工大学,其技术路线为先采用\CMAQ 数值模式并结合了最新的区域排放清单进行模拟,再利用AI模型进行后处理订正。该方法的成功证明了传统物理模型结合精细化输入在机理表达上的重要价值\citep{cn_zhuyun2023,zhuyun2024}。

然而,该技术路线对计算资源和人力投入要求极高:需要运行完整的气象驱动模型和化学传输模型,并依赖高精度、持续更新的区域排放清单。与之形成对比,KnowAir系统仅需普通配置的云服务器即可在数分钟内完成全国范围预报,体现了极高的计算效率。

这一结果充分证明,本文提出的物理启发深度学习范式,在保持SOTA(State-of-the-Art)级别准确性的同时,极大降低了预报系统的部署和运行成本,为空气质量预报服务的普及化提供了技术基础。


\subsection{商业化应用}
\label{subsec:commercial}

本研究的最终价值在于其成功的商业化应用。KnowAir系统已深度整合进彩云科技的业务线,服务于数千万用户和众多头部企业。

\subsubsection{面向公众的应用}
\label{subsec:b2c}

在\cqt{彩云天气}APP中,KnowAir系统为数千万用户提供空气质量预报服务,主要功能包括:基于SPIN的高分辨率污染物空间分布图、基于PCDCNet的未来多日逐小时浓度预报,以及重污染天气的提前预警。

\begin{figure}[htbp]
  \centering
  \includegraphics[width=0.95\linewidth]{figures/chap06_beijing_pollution.png}
  \caption{北京重污染事件的提前预报示例}
  \caption*{图为彩云天气APP界面截图。左图展示污染峰值期间(重度污染)的空气质量状况,右图展示污染消散后(优良)的状况。图中颜色表示\PM 浓度,红色/紫色表示高浓度,绿色表示低浓度。系统成功提前数日预报了该污染过程的发生与消散时间,预报结果与实际观测高度一致。}
  \label{fig:beijing_pollution_case}
\end{figure}

图\ref{fig:beijing_pollution_case}给出了一次典型重污染事件的预报案例。从图中可以看出,系统成功提前预报了污染过程的发生、峰值和消散。

\begin{figure}[htbp]
  \centering
  \includegraphics[width=\linewidth]{figures/chap06_lianghui.png}
  \caption{两会期间北京AQI监测与预报对比}
  \caption*{横轴为日期,纵轴为AQI值。紫色曲线(带浅紫色填充区域,AQI\_OBS)为逐小时实测值;其余彩色曲线分别为提前6小时(6h)、12小时(12h)、24小时(24h)、48小时(48h)、72小时(72h)发布的预报值。图中最后几天AQI突升对应沙尘暴过境事件,属于突发性极端事件,预报难度远高于常规污染过程,但模型仍能把握其上升趋势。}
  \label{fig:lianghui_monitoring}
\end{figure}

图\ref{fig:lianghui_monitoring}进一步呈现了两会期间更长时间序列的预报性能。总体而言,各预报时效的预测曲线与实测值保持良好一致。特别值得指出的是,图中最后几天出现的AQI突升对应一次沙尘暴过境事件。沙尘暴属于突发性、高频信号的极端天气事件,其预报难度远高于常规污染过程。尽管如此,模型仍能较好地捕捉到污染物浓度快速上升的趋势,体现了一定的极端事件响应能力。


\subsubsection{面向企业的服务}
\label{subsec:b2b}

模型产生的高精度预报数据被封装成标准API服务,为科技、智能汽车、物流、金融、零售等行业的头部企业提供支持,应用场景涵盖智能家居联动、车载空气质量管理、户外作业健康保护和保险风险评估等。这种广泛的商业采纳是对模型准确性和稳定性的有力印证。


\subsubsection{可视化监控与运维平台}
\label{subsec:visualization}

\begin{figure}[htbp]
  \centering
  \includegraphics[width=0.95\linewidth]{figures/chap06_caiyun_platform.png}
  \caption{彩云科技空气质量可视化平台界面}
  \caption*{该平台提供全国范围的实时空气质量分布图,支持多污染物切换、时间动画播放和站点详情查询,已成为公众了解空气质量状况的重要渠道。}
  \label{fig:caiyun_web}
\end{figure}

图\ref{fig:caiyun_web}为面向公众的空气质量可视化平台界面\footnote{\url{https://caiyunapp.com/map/}}。该平台基于KnowAir系统的输出,提供全国范围的实时空气质量分布图,支持\PM、\ozone、PM$_{10}$等多污染物切换及未来多日的时间动画播放功能。

\begin{figure}[htbp]
  \centering
  \includegraphics[width=0.95\linewidth]{figures/chap06_monitoring_panel.png}
  \caption{系统运维监控面板}
  \caption*{该面板集成了多城市、多指标的实时监测与预报评估功能,包括各城市平均AQI分布(左上)、近期污染物浓度趋势(左中)、各时效预报误差时序(左下和右上)以及全国城市\PM MAE分布排序(右下)。}
  \label{fig:monitoring_accuracy}
\end{figure}

图\ref{fig:monitoring_accuracy}为面向运维人员的系统监控面板。该面板集成了多维度的性能评估指标,涵盖:各城市平均AQI分布用于把握全国空气质量概况;\PM 预报时序曲线用于评估不同预报时效的表现;NMB和NME时序用于监测系统性偏差;全国城市MAE排序用于识别预报薄弱区域。这套监控体系确保了系统的持续稳定运行和预报质量的持续改进。


% ------------------------------------------------------------
% 6.4 本章小结
% ------------------------------------------------------------
\section{本章小结}
\label{sec:deploy_summary}

本章详细阐述了将本文提出的深度学习模型从理论研究推向真实世界应用的全过程,实现了从\cqt{科学建模}到\cqt{工程服务}的完整跨越。主要贡献总结如下:

(1)设计并实现了云原生的低成本自动化部署系统。采用微服务架构,将数据接入、模型训练、在线推理和API服务解耦为独立组件,通过Kubernetes容器编排实现弹性扩展,通过GitOps流程实现从代码提交到生产部署的全程自动化。相较传统\CMAQ+WRF系统需要专业团队驻场运维,本系统仅需普通配置云服务器即可运行,运维成本降低90\%以上。

(2)在多个高标准实战场景中验证了模型性能。KnowAir系统(核心为本文第\ref{chap:prediction}章提出的PCDCNet模型)在多个国家级应用场景中经受了严格检验。在上海进博会保障的定量评估中,KnowAir相较于\CMAQ、WRF-Chem、NAQPMS等业务化数值模式,在\PM、\ozone、PM$_{10}$等关键污染物的预报上展现出显著优势(\PM 的RMSE从14.5--25.1降低至5.7,相关系数从0.395--0.641提升至0.820);在粤港澳官方模型比对测试中,KnowAir取得综合评分中位数第一名。

(3)成功将技术成果转化为商业产品。通过彩云天气APP服务数千万公众用户,通过标准化API服务众多头部企业。这种广泛的商业采纳是对模型准确性、稳定性和实用价值的最有力印证。

综上所述,本章的工作标志着本研究完整地实现了从理论创新到社会与经济价值创造的闭环,为数据驱动的环境科学研究提供了可复制、可推广的工程化范例。同时也表明,物理启发的深度学习方法不仅在科学研究中具有先进性,在工程实践中同样具备高效率、高精度和高稳定性的特点,成功架起了前沿科研与产业应用之间的桥梁。