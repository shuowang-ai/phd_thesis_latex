% !TeX root = ../wangshuo_phdthesis.tex

\begin{resume}

  \section*{学术论文}

  \begin{achievements}
    \item \textbf{Shuo Wang}, Yanran Li, Jiang Zhang, Qingye Meng, Lingwei Meng, Fei Gao. PM$_{2.5}$-GNN: A Domain Knowledge Enhanced Graph Neural Network For PM2.5 Forecasting[C]. ACM SIGSPATIAL International Conference on Advances in Geographic Information Systems, 2020.
    \item \textbf{Shuo Wang}, Yun Cheng, Qingye Meng, Olga Saukh, Jiang Zhang, Jingfang Fan, Yuanting Zhang, Xingyuan Yuan, Lothar Thiele. PCDCNet: A Surrogate Model for Air Quality Forecasting with Physical-Chemical Dynamics and Constraints[J]. under review, 2025.
    \item \textbf{Shuo Wang}, Mengfan Teng, Yun Cheng, Lothar Thiele, Olga Saukh, Shuangshuang He, Yuanting Zhang, Jiang Zhang, Gangfeng Zhang, Xingyuan Yuan, Jingfang Fan. Physics-Guided Inductive Spatiotemporal Kriging for PM2.5 with Satellite Gradient Constraints[J]. under review, 2025.
    \item Gangfeng Zhang, \textbf{Shuo Wang} (共同第一作者), Jing Xu, Tim R. McVicar, Yun Cheng, Jiang Zhang, Cesar Azorin-Molina, Lorenzo Minola, Peijun Shi. A deep learning approach predicts O$_3$ increase and PM$_{2.5}$ declines under future high carbon emission scenario across the Northern China Plain[J]. Urban Climate, 2026.
    \item Ziqi Lin, \textbf{Shuo Wang}, Jing Xu, Peijun Shi, Yaoyao Ma, Yiwen Wang, Gangfeng Zhang. A Graph Neural Networks approach predicted spatiotemporal changes of Ozone Concentrations in the Yangtze River Delta (China)[J]. Environmental Research Communications, 2025.
    \item Jing Xu, \textbf{Shuo Wang}, Na Ying, Xiao Xiao, Jiang Zhang, Zhiling Jin, Yun Cheng, Gangfeng Zhang. Dynamic Graph Neural Network with Adaptive Edge Attributes for Air Quality Prediction: A Case Study in China[J]. Heliyon, 2023.
    \item Xiao Xiao, Zhiling Jin, \textbf{Shuo Wang}, Jing Xu, Ziyan Peng, Rui Wang, Wei Shao, Yilong Hui. A Dual-Path Dynamic Directed Graph Convolutional Network for Air Quality Prediction[J]. Science of The Total Environment, 2022.
    \item 林子琪, \textbf{王硕}, 许菁, 史培军, 马瑶瑶, 王怡雯, 张钢锋. 基于图神经网络的长三角臭氧浓度时空格局模拟[J]. 地理学报, 2025.
  \end{achievements}

  \section*{专利}

  \begin{achievements}
    \item 许菁, \textbf{王硕}, 营娜, 金志凌, 程云, 张江. 一种基于自适应动态图神经网络的空气质量预测方法:中国, 202210625446.8[P]. 2022-06-02.
  \end{achievements}

\end{resume}
