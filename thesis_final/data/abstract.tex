% !TeX root = ../wangshuo_phdthesis.tex
% 中英文摘要和关键字

\begin{abstract}
\bnusetup{
keywords = {地球系统科学, 数据驱动建模, 大气污染智能预测, 物理启发深度学习, 时空图神经网络},
}

大气污染是当前全球最严峻的环境与公共卫生挑战之一,每年导致约700万人过早死亡。我国\PM 年均浓度从2013年的72~\ug 降至2023年的30~\ug{},但与世界卫生组织指导值(5~\ug{})仍有较大差距;\PM 持续下降而\ozone 浓度逐年攀升的分化态势,更给协同治理带来了新的科学难题。如何实现高精度的空气质量预测、如何在无观测区域推断污染浓度分布、如何高效模拟不同减排政策下的空气质量演变,已成为支撑精细化污染防控的三项关键科学问题。

从系统科学视角审视,大气污染具备复杂系统的典型特征:排放、气象与化学演化三大子系统紧密耦合,跨区域传输沿地形廊道形成有向时变网络,多源数据在时空结构与物理内涵上差异显著。化学传输模式虽具物理可解释性,却受困于计算代价高昂、对排放清单高度依赖等瓶颈;纯数据驱动方法虽计算高效,却缺乏物理规律的显式建模,在极端情景下泛化能力不足。如何融合两者优势、构建兼具可解释性与高效性的智能建模框架,是当前大气环境领域的前沿科学问题。

针对上述困境,本研究提出物理启发的图神经网络建模范式。该范式以对流--扩散方程为物理基础,以图神经网络为核心建模工具,通过图结构的构建与约束函数的设计,将对流--扩散方程等物理先验知识系统性地嵌入深度学习框架。围绕空气质量预测、无观测区域浓度推断与减排情景模拟三类核心任务,本文整合地面监测、卫星遥感、气象与排放清单四类数据源,发展了一套物理与数据深度融合的空气质量智能建模方法体系。

在空气质量预测方面,本文提出\ModelPred 与\ModelSurr 两个模型。\ModelPred 利用风速向量投影定义图边权重,在图神经网络框架下显式构建\cqt{上风向影响下风向}的定向传输关系,首次将大气污染物沿风场传播的物理规律融入图结构设计。\ModelSurr 从对流--扩散方程出发,设计三模块架构分别对应化学反应、平流扩散与沉降累积三个物理过程,实现过程解耦与联合建模;同时提出领域一致性约束(DIC),将质量守恒等物理规律作为约束损失函数嵌入训练过程,保障预测结果的物理一致性。在京津冀及长三角区域72小时\PM 与\ozone 联合预报中,相比iTransformer、GC-LSTM等主流时空预测模型,RMSE降低13\%--23\%,重污染过程的峰值捕捉能力显著提升。

在无观测区域推断方面,本文提出\ModelInfer 模型,旨在基于稀疏地面监测数据重建高分辨率全域污染场。该模型构建扩散--平流双图并行机制,在图神经网络中实现对流--扩散过程的解耦表达;提出\cqt{以卫星AOD空间梯度为约束而非输入}的融合策略,通过掩码机制规避遥感缺测影响,实现全天候连续制图。在京津冀地区30\%站点缺测条件下,相比IGNNK、STGCN等基准模型,MAE降低约25\%。

在假设情景模拟方面,本文构建\ModelSim 模型,首次将污染物排放作为模型的输入引入,使模型既能预测污染物浓度,也能同时实现不同排放分布下的污染情景模拟。相比传统化学传输模式数小时的计算代价,本模型将单情景推理压缩至秒级。基于中国双碳排放路径数据集(DPEC),揭示了2025--2050年碳中和路径下\PM 持续下降而\ozone 普遍上升的反向演变规律,为协同减排决策提供科学依据。

在工程应用方面,本文基于云原生架构部署\SystemName 系统,该系统已服务于国家重大活动空气质量保障任务,并在中国环境监测总站官方模型比对中取得综合评分中位数第一名,验证了上述方法的实用性与可靠性。

本研究的核心贡献在于:提出物理启发的图神经网络建模范式,将图神经网络的关系建模能力与大气污染传输的网络化特征相结合,发展了预测、推断、模拟三位一体的空气质量智能建模方法体系。主要创新包括:(1)在图结构中显式编码\cqt{上风向影响下风向}的定向传输机制,并提出领域一致性约束(DIC)将物理守恒律嵌入模型训练;(2)提出以AOD空间梯度为约束的融合策略实现全天候推断;(3)首次将排放清单作为可控输入纳入深度学习框架,赋能假设情景模拟。完成从理论方法到业务系统的全链条落地,推动大气污染建模从\cqt{能预测}向\cqt{能推断、能模拟}跃升,为空气质量精细化管理提供智能技术支撑。本研究所发展的\cqt{物理先验驱动图结构与约束学习}方法论,可推广至地球科学乃至系统科学中涉及多源数据融合与复杂时空动力学建模的广泛问题,具有普适的理论与方法价值。

\end{abstract}

\begin{abstract*}
\bnusetup{
keywords* = {Earth System Science, Data-Driven Modeling, Intelligent Air Pollution Prediction, Physics-Inspired Deep Learning, Spatiotemporal Graph Neural Network},
}

Air pollution represents one of the most pressing environmental and public health challenges worldwide, causing approximately 7 million premature deaths annually. China's national average \PM concentration has declined from 72~\ug in 2013 to 30~\ug in 2023, yet a considerable gap remains compared to the WHO guideline of 5~\ug. Meanwhile, the divergent trends of declining \PM and rising \ozone pose new scientific challenges for synergistic pollution control. How to achieve accurate air quality prediction, how to infer pollution concentration in unmonitored areas, and how to efficiently simulate air quality evolution under different emission reduction policies have become three critical scientific problems supporting refined pollution prevention and control.

From a systems science perspective, atmospheric pollution exhibits typical characteristics of a complex system: emissions, meteorology, and chemical evolution form tightly coupled subsystems; cross-regional transport follows directed, time-varying networks shaped by terrain corridors; and multi-source data differ substantially in spatiotemporal structure and physical interpretation. Traditional chemical transport models possess physical interpretability but are constrained by high computational costs and strong emission dependence. Pure data-driven approaches, while computationally efficient, lack explicit physical modeling and demonstrate insufficient generalization under extreme events. How to integrate the respective advantages of physical knowledge and data-driven methods to construct an intelligent modeling framework with both interpretability and efficiency represents a frontier scientific challenge in the atmospheric environment field.

To address these challenges, this dissertation proposes a physics-inspired graph neural network (GNN) modeling paradigm. Built on the advection-diffusion equation as the physical foundation and graph neural networks as the core modeling tool, this paradigm systematically embeds physical prior knowledge---such as the advection-diffusion equation---into the deep learning framework through physics-informed graph construction and constraint function design. Targeting three core tasks---air quality prediction, pollution inference in unmonitored areas, and emission scenario simulation---this work integrates ground monitoring, satellite remote sensing, meteorological data, and emission inventories, developing a methodology that deeply fuses physical knowledge with data-driven modeling for air quality research.

For air quality prediction, \ModelPred and \ModelSurr are proposed. \ModelPred utilizes wind velocity projections to define graph edge weights, explicitly constructing ``upwind-to-downwind'' directional transport relationships within the GNN framework, pioneering the incorporation of wind-driven pollutant transport physics into graph structure design. \ModelSurr, grounded in the advection-diffusion equation, designs a three-module architecture corresponding to chemical reactions, advection-diffusion, and deposition-accumulation processes respectively, achieving process decomposition with joint modeling. Domain-Informed Constraints (DIC) are proposed to embed mass conservation and other physical laws as constraint loss functions into the training process, ensuring physical consistency of predictions. For 72-hour joint \PM and \ozone forecasting in the BTH and YRD regions, \RMSE is reduced by 13\%--23\% compared to mainstream spatiotemporal models such as iTransformer and GC-LSTM, with notably improved peak capture capability during heavy pollution episodes.

For pollution inference in unmonitored areas, \ModelInfer is proposed to reconstruct high-resolution regional pollution fields from sparse ground monitoring data. The model constructs a diffusion-advection dual-graph parallel mechanism to achieve decoupled representation of advection-diffusion processes within graph neural networks, and introduces an ``AOD spatial gradients as constraints rather than inputs'' fusion strategy that leverages satellite remote sensing gradient information to guide spatial inference while using masking mechanisms to handle data gaps, enabling all-weather continuous mapping. Under 30\% station missing conditions in the BTH region, \MAE is reduced by approximately 25\% compared to baselines such as IGNNK and STGCN.

For hypothetical scenario simulation, \ModelSim introduces pollutant emissions as model inputs for the first time, enabling the model to both predict pollutant concentrations and simultaneously simulate pollution scenarios under different emission distributions. Compared to traditional chemical transport models requiring hours of computation, this model compresses single-scenario inference to seconds. Using the China DPEC dataset, the model reveals opposing evolution patterns under 2025--2050 carbon neutrality pathways where \PM continues declining while \ozone generally rises, providing scientific basis for synergistic emission reduction decisions.

For engineering applications, the cloud-native \SystemName system is deployed, which has served air quality assurance for major national events and achieved the highest median composite score in official model comparison evaluations organized by CNEMC, validating the practicality and reliability of the proposed methods.

The core contributions of this research are: proposing a physics-inspired graph neural network modeling paradigm that combines the relational modeling capability of GNNs with the networked nature of atmospheric pollutant transport, developing an integrated air quality intelligent modeling methodology encompassing prediction, inference, and simulation. Key innovations include: (1) explicitly encoding ``upwind-to-downwind'' directional transport mechanisms in graph structures and proposing Domain-Informed Constraints (DIC) to embed physical conservation laws into model training; (2) proposing an AOD spatial gradient-as-constraint fusion strategy for all-weather inference; (3) incorporating emission inventories as controllable inputs into the deep learning framework for the first time to enable hypothetical scenario simulation. Full-chain implementation from theoretical methods to operational systems is completed, advancing air pollution modeling from ``capable of prediction'' to ``capable of inference and simulation,'' providing intelligent technology support for refined air quality management. The ``physics-prior-driven graph structure and constraint learning'' methodology developed in this research can be extended to a broad range of problems in earth science and systems science involving multi-source data fusion and complex spatiotemporal dynamics modeling, offering generalizable theoretical and methodological value.

\end{abstract*}