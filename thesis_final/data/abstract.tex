% !TeX root = ../wangshuo_phdthesis.tex
% 中英文摘要和关键字

\begin{abstract}
\bnusetup{
keywords = {地球系统科学, 数据驱动建模, 大气污染智能预测, 物理启发深度学习, 时空图神经网络},
}

大气污染是当前全球最严峻的环境与公共卫生挑战之一,每年导致约700万人过早死亡。我国\PM 年均浓度从2013年的72~\ug 降至2023年的30~\ug,但与世界卫生组织指导值(5~\ug)仍有较大差距;\PM 持续下降而\ozone 浓度逐年攀升的分化态势,更给协同治理带来了新的科学难题。精准的空气质量预测、无观测区域的污染推断以及减排政策的情景模拟,已成为支撑精细化污染防控的三项关键科学任务。

从系统科学视角审视,大气污染具备复杂系统的典型特征:排放、气象与化学演化三大子系统紧密耦合,跨区域传输沿地形廊道形成有向时变网络,多源数据在时空结构与物理内涵上差异显著。化学传输模式虽具物理可解释性,却受困于计算代价高等瓶颈;纯数据驱动方法虽计算高效,却缺乏物理规律的显式建模,泛化能力不足。如何融合两者优势,构建兼具可解释性与高效性的智能建模框架,是当前大气环境领域的前沿科学问题。

针对上述困境,本研究提出\cqt{物理启发的数据驱动建模}范式,以对流--扩散方程为物理基础,面向预测、推断与模拟三类核心任务,发展了一套物理与数据深度融合的空气质量建模方法体系。该范式的核心思想是在数据融合、模型架构与损失函数三个层面系统性地嵌入大气科学先验知识,实现物理可解释性与数据驱动灵活性的有机统一。

在多源数据融合方面,本文整合地面监测(中国环境监测总站1600余个国控站点)、卫星遥感(Himawari-8气溶胶光学厚度AOD)、气象再分析(ERA5/GFS)与排放清单(\MEIC/\DPEC)四类异构数据。针对各数据源在时空分辨率与物理含义上的差异,设计了统一的时空对齐方案;通过物理启发的图构建,在统一隐空间中实现多源信息的有效融合。

在空气质量预测方面,本文提出\ModelPred 与\ModelSurr 两个模型。\ModelPred 利用风速向量投影定义图边权重,首次在图神经网络框架下显式建模\cqt{上风向影响下风向}的定向传输规律,克服了传统方法各向同性假设的局限。\ModelSurr 从对流--扩散方程出发,设计\LID--\STD--\TAD 三模块架构分别对应化学反应、平流扩散与沉降累积过程,实现\cqt{过程解耦与联合建模};同时引入领域一致性约束\DIC 将质量守恒嵌入训练目标。在京津冀及长三角72小时\PM 与\ozone 联合预报中,相比GC-LSTM、STGCN等基准模型,RMSE降低13\%--23\%,重污染过程的峰值捕捉能力显著提升。

在无观测区域推断方面,本文提出\ModelInfer 模型,旨在基于稀疏的地面监测数据重建高分辨率的全域污染场。构建扩散--平流双图并行机制,分别建模浓度梯度驱动的各向同性扩散与风场驱动的各向异性平流,实现对流--扩散方程在图神经网络中的显式解耦表达。针对卫星AOD云遮挡与夜间缺测问题,创新性地提出\cqt{以空间梯度为约束而非输入}的融合策略,使模型在AOD缺失时平滑回退至物理驱动的推断模式,实现全天候连续制图。采用动态节点掩码训练策略,使模型具备对任意未观测位置的归纳式泛化能力。在京津冀地区30\%站点缺测条件下,MAE为9.5~\ug,相比IGNNK等基线误差降低约25\%。

在假设情景模拟方面,本文构建\ModelSim 模型,首次将排放清单作为可控输入变量纳入深度学习框架,实现多情景污染高效模拟。设计排放特征编码与浓度场解码的一对一映射架构,避免递归预测中的误差累积。相比WRF-CMAQ等传统模式数小时的计算,本模型将单情景推理压缩至秒级。基于中国双碳排放路径数据集(DPEC),揭示了2025--2050年碳中和路径下\PM 持续下降而\ozone 普遍上升的反向演变规律,为协同减排决策提供科学依据。

在工程应用方面,本文基于云原生架构部署\SystemName 系统,采用容器化微服务设计,集成数据采集、模型推理、可视化与预警推送等功能模块。该系统已服务于国家重大活动空气质量保障任务,并在中国环境监测总站官方模型比对中取得综合评分中位数第一名,验证了本研究方法的实用性与可靠性。

本研究的核心贡献在于:提出物理启发的数据驱动建模框架,在数据融合、模型结构与损失函数三个层面实现大气科学知识与深度学习的有机融合;发展了预测、推断、模拟三位一体的空气质量智能建模方法体系,推动大气污染建模从\cqt{能预测}向\cqt{能推断、能模拟}跃升;完成从理论方法到业务系统的全链条落地,为空气质量精细化管理提供智能技术支撑。

\end{abstract}

\begin{abstract*}
\bnusetup{
keywords* = {Earth System Science, Data-Driven Modeling, Intelligent Air Pollution Prediction, Physics-Constrained Deep Learning, Spatiotemporal Graph Neural Network},
}

Air pollution represents one of the most pressing environmental and public health challenges worldwide, causing approximately 7 million premature deaths annually. China's national average \PM concentration has declined from 72~\ug in 2013 to 30~\ug in 2023, achieving remarkable progress in pollution control. However, a considerable gap remains compared to the WHO guideline of 5~\ug. Meanwhile, the divergent trends of declining \PM and rising \ozone pose new scientific challenges for synergistic pollution control. Accurate air quality prediction, pollution inference in unmonitored areas, and scenario simulation for emission reduction policies have become three critical scientific tasks supporting refined pollution prevention and control.

From a systems science perspective, atmospheric pollution exhibits typical characteristics of a complex system: emissions, meteorology, and chemical evolution form tightly coupled subsystems; atmospheric chemistry displays markedly nonlinear dynamics; cross-regional transport follows directed, time-varying networks shaped by terrain corridors; and multi-source data differ substantially in spatiotemporal structure and physical interpretation. Traditional chemical transport models (such as WRF-CMAQ) possess physical interpretability but are constrained by high computational costs and strong emission dependence. Pure data-driven approaches, while computationally efficient, lack explicit physical modeling and demonstrate insufficient generalization under extreme events and out-of-distribution scenarios. How to integrate the respective advantages of physical knowledge and data-driven methods to construct an intelligent modeling framework with both interpretability and efficiency represents a frontier scientific challenge in the atmospheric environment field.

This dissertation proposes a physics-constrained data-driven modeling paradigm based on the advection-diffusion equation as the physical foundation. Following the ``theory$\rightarrow$data$\rightarrow$modeling$\rightarrow$application$\rightarrow$deployment'' pipeline, this work develops a methodological framework that deeply integrates physical knowledge with data-driven methods for three core tasks: prediction, inference, and simulation. The core idea is to systematically embed atmospheric science prior knowledge at three levels---data fusion, model architecture, and loss functions---achieving organic unification of physical interpretability and data-driven flexibility.

For multi-source data fusion, this work integrates ground monitoring (over 1,600 national stations from CNEMC), satellite remote sensing (Himawari-8 AOD), meteorological reanalysis (ERA5/GFS), and emission inventories (\MEIC/\DPEC). Addressing differences in spatiotemporal resolution, coverage, and physical meaning, a unified spatiotemporal alignment scheme is designed. Through physics-inspired graph construction based on geographic proximity and wind field direction, effective multi-source fusion is achieved within a unified latent space, providing high-quality input data for subsequent modeling.

For air quality prediction, \ModelPred and \ModelSurr are proposed, targeting lightweight real-time forecasting and mechanistic refined prediction respectively. \ModelPred utilizes wind velocity projections onto station connection directions to define graph edge weights, achieving explicit modeling of ``upwind-to-downwind'' directional transport within the GNN framework for the first time, overcoming the limitations of isotropic assumptions in traditional methods. \ModelSurr designs a three-module architecture (\LID--\STD--\TAD) based on the advection-diffusion equation, corresponding to chemical reactions, advection-diffusion, and deposition-accumulation respectively, achieving ``process decomposition with joint modeling.'' Domain-Informed Constraints (\DIC) embed mass conservation laws into training objectives. For 72-hour joint \PM and \ozone forecasting in BTH and YRD regions, \RMSE is reduced by 13\%--23\% compared to baselines such as GC-LSTM and STGCN, with notably improved peak capture capability during heavy pollution episodes.

For pollution inference in unmonitored areas, \ModelInfer aims to reconstruct high-resolution regional pollution fields from sparse ground monitoring data. A diffusion-advection dual-graph parallel mechanism is constructed to separately model concentration gradient-driven isotropic diffusion and wind field-driven anisotropic advection, achieving explicit decoupling of the advection-diffusion equation within graph neural networks. Addressing satellite AOD cloud occlusion and nighttime missing data, an innovative ``AOD spatial gradients as constraints rather than inputs'' fusion strategy is proposed, enabling smooth fallback to physics-driven inference when AOD is unavailable, achieving all-weather continuous mapping. Dynamic node masking training strategy enables inductive generalization for arbitrary unobserved locations. Under 30\% station missing conditions in the BTH region, \MAE of 9.5~\ug is achieved, reducing error by approximately 25\% compared to inductive baselines such as IGNNK.

For hypothetical scenario simulation, \ModelSim incorporates emission inventories as controllable input variables within deep learning for the first time, enabling efficient multi-scenario simulation. Addressing limitations of traditional methods that treat emissions as fixed boundary conditions, a one-to-one mapping architecture between emission feature encoding and concentration field decoding is designed, avoiding error accumulation in recursive prediction. Compared to traditional CTMs such as WRF-CMAQ requiring hours, this model compresses single-scenario inference to seconds, providing a feasible pathway for real-time scenario assessment. Using the China DPEC dataset, the model reveals opposing evolution patterns under 2025--2050 carbon neutrality pathways where \PM continues declining while \ozone generally rises, quantifying pollution synergy effects under different scenarios and providing scientific basis for synergistic emission reduction decisions.

For engineering applications, the cloud-native \SystemName system is deployed. The system adopts containerized microservice design with hourly rolling updates and automated operations, integrating data acquisition, model inference, visualization, and alert notification modules. The system has served air quality assurance for major national events and achieved the highest median score in official model comparison evaluations organized by CNEMC, validating the practicality and reliability of the proposed methods.

The core contributions include: proposing a physics-constrained data-driven modeling framework that systematically integrates atmospheric science with deep learning at three levels---data fusion, model architecture, and loss functions; developing an integrated air quality intelligent modeling methodology encompassing prediction, inference, and simulation, advancing air pollution modeling from ``capable of prediction'' to ``capable of inference and simulation''; completing full-chain implementation from theoretical methods to operational systems, providing credible intelligent technology support for refined air quality management.

\end{abstract*}