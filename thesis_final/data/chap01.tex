%!TEX encoding = UTF-8 Unicode
%!TEX program = xelatex

% ============================================================
% 第一章:绪论
% 基于复杂系统数据驱动建模的大气污染研究
% ============================================================

\chapter{绪论}
\label{chap:introduction}

\section{研究背景及意义}
\label{sec:background}

\subsection{大气污染问题的全球性挑战}
\label{subsec:global_challenge}

当前,大气污染已成为影响人类生存与发展的重大环境议题。世界卫生组织(World Health Organization, WHO)发布的监测数据显示,全球约有99\%的居民所处的空气质量条件未能达到该组织制定的健康指导标准\footnote{\url{https://www.who.int/news/item/04-04-2022-billions-of-people-still-breathe-unhealthy-air-new-who-data}}。每年因空气污染引发的过早死亡人数高达约700万\citep{world2021global,murray2020global}。在众多大气污染因子中,细颗粒物(Fine Particulate Matter, \PM)与近地面臭氧(Ground-Level Ozone, \ozone)由于其对公众健康造成的严重威胁,始终是科学研究和环境治理的重点关注对象。\PM 的空气动力学直径不超过2.5微米,这种极微小的粒径特性使其能够突破人体呼吸系统的防护屏障,抵达肺泡深处,进而随血液循环对各器官产生影响。大量流行病学研究证实,\PM 暴露与心血管病变、呼吸道疾患以及肺部恶性肿瘤的发生存在显著正相关\citep{burnett2018global}。\ozone 具有强烈的氧化性,持续暴露于高浓度环境会削弱呼吸系统机能,同时对农业生产造成减产威胁\citep{nuvolone2018effects,mills2018ozone}。

作为全球最大的发展中经济体,中国在应对大气污染问题上的探索具有重要的示范意义。伴随着工业化与城镇化进程的快速推进,经济高速增长的同时也积累了突出的空气质量问题。面对这一挑战,中央政府实施了层层递进的治理举措。2013年,国务院颁布《大气污染防治行动计划》(\cqt{大气十条})\footnote{\url{https://www.gov.cn/zwgk/2013-09/12/content_2486773.htm}},首次设立\PM 浓度的刚性考核目标\citep{china2013action}。2018年,《打赢蓝天保卫战三年行动计划》\footnote{\url{https://www.gov.cn/zhengce/content/2018-07/03/content_5303158.htm}}进一步强化治理力度。2023年,《空气质量持续改善行动计划》\footnote{\url{https://www.gov.cn/zhengce/content/202312/content_6919000.htm}}将\PM 削减确立为核心主线,推进NOx与VOCs协同减排,力图实现减污降碳双重目标\citep{jiang2021government}。值得一提的是,清华大学研发的中国多尺度排放清单(Multi-resolution Emission Inventory for China, MEIC)在污染源头解析与减排成效评估中发挥了重要的数据支撑作用\citep{zheng2018trends,geng2024efficacy}。历经十余年坚持不懈的努力,全国\PM 年均浓度从2013年的72 $\mu$g/m$^3$\footnote{\url{https://www.mee.gov.cn/gkml/sthjbgw/qt/201403/t20140325_269648.htm}}大幅下降至2023年的30 $\mu$g/m$^3$\footnote{\url{https://www.mee.gov.cn/hjzl/sthjzk/zghjzkgb/202406/P020240604551536165161.pdf}},重污染天气频次明显减少,空气质量改善速度在全球范围内首屈一指\citep{zhang2019drivers,geng2024efficacy,zhao2024challenges,cn_lishaolin2023daqishitiao}。

尽管如此,当前空气质量距WHO于2021年修订的\PM 年均浓度指导值(5 \ug)尚有相当差距\citep{world2021global}。就区域分布而言,京津冀及其周边、长三角、汾渭平原等敏感区域在秋冬季节仍面临重污染天气的侵扰\citep{chen2020influence,cn_yanli2021quyu};就污染类型而言,\ozone 问题的严峻程度与日俱增,呈现出\PM 稳步下行而\ozone 逐年攀升的差异化走势\citep{wang2020contrasting,liu2023drivers}。\PM 与\ozone 的联合管控遭遇深层次的科学难题:二者拥有NOx和VOCs这两类共同的前驱物质,在实际减排过程中呈现所谓\cqt{跷跷板效应}。其主要机理在于:随着\PM 浓度下降,气溶胶对太阳辐射的散射和吸收作用减弱,到达近地面的光合有效辐射增强,加速了光化学反应进程,从而促进\ozone 的生成\citep{qu2023underlying,li2023spatiotemporal,cn_lihong2019pm25o3}。这种错综复杂的大气化学关联使得针对单一污染物的减排策略往往难以取得理想效果,迫切需要研发能够表征多污染物相互作用机制的新型建模方法。与此同时,随着背景浓度的持续改善,重污染事件逐渐从周期性常态转变为偶发性、突发性特征,对污染预测的精度和时效性提出了更高要求。

从宏观战略角度审视,大气污染治理与碳达峰碳中和目标深度交织、相互影响。化石能源燃烧过程既排放CO$_2$,也产生SO$_2$、NOx、PM等空气污染物,减污与降碳在实现路径和政策手段上存在广阔的协同空间\citep{shi2022co,li2024synergistic}。2022年6月,生态环境部会同六部门联合发布《减污降碳协同增效实施方案》\footnote{\url{https://www.mee.gov.cn/xxgk2018/xxgk/xxgk03/202206/t20220617_985879.html}},明确提出\cqt{系统提升环境治理综合效能,实现环境、气候、经济效益协同共赢}的总体要求,标志着我国生态文明建设迈入减污降碳一体推进的新阶段\citep{cn_guochangsheng2023xietong}。在\cqt{2030年前实现碳达峰、2060年前实现碳中和}的战略部署下\footnote{\url{https://www.gov.cn/zhengce/202511/content_7047492.htm}},能源结构和产业格局必将经历深刻重塑。不同碳减排路径下空气质量将如何演进、怎样达成\PM 与\ozone 的协同改善,既是学术研究的前沿命题,也是关系国家发展的重大决策问题\citep{cheng2021pathways,sun2024air}。

综合以上分析,大气污染与公众健康、区域环境质量密切相关。在污染浓度整体下降但复合污染特征日益凸显的背景下,研发高精度、实时响应的大气污染建模与预测方法,对于推进空气质量科学研究与环境管理实践具有重要的理论和应用价值。

\subsection{大气污染的复杂系统特性}
\label{subsec:complex_system}

运用系统科学(System Science)的分析视角,大气污染体系远非简单的线性叠加系统可以描述,而是具备\cqt{开放复杂巨系统}(Open Complex Giant System)的典型特征\citep{xuesen1993new}。按照钱学森先生对开放复杂巨系统的界定,大气污染系统在以下维度上呈现典型特征:\textbf{开放性}——系统与外界持续进行物质(污染物排放与沉降)、能量(太阳辐射与地表热通量)和信息(监测与预报反馈)的交换;\textbf{复杂性}——排放、气象、化学三大子系统紧密耦合,形成非线性反馈回路;\textbf{巨量性}——涉及数以千计的监测站点、数百种化学物种、以及多源异构的海量观测数据;\textbf{涌现性}——区域性重污染事件表现为城市群的同步响应,非单个节点行为的简单叠加\citep{cn_wangshang2023}。系统把握这些复杂性表征,是突破现有建模瓶颈、发展新一代智能预测模型的必要前提。

\subsubsection{多要素强耦合}

大气污染物在时间和空间维度上的分布并非孤立的物理化学现象,而是排放源(Emissions)、气象条件(Meteorology)与化学演化(Chemistry)三大子系统紧密交织、相互作用的结果,各过程之间形成了复杂的反馈回路\citep{jacob2000heterogeneous,seinfeld2016atmospheric}。

气象要素从根本上制约着污染物的输送、混合与清除效率。当出现静稳型天气时,边界层高度显著压低,仿佛在城市上空形成一个\cqt{穹顶},致使污染物急速累积\citep{zhong2018feedback}。与之相反,高浓度气溶胶通过散射与吸收入射太阳辐射(即气溶胶--辐射相互作用),打破地表能量收支平衡,进一步削弱边界层的发展能力,形成\cqt{污染积聚$\to$辐射衰减$\to$边界层收缩$\to$污染加重}的恶性循环链条\citep{huang2014high,wang2016persistent}。相关研究揭示,气象因子对中国\PM 浓度的年际波动具有相当强的解释力\citep{zhang2019drivers,chen2020influence}。

人为排放活动的时空分布特征同样受到气象场的显著调制。夏季高温热浪期间居民空调用电负荷骤增,生物源VOCs的自然释放也随之加快,两者叠加效应极易催生光化学烟雾事件\citep{ma2019substantial};冬季取暖时段排放强度大幅增加,若再遇上静稳天气,便构成\PM 重污染的主要触发条件\citep{chen2020influence}。上述多要素耦合特性决定了建模方法必须在统一框架内综合表征排放、气象与监测数据等多元信息,而不能孤立地处理任何单一因子。

\subsubsection{动力学的高度非线性}

大气化学反应过程呈现出极为显著的非线性动力学行为,系统输入端(排放)与输出端(浓度)之间非简单的比例对应关系\citep{seinfeld2016atmospheric}。

\ozone 生成速率对其前驱物NOx与VOCs的非线性响应是大气化学非线性现象的标志性案例。研究证实,\ozone 的净生成取决于NOx与VOCs的相对配比,其等浓度响应曲线呈现明显的\cqt{山脊}形态\citep{sillman1999relation}。在VOC敏感区(如城市核心地带,VOCs/NOx比值偏低的情形),削减NOx反而可能因滴定作用减弱而推高\ozone 浓度;而在NOx敏感区(如郊区或远郊),削减NOx则可有效压低\ozone 水平\citep{xing2020deep,liu2023drivers}。只有当减排力度足够大、驱使系统跨越\cqt{分水岭}之后,继续降低NOx排放才能实现\ozone 的持续下降\citep{huang2020large}。类似地,二次气溶胶(涵盖硫酸盐、硝酸盐、铵盐及二次有机气溶胶)的形成涉及气相--液相--固相三态转化,受相对湿度、气溶胶酸碱度、氧化剂浓度等多重因素的非线性调节\citep{guo2014elucidating,wang2016persistent}。这种非线性特质意味着,基于简单线性外推的减排策略可能适得其反。深度神经网络作为通用的非线性函数逼近器,为刻画排放--浓度之间错综复杂的响应关系开辟了新的技术途径。

\subsubsection{复杂网络拓扑与涌现性}

污染物的跨区域迁移并非各向均匀的扩散过程,而是沿着特定地形廊道和大气流场形成有向性、时变性、长程相关的复杂传输网络\citep{quan2020regional,chen2020influence}。以京津冀区域为例,研究表明北京市\PM 污染中有30--50\%的比重源自周边区域的输入贡献\citep{chang2018assessment}。在典型\cqt{偏南气流输送型}重污染过程中,污染气团从河北中南部沿太行山东麓向北移动,途经石家庄、保定直达北京,形成绵延数百公里的\cqt{污染传输走廊}\citep{yin2025regional,quan2020regional,cn_liulin2017wrfchem}。这种传输表现出强烈的方向性和定向性,且随着大气流场的变化而动态调整路径。

借助网络科学的概念框架,监测站点之间的空间关联可被抽象为非欧几里得图结构:节点对应各监测点位,边权重反映传输强弱,天然适合表达\cqt{空间邻近但因山脉阻隔而传输微弱}等复杂的空间关系\citep{wang2020pm2,wu2020comprehensive}。图神经网络通过在此类图结构上执行信息传播与聚合运算,为非欧式空间数据建模提供了理想的计算框架\citep{kipf2017semi,velivckovic2018graph}。区域性重污染往往表现为城市群的\cqt{同步响应}现象,呈现出典型的涌现性特征:单个城市的治理成效不仅取决于本地减排行动,更受制于整个区域传输网络的系统性行为\citep{wang2017higher},必须运用图论和网络科学方法进行整体建模与分析。

\subsubsection{多源异构数据特征}

现代大气环境监测体系依托多种观测技术手段,汇聚形成了丰富但高度异构的数据资源\citep{geng2021tracking}。地面监测网络(如中国国家空气质量监测网下辖约1618个国控站点\footnote{\url{https://www.mee.gov.cn/xxgk2018/xxgk/xxgk06/202509/W020250905372029416580.pdf}})可提供精度较高但空间分布稀疏的污染物浓度实测值;卫星遥感平台(如MODIS搭载的MAIAC算法\footnote{\url{https://modis-land.gsfc.nasa.gov/MAIAC.html}}、静止卫星Himawari-8反演的气溶胶光学厚度AOD\footnote{\url{https://www.eorc.jaxa.jp/ptree/index.html}})能够获取空间连续但易受云层遮挡干扰的大气成分信息\citep{xiao2017full,wei2021himawari}。气象再分析资料(如ERA5)以格点形式提供驱动场数据\citep{hersbach2020era5};排放清单(如MEIC)描绘污染物前驱体的时空分布状况,但存在较大不确定性且更新周期较长\citep{zheng2018trends}。上述数据源在空间结构、时间分辨率、质量控制水平和物理内涵等方面差异明显。如何将多源异质信息在统一的潜在空间中实现有效融合与优势互补,构成了数据驱动建模亟待攻克的核心难题。

综上所述,大气污染体系的多要素耦合、动力学非线性、网络结构复杂以及数据多源异构等特性,共同界定了建模方法必须正视的核心挑战。如何在统一的理论框架内有效表征上述复杂性,是大气污染预测与模拟研究的关键科学问题。

\subsection{传统数值模型的原理与局限}
\label{subsec:traditional_models}

化学传输模型(Chemical Transport Models, CTMs)是大气环境科学领域模拟污染物时空演变过程的经典工具,其代表性模式包括社区多尺度空气质量模型(Community Multiscale Air Quality, CMAQ)\citep{byun2006review}、天气研究与预报--化学耦合模式(Weather Research and Forecasting model coupled with Chemistry, WRF-Chem)\citep{grell2005fully}以及全球大气化学传输模型(GEOS-Chem)\citep{bey2001global}等。历经数十年的迭代与完善,CTMs已成为空气质量业务预报、污染源头解析以及减排方案评估的核心技术支撑,在环境管理实践中发挥着难以替代的重要作用。

CTMs的数学基础是对流--扩散方程(Advection-Diffusion Equation),该方程从质量守恒定律出发,刻画污染物在大气中伴随风场输送(平流)、湍流混合(扩散)、化学转化、源排放以及沉降清除等过程的动态平衡关系(详见第\ref{chap:methodology}章)。CTMs通过数值方法在三维计算网格上求解该方程,显式地模拟污染物的排放、传输、化学转化与沉降过程,因此具备较好的物理可解释性与过程一致性。

然而,这种机理驱动的数值模型在实际应用中遭遇多重瓶颈制约。

\textbf{计算资源消耗巨大}是CTMs面临的首要障碍。化学传输模型需要在三维网格上对数十种污染物同步进行时间积分,每个时间步长都需完成平流、扩散、化学反应等多个子模块的求解。以公式(\ref{eq:advection_diffusion})中的化学反应项为例,其计算涉及包含数百个反应方程的化学机制(如CB6、SAPRC-07),运算量极为可观\citep{appel2021community}。一次常规的CMAQ区域模拟(例如针对京津冀及周边区域,采用4 km水平分辨率,模拟时长一周)需要在高性能计算平台上运行数小时乃至数天,难以满足高频次、实时化业务预报的时效要求。

\textbf{对排放清单输入的高度依赖}构成另一重要制约因素。公式(\ref{eq:advection_diffusion})中的排放源项$S$是CTMs的关键驱动输入,其准确程度直接影响模拟质量。然而,排放清单的编制周期漫长(通常滞后1--2年甚至更久)、空间分辨率受限,且在时间分解和行业拆分上存在较大不确定性\citep{thongthammachart2021integrated}。例如,交通源排放的逐时变化受实时交通流量影响显著,但清单通常仅能提供月均或日均的时间分配因子\citep{inventory},难以精细刻画真实的排放动态。这导致模式输出与地面实测之间常存在系统性偏差,在排放快速变化的特殊时段(如春节假期、森林火灾、秸秆焚烧季节等)偏差尤为突出。

\textbf{参数化方案的不确定性}同样制约着模式的预报精度。CTMs需要为化学反应、气溶胶演化、干湿沉降等过程选定特定的参数化方案(如化学机制CB06或SAPRC07、气溶胶方案AERO7、干沉降方案M3Dry等),不同方案组合对模拟结果影响显著\citep{appel2021community}。然而,最优参数化方案的选取往往依赖专家经验,且难以针对特定区域或季节进行动态适配。这种\cqt{固定方程的正向模拟}范式,缺乏从观测数据中逆向学习、自适应优化参数的机制。

\textbf{初始场同化技术尚不成熟}限制了CTMs的短临预报能力。与气象预报领域发展完善的资料同化系统相比,大气化学模式的初始场同化技术仍处于发展阶段。模式难以高效利用实时监测数据对公式(\ref{eq:advection_diffusion})中的浓度场$C$进行状态订正,导致预报技巧随预报时效延长而快速衰减——通常预报72小时后与实测的相关性显著下降\citep{sun2021improvement}。

\textbf{空间分辨率与观测尺度的不匹配}制约了CTMs的精细化应用能力。受计算资源约束,区域CTMs的水平分辨率通常为4--12 km,而地面监测站点代表的是局地尺度(约100 m量级)的污染物浓度。网格平均值与站点观测之间的\cqt{代表性误差}(Representativeness Error)难以完全消除,在地形复杂或排放空间异质性强的区域尤为明显。

\textbf{情景模拟的\cqt{组合爆炸}困境}制约了CTMs在决策支持中的应用效能。当需要评估不同减排方案的空气质量改善效果时,CTMs需要为每种排放情景重新执行完整模拟。假设需要评估5种污染物(SO$_2$、NOx、VOCs、NH$_3$、PM)各3个减排水平(0\%、30\%、50\%)的组合效应,则需运行$3^5=243$次独立模拟,计算代价极其高昂。这种\cqt{组合爆炸}问题严重限制了CTMs在交互式减排方案设计和实时决策辅助中的应用潜力。

综合来看,传统机理驱动的数值模型虽然具备较强的物理可解释性与过程一致性,但\textbf{\cqt{可解释却难以实时预测、可模拟却难以反向学习}}的内在矛盾,限制了其在精细化、实时化环境管理中的应用空间。如何在保持物理可解释性的同时突破计算效率与数据同化的双重瓶颈,成为大气污染建模领域亟需攻克的关键科学问题。

\subsection{AI与大数据时代的机遇与挑战}
\label{subsec:ai_opportunity}

近年来,深度学习技术与大数据方法的蓬勃发展为突破上述困境开辟了新的可能性路径。2019年,Reichstein等学者在\textit{Nature}期刊发表了一篇里程碑式的综述文章,正式提出\cqt{AI for Earth System Science}的研究范式\citep{reichstein2019deep}。该综述指出,深度学习能够自动从数据中提取时空特征以增进过程理解,借助混合建模(Hybrid Modeling)策略将物理过程模型与数据驱动方法有机结合,有望在保持机理可解释性的同时实现计算效率提升与数据融合能力增强。

地球系统科学数据具备典型的\cqt{4V}特征\citep{reichstein2019deep}:\textbf{Volume}(体量庞大)——多源观测、数值模式与再分析资料持续积累,全球气象再分析数据集ERA5已累积超过PB量级的数据规模;\textbf{Velocity}(更新迅速)——实时监测与预报需求不断提升,空气质量监测数据以小时级频率持续更新;\textbf{Variety}(类型多元)——涵盖格点场、站点观测、卫星遥感、排放清单等多模态异构数据;\textbf{Veracity}(可信度差异)——不同来源数据的精度与可靠程度存在显著差异。传统数值模型难以在这种高维动态数据空间中实现实时求解与多源融合,而数据驱动方法能够在学习复杂非线性映射的同时,有效捕捉多要素间的时空耦合关系。

近年来,以GraphCast\citep{lam2023learning}、Pangu-Weather\citep{bi2023accurate}、FourCastNet\citep{pathak2022fourcastnet}和Aurora\citep{bodnar2025foundation}为代表的AI气象大模型在全球天气预报领域展示出巨大潜力,标志着从\cqt{求解方程}向\cqt{从数据学习}的范式转变。GraphCast采用图神经网络架构\citep{battaglia2018relational},在0.25°分辨率下预测227个大气变量,单次推理耗时不足60秒,在1380个验证指标中约90\%的指标上优于欧洲中期天气预报中心(ECMWF)的高分辨率业务预报系统HRES\citep{lam2023learning}。Pangu-Weather引入三维地球特定Transformer架构,实现了相比数值模式约10000倍的计算加速\citep{bi2023accurate}。GenCast采用扩散模型架构生成集合预报,在97.2\%的验证指标上超越ECMWF业务集合预报系统\citep{price2025probabilistic}。Aurora作为新一代地球系统基础模型,拥有13亿参数规模,首次将预测范围从天气拓展至大气化学与空气质量领域\citep{bodnar2025foundation}。

然而,将上述AI气象大模型迁移应用于大气污染领域时,仍面临若干关键瓶颈。

\textbf{观测网络的稀疏性与非结构化特征}。气象变量(如温度、气压、风场)具有较强的空间连续性与平滑性,可借助高密度格点化再分析数据(如ERA5)开展模型训练。然而,空气质量监测站点分布稀疏且呈不规则布局,难以直接套用基于卷积神经网络(CNN)的格点模型。这种非结构化的观测网络特性为图神经网络(GNN)在稀疏空间数据上的应用提供了天然场景,也是本文选择图神经网络作为核心架构的重要考量。

\textbf{对初始场与再分析数据的路径依赖}。现有AI气象模型通常以再分析数据(如ERA5、CAMS)作为初始场输入,但这些数据本身存在数小时至数天的生产延迟\citep{kashinath2021physics}。以Aurora为例,虽然其支持空气质量预测功能,但初始场来源于哥白尼大气监测服务(CAMS)的再分析产品,在中国区域存在明显的系统性偏差且时效滞后。更关键的是,这些模型在架构设计上并不支持同化实时地面站点观测,难以对突发污染事件作出快速响应——而本文提出的模型正是针对站点级实时预测与应急响应需求量身设计的。

\textbf{物理一致性与可解释性的缺失}。纯数据驱动模型可能学习到与物理规律相悖的模式\citep{karniadakis2021physics}。在质量守恒约束缺失的情况下,模型可能预测出污染物\cqt{凭空产生}或\cqt{异常消失}的非物理结果,尤其在训练数据分布之外的极端情景下问题更为突出。这与公式(\ref{eq:advection_diffusion})所描述的质量守恒定律形成鲜明对照。物理约束的缺失不仅影响预测精度,更从根本上削弱了模型结果的可信度与可解释性。

\textbf{污染演化机制建模不完整}。现有AI预测模型大多采用自回归模式——依据历史污染物浓度序列预测未来浓度,而未显式建模公式(\ref{eq:advection_diffusion})中\cqt{排放$\to$传输$\to$化学转化$\to$沉降}的完整污染演化链条。这意味着模型缺乏对排放源项$S$的显式响应能力:当排放发生变化(如减排政策实施、突发事故或春节假期排放骤变)时,自回归模型无法准确捕捉这种外部驱动的变化,也就难以支撑\cqt{如果减排50\%结果如何}这类假设情景推演与政策评估需求。

\textbf{任务范围的局限性}。现有AI模型主要聚焦于时序预测单一任务,而大气污染治理实践需要的是一套完整的决策支持能力体系,包括空间推断(评估未设站区域的污染水平)和情景模拟(评估不同减排方案的改善效果)。仅依靠时序预测功能难以满足精细化环境管理的多元需求。

上述挑战表明:大气污染建模需要一种\textbf{融合领域知识与深度学习}的新范式——既保持数据驱动方法的计算效率与强拟合能力,又借助物理约束确保结果的合理性与可解释性;既能实现高精度的实时预测,又能支撑情景模拟与决策优化。这正是本文研究工作的核心目标与出发点。

\subsection{大气污染复杂系统建模的关键问题}
\label{subsec:challenges}

基于以上分析,大气污染作为典型的开放复杂巨系统,其数据驱动建模面临三个方面的核心挑战,这些挑战贯穿于预测、推断与模拟三类核心任务,同时构成了本研究的科学问题。

\textbf{挑战一:物理约束的有效嵌入}。如第\ref{subsec:complex_system}节所述,污染演化过程受排放、气象、化学反应等多要素共同驱动,存在显著的非线性交互与反馈机制。传统CTMs通过求解公式(\ref{eq:advection_diffusion})显式刻画物理过程,但计算效率受限;纯数据驱动模型虽然高效,却缺乏机理约束,容易产生违背物理定律的预测结果。这一挑战在不同任务中的具体表现为:在预测任务中,模型需保证质量守恒与时空连续性(第\ref{chap:prediction}章);在推断任务中,模型需刻画风场驱动的各向异性传输机制(第\ref{chap:inference}章);在模拟任务中,模型需建立排放源强与污染浓度之间的物理响应关系(第\ref{chap:simulation}章)。如何在数据驱动框架下有效嵌入这些物理先验知识,实现\cqt{数据拟合能力}与\cqt{物理一致性}的平衡,是理论上的核心难点。

\textbf{挑战二:多源异构数据的融合}。大气污染建模涉及地面监测、卫星遥感、气象再分析、排放清单等多种数据源,这些数据在空间结构、时间分辨率、覆盖特性和物理意义上存在显著差异。这一挑战在不同任务中的具体表现为:在预测任务中,需要融合风场信息构建有向图结构以刻画跨区域传输(第\ref{chap:prediction}章);在推断任务中,需要处理卫星AOD的非随机缺失并将其转化为空间约束(第\ref{chap:inference}章);在模拟任务中,需要桥接历史排放清单(MEIC)与未来情景数据(DPEC)的差异(第\ref{chap:simulation}章)。如何在统一的潜在空间中融合这些异构数据、处理系统性缺失并保持物理一致性,是数据处理的关键瓶颈。

\textbf{挑战三:模型的泛化与外推能力}。多数深度学习模型采用转导式学习范式,在训练样本分布内表现良好,但泛化能力不足\citep{wu2021inductive}。这一挑战在不同任务中的具体表现为:在预测任务中,模型需同时预测具有复杂光化学耦合关系的\PM 与\ozone(第\ref{chap:prediction}章);在推断任务中,模型需泛化至训练集中未出现的新位置(第\ref{chap:inference}章);在模拟任务中,模型需外推至未来气候与排放情景(第\ref{chap:simulation}章)。如何突破固定图拓扑与历史分布的限制,构建具有强泛化能力的时空学习框架,是模型设计的核心难题。

\subsection{研究意义与价值}
\label{subsec:significance}

针对上述挑战,本研究以复杂系统数据驱动建模为核心思想,面向大气污染的多任务场景——预测、推断与模拟——构建物理约束的时空学习与推理框架。研究意义体现在以下四个方面。

\textbf{科学价值:数据--机理融合的建模新范式}。本研究提出\cqt{物理约束的数据驱动建模}范式,将第\ref{chap:methodology}章公式(\ref{eq:advection_diffusion})所描述的平流、扩散、化学反应等物理过程显式嵌入深度学习架构,在模型结构设计与损失函数约束两个环节实现数据与机理的有效融合。该框架揭示了在复杂系统建模中平衡\cqt{数据拟合}与\cqt{物理一致性}的可行路径,为地球系统科学中的时空过程建模提供了新的方法论支撑。

\textbf{国家战略价值:支撑\cqt{双碳}目标与协同治理}。研究成果直接服务于国家\cqt{双碳}战略目标与\PM--\ozone 协同治理的技术体系建设。通过构建排放--浓度响应的代理模型与可微分优化框架,能够在数秒内完成传统CTMs需要数小时的情景模拟,定量评估不同减排方案对空气质量的改善效果,为国家及地方污染防治政策的制定提供高效、科学的决策支持工具。

\textbf{社会应用价值:从科学建模到智能服务}。本研究实现了空气质量智能预报系统的工程化部署,支撑从实时预报、空间推断到智能减排决策的全链条应用。系统已在\cqt{彩云天气}等平台上线运行,日均服务数千万用户查询请求,为公众健康防护提供及时信息;同时支撑了上海进博会等国家重大活动的空气质量保障工作,展现了研究成果的实际应用价值。

\textbf{方法论普适性:可迁移的时空建模框架}。本研究提出的核心方法——时空图网络建模传输过程、物理约束嵌入损失函数、多源数据在隐空间对齐——具有广泛的普适性。其建模思想可迁移至气象预报、水文模拟、交通流预测、传染病传播等其他涉及时空演化的复杂系统问题,为相关领域的数据驱动建模提供方法借鉴与技术参考。


\section{国内外研究现状}
\label{sec:literature}

\subsection{传统数值建模研究进展}
\label{subsec:numerical_models}

化学传输模型(CTMs)长期以来作为大气污染数值模拟的传统核心工具,其发展历程可追溯至20世纪70年代。在全球尺度上,具有代表性的模式系统包括GEOS-Chem\citep{bey2001global}、MOZART\citep{emmons2010description}以及CAM-chem\citep{lamarque2012cam}等;在区域尺度上,主流应用模式包括CMAQ\citep{byun2006review}、WRF-Chem\citep{grell2005fully}和CAMx\citep{emery2024comprehensive}等。这些模式通过数值求解第\ref{chap:methodology}章公式(\ref{eq:advection_diffusion})所描述的对流--扩散方程,从物理化学机理角度刻画污染物的形成、输送、转化与清除全过程。

CMAQ是目前应用最为广泛的区域空气质量模式之一,由美国环保署(EPA)开发并维护,采用模块化架构设计,涵盖气相化学(CB6、SAPRC机制)、气溶胶热力学(ISORROPIA)、云化学过程、干湿沉降等完整的过程模块\citep{appel2021community}。WRF-Chem由美国国家大气研究中心(NCAR)开发,将气象模式WRF与化学模块紧密耦合,可实现气象--化学的双向反馈模拟\citep{grell2005fully}。GEOS-Chem由哈佛大学主导开发,采用全球--区域嵌套技术架构,适用于跨区域乃至全球尺度的污染物传输过程研究\citep{bey2001global}。

在国内研究方面,嵌套网格空气质量预报模式系统(NAQPMS)由中国科学院大气物理研究所自主开发,采用多尺度嵌套技术与数据同化模块,已在APEC会议、G20峰会等重大活动的空气质量保障工作中发挥关键作用\citep{cn_wangzifa2006naqpms,wang2014modeling,cn_chenxueshun2015naqpms},体现了我国在大气污染数值模拟领域已具备独立自主的创新能力。

近年来,CTMs在以下方向取得重要进展:模式嵌套技术使区域模式能够与全球模式实现有效衔接\citep{wang2015implementation};高分辨率模拟能力不断提升,部分研究已实现公里级城市空气质量精细化模拟\citep{sicard2021high};排放清单持续完善更新,中国多尺度排放清单(MEIC)\citep{zheng2018trends}为模式运行提供了更精细的排放输入支撑。卢亚灵等\citep{cn_luyaling2021}系统梳理了空气质量预测技术的演变历程,指出从统计方法到数值模式再到人工智能的发展趋势。然而,如第\ref{subsec:traditional_models}节所述,CTMs面临的计算成本高、排放依赖强、实时性差等固有瓶颈始终未能根本突破,这为数据驱动方法的兴起创造了契机。

\subsection{数据驱动的大气污染建模研究}
\label{subsec:ai_airquality}

数据驱动的大气污染预测方法经历了从统计模型到深度学习的演进历程\citep{reichstein2019deep,karniadakis2021physics},近年来已成为\PM 和\ozone 浓度预报的重要技术手段。本节系统梳理该领域的研究现状,重点关注深度学习时空建模、图神经网络以及物理约束神经网络等前沿方向,并指出现有方法存在的关键不足。

\subsubsection{统计与传统机器学习方法}

早期研究工作主要采用统计模型和传统机器学习方法建立气象因子与污染浓度之间的经验关系。多元线性回归(MLR)、自回归积分滑动平均模型(ARIMA)和支持向量回归(SVR)等方法实现简单、计算成本低且具有较好的可解释性,但难以捕捉复杂的非线性关系,在长期预测任务上性能欠佳\citep{goyal2006statistical,wong2021using,leong2020prediction}。XGBoost和LightGBM等梯度提升集成方法在表格数据上表现优异,Ma等人\citep{ma2020application}将XGBoost与WRF-Chem进行集成,通过机器学习后处理校正数值模式的系统性偏差;Thongthammachart等人\citep{thongthammachart2021integrated}采用LightGBM融合多源特征实现区域\PM 预测。陈镇等\citep{cn_chenzhen2024zhujiang}运用SVR、随机森林等方法建立珠三角臭氧预测模型;曲悦等\citep{cn_quyue2019}对比了BP神经网络、CNN和LSTM模型在北京\PM 预测中的性能差异。然而,这类方法依赖人工特征工程,虽能快速拟合局地污染规律,但难以自动捕捉跨区域的污染传输关系和复杂的时空依赖结构\citep{xiao2018ensemble,chen2018machine}。

\subsubsection{深度学习时序建模方法}

随着深度学习技术的快速发展,循环神经网络(RNN)及其变体被广泛应用于大气污染时序预测任务\citep{cn_zhuyanmin2020shendu}。长短期记忆网络(LSTM)是该领域应用最多的模型架构,通过门控机制捕捉长程时序依赖关系并有效缓解梯度消失问题\citep{li2017long}。Chang等人\citep{chang2020lstm}采用聚合LSTM架构融合多站点信息进行北京\PM 预测。尹文君等\citep{cn_yinwenjun2015dbn}提出基于深度信念网络(DBN)的预报模型,利用多层RBM结构自动学习特征表示。门控循环单元(GRU)通过简化门控结构减少参数量,在保持预测性能的同时提升计算效率\citep{faraji2022integrated}。然而,RNN类模型采用序贯计算方式,难以实现并行化处理,且主要关注时间维度,将各站点独立建模或简单拼接,忽视了站点间的空间关联特性。

Transformer架构凭借自注意力机制实现的长程依赖建模和全并行计算能力,克服了RNN的序贯计算瓶颈,已成为空气质量预测的重要范式\citep{vaswani2017attention}。Liang等人\citep{liang2023airformer}提出AirFormer模型,采用分层Transformer架构处理大规模监测网络的72小时预报任务,其创新之处包括底层确定性阶段的新型自注意力机制和顶层随机性阶段的隐变量建模,有效处理了监测网络中的多尺度时空依赖问题。Yi等人\citep{yi2018deep}提出Deep Distributed Fusion Network,通过分布式融合架构整合空气质量、气象和天气预报数据,实现了多城市PM$_{2.5}$浓度的联合预测,展现了多源数据融合在深度学习框架中的潜力。陶辰亮\citep{cn_taochenliang2024}构建了融合循环神经网络、Transformer和梯度提升树的集成学习模型,结合SHAP可解释方法揭示了\ozone-\NOx-VOCs-\PM 之间的非线性交互关系。

近年来,通用时序预测模型的研究进展为大气污染预测提供了新的技术支撑。iTransformer\citep{liuitransformer}通过将注意力机制应用于变量维度而非时间维度,更有效地捕捉多变量时序数据中的跨通道依赖关系;TimeXer\citep{wang2024timexer}引入外生变量建模机制,使模型能够显式利用外部协变量增强预测性能,这一设计与大气污染预测中融合气象驱动的需求高度契合。在大气污染的具体应用中,TFB\citep{qiu2024tfb}构建了包含大气污染在内的多领域时序预测基准,对现有深度学习模型进行了全面评估;DUET\citep{qiu2025duet}通过双通道编码器分别建模污染物浓度的趋势成分和季节成分,有效处理了空气质量数据的复杂周期性模式。这些工作表明,通用时序预测框架经过领域适配后,能够在大气污染预测任务中取得具有竞争力的性能。

\subsubsection{图神经网络时空建模方法}

近年来,图神经网络(GNN)因其对非欧几里得空间数据的天然适用性,成为时空预测领域的主流技术架构\citep{zhou2020graph}。这类模型通过图结构(监测站点网络)引入归纳偏置,使用图卷积或改进的Transformer等空间模块刻画监测站点间的关联,配合时序卷积网络或LSTM等时间模块模拟污染物浓度的时间演变\citep{wu2020comprehensive}。

\textbf{核心GNN架构}。Yu等人\citep{yu2018spatio}提出时空图卷积网络(STGCN),通过图卷积捕捉空间依赖、一维卷积捕捉时序演化,建立了时空图建模的基础框架。Li等人\citep{li2018diffusion}提出扩散卷积循环神经网络(DCRNN),将信息传播建模为有向图上的扩散过程,其双向随机游走机制与大气污染物的扩散传输过程具有天然的相似性——直接对应公式(\ref{eq:advection_diffusion})中的扩散项。Wu等人\citep{wu2019graph}提出Graph WaveNet,引入自适应邻接矩阵学习机制和膨胀卷积,无需预定义图结构即可端到端学习节点间的隐式依赖关系。

\textbf{领域知识增强的图神经网络}。Qi等人\citep{qi2019hybrid}提出GC-LSTM模型,将图卷积嵌入LSTM单元内部实现时空联合建模。Wang等人\citep{wang2021modeling}提出注意力时序图卷积网络(ATGCN),通过站点级注意力机制自适应编码多种时空依赖类型。Chen等人\citep{chen2023group}提出群组感知图神经网络(GAGNN),构建城市图与城市群图的层级结构,通过可微分分组网络发现地理距离远但污染相关的城市间潜在依赖。王凯等\citep{cn_wangkai2023gcnlstm}提出GCN-LSTM城市臭氧预测模型,利用图卷积捕捉空间传输特征,结合LSTM提取时间依赖。汪宇晖\citep{cn_wangyuhui2023stgnn}提出自适应时空聚焦图卷积网络模型,通过多目标进化算法优化图结构,并结合长短周期时序分析进行\PM 预测。

\textbf{动态与自适应图}。预定义的静态图结构难以适应大气污染的动态特性。Xu等人\citep{xu2023dynamic}提出动态图神经网络(DGN-AEA),通过学习边属性生成自适应双向动态图,集成风场数据作为有向动态连接。Teng等人\citep{teng202372,teng2024new}采用聚合邻域时空信息的混合图深度网络,通过多尺度特征融合实现72小时实时预报。这些工作解决了静态图结构无法捕捉气象条件变化对污染传输影响的局限。

然而,现有时空图神经网络模型的成功主要局限于相对小规模的数据集,在全国范围的大尺度数据上往往面临扩展性问题。大部分深度模型在全国性数据集上的应用受限于其高计算复杂度和内存开销。

\subsubsection{物理约束的深度学习方法}

除从数据中挖掘模式外,另一个重要的发展方向是将大气物理和化学机理融入数据驱动模型,形成物理约束的深度学习框架\citep{reichstein2019deep,raissi2019physics,cn_zhawenshu2022pinn}。

\textbf{物理引导的图神经网络}。Hettige等人\citep{hettigeairphynet}提出AirPhyNet,是户外空气质量领域首个物理引导深度学习模型。其架构包含三个核心模块:GRU编码器将PM$_{2.5}$浓度编码为初始状态;GNN微分方程网络由物理规律约束,从连续性方程推导出扩散-平流微分方程;解码器生成最终预测。该模型将质量守恒原理直接嵌入网络结构,在稀疏数据和突变场景下展现出更强的鲁棒性。

Li等人\citep{li2023improving}提出物理启发深度图学习方法,采用欧拉2D网格系统,通过连续性方程编码流体物理,采用雷诺分解处理平流和扩散项。混合GCN与全残差深度网络架构能够生成物理一致的时空趋势,在外推场景下表现出显著优势。

\textbf{物理约束损失函数}。Li等人\citep{li2025knowledge}的物理信息深度学习框架将平流-扩散方程编码为软约束,通过在损失函数中引入物理一致性惩罚项,引导模型学习符合大气动力学规律的表示。关键发现是物理约束从根本上改变了学习动态,有效降低了多污染物联合预测的系统性偏差。石佳超等\citep{cn_shijiachao2018cmaqnn}将CMAQ模型预测值与前馈神经网络结合,构建长三角PM$_{2.5}$浓度快速响应模型。张斌等\citep{cn_zhangbin2020jicheng}融合多种深度学习模型对CMAQ预报进行误差订正。黄丛吾等\citep{cn_huangcongwu2018mos}采用极端随机树方法优化WRF-CMAQ-MOS模型。

\textbf{双分支物理-数据融合架构}。Tian等人\citep{tianair}提出Air-DualODE,是开放系统下物理引导的双分支神经ODE模型。该模型包括两个并行的动态分支:物理分支直接求解开放体系的边界感知扩散-平流方程,捕获由物理定律主导的污染物时空传播动态;数据驱动分支学习物理分支未能解释的额外依赖关系。两分支的隐表示在时间维度上对齐,通过融合模块加权合成最终预测,实现了物理可解释性与数据拟合能力的平衡。

\subsubsection{气象预报融合的关键缺失}

值得特别指出的是,\textbf{多数现有数据驱动模型未能充分利用未来气象预报信息},这是制约其业务化应用的关键缺陷。

在传统数值空气质量模式中,气象预报是必不可少的驱动输入。以CMAQ(Community Multiscale Air Quality)模式为例,其运行依赖WRF(Weather Research and Forecasting)模式提供的气象场预报\citep{byun2006review}。关键气象变量包括:天气形势(海平面气压、位势高度)决定大气稳定度和传输模式;化学驱动因子(温度、相对湿度)影响化学反应速率和二次气溶胶形成;传输因子(风速/风向)控制污染物扩散和平流;垂直混合(边界层高度PBLH)决定垂直稀释能力。CMAQ、WRF-Chem、GEOS-Chem等化学传输模式在预报阶段\textbf{固有地使用未来气象预报}作为输入。

与数值模式形成鲜明对比的是,多数机器学习模型仅使用历史气象观测数据。在许多方法中,气象要素往往被视为可有可无的辅助输入,通过浅层神经网络编码后在预测末段简单拼接进污染物特征中。这一局限的根本原因包括:(1)\textbf{时序错配}——机器学习模型使用历史气象观测与历史污染物浓度配对训练,预测时无法获取未来气象条件;(2)\textbf{时序依赖问题}——T+24h的空气质量取决于T+24h的气象条件而非T-24h的条件;(3)\textbf{预报时效衰减}——研究表明,无适当约束时模型性能随预报时间恶化,超过24小时后历史模式预测能力显著下降。

2022--2025年间已有若干重要工作开始着手解决此问题。Qiu等人\citep{qiu2023regional}提出过程参数化神经网络(PPN)模型,采用非对称编码器-解码器结构,在解码阶段输入气象变量、排放和上一时步PM$_{2.5}$预报,类似于化学传输模式使用四维数据同化(FDDA)的方式。Ma等人\citep{ma2025causal}提出CauAir模型,采用因果注意力机制显式建模气象协变量与空气质量的因果关系,凸显了气象预测信息在数据驱动空气质量预报中的关键作用。

\subsubsection{现有方法的共性问题}

综合上述分析,现有数据驱动模型仍存在以下共性问题:

\begin{enumerate}
    \item \textbf{气象预报融合不足}:多数模型未能有效利用未来气象预报信息,仅依赖历史气象观测,导致预报时效受限、长时预测精度衰减;
    \item \textbf{物理规律建模缺失}:缺乏对公式(\ref{eq:advection_diffusion})所描述的平流-扩散过程的显式建模,导致在极端情景或分布外数据上泛化能力不足;
    \item \textbf{排放响应与情景模拟能力有限}:缺乏对排放输入$E$的显式建模,无法响应排放变化或支持未来情景模拟,限制了模型在政策评估中的应用;
    \item \textbf{多污染物协同建模不足}:\PM 与\ozone 的协同预测研究有限,未能充分考虑二者共享前体物(NO$_x$、VOCs)的光化学耦合关系。
\end{enumerate}

\subsubsection{本文前期工作基础}

针对上述问题,本文作者在前期研究中开展了系统性探索。在图神经网络时空建模方面,本文作者提出PM$_{2.5}$-GNN模型\citep{wang2020pm2},首次将气象领域知识(风向、风速)融入图神经网络的边特征构建,基于物理地形约束构建图结构(污染物传输限于300km内且无山脉阻隔的城市间),设计基于风场的平流系数边权重,通过风驱动的消息传递机制显式建模跨区域污染传输。同时,本文作者构建并公开了KnowAir-DS数据集(已更新至KnowAir-DS-V2\footnote{\url{https://zenodo.org/records/15614907}},2016--2023,含O$_3$)\citep{wang2020pm2},为该领域后续研究提供了重要的数据基础。

在物理约束建模方面,本文作者进一步提出PCDCNet模型\citep{wang2025pcdcnet},将其设计为数值模型的代理(surrogate),通过在深度学习架构中融合物理-化学动力学知识实现高效预测。该模型明确将排放源、气象影响以及其他领域知识以约束形式纳入网络,结合图神经网络的空间传输建模、循环网络的时间累积建模,并引入光化学反应等局地相互作用的表示增强模块,在72小时\PM 与\ozone 浓度预报上取得了当前最先进的性能。

本文后续章节将在上述前期工作基础上,系统阐述物理约束数据驱动建模的完整框架,通过在数据融合、模型结构和损失函数三个环节嵌入大气科学领域知识,进一步解决现有方法的局限性。

\subsection{大气污染空间推断研究}
\label{subsec:inference_review}

由于监测站点分布稀疏且主要集中于城市区域,如何推断无观测区域的污染物浓度是大气环境领域的重要研究问题。全球范围内大部分人口生活在缺乏地面监测的区域\citep{southerland2022global},这使得空间推断成为污染暴露评估和健康风险分析的关键技术环节。

\textbf{传统空间插值方法}。反距离加权(IDW)和普通克里金(Ordinary Kriging)是最常用的空间插值技术\citep{li2011review}。这些方法基于空间自相关性假设,仅利用几何距离信息构建权重矩阵,忽略了地形、气象、排放源分布等因素的影响,在复杂地形区域或站点稀疏区域精度较低。回归克里金(Regression Kriging)和土地利用回归模型(LUR)通过引入辅助变量部分缓解了这一问题\citep{hoek2008review},但仍难以捕捉污染物传输的动态时空特性。协克里金(Co-Kriging)方法利用多变量间的空间相关性进行联合估计,但对变量间关系的线性假设限制了其在复杂大气化学体系中的适用性。

\textbf{卫星遥感反演方法}。利用卫星观测的气溶胶光学厚度(AOD)推算地面PM$_{2.5}$是目前实现全覆盖制图的主流手段\citep{geng2021tracking,xiao2018ensemble,van2016global}。Van Donkelaar等人\citep{van2016global}结合MODIS、MISR和SeaWiFS多源卫星AOD与GEOS-Chem模拟,构建了长时间序列的全球PM$_{2.5}$年均数据集。Geng等人开发的TAP(Tracking Air Pollution)数据库\citep{geng2021tracking}通过多源数据融合方法整合卫星AOD、地面观测与CTM模拟,实现了中国区域近实时PM$_{2.5}$反演。Wei等人\citep{wei2021reconstructing}采用时空统计方法重建了2000年以来中国1公里分辨率的高质量PM$_{2.5}$数据集,后续工作进一步将该方法扩展至全球尺度\citep{wei2023first}。Ma等人\citep{ma2015satellite}采用时空加权回归结合随机森林方法,有效处理了AOD与PM$_{2.5}$关系的时空异质性。然而,AOD产品受到云层遮挡、地表高反射率、气溶胶垂直分布假设和夜间无法观测等因素影响,存在严重的非随机缺失问题\citep{just2020gradient}。现有的深度学习模型多将AOD作为强制输入特征,导致在阴雨天或夜间模型完全失效,严重制约了实时空气质量评估的可用性。

\textbf{深度学习空间估计方法}。近年来,深度学习方法被广泛应用于空气质量空间估计。Lee等人\citep{lee2021hourly}采用深度神经网络融合静止卫星GOCI影像与UM再分析数据,实现了逐小时地面PM$_{2.5}$估计。Di等人\citep{di2019ensemble}提出集成神经网络方法,结合梯度提升、随机森林和神经网络的互补优势进行空间估计。Park等人\citep{park2020estimating}采用卷积神经网络处理网格化的多源输入数据,利用空间卷积核捕捉局部空间模式。Wei等人\citep{wei2023first}采用时空加权人工智能方法,首次实现了全球日均1公里分辨率无缝隙PM$_{2.5}$制图,为大范围空气质量评估提供了重要数据支撑。然而,这些方法多采用欧几里得网格结构,难以适应监测网络的不规则分布和动态变化。毛文静等\citep{cn_maowenjing2022lianxu}基于多层LSTM迭代预测模型,建立中国PM$_{2.5}$时空预报系统,实现未来24小时连续空间覆盖预报。

\textbf{归纳式图神经网络}。图神经网络为非规则分布的监测网络提供了更自然的建模框架。Wu等人提出的IGNNK(Inductive Graph Neural Networks for Spatiotemporal Kriging)\citep{wu2021inductive}为时空推断提供了新思路,通过随机子图采样和动态节点掩码训练使模型具备对未见节点的归纳式泛化能力,突破了传统转导式方法仅能预测训练时已见节点的局限。后续工作Increase\citep{zheng2023increase}引入因果注意力机制增强空间依赖建模,Kits\citep{xu2025kits}结合知识蒸馏提升未监测位置的预测性能,SATCN\citep{wu2021spatial}采用自适应时空卷积网络处理动态图结构。Appleby等人\citep{appleby2020kriging}将图卷积网络应用于空间插值任务,验证了GNN在克里金问题上的有效性。然而,这些方法主要面向交通流场景设计,在大气污染推断中缺少对排放驱动的物理建模、气象条件的动态影响以及卫星数据的有效融合机制。此外,现有方法通常假设目标节点在推断时已知其位置特征,未能解决完全无先验信息区域的浓度估计问题。

针对上述问题,本文后续章节将提出SPIN(Spatial Pollution Inference Network)模型,通过物理引导的图传播机制和多源数据自适应融合策略,实现对无监测区域的鲁棒空间推断。

\subsection{大气污染情景模拟研究}
\label{subsec:simulation_review}

在环境治理决策中,不仅需要\cqt{预测未来会发生什么},更需要\cqt{推演假设情景}(如:减排50\%后空气质量如何变化?碳中和目标下的空气质量如何演变?)。这类假设情景推理问题对模型提出了更高要求。

\textbf{基于CTMs的情景分析}。目前的排放情景分析和源贡献解析主要依赖化学传输模式。例如,CMAQ-ISAM(Integrated Source Apportionment Method)可量化各类排放源对污染物浓度的贡献\citep{appel2021community},CMAQ-DDM(Decoupled Direct Method)可计算浓度对排放的敏感性系数$\partial C_i / \partial E_i$,CAMx-OSAT/PSAT支持O$_3$和颗粒物的源解析。这些方法基于严格的物理化学方程,结果具有较强的科学可解释性。Hu等人\citep{hu2023changing}利用GEOS-Chem伴随模型分析了长三角地区\PM 和\ozone 对排放源的响应变化。Wang等人\citep{wang2019responses}采用WRF-CMAQ分析了气象条件和排放变化对\PM 和\ozone 浓度的响应关系。然而,这些分析每次都需要重新运行完整模拟,单次年度模拟在高性能集群上需要数天至数周的计算时间,计算代价极高,难以支撑实时决策和大规模情景扫描。

\textbf{长期情景模拟}。在气候变化与空气质量交互的长时间尺度上,CMIP6(Coupled Model Intercomparison Project Phase 6)\footnote{\url{https://wcrp-cmip.org/cmip-phases/cmip6/}} 提供了未来气候情景的标准化框架\citep{eyring2016overview}。Turnock等人\citep{turnock2020historical}基于CMIP6多模式集合分析了未来全球空气质量变化趋势,指出强减排情景(SSP1-26)下全球O$_3$和PM$_{2.5}$浓度均有显著下降潜力。然而,全球排放情景对中国的刻画较为粗糙,难以准确反映中国自2013年\cqt{大气十条}实施以来的快速减排进程。Tong等人\citep{tong2020dynamic}开发了中国未来排放动态评估模型DPEC(Dynamic Projection model for Emissions in China)\footnote{\url{http://meicmodel.org.cn/?page_id=1922}},通过衔接GCAM-China能源系统模型与MEIC排放清单模型,构建了与CMIP6 SSP-RCP情景矩阵衔接的中国本地化排放情景数据集,能够动态模拟不同社会经济情景、气候目标约束和污染控制政策组合下的未来排放变化。Cheng等人\citep{cheng2021pathways}结合DPEC情景开展了碳达峰碳中和路径下的空气质量预测,验证表明DPEC情景能更准确地重现2015--2020年中国排放变化趋势。Liu等人\citep{liu2021wind}采用与本文IGNN相似的\cqt{历史校准-未来预测}范式,基于WRF-CMAQ模式评估了SSP情景下风沙对华北地区颗粒物的长期影响,发现高排放情景下2050年沙尘事件将加剧并部分抵消人为减排效益;但该研究仅针对PM$_{10}$单一污染物和2050年单一目标年份,历史模拟精度亦有限。然而,CTM的高计算成本限制了长期集合模拟的数量和分辨率,数据驱动的快速情景评估工具具有重要应用价值。

\subsection{研究现状总结与存在问题}
\label{subsec:summary}

综合国内外研究进展,当前大气污染数据驱动建模领域仍存在以下四方面突出问题。

\textbf{机理模型可解释但实时性与精度受限}。传统化学传输模型以公式(\ref{eq:advection_diffusion})等物理--化学方程为基础,具备较强可解释性,但高计算成本限制了其实时应用;对排放清单和边界条件的依赖过强,导致在快速变化或极端天气条件下预测精度显著下降。

\textbf{数据驱动模型高效但缺乏物理一致性}。现有的深度学习模型在计算效率上具有显著优势,但缺乏对公式(\ref{eq:advection_diffusion})所描述物理规律的显式建模,在极端情景、分布外数据、长时序预测等任务上泛化能力不足;大多数模型不包含排放输入,无法响应排放变化或支持政策评估。

\textbf{情景模拟计算成本高昂}。传统灵敏度分析与情景模拟需反复运行数值模式,计算量巨大,存在\cqt{组合爆炸}问题。现有AI代理模型多采用自回归预测范式,缺乏对排放变化的响应能力,难以支撑\cqt{如果减排50\%结果如何}这类假设情景推演需求。

\textbf{系统落地能力不足,科研原型与应用脱节}。多数研究停留在离线实验阶段,缺乏稳定的数据流管理与云端服务能力,尚未实现从模型开发到业务化运行的闭环。

针对上述不足,本研究提出一种面向复杂系统的数据驱动建模框架,在统一潜在空间中融合多源观测与物理约束,构建从预测、推断到模拟的多任务建模体系,并通过云原生架构实现空气质量智能预报系统的工程化落地。


\section{研究框架与创新点}
\label{sec:framework}

\subsection{核心研究问题与挑战}
\label{subsec:core_problems}

本论文围绕大气污染复杂系统的数据驱动建模,聚焦于\textbf{预测}、\textbf{推断}与\textbf{模拟}三个核心研究问题,呈现\cqt{点$\rightarrow$面}与\cqt{当前$\rightarrow$未来}的递进关系:

\textbf{研究问题一(Q1):大气污染时序预测问题}(第\ref{chap:prediction}章)。给定历史空气质量观测序列、气象场数据和排放清单信息,如何构建物理约束的时空图神经网络模型,实现\PM 与\ozone 浓度的高精度联合预报?\textbf{核心挑战}:(1)长程传输建模——污染物跨区域迁移受风场主导,传统各向同性图结构难以刻画\cqt{上风向影响下风向}的定向传输规律;(2)物理一致性保障——纯数据驱动模型可能产生违背质量守恒的非物理预测,尤其在极端情景下泛化能力不足。

\textbf{研究问题二(Q2):大气污染空间推断问题}(第\ref{chap:inference}章)。由于地面监测站点分布稀疏且主要集中于城市区域,如何融合卫星遥感、气象再分析和排放清单等多源异构数据,对未设监测站点的区域进行高分辨率空间推断?\textbf{核心挑战}:(1)遥感AOD大面积缺失——云层遮挡导致卫星观测缺失率高达60\%以上,传统方法难以有效利用;(2)归纳式泛化——现有模型多采用转导式学习,仅能处理固定图拓扑,无法泛化至未见站点。

\textbf{研究问题三(Q3):未来情景模拟问题}(第\ref{chap:simulation}章)。面向碳达峰碳中和目标,如何利用历史数据训练的模型迁移至未来排放情景,预测不同减排路径下的空气质量演变?\textbf{核心挑战}:(1)数值模式计算慢——传统CTMs单次区域年尺度模拟需数百核时,难以支撑多情景大规模探索;(2)数据驱动方法难以响应排放变化——现有深度学习模型不包含排放输入,无法回答\cqt{减排后空气质量如何变化}这一关键问题。

\textbf{三个问题的共性挑战}。尽管上述问题在形式上各有侧重,但存在深层共性:(1)\textbf{时空复杂性}——大气污染的演化涉及多尺度时空耦合,传统方法难以同时捕捉局地化学反应与跨区域传输;(2)\textbf{物理--数据鸿沟}——机理模型可解释但计算昂贵,数据驱动方法高效但缺乏物理一致性;(3)\textbf{多源异构数据融合}——地面监测、卫星遥感、气象再分析、排放清单等数据源在时空分辨率、覆盖范围、观测误差等方面差异显著。如何在统一框架内有效应对这些共性挑战,是本研究的核心科学问题。

\subsection{研究范式}
\label{subsec:paradigm}

针对上述共性挑战,本论文提出\textbf{\cqt{物理约束的数据驱动建模}}核心范式:\textbf{在数据驱动的深度学习框架中融合大气科学领域知识}。

深度学习与传统统计机器学习的本质区别在于\textbf{表示学习}(Representation Learning)能力\citep{goodfellow2016deep}。传统方法依赖人工设计的特征工程,而深度学习能够自动从原始数据中提取层次化的特征表示。本文的核心思想是将大气科学领域知识融入表示学习的全过程——不仅约束模型的输入输出映射,更重要的是约束模型学习到的\textbf{隐空间表示}本身,使其具备物理意义与一致性。

具体而言,本文构建了\cqt{编码$\rightarrow$隐空间动力学$\rightarrow$解码}的统一建模框架(如图\ref{fig:framework}所示):通过\textbf{编码层}将多源异构数据(污染物浓度、气象变量、排放数据)映射到统一隐空间,在\textbf{隐空间动力学层}中通过空间模块与时间模块的交替建模学习时空演化规律,最后由\textbf{解码层}生成预测结果。物理约束在三个层次嵌入:编码层通过\textbf{图结构设计}编码风场等物理先验,隐空间层通过\textbf{图算子设计}模拟扩散与平流过程,解码层通过\textbf{损失函数}嵌入质量守恒等物理约束。这种三层架构的技术细节将在第\ref{chap:methodology}章详细阐述。

\begin{figure}[htbp]
    \centering
    \includegraphics[width=\textwidth]{figures/chap01_thesis_framework.pdf}
    \caption{本文研究框架}
    \caption*{框架涵盖四个层次:第二章建立物理约束的时空图神经网络建模范式(理论基础);第三至五章分别针对预测、推断、模拟三类问题提出定制化模型,各问题间体现\cqt{时间$\rightarrow$空间}与\cqt{当前$\rightarrow$未来}的维度迁移;第六章实现工程落地。}
    \label{fig:framework}
\end{figure}

这种范式的核心优势在于:表示学习使模型能够在隐空间中捕捉观测数据背后的本质规律,而物理约束确保学习到的表示具有科学意义。本文在预测、推断、模拟等不同任务上的一致有效性,验证了物理约束表示学习范式的普适性。

\subsection{解决方法与创新点}
\label{subsec:contributions_chap1}

基于上述研究范式,针对三个核心问题分别提出创新性解决方案:

\textbf{创新点一:物理约束的时空图神经网络预测框架}(第\ref{chap:prediction}章)。提出PM$_{2.5}$-GNN与PCDCNet模型,实现\PM 与\ozone 的72小时协同预报。设计LID--STD--TAD三模块架构,分别对应第\ref{chap:methodology}章公式(\ref{eq:advection_diffusion})中的化学反应与排放项、平流与扩散项、沉降与累积项,实现过程解耦与联合建模;利用风速向量投影定义有向边权重,突破传统GNN各向同性图结构的局限;提出领域一致性约束(DIC),将质量守恒与时空连续性嵌入训练目标。

\textbf{创新点二:多源数据融合与归纳式空间推断方法}(第\ref{chap:inference}章)。提出SPIN模型,实现无监测区域的全域制图。设计\cqt{以AOD空间梯度为约束而非输入}的融合策略,通过掩码机制规避缺测影响,实现全天候连续制图;采用扩散--平流双图并行传播机制刻画物理传输过程;引入动态节点掩码训练,赋予模型对未见站点的归纳式泛化能力,突破转导式学习仅能处理固定图拓扑的局限。

\textbf{创新点三:排放响应的深度学习情景模拟框架}(第\ref{chap:simulation}章)。提出IGNN模型,首次将排放清单作为可控变量纳入深度学习框架,融合多尺度排放数据(MEIC历史清单与DPEC未来情景),实现碳中和路径下2025--2060年的长期情景预测,揭示\PM 与\ozone 反向演变趋势与气候惩罚效应,为协同减排决策提供科学依据。

\textbf{工程落地}(第\ref{chap:deployment}章)。基于云原生架构构建KnowAir大气污染智能预报系统,实现模型的工程化部署与实时业务化服务,完成从科研原型到落地应用的闭环。

上述三个研究问题及其解决方案构成了\textbf{\cqt{理论基础$\rightarrow$科学问题$\rightarrow$方法创新$\rightarrow$工程落地}}的完整研究体系,各章引言将结合具体问题示意图详细阐述研究动机与技术挑战。

\section{本文组织结构}
\label{sec:organization}

本论文共分七章,按照\cqt{问题定义$\rightarrow$方法论基础$\rightarrow$预测建模$\rightarrow$空间推断$\rightarrow$情景模拟$\rightarrow$系统部署$\rightarrow$总结展望}的逻辑组织。

\textbf{第一章~绪论}。介绍研究背景,阐述大气污染的复杂系统特征,综述国内外研究现状,提出研究框架与创新点。

\textbf{第二章~物理约束的时空图神经网络建模范式}。系统阐述本文的方法论基础,包括图神经网络基础(谱方法与消息传递两大范式)、图神经网络在地球科学中的应用、物理启发的领域知识(对流-扩散方程与质量守恒约束)、以及本文提出的\cqt{编码$\rightarrow$隐空间动力学$\rightarrow$解码}统一建模框架,为后续应用章节提供统一的理论框架。

\textbf{第三章~物理约束时空图神经网络的大气污染预测}。针对现有模型缺乏物理一致性的问题,提出PM$_{2.5}$-GNN和PCDCNet模型,验证\PM 和\ozone 的72小时联合预报性能。

\textbf{第四章~基于多源数据融合的大气污染空间推断}。针对监测站点稀疏导致的空间盲区问题,提出SPIN模型,介绍扩散--平流双图构建与AOD梯度约束机制,验证不同缺测比例下的推断性能。

\textbf{第五章~未来污染情景模拟}。针对长期情景模拟问题,提出IGNN模型,首次将排放清单作为可控变量纳入深度学习框架,介绍排放数据融合方法,预测碳中和路径下的空气质量演变。

\textbf{第六章~系统部署与落地应用}。介绍从科研模型到业务系统的工程化落地,阐述云原生架构设计与实时数据处理流程,展示典型事件预报效果。

\textbf{第七章~总结与展望}。总结研究工作与主要贡献,分析局限性,展望物理机制可学习表达、不确定性量化、生成式建模等未来方向。