% =========================================================
% commands.tex - 博士论文全局符号与命令定义
% 基于复杂系统数据驱动建模的大气污染研究
%
% 版本: 3.0 (newcommand版)
% 修复内容: 所有命令使用 \newcommand 定义,\ensuremath 确保兼容性
% =========================================================

% =========================================================
% 0. 依赖包检测与加载 (Package Loading)
% =========================================================
\makeatletter
% xspace 用于智能空格
\@ifpackageloaded{xspace}{}{\usepackage{xspace}}

% amsmath 通常由模版加载
\@ifpackageloaded{amsmath}{}{\usepackage{amsmath}}

% [重要] amssymb 与 unicode-math 冲突,模版使用 unicode-math 时必须注释
% \@ifpackageloaded{amssymb}{}{\usepackage{amssymb}}

% pifont 用于支持 \ding{51} (打钩/打叉)
\@ifpackageloaded{pifont}{}{\usepackage{pifont}}
\makeatother

% =========================================================
% 1. 空气污染物 (Pollutants)
%    所有化学式均使用 \ensuremath 包裹
% =========================================================
% 1.1 颗粒物 (使用 \providecommand 避免与已有定义冲突)
\providecommand{\PM}{}
\renewcommand{\PM}{\ensuremath{\mathrm{PM}_{2.5}}\xspace}
\providecommand{\PMten}{\ensuremath{\mathrm{PM}_{10}}\xspace}
\providecommand{\PMcoarse}{\ensuremath{\mathrm{PM}_{10}}\xspace}

% 1.2 气态污染物
\providecommand{\ozone}{\ensuremath{\mathrm{O}_{3}}\xspace}
\providecommand{\NitrogenDioxide}{\ensuremath{\mathrm{NO}_{2}}\xspace}
\providecommand{\NitrogenOxides}{\ensuremath{\mathrm{NO}_{x}}\xspace}
\providecommand{\NO}{\ensuremath{\mathrm{NO}}\xspace}
\providecommand{\NOtwo}{\ensuremath{\mathrm{NO}_{2}}\xspace}
\providecommand{\NOx}{\ensuremath{\mathrm{NO}_{x}}\xspace}
\providecommand{\SulfurDioxide}{\ensuremath{\mathrm{SO}_{2}}\xspace}
\providecommand{\SOtwo}{\ensuremath{\mathrm{SO}_{2}}\xspace}
\providecommand{\CarbonMonoxide}{\ensuremath{\mathrm{CO}}\xspace}
\providecommand{\CO}{\ensuremath{\mathrm{CO}}\xspace}
\providecommand{\CarbonDioxide}{\ensuremath{\mathrm{CO}_{2}}\xspace}
\providecommand{\COtwo}{\ensuremath{\mathrm{CO}_{2}}\xspace}

% 1.3 前体物与其他化学物种
\providecommand{\VOC}{\ensuremath{\mathrm{VOC}}\xspace}
\providecommand{\VOCs}{\ensuremath{\mathrm{VOCs}}\xspace}
\providecommand{\BVOC}{\ensuremath{\mathrm{BVOC}}\xspace}
\providecommand{\BVOCs}{\ensuremath{\mathrm{BVOCs}}\xspace}
\providecommand{\Ammonia}{\ensuremath{\mathrm{NH}_{3}}\xspace}
\providecommand{\NH}{\ensuremath{\mathrm{NH}_{3}}\xspace}
\providecommand{\NHthree}{\ensuremath{\mathrm{NH}_{3}}\xspace}
\providecommand{\HNO}{\ensuremath{\mathrm{HNO}_{3}}\xspace}
\providecommand{\Hsotwo}{\ensuremath{\mathrm{H}_{2}\mathrm{SO}_{4}}\xspace}

% 1.4 二次污染物组分
\providecommand{\SOA}{\ensuremath{\mathrm{SOA}}\xspace}
\providecommand{\SNA}{\ensuremath{\mathrm{SNA}}\xspace}

% 1.5 气溶胶光学厚度
\providecommand{\AOD}{\ensuremath{\mathrm{AOD}}\xspace}

% =========================================================
% 2. 物理单位 (Units)
%    所有单位均使用 \ensuremath 包裹
% =========================================================
% 2.1 浓度单位
\providecommand{\ug}{\ensuremath{\mu\mathrm{g}/\mathrm{m}^3}\xspace}
\providecommand{\ugm}{\ensuremath{\mu\mathrm{g}\,\mathrm{m}^{-3}}\xspace}
\providecommand{\ppb}{\ensuremath{\mathrm{ppb}}\xspace}
\providecommand{\ppm}{\ensuremath{\mathrm{ppm}}\xspace}

% 2.2 长度与面积
\providecommand{\km}{\ensuremath{\mathrm{km}}\xspace}
\providecommand{\m}{\ensuremath{\mathrm{m}}\xspace}
\providecommand{\cm}{\ensuremath{\mathrm{cm}}\xspace}
\providecommand{\mm}{\ensuremath{\mathrm{mm}}\xspace}
\providecommand{\kmsq}{\ensuremath{\mathrm{km}^{2}}\xspace}

% 2.3 时间
\providecommand{\s}{\ensuremath{\mathrm{s}}\xspace}
\providecommand{\h}{\ensuremath{\mathrm{h}}\xspace}
\providecommand{\hr}{\ensuremath{\mathrm{h}}\xspace}

% 2.4 速度与通量
\providecommand{\mps}{\ensuremath{\mathrm{m/s}}\xspace}
\providecommand{\kmph}{\ensuremath{\mathrm{km/h}}\xspace}

% 2.5 质量与排放
\providecommand{\ton}{\ensuremath{\mathrm{ton}}\xspace}
\providecommand{\Tg}{\ensuremath{\mathrm{Tg}}\xspace}
\providecommand{\Gg}{\ensuremath{\mathrm{Gg}}\xspace}
\providecommand{\kg}{\ensuremath{\mathrm{kg}}\xspace}

% 2.6 气象单位
\providecommand{\Pa}{\ensuremath{\mathrm{Pa}}\xspace}
\providecommand{\hPa}{\ensuremath{\mathrm{hPa}}\xspace}
\providecommand{\K}{\ensuremath{\mathrm{K}}\xspace}
\providecommand{\degC}{\ensuremath{^{\circ}\mathrm{C}}\xspace}
\providecommand{\Wm}{\ensuremath{\mathrm{W/m}^{2}}\xspace}
\providecommand{\Wmsq}{\ensuremath{\mathrm{W}\,\mathrm{m}^{-2}}\xspace}
\providecommand{\degree}{\ensuremath{^{\circ}}\xspace}
\providecommand{\percent}{\ensuremath{\%}\xspace}

% =========================================================
% 3. 核心变量 (Core Variables)
%    对应符号表中的主要数学符号
% =========================================================
% 3.1 空气质量变量
\providecommand{\AirPoll}{\ensuremath{\mathbf{X}}\xspace}
\providecommand{\AirPollHat}{\ensuremath{\hat{\mathbf{X}}}\xspace}
\providecommand{\Concentration}{\ensuremath{C}\xspace}
\providecommand{\ConcentrationVec}{\ensuremath{\mathbf{C}}\xspace}

% 3.2 驱动变量
\providecommand{\Meteo}{\ensuremath{\mathbf{M}}\xspace}
\providecommand{\MeteoP}{\ensuremath{\mathbf{P}}\xspace}
\providecommand{\Emiss}{\ensuremath{\mathbf{E}}\xspace}
\providecommand{\EmissQ}{\ensuremath{\mathbf{Q}}\xspace}

% 3.3 遥感与辅助变量
\providecommand{\RemoteSensing}{\ensuremath{\mathbf{X}^{\mathrm{AOD}}}\xspace}
\providecommand{\AODMask}{\ensuremath{\mathbf{M}^{\mathrm{AOD}}}\xspace}
\providecommand{\AODObs}{\ensuremath{\mathbf{Y}^{\mathrm{AOD}}}\xspace}

% 3.4 隐状态与中间变量
\providecommand{\Hidden}{\ensuremath{\mathbf{H}}\xspace}
\providecommand{\HiddenZ}{\ensuremath{\mathbf{Z}}\xspace}
\providecommand{\Message}{\ensuremath{\mathbf{m}}\xspace}
\providecommand{\MessageAgg}{\ensuremath{\mathbf{Z}}\xspace}  % 聚合消息,避免与气象符号M冲突
\providecommand{\Feature}{\ensuremath{\mathbf{h}}\xspace}
\providecommand{\Embedding}{\ensuremath{\mathbf{e}}\xspace}

% 3.5 风场变量
\providecommand{\WindVec}{\ensuremath{\mathbf{u}}\xspace}
\providecommand{\WindU}{\ensuremath{u}\xspace}
\providecommand{\WindV}{\ensuremath{v}\xspace}
\providecommand{\WindW}{\ensuremath{w}\xspace}
\providecommand{\WindSpeed}{\ensuremath{|\mathbf{u}|}\xspace}

% 3.6 扩散与传输系数
\providecommand{\DiffCoef}{\ensuremath{K}\xspace}
\providecommand{\DiffCoefX}{\ensuremath{K_x}\xspace}
\providecommand{\DiffCoefY}{\ensuremath{K_y}\xspace}
\providecommand{\DiffCoefZ}{\ensuremath{K_z}\xspace}

% =========================================================
% 4. 图结构与时空参数 (Graph & Spatiotemporal)
% =========================================================
% 4.1 图结构基础
\providecommand{\Graph}{\ensuremath{\mathcal{G}}\xspace}
\providecommand{\Nodes}{\ensuremath{\mathcal{V}}\xspace}
\providecommand{\Edges}{\ensuremath{\mathcal{E}}\xspace}
\providecommand{\NodeSet}{\ensuremath{V}\xspace}
\providecommand{\EdgeSet}{\ensuremath{E}\xspace}
\providecommand{\NumNodes}{\ensuremath{|V|}\xspace}
\providecommand{\NumEdges}{\ensuremath{|E|}\xspace}

% 4.2 图矩阵
\providecommand{\Adj}{\ensuremath{\mathbf{A}}\xspace}
\providecommand{\AdjNorm}{\ensuremath{\tilde{\mathbf{A}}}\xspace}
\providecommand{\Laplacian}{\ensuremath{\mathbf{L}}\xspace}
\providecommand{\LaplacianNorm}{\ensuremath{\tilde{\mathbf{L}}}\xspace}
\providecommand{\Degree}{\ensuremath{\mathbf{D}}\xspace}
\providecommand{\Identity}{\ensuremath{\mathbf{I}}\xspace}
\providecommand{\Neighbor}[1]{\ensuremath{\mathcal{N}(#1)}\xspace}

% 4.3 双图结构 (SPIN模型)
\providecommand{\AdjDiff}{\ensuremath{\tilde{\mathbf{A}}^{\mathcal{D}}}\xspace}
\providecommand{\AdjAdv}{\ensuremath{\tilde{\mathbf{A}}^{\mathcal{A}}}\xspace}
\providecommand{\GraphSpatial}{\ensuremath{\mathcal{G}^{\mathcal{S}}}\xspace}
\providecommand{\GraphAdvection}{\ensuremath{\mathcal{G}^{\mathcal{A}}}\xspace}
\providecommand{\GraphDiffusion}{\ensuremath{\mathcal{G}^{\mathcal{D}}}\xspace}

% 4.4 边特征
\providecommand{\EdgeWeight}{\ensuremath{w_{ij}}\xspace}
\providecommand{\EdgeFeature}{\ensuremath{\mathbf{e}_{ij}}\xspace}
\providecommand{\Distance}{\ensuremath{d_{ij}}\xspace}

% 4.5 时间参数
\providecommand{\HistLen}{\ensuremath{T'}\xspace}
\providecommand{\PredLen}{\ensuremath{T}\xspace}
\providecommand{\LeadTime}{\ensuremath{\tau}\xspace}
\providecommand{\TimeStep}{\ensuremath{\Delta t}\xspace}
\providecommand{\TimeIndex}{\ensuremath{t}\xspace}

% =========================================================
% 5. 损失函数与优化 (Loss Functions & Optimization)
% =========================================================
% 5.1 损失函数符号
\providecommand{\Loss}{\ensuremath{\mathcal{L}}\xspace}
\providecommand{\LossTotal}{\ensuremath{\mathcal{L}_{\mathrm{total}}}\xspace}
\providecommand{\LossSup}{\ensuremath{\mathcal{L}_{\mathrm{sup}}}\xspace}
\providecommand{\LossDIC}{\ensuremath{\mathcal{L}_{\mathrm{DIC}}}\xspace}
\providecommand{\LossAOD}{\ensuremath{\mathcal{L}_{\mathrm{AOD}}}\xspace}
\providecommand{\LossInfer}{\ensuremath{\mathcal{L}_{\mathrm{infer}}}\xspace}
\providecommand{\LossInit}{\ensuremath{\mathcal{L}_{\mathrm{init}}}\xspace}
\providecommand{\LossPred}{\ensuremath{\mathcal{L}_{\ell_1}}\xspace}
\providecommand{\LossLone}{\ensuremath{\mathcal{L}_{\ell_1}}\xspace}
\providecommand{\LossLtwo}{\ensuremath{\mathcal{L}_{\ell_2}}\xspace}
\providecommand{\LossMSE}{\ensuremath{\mathcal{L}_{\mathrm{MSE}}}\xspace}
\providecommand{\LossMAE}{\ensuremath{\mathcal{L}_{\mathrm{MAE}}}\xspace}
\providecommand{\LossSmooth}{\ensuremath{\mathcal{L}_{\mathrm{smooth}}}\xspace}
\providecommand{\LossPhys}{\ensuremath{\mathcal{L}_{\mathrm{phys}}}\xspace}

% 5.2 优化参数
\providecommand{\Params}{\ensuremath{\Theta}\xspace}
\providecommand{\LearnRate}{\ensuremath{\alpha}\xspace}
\providecommand{\RegWeight}{\ensuremath{\lambda}\xspace}
\providecommand{\BatchSize}{\ensuremath{B}\xspace}
\providecommand{\Epoch}{\ensuremath{E}\xspace}

% 5.3 神经网络层
\providecommand{\Linear}{\ensuremath{\mathrm{Linear}}\xspace}
\providecommand{\MLP}{\ensuremath{\mathrm{MLP}}\xspace}
\providecommand{\GRUcell}{\ensuremath{\mathrm{GRU}}\xspace}
\providecommand{\LSTMcell}{\ensuremath{\mathrm{LSTM}}\xspace}
\providecommand{\Softmax}{\ensuremath{\mathrm{softmax}}\xspace}
\providecommand{\Sigmoid}{\ensuremath{\sigma}\xspace}
\providecommand{\ReLU}{\ensuremath{\mathrm{ReLU}}\xspace}
\providecommand{\Tanh}{\ensuremath{\tanh}\xspace}

% =========================================================
% 6. 模型名称 (Model Names)
% =========================================================
% 6.1 本文提出的模型
\providecommand{\ModelPred}{\ensuremath{\mathrm{PM}_{2.5}}\text{-GNN}\xspace}
\providecommand{\ModelSurr}{PCDCNet\xspace}
\providecommand{\ModelInfer}{SPIN\xspace}
\providecommand{\ModelSim}{IGNN\xspace}
\providecommand{\ModelControl}{PCDCNet-IR\xspace}
\providecommand{\SystemName}{KnowAir\xspace}

% 6.2 基线模型
\providecommand{\GCLSTM}{GC-LSTM\xspace}
\providecommand{\STGCN}{STGCN\xspace}
\providecommand{\AirFormer}{AirFormer\xspace}
\providecommand{\IGNNK}{IGNNK\xspace}
\providecommand{\XGBoost}{XGBoost\xspace}
\providecommand{\LightGBM}{LightGBM\xspace}
\providecommand{\LSTM}{LSTM\xspace}
\providecommand{\GRU}{GRU\xspace}
\providecommand{\Transformer}{Transformer\xspace}
\providecommand{\TCN}{TCN\xspace}

% 6.3 气象AI模型
\providecommand{\Pangu}{Pangu-Weather\xspace}
\providecommand{\GraphCast}{GraphCast\xspace}
\providecommand{\FourCastNet}{FourCastNet\xspace}
\providecommand{\Aurora}{Aurora\xspace}

% =========================================================
% 7. 数据源与区域 (Data Sources & Regions)
% =========================================================
% 7.1 排放清单
\providecommand{\MEIC}{MEIC\xspace}
\providecommand{\DPEC}{DPEC\xspace}
\providecommand{\EDGAR}{EDGAR\xspace}
\providecommand{\ABaCAS}{ABaCAS\xspace}

% 7.2 气象数据
\providecommand{\ERAFive}{ERA5\xspace}
\providecommand{\GFS}{GFS\xspace}
\providecommand{\CMIP}{CMIP6\xspace}
\providecommand{\CAMS}{CAMS\xspace}
\providecommand{\ECMWF}{ECMWF\xspace}
\providecommand{\IFS}{IFS\xspace}

% 7.3 遥感数据
\providecommand{\Himawari}{Himawari-8\xspace}
\providecommand{\MODIS}{MODIS\xspace}
\providecommand{\VIIRS}{VIIRS\xspace}

% 7.4 数值模式
\providecommand{\CMAQ}{CMAQ\xspace}
\providecommand{\WRFChem}{WRF-Chem\xspace}
\providecommand{\WRF}{WRF\xspace}
\providecommand{\GEOSChem}{GEOS-Chem\xspace}
\providecommand{\CAMx}{CAMx\xspace}
\providecommand{\NAQPMS}{NAQPMS\xspace}

% 7.5 监测网络
\providecommand{\CNEMC}{CNEMC\xspace}

% 7.6 研究区域
\providecommand{\BTHSA}{BTHSA\xspace}
\providecommand{\YRD}{YRD\xspace}
\providecommand{\PRD}{PRD\xspace}
\providecommand{\NCUA}{NCUA\xspace}
\providecommand{\FWP}{FWP\xspace}

% 7.7 数据集名称
\providecommand{\KnowAirOne}{KnowAir-DS-V1\xspace}
\providecommand{\KnowAirTwo}{KnowAir-DS-V2\xspace}
\providecommand{\KnowAirDS}{KnowAir-DS\xspace}

% =========================================================
% 8. 评价指标 (Evaluation Metrics)
%    所有指标使用 \ensuremath 包裹
% =========================================================
% 8.1 回归指标
\providecommand{\RMSE}{\ensuremath{\mathrm{RMSE}}\xspace}
\providecommand{\MAE}{\ensuremath{\mathrm{MAE}}\xspace}
\providecommand{\MSE}{\ensuremath{\mathrm{MSE}}\xspace}
\providecommand{\MAPE}{\ensuremath{\mathrm{MAPE}}\xspace}
\providecommand{\Rsquare}{\ensuremath{R^{2}}\xspace}
\providecommand{\Corr}{\ensuremath{r}\xspace}
\providecommand{\CorrCoef}{\ensuremath{\rho}\xspace}

% 8.2 分类/检测指标
\providecommand{\CSI}{\ensuremath{\mathrm{CSI}}\xspace}
\providecommand{\POD}{\ensuremath{\mathrm{POD}}\xspace}
\providecommand{\FAR}{\ensuremath{\mathrm{FAR}}\xspace}
\providecommand{\ACC}{\ensuremath{\mathrm{ACC}}\xspace}
\providecommand{\Precision}{\ensuremath{\mathrm{Precision}}\xspace}
\providecommand{\Recall}{\ensuremath{\mathrm{Recall}}\xspace}
\providecommand{\Fone}{\ensuremath{F_{1}}\xspace}

% 8.3 环境评估指标
\providecommand{\NMB}{\ensuremath{\mathrm{NMB}}\xspace}
\providecommand{\NME}{\ensuremath{\mathrm{NME}}\xspace}
\providecommand{\MFB}{\ensuremath{\mathrm{MFB}}\xspace}
\providecommand{\MFE}{\ensuremath{\mathrm{MFE}}\xspace}
\providecommand{\IA}{\ensuremath{\mathrm{IA}}\xspace}
\providecommand{\IOA}{\ensuremath{\mathrm{IOA}}\xspace}
\providecommand{\MDAeight}{\ensuremath{\mathrm{MDA8}}\xspace}

% =========================================================
% 9. 数学算子与常用符号 (Math Operators)
% =========================================================
% 9.1 微分算子
\providecommand{\Grad}{\ensuremath{\nabla}\xspace}
\providecommand{\Div}{\ensuremath{\nabla \cdot}\xspace}
\providecommand{\Curl}{\ensuremath{\nabla \times}\xspace}
\providecommand{\Lapl}{\ensuremath{\nabla^2}\xspace}
\providecommand{\PartialT}{\ensuremath{\frac{\partial}{\partial t}}\xspace}
\providecommand{\PartialX}{\ensuremath{\frac{\partial}{\partial x}}\xspace}
\providecommand{\PartialY}{\ensuremath{\frac{\partial}{\partial y}}\xspace}
\providecommand{\PartialZ}{\ensuremath{\frac{\partial}{\partial z}}\xspace}

% 9.2 差分与增量
\providecommand{\Diff}{\ensuremath{\Delta}\xspace}
\providecommand{\DiffX}{\ensuremath{\Delta X}\xspace}
\providecommand{\DiffE}{\ensuremath{\Delta E}\xspace}
\providecommand{\DiffC}{\ensuremath{\Delta C}\xspace}

% 9.3 模型映射函数
\providecommand{\ModelFunc}{\ensuremath{\mathcal{F}}\xspace}
\providecommand{\EncFunc}{\ensuremath{\mathcal{E}}\xspace}
\providecommand{\DecFunc}{\ensuremath{\mathcal{D}}\xspace}

% 9.4 集合与空间符号
\providecommand{\Real}{\ensuremath{\mathbb{R}}\xspace}
\providecommand{\Integer}{\ensuremath{\mathbb{Z}}\xspace}
\providecommand{\Natural}{\ensuremath{\mathbb{N}}\xspace}
\providecommand{\Expect}{\ensuremath{\mathbb{E}}\xspace}
\providecommand{\Prob}{\ensuremath{\mathbb{P}}\xspace}

% 9.5 范数与距离
\providecommand{\Lnorm}[1]{\ensuremath{\|\cdot\|_{#1}}\xspace}
\providecommand{\LoneNorm}{\ensuremath{\|\cdot\|_{1}}\xspace}
\providecommand{\LtwoNorm}{\ensuremath{\|\cdot\|_{2}}\xspace}
\providecommand{\FrobNorm}{\ensuremath{\|\cdot\|_{F}}\xspace}

% 9.6 特殊标记 (需要 pifont 宏包)
\providecommand{\cmark}{\ding{51}}
\providecommand{\xmark}{\ding{55}}

% 9.7 常用缩写
\providecommand{\ie}{\textit{i.e.}\xspace}
\providecommand{\eg}{\textit{e.g.}\xspace}
\providecommand{\etc}{\textit{etc.}\xspace}
\providecommand{\etal}{\textit{et al.}\xspace}
\providecommand{\vs}{\textit{vs.}\xspace}
\providecommand{\wrt}{w.r.t.\xspace}
\providecommand{\iid}{\textit{i.i.d.}\xspace}

% =========================================================
% 10. 模块名称与方法缩写 (Module Abbreviations)
% =========================================================
% 10.1 PCDCNet架构模块
\providecommand{\LID}{LID\xspace}
\providecommand{\STD}{STD\xspace}
\providecommand{\TAD}{TAD\xspace}
\providecommand{\DIC}{DIC\xspace}

% 10.2 通用神经网络模块
\providecommand{\GNN}{GNN\xspace}
\providecommand{\GCN}{GCN\xspace}
\providecommand{\GAT}{GAT\xspace}
\providecommand{\RNN}{RNN\xspace}
\providecommand{\CNN}{CNN\xspace}
\providecommand{\MHA}{MHA\xspace}
\providecommand{\FFN}{FFN\xspace}

% 10.3 化学机制
\providecommand{\CBsix}{CB6\xspace}
\providecommand{\SAPRC}{SAPRC\xspace}
\providecommand{\ISORROPIA}{ISORROPIA\xspace}

% 10.4 数据同化方法
\providecommand{\ISAM}{ISAM\xspace}
\providecommand{\DDM}{DDM\xspace}
\providecommand{\PSAT}{PSAT\xspace}
\providecommand{\OSAT}{OSAT\xspace}

% =========================================================
% 11. 物理过程术语 (Physical Process Terms)
% =========================================================
% 11.1 大气过程
\providecommand{\Advection}{\ensuremath{\mathrm{Adv}}\xspace}
\providecommand{\Diffusion}{\ensuremath{\mathrm{Diff}}\xspace}
\providecommand{\Emission}{\ensuremath{\mathrm{Emis}}\xspace}
\providecommand{\Deposition}{\ensuremath{\mathrm{Dep}}\xspace}
\providecommand{\DryDep}{\ensuremath{\mathrm{Dry}}\xspace}
\providecommand{\WetDep}{\ensuremath{\mathrm{Wet}}\xspace}
\providecommand{\ChemReaction}{\ensuremath{\mathrm{Chem}}\xspace}

% 11.2 边界层参数
\providecommand{\PBL}{\ensuremath{\mathrm{PBL}}\xspace}
\providecommand{\PBLH}{\ensuremath{h_{\mathrm{PBL}}}\xspace}

% =========================================================
% 12. 情景与政策术语 (Scenarios & Policies)
% =========================================================
% 12.1 排放情景
\providecommand{\RCP}{\ensuremath{\mathrm{RCP}}\xspace}
\providecommand{\SSP}{\ensuremath{\mathrm{SSP}}\xspace}
\providecommand{\BAU}{\ensuremath{\mathrm{BAU}}\xspace}

% 12.2 政策目标
\providecommand{\CarbonPeak}{碳达峰\xspace}
\providecommand{\CarbonNeutral}{碳中和\xspace}

% =========================================================
% 13. 文献引用格式辅助
%     用于表格中的简化引用
% =========================================================
\providecommand{\citemark}[1]{\textsuperscript{#1}}

% =========================================================
% 文件结束
% =========================================================
