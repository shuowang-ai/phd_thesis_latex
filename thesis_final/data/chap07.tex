% ============================================================
% 第七章 总结与展望
% 基于复杂系统数据驱动建模的大气污染研究
% ============================================================

\chapter{总结与展望}
\label{chap:conclusion}


% ------------------------------------------------------------
% 7.1 研究工作总结
% ------------------------------------------------------------
\section{研究工作总结}
\label{sec:summary}

本论文围绕\cqt{大气污染复杂系统的数据驱动建模}这一核心主题,针对空气质量预测、推断与模拟三大核心任务,构建了\cqt{理论$\rightarrow$数据$\rightarrow$建模$\rightarrow$应用$\rightarrow$部署}的完整研究体系。

大气污染作为典型的开放复杂巨系统,具有多要素强耦合、高度非线性、复杂网络拓扑与多源异构数据等本质特征。化学传输模型虽具备物理可解释性,但面临计算成本高昂、对排放清单高度依赖等瓶颈;现有数据驱动方法虽然计算高效,却普遍存在物理一致性弱、难以支撑情景模拟等不足。本研究的核心思想是将大气科学领域知识融入表示学习全过程,构建\cqt{观测空间$\leftrightarrow$隐空间$\leftrightarrow$观测空间}的完整建模框架,从根本上弥合机理建模与数据驱动之间的鸿沟。各章工作具体总结如下。

物理启发的时空图神经网络建模范式(第\ref{chap:methodology}章)。系统阐述了本文的方法论基础,包括时空数据的图表示、图神经网络基础、物理启发融合方法三大核心内容,为后续应用章节提供统一的理论框架。

物理启发时空图神经网络的大气污染预测(第\ref{chap:prediction}章)。针对大气污染时空依赖复杂、物理一致性不足等问题,首先提出PM$_{2.5}$-GNN模型,验证了将风场信息编码为有向图边权的有效性,在多个城市群的\PM 预测中建立了基线;在此基础上进一步提出PCDCNet预测框架,在隐空间中设计过程解耦的三模块结构(LID--STD--TAD),分别对应对流--扩散方程中的化学反应与排放项、平流与扩散项、沉降与累积项,通过物理一致性约束将质量守恒嵌入表示学习过程。在京津冀与长三角区域的\PM 与\ozone 预测中,相较现有最优方法RMSE降低约13--23\%,实现了高精度的72小时协同预报。

基于多源数据融合的大气污染空间推断(第\ref{chap:inference}章)。针对地面监测站点稀疏、遥感AOD数据缺失严重等问题,提出SPIN推断框架。模型采用扩散--平流双图并行传播机制刻画物理传输过程,设计AOD空间梯度约束的掩码机制规避缺测影响;采用动态节点掩码训练策略赋予模型归纳式泛化能力。在30\%站点缺测条件下MAE达到9.5 $\mu$g/m$^3$,较基线方法误差降低约25\%。

未来污染情景模拟(第\ref{chap:simulation}章)。面向碳达峰碳中和战略目标,提出IGNN模型,首次将排放清单作为可控变量纳入深度学习框架,实现2025--2050年多情景空气质量预测,揭示了\PM 与\ozone 的反向演变趋势及气候惩罚效应;同时发现不同城市呈现差异化响应特征,例如北京、太原等VOC控制区的\ozone 浓度呈微弱下降趋势,而多数城市\ozone 持续上升,体现了排放结构与光化学机制的区域异质性。在与传统物理化学模型(CMAQ、WRF-Chem)的对比中,IGNN在典型污染事件模拟上展现出显著优势(\PM:相关系数从0.36--0.38提升至0.84;\ozone:相关系数从0.47--0.57提升至0.90),同时将单情景推理从数小时压缩至秒级,为大规模情景探索提供了可行路径。

系统部署与案例分析(第\ref{chap:deployment}章)。基于云原生架构构建KnowAir智能预报平台(核心为第\ref{chap:prediction}章提出的PCDCNet模型),完成从科研原型到业务系统的工程化部署。系统仅需普通配置云服务器即可在3分钟内完成全国72小时预报。在上海进博会保障等实战应用中,KnowAir与\CMAQ、WRF-Chem、NAQPMS等数值模式进行了直接对比,在\PM、\ozone 等关键污染物预报上展现出显著优势;在粤港澳官方模型比对测试中取得综合评分中位数第一名,并成功转化为商业产品服务数千万用户。


% ------------------------------------------------------------
% 7.2 主要创新点与贡献
% ------------------------------------------------------------
\section{主要创新点与贡献}
\label{sec:contributions}

本论文从方法论角度提炼出\cqt{物理启发的数据驱动建模}核心范式,主要创新点总结如下:

创新点一:物理启发的时空图神经网络预测框架(第\ref{chap:prediction}章)。设计LID--STD--TAD三模块架构,分别对应公式(\ref{eq:advection_diffusion})中的化学反应与排放项、平流与扩散项、沉降与累积项,实现过程解耦与联合建模;利用风速向量投影定义有向边权重,突破传统GNN各向同性图结构的局限,使模型表征\cqt{上风向影响下风向}的定向传输规律;提出领域一致性约束(DIC),将质量守恒嵌入训练目标,提升极端情景下的物理合理性。

创新点二:多源数据融合与归纳式空间推断方法(第\ref{chap:inference}章)。提出\cqt{以AOD空间梯度为约束而非输入}的融合策略,利用AOD与地面\PM 浓度之间的物理关联性——AOD反映大气柱气溶胶光学厚度,其空间梯度蕴含污染物水平分布的物理先验——将该先验以损失函数约束的形式融入模型训练,通过掩码机制规避遥感数据大面积缺测的影响,实现全天候连续制图;采用扩散--平流双图并行传播机制刻画物理传输过程;引入动态节点掩码训练,赋予模型对未见站点的归纳式泛化能力,突破转导式学习仅能处理固定图拓扑的局限。

创新点三:排放响应的深度学习情景模拟框架(第\ref{chap:simulation}章)。首次将排放清单作为可控变量纳入深度学习框架,通过IGNN模型构建从排放源到污染物浓度的端到端映射:利用融合图结构将多尺度排放数据(MEIC历史清单与DPEC未来情景)与气象场、站点观测进行联合编码,使模型学习排放强度变化对大气污染浓度场的响应关系,实现碳中和路径下2025--2050年的长期情景预测,揭示\PM 与\ozone 反向演变趋势与气候惩罚效应,为协同减排决策提供科学依据。

从学科贡献来看,本研究推动大气污染模型从\cqt{能预测}走向\cqt{能推断、能模拟、能决策}。更一般地,本文提炼出一套面向复杂系统的数据驱动建模框架:\cqt{领域知识$\rightarrow$图结构构建$\rightarrow$物理启发表示学习$\rightarrow$任务导向解码},其中图网络作为核心建模工具,具备天然适配空间拓扑结构、支持异构多源数据融合、能够显式编码物理传输机制(如风场驱动的有向传播、扩散--平流双通道)等优势。该框架不局限于大气污染领域,可迁移至气象预报、水文模拟、交通流预测等涉及时空演化与网络拓扑耦合的复杂系统问题。


% ------------------------------------------------------------
% 7.3 研究局限性
% ------------------------------------------------------------
\section{研究局限性}
\label{sec:limitations}

尽管本研究取得了重要进展,但仍存在以下局限性。

物理--化学机理的显式性有限。当前模型虽引入了守恒与一致性约束,但主要以软约束形式作用于隐空间表示。化学反应速率、传输方程等核心机理尚未能显式可学习表达,在极端或罕见工况下的外推能力有待提升。

不确定性刻画能力不足。现有模型主要输出点估计结果,对预测区间、置信度与极端事件风险的量化能力有限。多源输入数据本身存在的观测误差与系统偏差如何传递至预测结果,尚未在模型中得到充分刻画。

泛化范围需进一步扩展。本研究主要验证于中国典型区域,对于不同气候带、地形条件与排放结构的区域,模型的迁移能力尚需系统评估。

跨尺度因果机制的理论认识不足。本研究的Embed--隐空间动力学--Readout架构实现了信息压缩,但该架构是否真正捕捉了涌现的宏观因果规律,目前仍缺乏理论验证;编码器设计也未以最大化因果效应为优化目标。


% ------------------------------------------------------------
% 7.4 未来研究展望
% ------------------------------------------------------------
\section{未来研究展望}
\label{sec:outlook}

基于上述分析,本研究提出以下未来研究方向。

因果涌现视角下的宏微观协同建模。因果涌现(Causal Emergence)是指复杂系统经过粗粒化后,宏观尺度的因果效应可以超越微观尺度的现象,即宏观描述比微观描述具有更强的因果决定性\citep{hoel2013quantifying}。未来可引入因果涌现理论框架\citep{hoel2013quantifying,zhang2022neural},将有效信息(EI)作为隐空间表示学习的优化目标,使编码器自动发现因果效应最强的宏观变量;构建宏微观双向因果架构,探索多尺度因果涌现的相变规律。

物理--化学机制的可学习表达。探索神经偏微分方程(Neural PDEs)、神经算子与物理符号回归等技术,将大气扩散方程、化学反应动力学方程嵌入神经网络架构;利用符号回归从数据中发现简洁的解析表达式,自动提取化学反应速率常数等可解释参数。

多源数据同化与不确定性估计。引入贝叶斯深度学习框架估计隐空间表示及预测结果的后验分布;采用流匹配与扩散模型学习污染场的概率分布,生成多条可能演化轨迹;结合变分数据同化与深度学习构建可在线订正的实时预测系统。

生成式建模与极端情景推演。利用生成扩散模型对历史极端污染事件进行建模,增强模型对罕见高污染事件的模拟与预警能力;通过条件生成实现不同政策干预下的空气质量响应评估。

跨区域迁移与全球服务化推广。基于领域自适应与元学习技术开发可快速适配新区域的模型迁移框架;结合Copernicus大气监测服务、全球排放清单等开放数据源,将系统扩展至全球主要城市群。

智能决策与数字孪生地球。构建面向城市空气质量治理的数字孪生系统,基于\cqt{前向模拟+情景评估}框架实现排放--浓度--健康--政策的全链条闭环分析,结合强化学习探索最优减排路径。