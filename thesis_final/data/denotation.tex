% !TeX root = ../wangshuo_phdthesis.tex

\begin{denotation}[3cm]

% 1. 核心变量
\item[\(\mathbf{X}_t, \hat{\mathbf{X}}_t\)] 时间 \(t\) 的空气质量观测值与预测值(单位:\(\mu\)g/m\(^3\)),$\mathbf{X} \in \mathbb{R}^{N \times D_X}$,包含 PM\(_{2.5}\)、PM\(_{10}\)、O\(_3\)、NO\(_2\) 等污染物。
\item[\(\mathbf{M}_t, \mathbf{E}_t\)] 时间 \(t\) 的气象变量与排放变量集合。
    \begin{itemize}
        \item \(\mathbf{M}_t\) 包含:气温 ($M_{\text{t2m}}$)、露点 ($M_{\text{d2m}}$)、边界层高度 ($M_{\text{blh}}$)、风速 ($M_{u}$, $M_{v}$) 等
        \item \(\mathbf{E}_t\) 包含:NO\(_x\) ($E_{\text{NO}_x}$)、VOC ($E_{\text{VOC}}$)、SO\(_2\) ($E_{\text{SO}_2}$) 等
    \end{itemize}
\item[\(\mathbf{X}^{F}_t\)] 污染物浓度的连续空间场(Field),用于网格化推断与模拟任务。
\item[\(\mathbf{X}^{\text{AOD}}\)] 遥感观测的气溶胶光学厚度(Aerosol Optical Depth),作为空间约束。附带节点级有效性掩码 $\boldsymbol{\omega}^{\text{AOD}} \in \{0,1\}^{N}$($\omega_i^{\text{AOD}} = 1$表示有效,$0$表示云遮挡缺失),以及由此导出的边级掩码$\Omega_{ij}^{\text{AOD}} = \omega_i^{\text{AOD}} \cdot \omega_j^{\text{AOD}}$。

% 2. 图结构与时空参数
\item[\(\mathcal{G} = (\mathcal{V}, \mathcal{E})\)] 图结构,\(\mathcal{V}\) 为站点/城市节点集,\(\mathcal{E}\) 为基于风场或距离构建的边集。
\item[\(\mathcal{G}^{\text{s}}, \mathcal{G}^{\text{g}}\)] 站点图与网格图(第\ref{chap:inference}章)。$\mathcal{G}^{\text{s}}$以$N^{\text{s}}$个监测站点为节点,$\mathcal{G}^{\text{g}}$以$N^{\text{g}}$个$0.25^{\circ}$网格像素为节点,两者采用相同的模型架构与图核构建方法,但各自独立训练,所得邻接矩阵维度不同。
\item[\(N^{\text{s}}, N^{\text{g}}\)] 站点图与网格图的节点数(第\ref{chap:inference}章)。$N^{\text{s}} = |\mathcal{V}^{\text{s}}|$为监测站点数,$N^{\text{g}} = |\mathcal{V}^{\text{g}}|$为网格像素数。
\item[\(\mathbf{A}, \tilde{\mathbf{A}}\)] 图邻接矩阵与归一化后的邻接矩阵。
\item[\(A^{\text{s}}, A^{\text{g}}\)] 站点图与网格图的基础邻接矩阵(第\ref{chap:inference}章),基于节点间测地距离$d_{ij}$按高斯核构建,$A^{\text{s}} \in \mathbb{R}^{N^{\text{s}} \times N^{\text{s}}}$,$A^{\text{g}} \in \mathbb{R}^{N^{\text{g}} \times N^{\text{g}}}$。
\item[\(\mathbf{L}, \tilde{\mathbf{L}}\)] 图拉普拉斯矩阵$\mathbf{L} = \mathbf{D} - \mathbf{A}$与归一化拉普拉斯矩阵。
\item[\(K\)] 大气扩散系数,用于对流-扩散方程的扩散项。
\item[\(t=0\)] 起报时刻,\(t \leq 0\) 为历史时刻,\(t>0\) 为未来预测时刻。
\item[\(T', T\)] 历史输入窗口与未来预测窗口长度。

% 3. 模型与物理算子
\item[\(\mathcal{F}_\Theta\)] 参数为 \(\Theta\) 的深度学习模型,学习真实空气质量生成过程 \(f(\mathbf{M}, \mathbf{E})\) 的映射。
\item[\(\mathcal{F}^{\text{s}}_{\Theta^{\text{s}}}, \mathcal{F}^{\text{g}}_{\Theta^{\text{g}}}\)] 站点推断模型与网格推断模型(第\ref{chap:inference}章)。两者采用相同的SPIN四阶段架构,但各自独立训练,$\Theta^{\text{s}} \neq \Theta^{\text{g}}$。
\item[\(C\)] 污染物浓度场的点值形式,用于对流-扩散方程表述。
\item[\(S, D\)] 对流-扩散方程中的排放源项与沉降汇项。
\item[\(\mathbf{u}\)] 风速矢量场,用于计算平流传输。
\item[\(\mathbf{H}^{(l)}, \mathbf{h}_i\)] 第$l$层隐表示,$\mathbf{h}_i$为节点$i$的隐向量。
\item[\(\mathcal{V}_{\text{obs}}, \mathcal{V}_{\text{target}}\)] 观测节点集与目标推断节点集。
\item[\(\nabla\)] 空间梯度算子,用于扩散/平流约束。
\item[\(\mathcal{L}\)] 总损失函数,由监督损失与物理约束损失组成。
\item[\(\mathcal{L}_{\mathrm{Pred}}, \mathcal{L}_{\mathrm{DIC}}\)] 预测损失与领域一致性约束损失(第\ref{chap:prediction}章)。
\item[\(\mathcal{L}_{\mathrm{infer}}, \mathcal{L}_{\mathrm{init}}, \mathcal{L}_{\mathrm{AOD}}\)] 推断损失、初始场约束损失与AOD梯度约束损失(第\ref{chap:inference}章)。

% 4. 缩略语
\item[AOD] Aerosol Optical Depth (气溶胶光学厚度)。
\item[CTM] Chemical Transport Model (化学传输模式),如 CMAQ、WRF-Chem。
\item[MEIC] Multi-resolution Emission Inventory for China (中国多尺度排放清单)。
\item[DPEC] Dynamic Projection model for Emissions in China (中国未来排放动态评估模型)。
\item[ERA5 / GFS] ECMWF第五代再分析数据 / 美国全球预报系统。
\item[GNN / GCN] Graph Neural Network / Graph Convolutional Network (图神经网络/图卷积网络)。
\item[GRU / LSTM] Gated Recurrent Unit / Long Short-Term Memory (门控循环单元/长短期记忆网络)。
\item[PCDCNet] Physical-Chemical Dynamics and Constraints Network,本文提出的预测模型。
\item[DIC] Domain-Informed Constraints (领域一致性约束),基于质量守恒设计。
\item[SPIN] Spatiotemporal Physics-Guided Inference Network,本文提出的推断模型。
\item[IGNN] Integrated Graph Neural Network,本文提出的情景模拟模型。
\item[KnowAir] 本文构建的空气质量智能预报系统。

\end{denotation}
