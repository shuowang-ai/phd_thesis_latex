% ============================================================
% 第三章 物理约束时空图神经网络的大气污染预测
% 基于复杂系统数据驱动建模的大气污染研究
% ============================================================

\chapter{物理约束时空图神经网络的大气污染预测}
\label{chap:prediction}

空气质量预报对于政府应急决策、企业生产调度和公众健康防护具有重要意义。本章基于第\ref{chap:methodology}章所建立的物理约束时空图神经网络建模范式,针对多站点多污染物协同预测问题,提出PM$_{2.5}$-GNN与PCDCNet两个递进式模型,实现高精度的72小时\PM 与\ozone 联合预报。

% ------------------------------------------------------------
% 3.1 引言
% ------------------------------------------------------------
\section{引言}
\label{sec:pred_intro}

空气质量预报(Air Quality Forecasting, AQF)旨在预判未来24至72小时的污染物浓度变化,为重污染天气预警、应急响应决策以及公众健康防护提供科学依据。正如第\ref{chap:introduction}章所述,大气污染作为一个复杂系统演化过程(详见第\ref{subsec:complex_system}节),其动态行为受排放源、气象条件以及物理化学过程的多重交互影响,呈现显著的非线性特征和时空异质性。大气污染时序预测问题可表述为:给定历史空气质量观测序列、气象场数据和排放清单信息,预测未来多时步的污染物浓度。

\textbf{核心挑战与现有方法局限}。实现高精度的空气质量预报面临以下三方面挑战:

(1)\textbf{物理一致性保障}。纯数据驱动模型可能产生违背质量守恒等物理定律的预测结果,尤其在长时序预测和极端污染事件中表现不稳定。现有方法中,数值化学传输模式(如CMAQ、WRF-Chem)虽具完备的物理化学方程组,但高昂的计算代价制约了其实时应用;早期深度学习方法(如LSTM、CNN)虽计算高效,却将物理过程视为\cqt{黑盒},缺乏显式物理约束。如何将公式(\ref{eq:advection_diffusion})中的物理约束有效嵌入深度学习框架,是保障预测结果物理合理性的关键。

(2)\textbf{风场驱动的跨区域传输建模}。污染物的跨区域迁移受风场主导,呈现强烈的方向性和时变性。然而,现有时空图神经网络多采用基于欧氏距离的静态图结构,未能有效融合风场信息,无法刻画污染物\cqt{上风向影响下风向}的定向传输规律。如何将气象数据中的风场信息有效融入图神经网络的边特征构建,实现物理驱动的有向图结构动态更新,是多源数据融合的核心问题。

(3)\textbf{多污染物协同预测}。\PM 与\ozone 共享NOx和VOCs等前体物,二者在光化学过程中存在复杂的非线性耦合关系——在特定条件下呈现\cqt{跷跷板效应}。然而,多数现有模型仅针对\PM 单一污染物,未能建模\PM 与\ozone 之间的光化学耦合关系。如何在统一的网络架构内同时建模一次污染物与二次污染物的生成与消耗过程,考验模型的泛化与表征能力。

针对上述挑战,本章提出两个递进式模型:PM$_{2.5}$-GNN首次将风场信息融入图神经网络的边特征构建,实现对定向传输的显式建模;PCDCNet在此基础上进一步引入物理化学动力学约束,通过LID--STD--TAD模块化架构实现过程解耦与联合建模。


% ------------------------------------------------------------
% 3.2 问题定义
% ------------------------------------------------------------
\section{问题定义}
\label{sec:pred_problem}

本章聚焦的核心科学问题是:\textbf{基于历史空气质量观测、气象条件和排放数据,预测未来$T$个时间步的多污染物浓度分布}。图\ref{fig:prediction_problem}展示了该问题的整体框架。

\begin{figure}[htbp]
    \centering
    \includegraphics[width=\textwidth]{figures/fig_chap01_Q1_prediction.pdf}
    \caption{大气污染时序预测问题示意图}
    \caption*{\textbf{输入}:历史空气质量观测$\mathbf{X}^{-T'+1:0}$(浅蓝色背景)、气象场$\mathbf{M}^{-T'+1:T}$、排放数据$\mathbf{E}^{-T'+1:T}$;\textbf{输出}:未来$T$个时间步的\PM 与\ozone 浓度预测$\hat{\mathbf{X}}^{1:T}$(浅粉色背景);\textbf{模型}:基于空间距离构建图结构$\mathcal{G}$,通过PM$_{2.5}$-GNN/PCDCNet建模节点间的时空相关性。}
    \label{fig:prediction_problem}
\end{figure}

如图所示,大气污染时序预测问题的输入-输出结构如下:\textbf{输入}包括三类数据——历史空气质量观测序列$\mathbf{X}^{-T'+1:0}$(图中左侧浅蓝色背景区域,从$t=-T'+1$至$t=0$共$T'$个时间步)、气象场数据$\mathbf{M}$(包括风速风向、温度、湿度、边界层高度等)以及排放清单$\mathbf{E}$;\textbf{输出}为未来$T$个时间步的\PM 与\ozone 浓度预测序列$\hat{\mathbf{X}}^{1:T}$(图中右侧浅粉色背景区域,从$t=1$至$t=T$共$T$个时间步)。图中蓝色圆点代表监测站点/城市节点,节点间的连线表示基于空间距离构建的图结构。

\textbf{时间坐标定义。}在空气质量预报问题中,我们将\textbf{起报时刻}(Forecast Initialization Time)定义为$t=0$,它是最后一个可获取空气质量观测数据的时间戳。以此为参照,$t \leq 0$表示历史时刻($t=0, -1, -2, \ldots, -T'+1$共$T'$个时间步),$t>0$表示未来时刻($t=1, 2, \ldots, T$共$T$个待预测时间步)。\textbf{历史窗口}$T'$用于捕捉污染物浓度的时序演化规律,\textbf{预测窗口}$T$决定了预报的时间范围(Lead Time),在此期间模型需要未来的气象预报和排放数据作为驱动输入。时间分辨率与$t$的具体对应关系详见第\ref{subsec:knowair_dataset}节的数据集描述。

形式化地,设研究区域包含$N$个城市或监测站点,构成图结构$\mathcal{G} = (\mathcal{V}, \mathcal{E})$,其中$\mathcal{V}$为节点集合($|\mathcal{V}|=N$),$\mathcal{E}$为边集合。预测问题可形式化表述为:

\begin{equation}
\hat{\mathbf{X}}^{1:T} = \mathcal{F}_\Theta\left(\mathbf{X}^{-T'+1:0}, \mathbf{M}^{-T'+1:T}, \mathbf{E}^{-T'+1:T}, \mathcal{G} \right)
\label{eq:pred_problem}
\end{equation}

\noindent 式中,$\mathbf{X} \in \mathbb{R}^{N \times D_X}$表示空气质量观测($D_X$为污染物种类数,本文主要关注\PM 和\ozone),$\mathbf{M} \in \mathbb{R}^{N \times D_M}$表示气象变量(包括风速、风向、温度、湿度、边界层高度等),$\mathbf{E} \in \mathbb{R}^{N \times D_E}$表示排放数据(包括\NOx、VOC、SO$_2$等),$T'$为历史窗口长度,$\Theta$为模型参数。

有别于传统的自回归预测(仅依赖历史浓度),本文的问题设定显式地将\textbf{未来气象预报}和\textbf{未来排放}作为输入变量纳入模型框架。这种设计思路与数值模式(如CMAQ)的输入范式保持一致,使得模型能够对排放变化作出响应,从而支撑后续的情景模拟应用。如图\ref{fig:input_output_paradigm}所示,这种范式的核心优势在于:模型不仅能够学习历史演化模式,更能充分利用未来气象预报信息进行物理驱动的前向预测,从而实现真正意义上的\cqt{预测}而非简单的\cqt{外推}。

\begin{figure}[htbp]
  \centering
  \includegraphics[width=0.9\linewidth]{figures/iteration_framework.pdf}
  \caption{不同预测模型的输入输出范式对比}
  \caption*{XGBoost等机器学习方法对每个时间步独立预测,缺乏时序建模能力;传统自回归模型(如LSTM)仅依赖历史观测进行外推;Transformer模型虽能捕捉长程历史依赖,但仍缺乏未来信息;而本文PCDCNet通过滑动窗口融合历史观测、未来气象预报和排放数据,对齐数值模式CMAQ的代理模型范式,实现气象驱动的前向预测。}
  \label{fig:input_output_paradigm}
\end{figure}


% ------------------------------------------------------------
% 3.3 数据与预处理
% ------------------------------------------------------------
\section{数据与预处理}
\label{sec:pred_data}

本节系统介绍用于模型训练与验证的多源数据集,涵盖空气质量观测、气象再分析/预报数据、排放清单以及图结构构建方法。这些数据共同构成了本文\cqt{物理约束数据驱动建模}范式的数据基础。

\subsection{空气质量数据($\mathbf{X}$)}
\label{subsec:aq_data}

空气质量数据(记为$\mathbf{X} \in \mathbb{R}^{N \times D_X}$,$D_X$为污染物种类数)来源于中国国家环境监测总站(China National Environmental Monitoring Centre, CNEMC)发布的城市空气质量实时监测数据\footnote{\url{https://www.cnemc.cn/}}。研究区域覆盖两个重点区域:

\textbf{京津冀及周边地区(Beijing-Tianjin-Hebei and Surrounding Areas, BTHSA)}:涵盖\cqt{2+26}城市群\footnote{\cqt{2+26}城市群是指京津冀大气污染传输通道城市,其中\cqt{2}指北京、天津两个直辖市,\cqt{26}指河北省8市(石家庄、唐山、廊坊、保定、沧州、衡水、邢台、邯郸)、山西省4市(太原、阳泉、长治、晋城)、山东省7市(济南、淄博、济宁、德州、聊城、滨州、菏泽)、河南省7市(郑州、开封、安阳、鹤壁、新乡、焦作、濮阳),共28个城市。详见生态环境部官方网站:\url{https://www.mee.gov.cn/}。},共152个国控监测站点\citep{wang2025physics}。该区域属温带季风气候,冬季受西北气流控制、静稳天气频发导致污染物累积,夏季高温高湿有利于二次气溶胶生成。作为我国大气污染防治的核心区域,工业布局密集、排放强度高,秋冬季重污染频发,是检验模型极端情景预测能力的理想场所。

\textbf{长江三角洲地区(Yangtze River Delta, YRD)}:涵盖上海、江苏、浙江等省市的主要城市,共约175个国控监测站点。该区域属亚热带季风气候,夏季高温多雨、太阳辐射强烈,有利于光化学反应生成\ozone;梅雨季节湿度大、边界层低,易形成区域性污染。经济发展水平高、城镇化程度深,\ozone 污染问题日益突出,是检验模型多污染物协同预测能力的典型区域。

时间范围为2015年1月至2023年12月,时间分辨率为1小时。监测指标包括\PM、PM$_{10}$、\ozone、NO$_2$、SO$_2$、CO等六种常规污染物,本文主要关注\PM 和\ozone 两种污染物。数据经过严格的质量控制处理,具体包括:(1)异常值剔除——将超过物理合理范围(如\PM $>$ 1000 $\mu$g/m$^3$)的值标记为缺失;(2)缺失值插补——对于短期缺失($<$6小时)采用线性插值,长期缺失($\geq$6小时)采用同一站点历史同期均值填补。


\subsection{气象数据($\mathbf{M}$)}
\label{subsec:meteo_data}

气象驱动数据(记为$\mathbf{M} \in \mathbb{R}^{N \times D_M}$,$D_M=9$)来源于两个互补的数据集,分别服务于离线训练和在线部署:

\textbf{离线训练阶段}采用欧洲中期天气预报中心(European Centre for Medium-Range Weather Forecasts, ECMWF)发布的ERA5再分析数据集\citep{hersbach2020era5}。ERA5是目前全球质量最高的大气再分析产品之一,融合了多源观测与数值模式,提供了物理一致的历史气象场。选取的变量包括:2米气温(t2m)、2米露点温度(d2m)、相对湿度(r)、地表气压(sp)、行星边界层高度(blh)、总降水量(tp)、100米风速U/V分量(u100, v100)、平均地表向下短波辐射通量(msdwswrf)、K指数等。空间分辨率为0.25°×0.25°,时间分辨率为1小时。

\textbf{在线部署阶段}采用全球预报系统(Global Forecast System, GFS)的业务预报产品\footnote{\url{https://www.ncdc.noaa.gov/data-access/model-data/model-datasets/global-forcast-system-gfs}}。GFS每日更新4次,提供未来15天的气象预报,具有较高的时效性,适合支撑实时预报业务运行。

表\ref{tab:pm25gnn_node_attributes}列出了用于建模的主要气象变量及其物理意义。这些变量涵盖了影响污染物演化的关键气象因子:边界层高度决定垂直混合能力,风场决定平流传输方向与强度,温度和辐射影响光化学反应速率,降水通过湿沉降去除污染物。

\begin{table}[htbp]
    \centering
    \caption{节点属性中包含的气象变量($\mathbf{M}$)}
    \caption*{列出了气象特征向量$\mathbf{M} \in \mathbb{R}^{N \times D_M}$中包含的$D_M=9$个变量及其物理意义,涵盖边界层、风场、温湿度、降水、辐射等影响污染物演化的关键因子。}
    \label{tab:pm25gnn_node_attributes}
    \begin{tabular}{@{}lllp{5.5cm}@{}}
        \toprule
        \textbf{变量名称} & \textbf{符号} & \textbf{单位} & \textbf{物理意义} \\
        \midrule
        行星边界层高度 & $M_{\text{blh}}$ & m & 决定污染物垂直稀释能力,与\PM 呈负相关\citep{su2018relationships} \\
        K指数 & $M_{\text{K}}$ & K & 表征大气层结稳定度,高K指数利于污染扩散 \\
        U方向风分量 & $M_{u}$ & m/s & 控制东西向平流传输 \\
        V方向风分量 & $M_{v}$ & m/s & 控制南北向平流传输 \\
        2米气温 & $M_{\text{t2m}}$ & K & 影响化学反应速率和边界层发展 \\
        相对湿度 & $M_{\text{rh}}$ & \% & 高湿度促进二次气溶胶生成\citep{tai2010correlations} \\
        总降水量 & $M_{\text{tp}}$ & m & 湿沉降去除污染物\citep{pye2009effect} \\
        地表气压 & $M_{\text{sp}}$ & Pa & 与大气稳定度相关\citep{li2017characteristics} \\
        短波辐射 & $M_{\text{rad}}$ & W/m$^2$ & 驱动光化学反应生成\ozone \\
        \bottomrule
    \end{tabular}
\end{table}


\subsection{排放清单数据($\mathbf{E}$)}
\label{subsec:emis_data}

排放数据(记为$\mathbf{E} \in \mathbb{R}^{N \times D_E}$,$D_E=6$)是区分本文方法与传统自回归预测模型的关键输入。传统深度学习预测模型通常仅依赖历史浓度和气象数据,缺乏对排放源的显式表征,因此无法响应排放变化或支持情景模拟。本文通过融合多源排放清单,将排放数据作为动态驱动变量纳入预测框架。

\textbf{MEIC清单}:由清华大学开发的中国多尺度排放清单(Multi-resolution Emission Inventory for China)\citep{zheng2018trends,li2017anthropogenic}\footnote{\url{http://meicmodel.org.cn/}},提供月尺度的行业分部门排放数据。覆盖的污染物种类包括:\NOx(氮氧化物)、VOC(挥发性有机物)、SO$_2$(二氧化硫)、NH$_3$(氨)、PM$_{2.5}$一次排放、PM$_{10}$一次排放等。覆盖的排放部门包括:电力、工业、交通、居民、农业等。空间分辨率为0.25°×0.25°。

\textbf{时间分配处理}:MEIC清单提供的是月均排放量,而空气质量预测需要小时尺度的排放输入。参照\cite{inventory}的方法,采用时间分配因子将月均排放降尺度至小时分辨率。时间分配因子综合考虑了以下周期性变化:(1)\textbf{日变化}——交通排放呈现早晚高峰,工业排放相对平稳,居民排放集中在早晚;(2)\textbf{周变化}——周末与工作日的排放强度差异;(3)\textbf{季节变化}——采暖季与非采暖季的排放差异。

图\ref{fig:temporal_nox}展示了北京和上海典型监测站点的\NOx 排放日变化特征。可以看到,城区站点(1001A、1147A)的\NOx 排放呈现显著的早晚高峰,对应交通排放特征;郊区站点(1010A、3272A)的排放水平相对较低,日变化幅度也较小。这种时空异质性的准确刻画对于模型学习排放-浓度响应关系至关重要。

\begin{figure}[htbp]
  \centering
  \includegraphics[width=\linewidth]{figures/temporal_nox.png}
  \caption{北京和上海典型监测站点的\NOx 排放日变化特征}
  \caption*{城区站点(1001A北京城区、1147A上海城区)呈现显著的早晚交通高峰;郊区站点(1010A北京郊区、3272A上海郊区)排放水平相对较低。数据已转换为UTC+0时间。}
  \label{fig:temporal_nox}
\end{figure}

图\ref{fig:spatial_emissions}展示了研究区域主要污染物排放的空间分布特征。可以观察到:(1)工业\NOx 排放集中在钢铁、化工等重工业聚集区;(2)交通排放沿高速公路网络和城市群分布,与人口密度高度相关;(3)电力部门排放集中在火电厂周边。这种空间异质性是污染物时空分布差异的重要驱动因素。

\begin{figure}[htbp]
  \centering
  \includegraphics[width=\linewidth]{figures/spatial_emissions.pdf}
  \caption{研究区域主要污染物排放的空间分布}
  \caption*{从左至右依次为:(a)工业部门\NOx 排放;(b)交通部门\PM 排放;(c)交通部门\NOx 排放;(d)电力部门\PM 排放。高排放区域主要集中在京津冀和长三角的城市密集区与工业聚集区。}
  \label{fig:spatial_emissions}
\end{figure}

表\ref{tab:emission_variables}总结了本文使用的排放变量。这些变量覆盖了\PM 和\ozone 的主要前驱物,为模型学习\cqt{排放$\to$传输$\to$化学转化$\to$浓度}的完整响应链提供了数据基础。

\begin{table}[htbp]
    \centering
    \caption{排放清单变量($\mathbf{E}$)及其与污染物的关系}
    \caption*{排放特征向量$\mathbf{E} \in \mathbb{R}^{N \times D_E}$包含$D_E=6$种排放物种,涵盖\PM 和\ozone 的主要前驱物。}
    \label{tab:emission_variables}
    \begin{tabular}{@{}lllp{5cm}@{}}
        \toprule
        \textbf{排放物种} & \textbf{符号} & \textbf{单位} & \textbf{与目标污染物的关系} \\
        \midrule
        PM$_{2.5}$一次排放 & $E_{\text{PM}_{2.5}}$ & ton & 直接贡献于\PM 浓度 \\
        PM$_{10}$一次排放 & $E_{\text{PM}_{10}}$ & ton & 直接贡献于PM$_{10}$浓度,部分转化为\PM \\
        \NOx & $E_{\text{NO}_x}$ & ton & \ozone 和硝酸盐(\PM 组分)的前驱物 \\
        VOC & $E_{\text{VOC}}$ & ton & \ozone 和二次有机气溶胶(\PM 组分)的前驱物 \\
        SO$_2$ & $E_{\text{SO}_2}$ & ton & 硫酸盐(\PM 组分)的前驱物 \\
        NH$_3$ & $E_{\text{NH}_3}$ & ton & 铵盐(\PM 组分)的前驱物 \\
        \bottomrule
    \end{tabular}
\end{table}


\subsection{图结构构建($\mathcal{G}$)}
\label{subsec:graph_construction}

基于城市站点的地理位置和地形信息构建图结构$\mathcal{G} = (\mathcal{V}, \mathcal{E})$,其中$\mathcal{V}$为节点集合($|\mathcal{V}|=N$个城市),$\mathcal{E}$为边集合。有别于传统方法仅基于距离构建图的做法,本文综合考虑了污染物传输的物理约束。

\textbf{节点定义}:选取地级市作为图的基本单元。每个城市的位置和污染物浓度由该城市所有监测站点的平均值表示。这种聚合方式聚焦于城市间的长程传输关系,符合研究目标定位。

\textbf{边建立规则}:两城市$i$和$j$之间建立边,当且仅当满足以下两个条件:

\begin{equation}
\begin{aligned}
  A_{ij} &= H(d_\theta - d_{ij}) \cdot H(m_\theta - m_{ij}), \quad \text{其中} \\
  d_{ij} &= ||\rho_i-\rho_j||_2 \\
  m_{ij} &= \sup_{\lambda \in(0,1)} \left\{h(\lambda \rho_i  + (1-\lambda)\rho_j)- \max\left\{h(\rho_i),h(\rho_j)\right\}\right\}
  \label{eq:physical_constraints}
 \end{aligned}
\end{equation}

\noindent 式中$\rho_i$为节点$i$的地理坐标(经度、纬度),$h(\rho)$为位置$\rho$的海拔高度,$||\cdot||_2$为欧氏距离,$H(\cdot)$为Heaviside阶跃函数。$d_\theta=300$km为距离阈值,$m_\theta=1200$m为海拔阈值。

上述规则的物理含义是:(1)\textbf{距离约束}——污染物的平流传输受限于风速和时间尺度,300km大致对应72小时内约4m/s风速的传输距离;(2)\textbf{地形约束}——高山对污染物传输形成物理屏障,当两城市间存在高于两地海拔1200m以上的山脉时,传输通道被阻断。

图\ref{fig:multi-image}展示了BTHSA和YRD两个区域的城市网络图结构。可以看到,太行山脉有效阻隔了山西与河北之间的部分传输通道,这与实际的污染传输规律相吻合。

\begin{figure}[htbp]
  \centering
  \subcaptionbox{京津冀及周边地区(BTHSA)城市网络\label{fig:graph-bthsa}}
    {\includegraphics[width=0.45\linewidth]{figures/map_graph_BTHSA.png}}
  \subcaptionbox{长江三角洲地区(YRD)城市网络\label{fig:graph-yrd}}
    {\includegraphics[width=0.45\linewidth]{figures/map_graph_YRD.png}}
  \caption{研究区域的城市网络图结构}
  \caption*{节点代表城市,边代表城市间的潜在污染传输通道。边的建立综合考虑了地理距离($<$300km)和地形阻隔因素(山脉海拔差$<$1200m)。}
  \label{fig:multi-image}
\end{figure}


\subsection{KnowAir-DS数据集}
\label{subsec:knowair_dataset}

基于上述多源数据,我们构建并公开发布了\textbf{KnowAir-DS-V1}数据集\citep{wang2020pm2}\footnote{\url{https://github.com/shuowang-ai/PM2.5-GNN}}及其扩展版本\textbf{KnowAir-DS-V2}\footnote{\url{https://zenodo.org/records/15614907}},为大气污染预测研究提供标准化的数据基础设施。

\textbf{KnowAir-DS-V1}(2015-2018):研究范围为$103^{\circ}\text{E}$--$122^{\circ}\text{E}$、$28^{\circ}\text{N}$--$42^{\circ}\text{N}$,覆盖中国中东部184个城市(图\ref{fig:knowair_v1_study_area}),包含\PM 浓度和8种气象变量,时间分辨率为3小时,连边阈值为300km,主要用于评估PM$_{2.5}$-GNN的单污染物预测能力。以3小时分辨率为例,$t=1$对应起报后3小时的第一帧预测,$t=24$对应起报后72小时的最后一帧预测。

\begin{figure}[htbp]
  \centering
  \includegraphics[width=0.7\linewidth]{figures/knowair_ds_v1_study_area.jpg}
  \caption{KnowAir-DS-V1数据集研究区域}
  \caption*{研究范围为$103^{\circ}\text{E}$--$122^{\circ}\text{E}$、$28^{\circ}\text{N}$--$42^{\circ}\text{N}$,覆盖中国中东部184个城市,节点为地级市,连边阈值为300km,时间分辨率为3小时。该数据集用于PM$_{2.5}$-GNN模型的训练与评估。}
  \label{fig:knowair_v1_study_area}
\end{figure}

\textbf{KnowAir-DS-V2}(2016-2023):覆盖BTHSA(152个站点)和YRD(175个站点)共327个监测站点,包含2种空气质量变量(\PM、\ozone)、8种气象变量和6种排放变量,时间分辨率为1小时,连边阈值为200km,总计70,128小时的观测记录。数据集按时间划分为:训练集(2016-2019)、验证集(2020-2021)、测试集(2022-2023)。

\textbf{多尺度建模的统一范式}。值得注意的是,KnowAir-DS-V1与V2在空间尺度上存在显著差异:V1以城市为节点、采用较大的连边阈值(300km)和较粗的时间分辨率(3小时),适用于跨省域的大尺度传输建模;V2以监测站点为节点、采用较小的连边阈值(200km)和更精细的时间分辨率(1小时),适用于城市群内部的精细化预测。尽管空间尺度和时间粒度不同,但二者背后的建模思想是统一的——均基于第\ref{chap:methodology}章提出的物理约束时空图神经网络范式,核心组件(如平流系数、消息传递机制、物理一致性约束)在不同尺度下具有良好的可迁移性。这种多尺度适应性为模型在不同应用场景下的灵活部署提供了理论和实践基础。

与现有数据集(如KDD CUP Fresh Air Dataset\footnote{\url{https://www.kdd.org/kdd2018/kdd-cup}})相比,KnowAir-DS系列数据集具有以下优势:(1)空间覆盖范围更广,支持跨区域传输研究;(2)时间跨度更长,支持跨年度泛化评估;(3)融合了排放清单数据,支持排放响应建模;(4)提供了标准化的图结构信息,便于研究复现。


% ------------------------------------------------------------
% 3.4 方法一:基于风场驱动图网络的PM2.5传输建模(PM2.5-GNN)
% ------------------------------------------------------------
\section{PM$_{2.5}$-GNN:风场驱动的图网络模型}
\label{sec:pm25gnn}

细颗粒物(\PM)具有显著的跨区域传输特性。研究表明,在季风气候条件下,污染物可在72小时内传输数百公里\citep{pongkiatkul2007assessment,wang2015long,hao2019transport}。图\ref{fig:pm25_charateristics}展示了\PM 演化过程的典型特征:污染物沿风向从上游城市传输至下游城市(蓝色箭头),同时在局地发生扩散累积(虚线)。这种\cqt{传输-扩散}的双重动力学特性,要求预测模型必须具备捕捉长距离、有向时空依赖的能力。

\begin{figure}[htbp]
  \centering
  \includegraphics[width=0.6\linewidth]{figures/pm25_charateristics.png}
  \caption{\PM 演化过程的特征示意图}
  \caption*{颜色表示浓度高低;蓝色箭头表示风驱动的跨区域传输;虚线表示局地扩散累积;山脉构成传输屏障。预测模型需要同时捕捉这些过程。}
  \label{fig:pm25_charateristics}
\end{figure}

现有的时空图神经网络(如GC-LSTM\citep{qi2019hybrid})通常基于地理距离构建无向图,假设节点间的影响是相互对称的。然而,大气中的\textbf{平流输送(Advection)}具有强烈的方向性:上游城市对下游城市的影响显著,反之则不然。针对这一问题,我们提出了PM$_{2.5}$-GNN\citep{wang2020pm2}模型,通过物理启发式的图结构设计显式编码风场信息,实现对\PM 传输过程的精确建模。


\subsection{图结构设计}
\label{subsec:pm25gnn_graph}

为了显式建模风场驱动的传输过程,我们设计了包含\textbf{节点属性}、\textbf{边属性}和\textbf{有向邻接矩阵}的动态知识图谱。

\textbf{(1)节点属性。}除了历史\PM 浓度外,我们将边界层高度(blh)、相对湿度、风速等气象因子作为节点特征。这些变量与\PM 的局地累积和垂直扩散密切相关。节点$i$在时刻$t$的特征向量定义为:
\begin{equation}
    \boldsymbol{\xi}_i^t = [\hat{\mathbf{X}}_i^{t-1}, \mathbf{M}_i^t]
\end{equation}
\noindent 式中$\hat{\mathbf{X}}_i^{t-1}$为上一时刻的\PM 浓度预测值(或观测值),$\mathbf{M}_i^t$为当前时刻的气象特征向量。

\textbf{(2)边属性与平流系数。}对于边特征,我们创新性地引入了\textbf{平流系数(Advection Coefficient)},将风场对污染传输的影响显式编码到图结构中。如图\ref{fig:advection_coeff}所示,设源节点$j$指向目标节点$i$,源节点的风速为$|v|$,两节点间距离为$d$,源节点风向$\beta$与两节点连线方向$\gamma$的夹角为$\alpha = |\gamma - \beta|$,则平流系数$S$定义为:

\begin{equation}
    S_{j \to i} = \text{ReLU}\left(\frac{|v|}{d} \cos(\alpha)\right)
\label{eq:advection_coeff}
\end{equation}

\begin{figure}[htbp]
  \centering
  \includegraphics[width=0.4\linewidth]{figures/diffusion.png}
  \caption{平流系数的计算示意图}
  \caption*{源节点$j$的风速为$|v|$,风向为$\beta$;$\gamma$为$j$指向$i$的连线方向;$\alpha$为二者夹角;$d$为两节点距离。当风从$j$吹向$i$时($\cos\alpha > 0$),平流系数为正。}
  \label{fig:advection_coeff}
\end{figure}

利用ReLU函数,我们确保了仅当风从$j$吹向$i$时(即$\cos\alpha > 0$),边权重才为正,从而在物理上强制了传输的方向性约束。这一设计使得模型能够理解\cqt{上风向影响下风向}的定向传输规律。公式的物理含义是:传输强度与源节点风速成正比,与传输距离成反比,且仅在风向与连线方向一致时有效。

表\ref{tab:pm25gnn_edge_attributes}列出了边属性中包含的所有变量。除平流系数外,还包括原始的风速、距离、角度信息,为模型提供了冗余特征以增强学习能力。

\begin{table}[htbp]
    \centering
    \caption{边属性($\mathbf{A}$)中包含的变量}
    \caption*{边属性向量$\mathbf{A}_{j \to i} \in \mathbb{R}^{5}$包含5个特征,用于刻画从源节点$j$到目标节点$i$的传输特性。}
    \label{tab:pm25gnn_edge_attributes}
    \begin{tabular}{@{}lllp{4.5cm}@{}}
        \toprule
        \textbf{变量名称} & \textbf{符号} & \textbf{单位} & \textbf{说明} \\
        \midrule
        源节点风速 & $A_{|v|}$ & km/h & 驱动传输的动力 \\
        节点间距离 & $A_{d}$ & km & 传输的空间尺度 \\
        源节点风向 & $A_{\beta}$ & ($^{\circ}$) & 气象学风向定义 \\
        连线方向 & $A_{\gamma}$ & ($^{\circ}$) & 从源节点指向目标节点 \\
        平流系数 & $A_{S}$ & - & 公式\eqref{eq:advection_coeff}计算 \\
        \bottomrule
    \end{tabular}
\end{table}


\subsection{模型架构}
\label{subsec:pm25gnn_model}

在构建的有向图上,我们采用\textbf{消息传递神经网络(Message Passing Neural Network, MPNN)}框架\citep{gilmer2017neural,battaglia2018relational}来模拟污染流的输送过程。模型架构如图\ref{fig:pm25gnn_model}所示,由知识增强的图神经网络(GNN)和时空门控循环单元(GRU)两个核心组件构成。

\begin{figure}[htbp]
  \centering
  \includegraphics[width=0.8\linewidth]{figures/pm25gnn_model.png}
  \caption{PM$_{2.5}$-GNN模型架构示意图}
  \caption*{模型由三个核心组件构成:(1)知识增强的消息传递模块,利用风驱边权重计算邻居节点的污染传输通量;(2)时空GRU单元,融合空间聚合信息与历史隐状态;(3)输出层,预测下一时刻的\PM 浓度。橙色箭头表示输入流(import),蓝色箭头表示输出流(export)。}
  \label{fig:pm25gnn_model}
\end{figure}

节点$i$在时刻$t$的状态更新过程包含三个步骤:

\textbf{步骤1:消息生成。}根据源节点$j$和目标节点$i$的状态,以及有向边属性$\mathbf{A}_{j \to i}^t$,计算从$j$到$i$的传输消息:
\begin{equation}
    e_{j \to i}^t = \Psi([\boldsymbol{\xi}_j^t, \boldsymbol{\xi}_i^t, \mathbf{A}_{j \to i}^t])
\end{equation}
\noindent 式中$\Psi$为可学习的消息函数,由两层MLP实现。边属性$\mathbf{A}_{j \to i}^t$包含了平流系数等物理信息,使消息生成过程具有物理意义。

\textbf{步骤2:净通量聚合。}与传统GNN仅聚合输入消息不同,我们计算每个节点的\textbf{净通量}——即输入流与输出流之差:
\begin{equation}
    \zeta_i^t = \Phi\left(\sum_{j \in \mathcal{N}(i)} (e_{j \to i}^t - e_{i \to j}^t)\right)
\label{eq:message_agg}
\end{equation}
\noindent 式中$e_{j \to i}^t$代表从邻居$j$流入节点$i$的污染物(输入流),$e_{i \to j}^t$代表从节点$i$流出到邻居$j$的污染物(输出流),二者之差表示该节点的净通量积累。$\Phi$为单层MLP聚合函数。

这种\cqt{输入减输出}的设计具有明确的物理意义:根据质量守恒定律,节点$i$的浓度变化应等于净输入通量。通过显式建模输入与输出的差值,模型能够更准确地刻画污染物的\cqt{收支平衡}。

\textbf{步骤3:时空状态更新。}将聚合的空间信息$\zeta_i^t$与节点特征$\boldsymbol{\xi}_i^t$拼接后,输入到GRU单元中,与上一时刻的隐状态融合,完成时空状态的更新:
\begin{equation}
\begin{aligned}
x_i^t &= [\boldsymbol{\xi}_i^t, \zeta_i^t] \\
z_i^t &= \sigma(W_z \cdot [h_i^{t-1}, x_i^t]) \\
r_i^t &= \sigma(W_r \cdot [h_i^{t-1}, x_i^t]) \\
\tilde{h}_i^t &= \tanh(W \cdot [r_i^t \odot h_i^{t-1}, x_i^t]) \\
h_i^t &= (1 - z_i^t) \odot h_i^{t-1} + z_i^t \odot \tilde{h}_i^t
\end{aligned}
\label{eq:grucell}
\end{equation}
\noindent 式中$W_z$、$W_r$、$W$为可学习参数,$\sigma$为sigmoid激活函数,$\odot$为逐元素乘法。GRU的门控机制使模型能够自适应地选择保留历史信息还是更新新信息,有效捕捉污染物的长时序累积与衰减过程。

最后,通过输出层预测\PM 浓度:
\begin{equation}
    \hat{\mathbf{X}}_i^t = \Omega(h_i^t)
\end{equation}
\noindent 式中$\Omega$为单层MLP。

算法\ref{alg:pm25_gnn}给出了PM$_{2.5}$-GNN的完整训练与推理流程。

\begin{algorithm}[htbp]
\caption{PM$_{2.5}$-GNN的训练与推理流程}
\caption*{算法包含初始化、自回归预测循环、MSE损失计算和参数更新四个阶段。预测循环中依次进行节点特征构建、净通量聚合、GRU状态更新和浓度预测。}
\label{alg:pm25_gnn}
\begin{algorithmic}[1]
    \REQUIRE 初始观测值 $\mathbf{X}^{0}$,未来$T$步的节点属性 $\{\mathbf{M}^t\}_{t=1}^T$ 和边属性 $\{\mathbf{A}^t\}_{t=1}^T$,图结构 $\mathcal{G}=(\mathcal{V}, \mathcal{E})$,已初始化参数 $\Theta$,学习率 $\alpha$
    \ENSURE 优化后的参数 $\Theta$(训练),预测序列 $\{\hat{\mathbf{X}}^t\}_{t=1}^T$(推理)

    \STATE 初始化所有节点的GRU隐状态 $\{\mathbf{h}_i^0\}_{i \in \mathcal{V}} \leftarrow \mathbf{0}$
    \STATE 初始化预测序列 $\hat{\mathbf{X}}^0 \leftarrow \mathbf{X}^0$
    
    \FOR{$t \leftarrow 1$ to $T$}
        \FOR{每个节点 $i \in \mathcal{V}$}
            \STATE $\boldsymbol{\xi}_i^t \leftarrow [\hat{\mathbf{X}}_i^{t-1}, \mathbf{M}_i^t]$ \COMMENT{构建节点输入特征}
            \STATE $\zeta_i^t \leftarrow \Phi(\sum_{j\in \mathcal{N}(i)}(\Psi([\boldsymbol{\xi}_j^t, \boldsymbol{\xi}_i^t, \mathbf{A}_{j\to i}^t]) - \Psi([\boldsymbol{\xi}_i^t, \boldsymbol{\xi}_j^t, \mathbf{A}_{i\to j}^t])))$ \COMMENT{净通量聚合}
            \STATE $\mathbf{h}_i^t \leftarrow \text{GRUcell}([\boldsymbol{\xi}_i^t, \zeta_i^t], \mathbf{h}_i^{t-1})$ \COMMENT{时空状态更新}
            \STATE $\hat{\mathbf{X}}_i^t \leftarrow \Omega(\mathbf{h}_i^t)$ \COMMENT{浓度预测}
        \ENDFOR
    \ENDFOR

    \STATE $\mathcal{L} \leftarrow \frac{1}{T} \sum_{t=1}^{T}\frac{1}{N} \sum_{i=1}^{N}(\hat{\mathbf{X}}_i^t - \mathbf{X}_i^t)^{2}$ \COMMENT{MSE损失}
    \STATE $\Theta \leftarrow \Theta - \alpha \frac{\partial \mathcal{L}}{\partial \Theta}$
    \STATE \RETURN $\Theta$(训练)\textbf{或} $\{\hat{\mathbf{X}}^t\}_{t=1}^T$(推理)
\end{algorithmic}
\end{algorithm}


\subsection{局限性分析}
\label{subsec:pm25gnn_limitation}

PM$_{2.5}$-GNN在KnowAir-DS数据集上的实验表明,相比于GC-LSTM等基线模型,其在72小时\PM 长时序预测中取得了显著优势(详见\ref{subsec:pm25gnn_results}节),特别是能够准确捕捉风向下游城市的污染峰值滞后现象。然而,随着环境治理需求的不断深化,PM$_{2.5}$-GNN逐渐显露出若干关键局限:

\textbf{(1)污染物覆盖单一。}PM$_{2.5}$-GNN仅针对\PM 建模,未覆盖\ozone 等与\PM 存在复杂耦合关系的重要污染物。而在实际治理中,\PM 与\ozone 的\cqt{跷跷板效应}日益凸显——\PM 浓度下降导致气溶胶对太阳辐射的散射减弱,近地面辐射增强反而加速光化学反应、促进\ozone 生成\citep{qu2023underlying}。单一污染物的预测已无法满足协同治理的需求。

\textbf{(2)缺乏化学机制约束。}模型侧重于物理传输过程(平流、扩散),缺乏对光化学反应(如VOC和\NOx 在太阳辐射下生成\ozone)的显式物理-化学约束。这导致模型在\ozone 预测任务上表现不佳,且在极端情景下可能产生违背化学机理的预测结果。

\textbf{(3)无法响应排放变化。}模型仅利用历史浓度和气象作为输入,未纳入排放清单(Emissions)。这使得模型无法回答\cqt{如果减排50\%,空气质量会如何变化}这一关键的政策制定问题,也无法支持后续的情景模拟应用(第\ref{chap:simulation}章)。

\textbf{(4)物理约束缺失。}模型训练仅优化预测损失,未引入质量守恒等物理约束。在长时序自回归预测中,误差可能累积发散,产生非物理结果。

鉴于上述局限,我们需要构建一个更全面、更具物理可解释性的模型体系。下一节将介绍针对这些问题的解决方案——PCDCNet模型。


% ------------------------------------------------------------
% 3.5 方法二:物理化学动力学约束网络(PCDCNet)
% ------------------------------------------------------------
\section{PCDCNet:物理化学动力学约束模型}
\label{sec:pcdcnet}

为解决PM$_{2.5}$-GNN的上述局限,我们提出了\textbf{PCDCNet(Physical-Chemical Dynamics and Constraints Network)}\citep{wang2025pcdcnet}。该模型定位为数值模式(如CMAQ)的\textbf{深度学习代理模型(Surrogate Model)},在继承PM$_{2.5}$-GNN风场驱动图结构设计的基础上,进行了以下关键升级:

\begin{itemize}
    \item \textbf{多污染物联合预测}:同时预报\PM 与\ozone 浓度,并建模其协同效应;
    \item \textbf{排放响应建模}:融合排放清单作为动态输入,实现对排放变化的显式响应;
    \item \textbf{物理化学动力学建模}:设计LID-STD-TAD三模块架构,分别建模局地化学反应、空间传输和时间累积过程;
    \item \textbf{领域知识约束}:引入DIC约束,将质量守恒嵌入训练目标。
\end{itemize}


\subsection{设计理念}
\label{subsec:pcdcnet_design}

传统的AI预测模型通常采用自回归形式$\hat{\mathbf{X}}^{t+1} = f(\mathbf{X}^{1:t}, \mathbf{M}^{1:t})$,仅依赖历史观测和气象数据。这种范式存在两个根本性问题:(1)无法利用未来气象预报信息,导致预报时效受限;(2)无法响应排放变化,无法支持情景模拟。

PCDCNet旨在模拟数值模式(如CMAQ)的演化过程,其输入输出范式对齐了数值模式的设计:
\begin{equation}
    \hat{\mathbf{X}}^{t+1} = \mathcal{F}_\Theta\left(\mathbf{X}^t, \mathbf{M}^{t+1}, \mathbf{E}^{t+1} \right)
\label{eq:pcdcnet_formulation}
\end{equation}

即利用当前的污染状态$\mathbf{X}^t$、未来的气象条件$\mathbf{M}^{t+1}$和未来的排放强度$\mathbf{E}^{t+1}$来推演下一时刻的浓度。这种设计使得模型具备了对排放变化的敏感性,从而支持:

\textbf{(1)排放情景模拟}——给定不同排放路径(如减排30\%、50\%、70\%),预测空气质量的演变趋势(详见第\ref{chap:simulation}章)。

图\ref{fig:input_output_paradigm}与表\ref{tab:methods_comparison_vertical}对比了不同预测模型的输入输出范式。可以看到,PCDCNet是唯一同时融合历史观测、未来气象预报和排放数据的模型,与数值模式CMAQ的范式最为接近。

\subsection{模型总体架构}
\label{subsec:pcdcnet_architecture}

PCDCNet遵循第\ref{chap:methodology}章图\ref{fig:unified_framework}所示的\cqt{编码$\rightarrow$隐空间动力学$\rightarrow$解码}统一框架。如图\ref{fig:overall_framework}所示,模型通过三个专门设计的动力学模块来解耦大气过程,分别对应大气污染演化的三类核心物理机制:

\begin{figure}[htbp]
  \centering
  \includegraphics[width=\linewidth]{figures/overall_framework.pdf}
  \caption{PCDCNet模型总体架构}
  \caption*{模型由三个核心动力学模块构成:LID(局地交互动力学)建模本地化学反应与排放响应,STD(空间传输动力学)建模风驱平流与扩散传输,TAD(时间积累动力学)建模长时序累积与沉降效应。最终输出通过残差预测方式生成,并引入DIC约束确保物理一致性。}
  \label{fig:overall_framework}
\end{figure}

\textbf{LID(Local Interaction Dynamics)}:建模发生在局地的化学反应和排放生成过程,对应公式\eqref{eq:advection_diffusion}中的化学反应项$R$和排放源项$S$;

\textbf{STD(Spatial Transport Dynamics)}:建模污染物的空间传输过程,对应公式\eqref{eq:advection_diffusion}中的平流项和扩散项;

\textbf{TAD(Temporal Accumulation Dynamics)}:建模污染物的时间累积与沉降过程,对应公式\eqref{eq:advection_diffusion}中的沉降项$D$和浓度随时间的累积。

这种\cqt{过程解耦+联合建模}的设计使得每个模块具有明确的物理对应,既保留了深度学习的灵活拟合能力,又增强了模型的可解释性。


% 嵌入层:多源数据融合
在进入三个动力学模块之前,PCDCNet首先通过嵌入层(Embed)将多源异构数据映射到统一的隐空间。设时刻$t$的输入包括上一时刻的污染物浓度$\hat{\mathbf{X}}^{t-1}$、当前时刻的气象变量$\mathbf{M}^t$和排放数据$\mathbf{E}^t$,嵌入过程为:

\begin{equation}
    \mathbf{H}^t = \text{Linear}([\hat{\mathbf{X}}^{t-1}, \mathbf{M}^t, \mathbf{E}^t])
\end{equation}

\noindent 式中$[\cdot, \cdot, \cdot]$表示特征拼接,$\text{Linear}$为线性投影层。这一步骤将维度不同的污染物、气象、排放数据统一投影到$d$维隐空间中,为后续的动力学建模提供统一的特征表示。

在训练阶段,当$t < 1$(历史编码阶段)时,$\hat{\mathbf{X}}^{t-1}$使用真实观测值$\mathbf{X}^{t-1}$;当$t \geq 1$(预测阶段)时,使用上一时刻的预测值。这种设计使模型能够在训练时学习真实的\cqt{状态-演化}映射,在推理时进行自回归预测。


% 局地交互动力学模块(LID)
\textbf{局地交互动力学(Local Interaction Dynamics, LID)}模块主要模拟发生在网格或站点内部的\textbf{光化学反应}与\textbf{排放生成}过程。

在大气化学中,\ozone 的生成依赖于\NOx 和VOC在太阳辐射下的光化学反应链,其速率受温度、辐射强度等因素的非线性调制。\PM 的浓度则受到一次排放(直接排放的颗粒物)和二次生成(气态前驱物转化)的双重贡献。这些过程发生在每个站点局地,不涉及站点间的空间传输。

LID模块通过多层感知机(MLP)学习这种复杂的非线性响应:
\begin{equation}
    \mathbf{E}^t = \text{MLP}(\text{RMSNorm}(\mathbf{H}^t))
\end{equation}
\begin{equation}
    \mathbf{H}^t = \mathbf{H}^t + \mathbf{E}^t
\end{equation}

\noindent 式中MLP包含Linear层、RMSNorm归一化\citep{zhang2019root}、SiLU激活函数和Dropout正则化。残差连接确保梯度稳定传播。

通过将排放$\mathbf{E}$显式作为输入,LID模块能够学习:(1)\NOx 与VOC在不同气象条件(光照$rad$、温度$T$)下生成\ozone 的复杂非线性函数;(2)一次\PM 的直接排放贡献;(3)气态前驱物(SO$_2$、\NOx、NH$_3$)向二次气溶胶转化的响应关系。


% 空间传输动力学模块(STD)
\textbf{空间传输动力学(Spatial Transport Dynamics, STD)}模块继承并升级了PM$_{2.5}$-GNN的思想,利用图神经网络模拟污染物的\textbf{平流}与\textbf{扩散}传输。

基于城市网络图$\mathcal{G}$,STD模块通过图卷积操作聚合邻居节点的信息:
\begin{equation}
    \mathbf{S}^t = \text{Linear}\left(\tilde{\mathbf{L}}\mathbf{H}^t\right) = \text{Linear}\left((\mathbf{I} - \mathbf{D}^{-1/2}\mathbf{A}\mathbf{D}^{-1/2})\mathbf{H}^t\right)
\label{eq:std_gcn}
\end{equation}
\begin{equation}
    \mathbf{H}^t = \mathbf{H}^t + \mathbf{S}^t
\end{equation}

\noindent 式中$\tilde{\mathbf{L}}$为归一化图拉普拉斯矩阵,$\mathbf{A}$为邻接矩阵,$\mathbf{D}$为度矩阵。这种图卷积操作在物理上对应于大气扩散方程的空间离散化\citep{kipf2017semi,li2023improving}——拉普拉斯算子正是扩散方程中的核心算子。

与PM$_{2.5}$-GNN相比,STD模块的主要升级在于:(1)采用归一化拉普拉斯而非自定义的消息传递,提高了计算效率和数值稳定性;(2)引入了领域知识约束(DIC),确保传输过程满足质量守恒原则(详见\ref{subsec:dic}节)。


% 时间积累动力学模块(TAD)
\textbf{时间积累动力学(Temporal Accumulation Dynamics, TAD)}模块利用门控循环单元(GRU)捕捉污染物的长时序\textbf{累积}与\textbf{沉降(Deposition)}效应:
\begin{equation}
    \mathbf{Z}^t = \text{GRUcell}(\mathbf{H}^t, \mathbf{Z}^{t-1})
\end{equation}
\begin{equation}
    \mathbf{H}^t = \mathbf{H}^t + \mathbf{Z}^t
\end{equation}

TAD模块整合了LID产生的化学变化和STD产生的物理传输,通过GRU的门控机制自适应地决定信息的保留与遗忘,从而捕捉:

\textbf{(1)污染累积}:在静稳天气下,边界层高度降低,污染物难以垂直扩散,导致浓度逐日累积;

\textbf{(2)沉降衰减}:污染物通过干沉降(重力沉降、湍流扩散)和湿沉降(降水冲刷)从大气中去除;

\textbf{(3)长时序依赖}:捕捉跨越多个时间步的污染演化规律,如周期性变化(日变化、周变化)和趋势变化。


% 领域知识约束(DIC)
\label{subsec:dic}
为了防止纯数据驱动模型产生违反物理规律的预测(如质量不守恒、负浓度、非物理突变等),我们设计了\textbf{领域知识约束(Domain-Informed Constraints, DIC)}损失函数。

DIC约束基于大气污染物的\textbf{质量守恒}原则。在第\ref{chap:methodology}章公式\eqref{eq:advection_diffusion}所描述的对流-扩散方程中,污染物的时空演化满足质量守恒——即污染物不会凭空产生或消失,只能通过排放源产生、通过沉降去除或在空间中传输。

具体而言,DIC约束的核心是\textbf{空间守恒约束}:在某一时刻,通过STD模块传输的污染物净通量在全局范围内应趋近于零(不考虑化学源汇):
\begin{equation}
    \sum_{v \in \mathcal{V}} \nabla \hat{\mathbf{X}}_{\mathbf{S}}^t = 0
\label{eq:dic_spatial}
\end{equation}
\noindent 式中$\nabla \hat{\mathbf{X}}_{\mathbf{S}}^t = \text{Linear}(\mathbf{S}^t)$表示由STD模块产生的浓度变化量(传输贡献)。这一约束确保传输过程是\cqt{零和博弈}——一个节点的流入必然对应其他节点的流出。

基于此,DIC损失定义为:
\begin{equation}
    \mathcal{L}_{\mathrm{DIC}} = \frac{1}{|\mathcal{V}|} \sum_v \left| \sum_{v' \in \mathcal{N}(v)} \nabla \hat{\mathbf{X}}_{\mathbf{S}, v \to v'}^t \right|
\label{eq:dic_loss}
\end{equation}

DIC约束强制模型在进行空间传输预测时满足质量守恒定律。这种软约束方式允许模型在优化预测精度与满足物理规律之间取得平衡。


% 读出层与残差预测
PCDCNet采用\textbf{残差预测}方式生成最终输出,即预测浓度的变化量而非绝对值:
\begin{equation}
    \Delta \hat{\mathbf{X}}^t = \text{Linear}(\mathbf{H}^t)
\end{equation}
\begin{equation}
    \hat{\mathbf{X}}^t = \hat{\mathbf{X}}^{t-1} + \Delta \hat{\mathbf{X}}^t
\label{eq:residual_pred}
\end{equation}

这种设计符合大气污染演化的物理特性——浓度的变化通常是渐进的,残差形式有助于:(1)稳定长时序预测,避免绝对值预测中的漂移问题;(2)使模型聚焦于学习\cqt{变化量},与公式\eqref{eq:advection_diffusion}中$\partial C / \partial t$的物理意义对应;(3)天然保持预测的时序连续性,减少非物理突变。


\subsection{损失函数与训练策略}
\label{subsec:pcdcnet_training}

模型的总损失函数由预测损失与DIC约束损失组成:
\begin{equation}
    \mathcal{L} = \mathcal{L}_{\mathrm{Pred}} + \lambda \mathcal{L}_{\mathrm{DIC}}
\label{eq:total_loss_chap2}
\end{equation}

\noindent 式中预测损失采用L1损失(平均绝对误差):
\begin{equation}
    \mathcal{L}_{\mathrm{Pred}} = \frac{1}{NT} \sum_{n=1}^{N} \sum_{t=1}^{T} |\hat{\mathbf{X}}_n^t - \mathbf{X}_n^t|
\end{equation}

$\lambda$为约束权重超参数,用于平衡预测精度与物理一致性。实验表明(见\ref{subsec:pcdcnet_ablation}节),$\lambda$在$10^{-3}$至$10^{-2}$区间时模型性能最优。

算法\ref{alg:algorithm}给出了PCDCNet的完整训练与推理流程。

\begin{algorithm}[htbp]
\caption{PCDCNet的训练与推理流程}
\caption*{算法包含历史编码和预测两个阶段。每个时间步依次执行:嵌入层融合多源数据、LID模块处理局地化学反应、STD模块处理空间传输、TAD模块处理时间累积、残差预测生成浓度。训练时同时优化预测损失和DIC约束损失。}
\label{alg:algorithm}
\begin{algorithmic}[1]
    \REQUIRE 训练数据集 $\mathcal{D} = \{D_n\}_{n=1}^N$,初始化参数 $\Theta$,学习率 $\alpha$,DIC权重 $\lambda$
    \ENSURE 优化后的参数 $\Theta$(训练),预测序列 $\{\hat{\mathbf{X}}\}_{t=1}^T$(推理)

    \FOR{每个样本 $D_k = (\mathbf{X}, \mathbf{M}, \mathbf{E}) \in \mathcal{D}$}
        \STATE 初始化 $\mathbf{Z}^{-T'+1} \leftarrow \mathbf{0}$
        \FOR{$t \leftarrow -T'+2$ to $T$}
            \IF{$t < 1$}
                \STATE $\mathbf{H}^t \leftarrow \text{Linear}([\mathbf{X}^{t-1}, \mathbf{M}^t, \mathbf{E}^t])$ \COMMENT{历史编码阶段:使用真实观测}
            \ELSE
                \STATE $\mathbf{H}^t \leftarrow \text{Linear}([\hat{\mathbf{X}}^{t-1}, \mathbf{M}^t, \mathbf{E}^t])$ \COMMENT{预测阶段:使用预测值}
            \ENDIF
            
            \STATE $\mathbf{E}^t \leftarrow \text{MLP}(\mathbf{H}^t); \quad \mathbf{H}^t \mathrel{+}= \mathbf{E}^t$ \COMMENT{LID模块:局地化学反应}
            \STATE $\mathbf{S}^t \leftarrow \text{GraphConv}(\mathbf{H}^t, \mathcal{G}); \quad \mathbf{H}^t \mathrel{+}= \mathbf{S}^t$ \COMMENT{STD模块:空间传输}
            \STATE $\mathbf{Z}^t \leftarrow \text{GRUcell}(\mathbf{H}^t, \mathbf{Z}^{t-1}); \quad \mathbf{H}^t \mathrel{+}= \mathbf{Z}^t$ \COMMENT{TAD模块:时间累积}
            \STATE $\Delta \hat{\mathbf{X}}^{t} \leftarrow \text{Linear}(\mathbf{H}^t); \quad \hat{\mathbf{X}}^{t} \leftarrow \hat{\mathbf{X}}^{t-1} + \Delta \hat{\mathbf{X}}^{t}$ \COMMENT{残差预测}
            
            \IF{$t \geq 1$}
                \STATE 存储 $\hat{\mathbf{X}}^t$
                \STATE $\nabla \hat{\mathbf{X}}_{\mathbf{S}}^t \leftarrow \text{Linear}(\mathbf{S}^t)$ \COMMENT{提取传输贡献}
                \STATE $\mathcal{L}_{\mathrm{DIC}} \mathrel{+}= \text{DIC}(\nabla \hat{\mathbf{X}}_{\mathbf{S}}^{t-1}, \nabla \hat{\mathbf{X}}_{\mathbf{S}}^t)$ \COMMENT{累加DIC损失}
            \ENDIF
        \ENDFOR
    \ENDFOR

    \STATE $\mathcal{L} \leftarrow \mathcal{L}_{\mathrm{Pred}} + \lambda \mathcal{L}_{\mathrm{DIC}}$
    \STATE $\Theta \leftarrow \Theta - \alpha \frac{\partial \mathcal{L}}{\partial \Theta}$
    \STATE \RETURN $\Theta$(训练)\textbf{或} $\{\hat{\mathbf{X}}^t\}_{t=1}^T$(推理)
\end{algorithmic}
\end{algorithm}


% ------------------------------------------------------------
% 3.6 实验与结果
% ------------------------------------------------------------
\section{实验与结果}
\label{sec:pred_experiment}

本节通过系统的实验验证PM$_{2.5}$-GNN和PCDCNet的有效性。实验按照递进关系组织:首先在KnowAir-DS数据集上评估PM$_{2.5}$-GNN的\PM 单污染物预测能力(\ref{subsec:pm25gnn_results}节);随后在KnowAir-DS-V2数据集上评估PCDCNet的多污染物联合预测能力(\ref{subsec:pcdcnet_main_results}节);最后通过消融实验深入探讨模型各组件的贡献(\ref{subsec:pcdcnet_ablation}节)。


\subsection{实验设置}
\label{subsec:exp_setup}

\textbf{(1)数据集划分。}本章实验使用两个数据集:

\textbf{KnowAir-DS-V1}(2015-2018):用于评估PM$_{2.5}$-GNN的\PM 单指标预测能力。数据按时间划分为三个子数据集,用于评估不同场景下的模型性能:

\begin{itemize}
    \item \textbf{全年评估(Dataset 1)}:训练集2015/1/1--2016/12/31,验证集2017年,测试集2018年,评估模型在一般场景下的预测能力;
    \item \textbf{采暖季评估(Dataset 2)}:仅使用11月至次年2月的数据,评估模型在重污染高发期的预测能力;
    \item \textbf{滚动预测(Dataset 3)}:训练集2016/9/1--2016/11/30,验证集2016/12,测试集2017/1,模拟在线系统的滚动预测场景。
\end{itemize}

\textbf{KnowAir-DS-V2}(2016-2023):用于评估PCDCNet的多污染物预测能力。训练集2016-2019年,验证集2020-2021年,测试集2022-2023年。表\ref{tab:pm25gnn_dataset_split}汇总了数据集划分方案。

\begin{table}[htbp]
    \centering
    \caption{数据集划分方案}
    \caption*{KnowAir-DS-V1数据集用于PM$_{2.5}$-GNN评估,包含全年、采暖季和滚动预测三种场景;KnowAir-DS-V2数据集用于PCDCNet多污染物评估。}
    \label{tab:pm25gnn_dataset_split}
    \begin{tabular}{@{}llccc@{}}
        \toprule
        \textbf{数据集} & \textbf{实验场景} & \textbf{训练集} & \textbf{验证集} & \textbf{测试集} \\
        \midrule
        \multirow{3}{*}{KnowAir-DS-V1} & 全年评估 & 2015/1--2016/12 & 2017年 & 2018年 \\
        & 采暖季评估 & 2015/11--2016/2 & 2016/11--2017/2 & 2017/11--2018/2 \\
        & 滚动预测 & 2016/9--2016/11 & 2016/12 & 2017/1 \\
        \midrule
        KnowAir-DS-V2 & 多污染物评估 & 2016--2019 & 2020--2021 & 2022--2023 \\
        \bottomrule
    \end{tabular}
\end{table}

\textbf{(2)评价指标。}采用以下指标评估模型性能\citep{evaluation}:

\begin{itemize}
    \item \textbf{RMSE}(均方根误差):$\text{RMSE} = \sqrt{\frac{1}{n}\sum_{i=1}^{n}(y_i - \hat{y}_i)^2}$,对大误差敏感;
    \item \textbf{MAE}(平均绝对误差):$\text{MAE} = \frac{1}{n}\sum_{i=1}^{n}|y_i - \hat{y}_i|$,反映平均预测偏差;
    \item \textbf{CSI}(临界成功指数):$\text{CSI} = \frac{\text{hits}}{\text{hits} + \text{misses} + \text{false alarms}}$,综合评估预警能力;
    \item \textbf{POD}(命中率):$\text{POD} = \frac{\text{hits}}{\text{hits} + \text{misses}}$,反映污染事件的捕捉能力;
    \item \textbf{FAR}(虚警率):$\text{FAR} = \frac{\text{false alarms}}{\text{hits} + \text{false alarms}}$,反映误报情况。
\end{itemize}

其中,CSI、POD、FAR的计算以75 $\mu$g/m$^3$为阈值(中国《环境空气质量标准》良好等级分界点\citep{evaluation}),将预测和观测二值化为\cqt{污染/非污染}后统计。

\textbf{(3)基线方法。}将模型与以下方法进行对比:

\begin{itemize}
    \item \textbf{机器学习基线}:MLP、XGBoost\citep{chen2016xgboost}、LightGBM\citep{ke2017lightgbm};
    \item \textbf{通用时序模型}:LSTM\citep{hochreiter1997long}、GRU\citep{cho2014learning}、iTransformer\citep{liuitransformer}、TimeXer\citep{wang2024timexer};
    \item \textbf{空气质量专用模型}:GC-LSTM\citep{qi2019hybrid}、AirPhyNet\citep{hettigeairphynet}。
\end{itemize}

表\ref{tab:methods_comparison_vertical}对比了各模型的原生能力特征。可以看到,PCDCNet是唯一同时具备时序建模、多变量预测、外源变量融合、空间建模和物理约束的方法,与数值模式CMAQ的能力对齐。

\begin{table}[htbp]
\centering
\caption{基线模型原生能力对比}
\caption*{各列分别代表:AQF--是否为空气质量预报专用模型;Temp--是否建模时间依赖性;MultiV--是否支持多变量预测;Exog--是否能融合未来外源变量;Spat--是否建模空间相关性;Phy--是否融合物理约束。PCDCNet是唯一具备全部能力的方法。}
\label{tab:methods_comparison_vertical}
\begin{tabular}{@{}llcccccc@{}}
\toprule
\textbf{模型类别} & \textbf{模型} & \textbf{AQF} & \textbf{Temp} & \textbf{MultiV} & \textbf{Exog} & \textbf{Spat} & \textbf{Phy} \\
\midrule
\multirow{2}{*}{机器学习基线} & XGBoost & \xmark & \xmark & \xmark & \cmark & \xmark & \xmark \\
& LightGBM & \xmark & \xmark & \xmark & \cmark & \xmark & \xmark \\
\midrule
\multirow{2}{*}{通用时序模型} & iTransformer & \xmark & \cmark & \cmark & \xmark & \xmark & \xmark \\
& TimeXer & \xmark & \cmark & \cmark & \xmark & \xmark & \xmark \\
\midrule
\multirow{3}{*}{空气质量模型} & GC-LSTM & \cmark & \cmark & \xmark & \xmark & \cmark & \xmark \\
& PM$_{2.5}$-GNN & \cmark & \cmark & \xmark & \cmark & \cmark & \xmark \\
& AirPhyNet & \cmark & \cmark & \xmark & \xmark & \cmark & \cmark \\
\midrule
\multirow{2}{*}{本文方法} & CMAQ (参考) & \cmark & \cmark & \cmark & \cmark & \cmark & \cmark \\
& \textbf{PCDCNet} & \cmark & \cmark & \cmark & \cmark & \cmark & \cmark \\
\bottomrule
\end{tabular}
\end{table}

\textbf{(4)实现细节。}模型基于PyTorch实现,图计算使用PyTorch Geometric(PyG)库\citep{fey2019fast}。训练在NVIDIA 4070S GPU上进行。主要超参数设置:隐层维度$d=32$,历史窗口$T'=24$(72小时),预测窗口$T=24$(72小时),学习率$10^{-4}$,批大小32,训练轮数100,早停patience为10。优化器使用Adam,学习率调度使用ReduceLROnPlateau。


\subsection{PM$_{2.5}$-GNN实验结果}
\label{subsec:pm25gnn_results}

表\ref{tab:pm25gnn_overall_performance}展示了PM$_{2.5}$-GNN与基线模型在KnowAir-DS-V1数据集三个子数据集上的整体性能对比。每个指标报告10次实验的均值和标准差。

\begin{table*}[htbp]
    \centering
    \caption{PM$_{2.5}$-GNN与基线模型在KnowAir-DS-V1数据集上的性能对比}
    \caption*{报告的指标为72小时预测的平均值,包括RMSE、MAE(单位:$\mu$g/m$^3$)和预警指标CSI、POD、FAR(单位:\%)。每个指标报告10次实验的均值$\pm$标准差,最优结果已加粗显示。}
    \label{tab:pm25gnn_overall_performance}
    \resizebox{\textwidth}{!}{%
    \begin{tabular}{@{}clcccccc@{}}
        \toprule
        \textbf{数据集} & \textbf{指标} & \textbf{MLP} & \textbf{LSTM} & \textbf{GRU} & \textbf{GC-LSTM} & \textbf{nodesFC-GRU} & \textbf{PM$_{2.5}$-GNN} \\
        \midrule
        \multirow{5}{*}{全年} 
        & RMSE ($\mu$g/m$^3$) & $22.98\pm0.98$ & $21.07\pm0.38$ & $21.13\pm0.37$ & $20.90\pm0.40$ & $20.28\pm0.29$ & \textbf{20.16}$\pm$\textbf{0.48} \\
        & MAE ($\mu$g/m$^3$) & $18.37\pm0.94$ & $16.68\pm0.39$ & $16.77\pm0.37$ & $16.53\pm0.41$ & $15.98\pm0.30$ & \textbf{15.91}$\pm$\textbf{0.49} \\
        & CSI (\%) & $40.77\pm2.69$ & $44.87\pm1.09$ & $44.71\pm0.99$ & $45.64\pm1.10$ & $47.61\pm0.92$ & \textbf{47.91}$\pm$\textbf{1.65} \\
        & POD (\%) & $51.43\pm5.68$ & $56.43\pm2.43$ & $56.17\pm2.45$ & $57.98\pm2.51$ & $59.79\pm2.11$ & \textbf{60.33}$\pm$\textbf{3.42} \\
        & FAR (\%) & $32.80\pm4.29$ & $31.21\pm1.68$ & $31.16\pm1.80$ & $31.65\pm1.73$ & $29.87\pm1.43$ & \textbf{29.83}$\pm$\textbf{2.36} \\
        \midrule
        \multirow{5}{*}{采暖季} 
        & RMSE ($\mu$g/m$^3$) & $35.55\pm2.76$ & $33.53\pm1.04$ & $33.09\pm1.00$ & $33.20\pm1.23$ & $33.07\pm1.03$ & \textbf{32.11}$\pm$\textbf{1.47} \\
        & MAE ($\mu$g/m$^3$) & $28.67\pm2.52$ & $26.90\pm1.04$ & $26.54\pm0.97$ & $26.57\pm1.22$ & $26.40\pm0.97$ & \textbf{25.68}$\pm$\textbf{1.42} \\
        & CSI (\%) & $45.52\pm5.49$ & $49.75\pm2.09$ & $49.83\pm1.79$ & $50.13\pm2.50$ & $48.79\pm1.38$ & \textbf{51.35}$\pm$\textbf{2.53} \\
        & POD (\%) & $60.85\pm9.17$ & $64.94\pm3.30$ & $64.58\pm3.03$ & $64.54\pm3.49$ & $61.29\pm2.07$ & \textbf{66.24}$\pm$\textbf{4.56} \\
        & FAR (\%) & $34.56\pm6.21$ & $31.88\pm2.28$ & $31.31\pm2.44$ & $30.73\pm2.80$ & \textbf{29.37}$\pm$\textbf{2.60} & $30.11\pm3.67$ \\
        \midrule
        \multirow{5}{*}{滚动预测} 
        & RMSE ($\mu$g/m$^3$) & $50.70\pm4.57$ & $46.19\pm2.04$ & $46.06\pm2.03$ & $45.71\pm2.38$ & $47.97\pm1.67$ & \textbf{44.36}$\pm$\textbf{2.85} \\
        & MAE ($\mu$g/m$^3$) & $41.89\pm4.22$ & $37.97\pm1.94$ & $37.94\pm1.92$ & $37.46\pm2.29$ & $39.03\pm1.65$ & \textbf{36.32}$\pm$\textbf{2.81} \\
        & CSI (\%) & $52.44\pm3.81$ & $58.85\pm2.36$ & $59.16\pm1.87$ & $58.98\pm2.47$ & $58.84\pm1.60$ & \textbf{60.57}$\pm$\textbf{2.78} \\
        & POD (\%) & $74.16\pm7.25$ & $81.03\pm3.14$ & $83.32\pm1.95$ & $81.92\pm2.91$ & $79.40\pm1.71$ & \textbf{83.94}$\pm$\textbf{3.34} \\
        & FAR (\%) & $35.25\pm5.32$ & $31.71\pm2.38$ & $32.86\pm2.37$ & $32.18\pm2.36$ & \textbf{30.51}$\pm$\textbf{2.28} & $31.37\pm3.63$ \\
        \bottomrule
    \end{tabular}}
\end{table*}

从表\ref{tab:pm25gnn_overall_performance}可以得出以下结论:

\textbf{(1)PM$_{2.5}$-GNN在所有三个数据集上均取得最优或接近最优的性能},验证了风场驱动图结构设计的有效性。相比于仅使用时序信息的LSTM/GRU,PM$_{2.5}$-GNN的RMSE降低约4-6\%;相比于使用无向图的GC-LSTM,RMSE降低约2-3\%。

\textbf{(2)在采暖季(重污染高发期)场景下,PM$_{2.5}$-GNN的优势更为明显}。CSI指标从GC-LSTM的50.13\%提升至51.35\%,POD从64.54\%提升至66.24\%。这表明风场驱动的有向图设计有助于捕捉重污染事件中的跨区域传输,提高预警准确率。

\textbf{(3)在滚动预测场景下,PM$_{2.5}$-GNN展现出最强的泛化能力}。该场景使用最近3个月数据预测下个月,最接近实际业务系统的运行方式。PM$_{2.5}$-GNN的RMSE(44.36)显著低于其他方法,表明其在数据分布变化时仍能保持稳定性能。

\textbf{(4)nodesFC-GRU的过拟合问题}。nodesFC-GRU用全连接层替代GNN,虽然训练损失最低,但测试性能较差,表现出明显的过拟合。这说明GNN的归纳偏置(仅聚合邻居信息)有助于防止过拟合,提高泛化能力。

图\ref{fig:pm25gnn_leadtime}展示了不同预测时效下模型性能的变化趋势。

\begin{figure}[htbp]
  \centering
  \includegraphics[width=\linewidth]{figures/pm25gnn_leadtime.png}
  \caption{不同预测时效下各模型的RMSE和Pearson相关系数变化曲线}
  \caption*{基于滚动预测数据集。随着预测时效增加,所有模型的误差均呈上升趋势,但PM$_{2.5}$-GNN的误差增长速率最为平缓。在72小时预测末端,PM$_{2.5}$-GNN仍保持最优性能。}
  \label{fig:pm25gnn_leadtime}
\end{figure}

从图中可以看到:(1)MLP(无记忆模块)在首个预测步性能尚可,但随后快速衰减,说明时序建模的必要性;(2)LSTM、GRU、GC-LSTM的性能曲线较为接近,但GC-LSTM通过引入空间信息获得了一定提升;(3)PM$_{2.5}$-GNN在所有时效上均保持最优,且误差增长最为平缓,表明风场驱动的图结构有效提升了长时序预测的稳定性。

表\ref{tab:pm25gnn_ablation_study}展示了PM$_{2.5}$-GNN的消融实验结果,验证了边界层高度特征和净通量(export项)设计的贡献。

\begin{table}[htbp]
    \centering
    \caption{PM$_{2.5}$-GNN消融实验结果}
    \caption*{\cqt{无blh特征}表示从节点属性中移除边界层高度;\cqt{无export项}表示在消息聚合中仅使用输入流,不减去输出流。结果表明两项设计均对模型性能有显著贡献。}
    \label{tab:pm25gnn_ablation_study}
    \begin{tabular}{@{}clccc@{}}
        \toprule
        \textbf{数据集} & \textbf{指标} & \textbf{完整模型} & \textbf{无blh特征} & \textbf{无export项} \\
        \midrule
        \multirow{3}{*}{全年} & RMSE & \textbf{20.16}$\pm$\textbf{0.48} & $20.46\pm0.43$ & $20.98\pm0.33$ \\
        & MAE & \textbf{15.91}$\pm$\textbf{0.49} & $16.12\pm0.44$ & $16.67\pm0.35$ \\
        & CSI & \textbf{47.91}$\pm$\textbf{1.65\%} & $46.70\pm1.48\%$ & $45.41\pm1.17\%$ \\
        \midrule
        \multirow{3}{*}{采暖季} & RMSE & \textbf{32.11}$\pm$\textbf{1.47} & $33.25\pm1.65$ & $32.70\pm1.31$ \\
        & MAE & \textbf{25.68}$\pm$\textbf{1.42} & $26.67\pm1.59$ & $26.16\pm1.27$ \\
        & CSI & \textbf{51.35}$\pm$\textbf{2.53\%} & $49.42\pm2.90\%$ & $50.41\pm2.43\%$ \\
        \midrule
        \multirow{3}{*}{滚动预测} & RMSE & \textbf{44.36}$\pm$\textbf{2.85} & $46.12\pm3.38$ & $44.80\pm2.59$ \\
        & MAE & \textbf{36.32}$\pm$\textbf{2.81} & $38.04\pm3.38$ & $36.78\pm2.53$ \\
        & CSI & \textbf{60.57}$\pm$\textbf{2.78\%} & $58.72\pm3.15\%$ & $60.12\pm2.38\%$ \\
        \bottomrule
    \end{tabular}
\end{table}

消融实验表明:

\textbf{(1)边界层高度是关键特征}。移除blh后,所有数据集上的性能均显著下降,RMSE上升1-4\%。这验证了边界层高度与\PM 浓度的强负相关性\citep{su2018relationships,luan2018quantifying}——低边界层导致污染物难以垂直扩散,在近地面累积。

\textbf{(2)净通量设计有效}。移除export项(即仅聚合输入流)后,性能有所下降,尤其在全年评估数据集上RMSE上升4\%。这验证了\cqt{输入减输出}设计的物理合理性——它更准确地刻画了节点的\cqt{收支平衡}。


\subsection{PCDCNet主实验结果}
\label{subsec:pcdcnet_main_results}

表\ref{tab:performance_vertical}展示了PCDCNet与基线模型在BTHSA和YRD两个区域72小时预报任务中的性能对比。报告的指标为所有时间步的平均RMSE和MAE。

\begin{table*}[htbp]
\centering
\caption{PCDCNet与基线模型在72小时预报任务中的性能对比}
\caption*{报告的指标为所有时间步的平均RMSE和MAE(单位:$\mu$g/m$^3$)。对比方法涵盖机器学习基线、通用时序模型和空气质量专用模型。最优结果已加粗,次优结果已加下划线。}
\label{tab:performance_vertical}
\begin{tabular}{@{}llcccc@{}}
\toprule
\multirow{2}{*}{\textbf{区域}} & \multirow{2}{*}{\textbf{模型}} & \multicolumn{2}{c}{\textbf{RMSE} $\downarrow$} & \multicolumn{2}{c}{\textbf{MAE} $\downarrow$} \\
\cmidrule(lr){3-4} \cmidrule(lr){5-6}
& & \textbf{PM$_{2.5}$} & \textbf{O$_3$} & \textbf{PM$_{2.5}$} & \textbf{O$_3$} \\
\midrule
\multirow{8}{*}{BTHSA} & XGBoost & 35.47 & \underline{27.53} & 26.05 & \underline{20.68} \\
& LightGBM & 35.48 & 27.97 & 26.07 & 21.05 \\
& GC-LSTM & 32.95 & 35.12 & 23.09 & 26.82 \\
& PM$_{2.5}$-GNN & 31.51 & 31.85 & 23.50 & 22.90 \\
& iTransformer & \underline{31.17} & 29.90 & 20.72 & 22.58 \\
& AirPhyNet & 50.49 & 68.04 & 40.26 & 51.53 \\
& TimeXer & 31.51 & 30.07 & \underline{20.60} & 22.87 \\
\cmidrule(lr){2-6}
& \textbf{PCDCNet (本文)} & \textbf{24.13} & \textbf{22.45} & \textbf{15.46} & \textbf{16.73} \\
\midrule
\multirow{8}{*}{YRD} & XGBoost & 22.07 & \underline{27.71} & 16.64 & \underline{20.99} \\
& LightGBM & 22.06 & 28.07 & 16.74 & 21.33 \\
& GC-LSTM & 18.77 & 34.65 & 13.57 & 26.31 \\
& PM$_{2.5}$-GNN & 18.17 & 31.46 & 13.24 & 24.05 \\
& iTransformer & 17.92 & 30.53 & \underline{12.86} & 23.28 \\
& AirPhyNet & 28.53 & 56.96 & 22.90 & 42.76 \\
& TimeXer & \underline{17.89} & 30.47 & \underline{12.86} & 23.28 \\
\cmidrule(lr){2-6}
& \textbf{PCDCNet (本文)} & \textbf{15.55} & \textbf{23.03} & \textbf{10.97} & \textbf{17.27} \\
\bottomrule
\end{tabular}
\end{table*}

从表\ref{tab:performance_vertical}可以得出以下结论:

\textbf{(1)PCDCNet在所有指标上均取得最优性能}。在BTHSA区域,\PM 的RMSE从次优方法iTransformer的31.17降至24.13,降幅达\textbf{22.6\%};\ozone 的RMSE从次优方法XGBoost的27.53降至22.45,降幅达\textbf{18.5\%}。在YRD区域,\PM 的RMSE降低\textbf{13.1\%},\ozone 的RMSE降低\textbf{16.9\%}。

\textbf{(2)PM$_{2.5}$-GNN在\ozone 预测上表现不佳}。PM$_{2.5}$-GNN的\ozone RMSE(BTHSA: 31.85, YRD: 31.46)显著高于PCDCNet,验证了我们在\ref{subsec:pm25gnn_limitation}节分析的局限性——缺乏化学机制约束和排放输入导致其难以捕捉\ozone 的光化学生成过程。

\textbf{(3)通用时序模型在\ozone 预测上优于专用空气质量模型}。iTransformer和TimeXer在\ozone 预测上的表现优于GC-LSTM和PM$_{2.5}$-GNN,但仍显著落后于PCDCNet。这表明:(a)\ozone 预测需要捕捉复杂的非线性依赖,Transformer架构在此有一定优势;(b)仅依靠通用时序建模能力仍不足够,需要融合排放和物理约束。

\textbf{(4)AirPhyNet性能较差的原因分析}。AirPhyNet虽然引入了基于Neural ODE的物理约束,但其设计假设较强(仅使用风场和历史浓度),缺乏排放输入和多气象变量融合。在本文的多污染物、长时序预测任务中,AirPhyNet出现了严重的误差发散问题(\PM RMSE达50.49,\ozone RMSE达68.04)。

\textbf{(5)XGBoost在\ozone 预测上的相对优势}。XGBoost在\ozone 预测上取得次优性能,这是因为它直接利用了排放数据和未来气象预报作为输入(表\ref{tab:methods_comparison_vertical}中的Exog能力)。然而,由于缺乏时序建模和空间建模能力,其\PM 预测性能较差。

图\ref{fig:leadtime}展示了不同预测时效下PCDCNet与基线模型的性能对比。

\begin{figure}[htbp]
  \centering
  \includegraphics[width=\linewidth]{figures/leadtime.png}
  \caption{不同预测时效下各模型的MAE变化曲线}
  \caption*{左图为BTHSA区域的\PM 预测,右图为YRD区域的\ozone 预测。PCDCNet在所有时效上均保持最优性能,且误差随时效增长的速率最为平缓,表明DIC约束有效抑制了长时序预测中的误差累积。}
  \label{fig:leadtime}
\end{figure}

从图中可以观察到:

\textbf{(1)PCDCNet在所有时效上均保持最优}。从6小时到72小时,PCDCNet的MAE曲线始终位于最下方,且与其他方法的差距随时效增加而扩大。

\textbf{(2)误差增长速率差异显著}。GC-LSTM和AirPhyNet的误差随时效快速增长,在48小时后出现加速发散;而PCDCNet的误差增长近似线性,表明DIC约束有效抑制了误差累积。

\textbf{(3)Transformer模型在中长时效表现稳定}。iTransformer和TimeXer的误差增长速率介于GC-LSTM和PCDCNet之间,验证了自注意力机制在长程依赖建模中的优势。


\subsection{消融实验}
\label{subsec:pcdcnet_ablation}

为验证各模块和设计选择的贡献,我们设计了系统的消融实验。

\textbf{(1)核心模块消融}。分别移除LID、STD、TAD模块,评估其对预测性能的影响。图\ref{fig:ablation}展示了消融实验结果。

\begin{figure}[htbp]
  \centering
  \includegraphics[width=0.8\linewidth]{figures/ablation.png}
  \caption{PCDCNet消融实验结果}
  \caption*{左图为\PM 预测性能,右图为\ozone 预测性能。w/o Emission表示移除排放数据输入;w/o DIC表示移除领域知识约束;w/o LID/STD/TAD分别表示移除对应的动力学模块。移除任何模块均导致性能下降。}
  \label{fig:ablation}
\end{figure}

消融实验的关键发现包括:

\textbf{排放数据对\ozone 预测至关重要}:移除排放输入(w/o Emission)后,\ozone 的MAE上升约\textbf{15\%},而\PM 的影响相对较小(上升约5\%)。这验证了排放清单对于捕捉二次污染物(\ozone)生成的关键作用——\ozone 的生成强烈依赖于前驱物\NOx 和VOC的排放量,而\PM 中有相当比例是一次排放,对排放变化的响应相对直接。

\textbf{DIC约束提升长时序稳定性}:移除DIC约束(w/o DIC)后,模型在短时效($<$24小时)的性能变化不大,但在72小时预测末期的误差显著增大。这证明物理约束的主要作用是抑制长时序预测中的误差累积,确保预测结果的物理合理性。

\textbf{三个动力学模块缺一不可}:LID、STD、TAD分别对应化学生成、空间传输、时间累积三类物理过程,移除任一模块都会导致性能下降。其中,移除TAD模块的影响最大(\PM MAE上升约10\%),表明时序累积建模对于捕捉污染物浓度的演化趋势至关重要。

\textbf{(2)排放物种消融}。进一步分析不同排放物种对预测性能的贡献。图\ref{fig:emis_ablation}展示了分别移除\NOx、VOC、SO$_2$等排放变量后的性能变化。

\begin{figure}[htbp]
  \centering
  \includegraphics[width=0.8\linewidth]{figures/emis_ablation.png}
  \caption{不同排放物种对预测性能的贡献分析}
  \caption*{分别移除\NOx、VOC、SO$_2$等排放变量后,评估对\PM 和\ozone 预测的影响。结果表明,\NOx 和VOC对\ozone 预测贡献最大,SO$_2$对\PM 预测贡献显著。}
  \label{fig:emis_ablation}
\end{figure}

排放物种消融的关键发现包括:

\textbf{\NOx 和VOC是\ozone 预测的关键排放}:移除\NOx 后\ozone MAE上升约8\%,移除VOC后上升约6\%。这与\ozone 的光化学生成机理一致——\NOx 和VOC是\ozone 的两个主要前驱物。

\textbf{SO$_2$对\PM 预测贡献显著}:移除SO$_2$后\PM MAE上升约5\%,这是因为SO$_2$氧化生成的硫酸盐是\PM 的重要组分。

\textbf{NH$_3$的影响相对较小}:移除NH$_3$后性能变化不大,可能是因为在当前模型分辨率下,氨盐生成的非线性过程难以被充分捕捉。

\textbf{(3)隐层维度敏感性分析}。评估不同隐层维度$d$对模型性能和计算效率的影响。图\ref{fig:ablation_sensitivity}左图展示了隐层维度为16、32、64时的性能对比。

\begin{figure}[htbp]
  \centering
  \includegraphics[width=0.8\linewidth]{figures/ablation.png}
  \caption{参数敏感性分析}
  \caption*{左图:隐层维度敏感性分析,$d=32$取得最优性能,在模型容量与泛化能力之间达到平衡;右图:核心模块消融实验结果。}
  \label{fig:ablation_sensitivity}
\end{figure}

结果表明:$d=16$时模型容量不足,欠拟合导致性能下降;$d=64$时参数量增加但性能未见提升,存在轻微过拟合风险;$d=32$在模型容量与泛化能力之间达到最优平衡。


% DIC约束效果分析
为深入理解DIC约束的作用机制,我们分析了不同约束权重$\lambda$对模型性能的影响。图\ref{fig:loss}展示了训练过程中预测损失与DIC损失的变化曲线。

\begin{figure}[htbp]
  \centering
  \includegraphics[width=\linewidth]{figures/loss.png}
  \caption{训练过程中的损失曲线分析}
  \caption*{上排:不同$\lambda$值下的训练/验证预测损失;下排:对应的DIC损失(空间约束、时间约束、总约束)。当$\lambda=0$时,DIC损失虽未被直接优化,但仍呈下降趋势,表明STD模块具有隐式学习质量守恒的能力;当$\lambda>0$时,DIC损失下降更快且收敛值更低,验证了显式约束的有效性。}
  \label{fig:loss}
\end{figure}

DIC约束效果分析的关键发现包括:

\textbf{(1)STD模块具有隐式学习质量守恒的能力}。即使$\lambda=0$(不使用DIC约束),DIC损失在训练过程中也呈下降趋势。这表明图卷积操作本身具有一定的\cqt{守恒倾向}——由归一化拉普拉斯算子定义的信息传播在数学上与扩散方程相关。

\textbf{(2)显式DIC约束显著提升物理一致性}。当$\lambda>0$时,DIC损失的收敛值显著降低(从$\lambda=0$的约0.15降至$\lambda=10$的约0.05),表明显式约束有效引导模型学习更符合物理规律的传输模式。

\textbf{(3)$\lambda$过大会损害预测精度}。当$\lambda$过大时(如$\lambda=100$,未在图中显示),约束过强限制了模型的拟合能力,导致预测损失上升。最优$\lambda$值在$10^{-3}$至$10^{-2}$区间,此时模型在预测精度与物理一致性之间达到平衡。

\textbf{(4)DIC约束改善泛化能力}。对比训练损失和验证损失可以发现:$\lambda=0$时存在明显的过拟合现象(训练损失低但验证损失相对较高);$\lambda>0$时过拟合得到缓解,验证损失与训练损失的差距缩小。这证明DIC约束具有正则化效果,有助于提升模型的泛化能力。


\subsection{案例分析}
\label{subsec:case_study}

为直观展示模型的预测能力,我们选取典型污染事件进行案例分析。

\textbf{案例:华北秋冬季重污染过程}。图\ref{fig:case_beijing}展示了2022年11月某次重污染过程中PCDCNet的预测表现。该过程持续约5天,\PM 峰值浓度超过200 $\mu$g/m$^3$。

\begin{figure}[htbp]
  \centering
  \includegraphics[width=\linewidth]{figures/beijing_pollution.png}
  \caption{华北重污染过程预测案例}
  \caption*{左图为PCDCNet 72小时预测结果,右图为实际观测。模型成功提前72小时预警污染过程的启动、峰值时间与消退趋势,预测曲线与观测高度吻合。值得强调的是,该结果来源于系统业务化运行期间的实时截图,所涉及时段的数据在模型训练阶段从未出现,充分验证了模型在真实应用场景下的泛化能力。}
  \label{fig:case_beijing}
\end{figure}

从案例中可以看到:(1)PCDCNet成功提前72小时预警污染过程的启动;(2)预测的峰值时间与实际观测吻合,误差在3小时以内;(3)污染消退的时间和速率也被准确捕捉。这种\cqt{全过程}的准确预测对于环境管理具有重要的实用价值。


% ------------------------------------------------------------
% 3.7 本章小结
% ------------------------------------------------------------
\section{本章小结}
\label{sec:pred_summary}

本章聚焦于大气污染的时空预测问题,完成了从单一物理传输建模到多过程物理化学耦合建模的跨越。通过PM$_{2.5}$-GNN和PCDCNet两个模型的递进式研究,系统探索了如何在深度学习框架中融合大气科学领域知识。主要贡献总结如下:

\textbf{(1)提出了基于风场驱动图网络的PM$_{2.5}$-GNN模型}。通过在图神经网络中显式编码风场信息,设计了平流系数公式(公式\eqref{eq:advection_coeff})和净通量聚合机制(公式\eqref{eq:message_agg}),有效解决了\PM 的跨区域定向传输预测难题。实验表明,PM$_{2.5}$-GNN相比GC-LSTM等基线模型,在72小时长时序预测和重污染事件预警方面具有显著优势,验证了风场驱动有向图设计的有效性。

\textbf{(2)针对PM$_{2.5}$-GNN的局限性,提出了PCDCNet模型}。作为数值模式CMAQ的深度学习代理模型,PCDCNet具备以下创新特征:

\begin{itemize}
    \item \textbf{排放响应建模}:将排放清单作为动态输入变量纳入模型框架,实现对排放变化的显式响应能力,为后续的情景模拟应用奠定基础;
    \item \textbf{LID-STD-TAD三模块架构}:分别对应公式\eqref{eq:advection_diffusion}中的化学反应与排放项(R、E)、平流与扩散项、沉降项与时间累积(D),实现\cqt{过程解耦+联合建模},增强了模型的物理可解释性;
    \item \textbf{领域知识约束(DIC)}:将质量守恒原则嵌入损失函数,引导模型学习物理一致的传输模式,有效抑制长时序预测中的误差累积。
\end{itemize}

\textbf{(3)验证了物理知识约束在深度学习中的有效性}。系统的消融实验和约束效果分析表明:(a)在损失函数中引入质量守恒等物理约束,能显著提升模型的泛化能力和长时序稳定性;(b)排放数据对于\ozone 等二次污染物的预测至关重要;(c)DIC约束具有正则化效果,有助于缓解过拟合。

\textbf{(4)构建并公开了KnowAir-DS系列数据集}。为大气污染预测研究提供了融合多源观测、气象再分析和排放清单的标准化数据基础,推动该领域的开放研究。

在京津冀和长三角区域的72小时多污染物预报任务中,PCDCNet相较现有最优方法,\PM 预测的RMSE降低\textbf{13-23\%},\ozone 预测的RMSE降低\textbf{17-19\%}。这一性能提升验证了\cqt{物理约束数据驱动建模}范式的有效性。

从方法论角度,本章的核心贡献在于:将第\ref{chap:methodology}章公式\eqref{eq:advection_diffusion}所描述的大气物理化学过程(平流、扩散、化学反应、排放、沉降)显式地映射到深度学习架构设计中,构建了\cqt{Embed$\to$LID$\to$STD$\to$TAD$\to$Readout}的完整建模框架。这种\cqt{观测空间$\leftrightarrow$隐空间$\leftrightarrow$观测空间}的表示学习范式,不仅约束了模型的输入输出映射,更重要的是约束了模型学习到的隐空间表示本身,使其具备物理意义与一致性。

PCDCNet的成功构建,不仅实现了高精度的实时预报,更为后续章节开展基于多源数据融合的\textbf{空间推断}(第\ref{chap:inference}章)和基于排放清单的\textbf{情景模拟}(第\ref{chap:simulation}章)奠定了坚实的建模基础。