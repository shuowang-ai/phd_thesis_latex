% !TeX encoding = UTF-8
% !TeX program = xelatex
% !TeX spellcheck = en_US

\documentclass[degree=doctor,fontset=fandol]{bnuthesis}
  % 学位 degree:
  %   doctor | master | bachelor
  % 学位类型 degree-type:
  %   academic(默认)| professional
  % 语言 language
  %   chinese(默认)
  % 字体库 fontset
  %   windows | mac | fandol | ubuntu
  % Overleaf云端编译请使用 fontset=fandol
  % 建议终版使用 Windows 平台的字体编译


% 论文基本配置,加载宏包等全局配置
% !TeX root = ./wangshuo_phdthesis.tex

% 论文基本信息配置

\bnusetup{
  %******************************
  % 注意:
  %   1. 配置里面不要出现空行
  %   2. 不需要的配置信息可以删除
  %   3. 建议先阅读文档中所有关于选项的说明
  %******************************
  %
  % 输出格式
  %   选择打印版(print)或用于提交的电子版(electronic),前者会插入空白页以便直接双面打印
  %
  output = electronic,
  %
  % 标题
  %   可使用“\\”命令手动控制换行
  %
  title  = {大气污染复杂系统的数据驱动建模:\\预测、推断与模拟},
  title* = {Data-Driven Modeling of Air Pollution as a Complex System:\\Prediction, Inference, and Simulation},
  %
  % 学科门类
  %   1. 学术型
  %      - 中文
  %        需注明所属的学科门类,例如:
  %        哲学、经济学、法学、教育学、文学、历史学、理学、工学、农学、医学、
  %        军事学、管理学、艺术学
  %      - 英文
  %        博士:Doctor of Philosophy
  %        硕士:
  %          哲学、文学、历史学、法学、教育学、艺术学门类,公共管理学科
  %          填写“Master of Arts“,其它填写“Master of Science”
  %   2. 专业型
  %      直接填写专业学位的名称,例如:
  %      教育博士、工程硕士等
  %      Doctor of Education, Master of Engineering
  %   3. 本科生不需要填写
  %
  % degree-category  = {教育硕士},
  % degree-category* = {Master of Education},
  %
  % 培养单位
  %   填写所属院系的全名
  %
  department = {系统科学学院},
  %
  % 学科
  %   1. 研究生学术型学位,获得一级学科授权的学科填写一级学科名称,其他填写二级学科名称
  %   2. 本科生填写专业名称,第二学位论文需标注“(第二学位)”
  %
  discipline  = {系统分析与集成},
  discipline* = {System Analysis and Integration},
  %
  % 专业领域
  %   1. 设置专业领域的专业学位类别,填写相应专业领域名称
  %   2. 2019 级及之前工程硕士学位论文,在 `engineering-field` 填写相应工程领域名称
  %   3. 其他专业学位类别的学位论文无需此信息
  %
  % professional-field  = {学科教学(数学)},
  % professional-field* = {Subject Teaching(Mathematics)},
  %
  % 姓名
  %
  author  = {王硕},
  author* = {Wang Shuo},
  student-id = {202031250018},
  %
  % 导师
  %   中文姓名和职称之间以英文逗号“,”分开,下同
  %
  supervisor  = {张江, 教授},
  supervisor* = {Professor Zhang Jiang},
  supervisor-department = {系统科学学院},
  %
  % 副导师
  %
  % associate-supervisor  = {陈文光, 教授},
  % associate-supervisor* = {Professor Chen Wenguang},
  %
  % 联合导师
  %
  % co-supervisor  = {某某某, 教授},
  % co-supervisor* = {Professor Mou Moumou},
  %
  % 日期
  %   使用 ISO 格式;默认为当前时间
  %
  date = {2026-02-01},
  %
  % 是否在中文封面后的空白页生成书脊(默认 false),仅博士需要。
  %
  include-spine = true,
  spine-school = {北京师范大学},
  %
  % 密级和年限
  %   秘密, 机密, 绝密
  %
  % secret-level = {秘密},
  % secret-year  = {10},
  %
}

% 载入所需的宏包

% 定理类环境宏包
\usepackage{amsthm}
% 也可以使用 ntheorem
% \usepackage[amsmath,thmmarks,hyperref]{ntheorem}

\bnusetup{
  %
  % math-style = GB,  % GB | ISO | TeX
  % 数学公式字体,如果认为xits字体不满意的话,可尝试更改为newcm.
  math-font  = xits,  % stix | xits | libertinus | newcm 
  % 图表编号时是否带有章节序号,默认为不带有false
  figurestables-chapternumber=false,  %false / true
}

% 可以使用 nomencl 生成符号和缩略语说明
% \usepackage{nomencl}
% \makenomenclature

% 表格加脚注
\usepackage{threeparttable}

% 表格中支持跨行
\usepackage{multirow}

% 固定宽度的表格。
% \usepackage{tabularx}

% 跨页表格
\usepackage{longtable}

% 算法
\usepackage{algorithm}
\usepackage{algorithmic}

% 量和单位
\usepackage{siunitx}

\usepackage{xspace}
\usepackage{graphicx}
\usepackage{subcaption}
\usepackage{amsmath}
\usepackage{makecell}
\usepackage{booktabs}   % 用于三线表
\usepackage{multirow}   % 用于多行单元格
\usepackage{pifont}     % 用于显示对勾和叉号
\usepackage{comment}
\usepackage{csquotes}   % 中文引号

% 参考文献使用 BibTeX + natbib 宏包
% 顺序编码制
\usepackage[sort]{natbib}
\bibliographystyle{bnuthesis-numeric}

% 著者-出版年制
% \usepackage{natbib}
% \bibliographystyle{bnuthesis-author-year}

% 本科生参考文献的著录格式
% \usepackage[sort]{natbib}
% \bibliographystyle{bnuthesis-bachelor}

% 参考文献使用 BibLaTeX 宏包
% \usepackage[style=bnuthesis-numeric]{biblatex}
% \usepackage[style=bnuthesis-author-year]{biblatex}
% \usepackage[style=gb7714-2015]{biblatex}
% \usepackage[style=apa]{biblatex}
% \usepackage[style=mla-new]{biblatex}
% 声明 BibLaTeX 的数据库
% \addbibresource{ref/refs.bib}

% 定义所有的图片文件在 figures 子目录下
\graphicspath{{figures/}}

% 数学命令
\makeatletter
\newcommand\dif{%  % 微分符号
  \mathop{}\!%
  \ifbnu@math@style@TeX
    d%
  \else
    \mathrm{d}%
  \fi
}
\makeatother

% 处理中文引号问题
\newcommand{\cqt}[1]{“#1”}  % Chinese Quote,使用正确的中文引号对

% =========================================================
% 核心命令定义(直接嵌入以确保 Overleaf 兼容性)
% =========================================================
% 污染物
\def\PM{\ensuremath{\mathrm{PM}_{2.5}}\xspace}
\def\PMten{\ensuremath{\mathrm{PM}_{10}}\xspace}
\def\ozone{\ensuremath{\mathrm{O}_{3}}\xspace}
\def\NOx{\ensuremath{\mathrm{NO}_{x}}\xspace}
\def\NOtwo{\ensuremath{\mathrm{NO}_{2}}\xspace}
\def\VOCs{\ensuremath{\mathrm{VOCs}}\xspace}
\def\CO{\ensuremath{\mathrm{CO}}\xspace}
\def\SOtwo{\ensuremath{\mathrm{SO}_{2}}\xspace}
\def\AOD{\ensuremath{\mathrm{AOD}}\xspace}

% 模型名称
\def\ModelPred{\ensuremath{\mathrm{PM}_{2.5}}\text{-GNN}\xspace}
\def\ModelSurr{PCDCNet\xspace}
\def\ModelInfer{SPIN\xspace}
\def\ModelSim{IGNN\xspace}
\def\SystemName{KnowAir\xspace}

% 损失函数
\def\LossAOD{\ensuremath{\mathcal{L}_{\mathrm{AOD}}}\xspace}
\def\LossInfer{\ensuremath{\mathcal{L}_{\mathrm{infer}}}\xspace}
\def\LossInit{\ensuremath{\mathcal{L}_{\mathrm{init}}}\xspace}
\def\LossDIC{\ensuremath{\mathcal{L}_{\mathrm{DIC}}}\xspace}

% 评价指标
\def\RMSE{\ensuremath{\mathrm{RMSE}}\xspace}
\def\MAE{\ensuremath{\mathrm{MAE}}\xspace}
\def\Rsquare{\ensuremath{R^{2}}\xspace}

% 数据源
\def\MEIC{MEIC\xspace}
\def\ERAFive{ERA5\xspace}
\def\CMAQ{CMAQ\xspace}
\def\GNN{GNN\xspace}
\def\GRU{GRU\xspace}
\def\LSTM{LSTM\xspace}

% hyperref 宏包在最后调用
\usepackage{hyperref}

% 只在"生成 PDF 书签"时生效的替换规则
\pdfstringdefDisableCommands{%
  % 你前面已经可以在这里放 \mathrm / \symcal 之类
  % \def\mathrm#1{#1}%
  % \def\symcal#1{#1}%

  % 下面是关键:告诉 hyperref,书签里这些宏长这样
  \def\PM{PM2.5}%
  \def\ozone{O3}%
  \def\ModelPred{PM2.5-GNN}%
  \def\LossAOD{Loss-AOD}%
  \def\LossInfer{Loss-Infer}%
  \def\LossInit{Loss-Init}%
}


% =========================================================
% commands.tex - 博士论文全局符号与命令定义
% 基于复杂系统数据驱动建模的大气污染研究
%
% 版本: 3.0 (newcommand版)
% 修复内容: 所有命令使用 \newcommand 定义,\ensuremath 确保兼容性
% =========================================================

% =========================================================
% 0. 依赖包检测与加载 (Package Loading)
% =========================================================
\makeatletter
% xspace 用于智能空格
\@ifpackageloaded{xspace}{}{\usepackage{xspace}}

% amsmath 通常由模版加载
\@ifpackageloaded{amsmath}{}{\usepackage{amsmath}}

% [重要] amssymb 与 unicode-math 冲突,模版使用 unicode-math 时必须注释
% \@ifpackageloaded{amssymb}{}{\usepackage{amssymb}}

% pifont 用于支持 \ding{51} (打钩/打叉)
\@ifpackageloaded{pifont}{}{\usepackage{pifont}}
\makeatother

% =========================================================
% 1. 空气污染物 (Pollutants)
%    所有化学式均使用 \ensuremath 包裹
% =========================================================
% 1.1 颗粒物 (使用 \providecommand 避免与已有定义冲突)
\providecommand{\PM}{}
\renewcommand{\PM}{\ensuremath{\mathrm{PM}_{2.5}}\xspace}
\providecommand{\PMten}{\ensuremath{\mathrm{PM}_{10}}\xspace}
\providecommand{\PMcoarse}{\ensuremath{\mathrm{PM}_{10}}\xspace}

% 1.2 气态污染物
\providecommand{\ozone}{\ensuremath{\mathrm{O}_{3}}\xspace}
\providecommand{\NitrogenDioxide}{\ensuremath{\mathrm{NO}_{2}}\xspace}
\providecommand{\NitrogenOxides}{\ensuremath{\mathrm{NO}_{x}}\xspace}
\providecommand{\NO}{\ensuremath{\mathrm{NO}}\xspace}
\providecommand{\NOtwo}{\ensuremath{\mathrm{NO}_{2}}\xspace}
\providecommand{\NOx}{\ensuremath{\mathrm{NO}_{x}}\xspace}
\providecommand{\SulfurDioxide}{\ensuremath{\mathrm{SO}_{2}}\xspace}
\providecommand{\SOtwo}{\ensuremath{\mathrm{SO}_{2}}\xspace}
\providecommand{\CarbonMonoxide}{\ensuremath{\mathrm{CO}}\xspace}
\providecommand{\CO}{\ensuremath{\mathrm{CO}}\xspace}
\providecommand{\CarbonDioxide}{\ensuremath{\mathrm{CO}_{2}}\xspace}
\providecommand{\COtwo}{\ensuremath{\mathrm{CO}_{2}}\xspace}

% 1.3 前体物与其他化学物种
\providecommand{\VOC}{\ensuremath{\mathrm{VOC}}\xspace}
\providecommand{\VOCs}{\ensuremath{\mathrm{VOCs}}\xspace}
\providecommand{\BVOC}{\ensuremath{\mathrm{BVOC}}\xspace}
\providecommand{\BVOCs}{\ensuremath{\mathrm{BVOCs}}\xspace}
\providecommand{\Ammonia}{\ensuremath{\mathrm{NH}_{3}}\xspace}
\providecommand{\NH}{\ensuremath{\mathrm{NH}_{3}}\xspace}
\providecommand{\NHthree}{\ensuremath{\mathrm{NH}_{3}}\xspace}
\providecommand{\HNO}{\ensuremath{\mathrm{HNO}_{3}}\xspace}
\providecommand{\Hsotwo}{\ensuremath{\mathrm{H}_{2}\mathrm{SO}_{4}}\xspace}

% 1.4 二次污染物组分
\providecommand{\SOA}{\ensuremath{\mathrm{SOA}}\xspace}
\providecommand{\SNA}{\ensuremath{\mathrm{SNA}}\xspace}

% 1.5 气溶胶光学厚度
\providecommand{\AOD}{\ensuremath{\mathrm{AOD}}\xspace}

% =========================================================
% 2. 物理单位 (Units)
%    所有单位均使用 \ensuremath 包裹
% =========================================================
% 2.1 浓度单位
\providecommand{\ug}{\ensuremath{\mu\mathrm{g}/\mathrm{m}^3}\xspace}
\providecommand{\ugm}{\ensuremath{\mu\mathrm{g}\,\mathrm{m}^{-3}}\xspace}
\providecommand{\ppb}{\ensuremath{\mathrm{ppb}}\xspace}
\providecommand{\ppm}{\ensuremath{\mathrm{ppm}}\xspace}

% 2.2 长度与面积
\providecommand{\km}{\ensuremath{\mathrm{km}}\xspace}
\providecommand{\m}{\ensuremath{\mathrm{m}}\xspace}
\providecommand{\cm}{\ensuremath{\mathrm{cm}}\xspace}
\providecommand{\mm}{\ensuremath{\mathrm{mm}}\xspace}
\providecommand{\kmsq}{\ensuremath{\mathrm{km}^{2}}\xspace}

% 2.3 时间
\providecommand{\s}{\ensuremath{\mathrm{s}}\xspace}
\providecommand{\h}{\ensuremath{\mathrm{h}}\xspace}
\providecommand{\hr}{\ensuremath{\mathrm{h}}\xspace}

% 2.4 速度与通量
\providecommand{\mps}{\ensuremath{\mathrm{m/s}}\xspace}
\providecommand{\kmph}{\ensuremath{\mathrm{km/h}}\xspace}

% 2.5 质量与排放
\providecommand{\ton}{\ensuremath{\mathrm{ton}}\xspace}
\providecommand{\Tg}{\ensuremath{\mathrm{Tg}}\xspace}
\providecommand{\Gg}{\ensuremath{\mathrm{Gg}}\xspace}
\providecommand{\kg}{\ensuremath{\mathrm{kg}}\xspace}

% 2.6 气象单位
\providecommand{\Pa}{\ensuremath{\mathrm{Pa}}\xspace}
\providecommand{\hPa}{\ensuremath{\mathrm{hPa}}\xspace}
\providecommand{\K}{\ensuremath{\mathrm{K}}\xspace}
\providecommand{\degC}{\ensuremath{^{\circ}\mathrm{C}}\xspace}
\providecommand{\Wm}{\ensuremath{\mathrm{W/m}^{2}}\xspace}
\providecommand{\Wmsq}{\ensuremath{\mathrm{W}\,\mathrm{m}^{-2}}\xspace}
\providecommand{\degree}{\ensuremath{^{\circ}}\xspace}
\providecommand{\percent}{\ensuremath{\%}\xspace}

% =========================================================
% 3. 核心变量 (Core Variables)
%    对应符号表中的主要数学符号
% =========================================================
% 3.1 空气质量变量
\providecommand{\AirPoll}{\ensuremath{\mathbf{X}}\xspace}
\providecommand{\AirPollHat}{\ensuremath{\hat{\mathbf{X}}}\xspace}
\providecommand{\Concentration}{\ensuremath{C}\xspace}
\providecommand{\ConcentrationVec}{\ensuremath{\mathbf{C}}\xspace}

% 3.2 驱动变量
\providecommand{\Meteo}{\ensuremath{\mathbf{M}}\xspace}
\providecommand{\MeteoP}{\ensuremath{\mathbf{P}}\xspace}
\providecommand{\Emiss}{\ensuremath{\mathbf{E}}\xspace}
\providecommand{\EmissQ}{\ensuremath{\mathbf{Q}}\xspace}

% 3.3 遥感与辅助变量
\providecommand{\RemoteSensing}{\ensuremath{\mathbf{X}^{\mathrm{AOD}}}\xspace}
\providecommand{\AODMask}{\ensuremath{\mathbf{M}^{\mathrm{AOD}}}\xspace}
\providecommand{\AODObs}{\ensuremath{\mathbf{Y}^{\mathrm{AOD}}}\xspace}

% 3.4 隐状态与中间变量
\providecommand{\Hidden}{\ensuremath{\mathbf{H}}\xspace}
\providecommand{\HiddenZ}{\ensuremath{\mathbf{Z}}\xspace}
\providecommand{\Message}{\ensuremath{\mathbf{m}}\xspace}
\providecommand{\MessageAgg}{\ensuremath{\mathbf{Z}}\xspace}  % 聚合消息,避免与气象符号M冲突
\providecommand{\Feature}{\ensuremath{\mathbf{h}}\xspace}
\providecommand{\Embedding}{\ensuremath{\mathbf{e}}\xspace}

% 3.5 风场变量
\providecommand{\WindVec}{\ensuremath{\mathbf{u}}\xspace}
\providecommand{\WindU}{\ensuremath{u}\xspace}
\providecommand{\WindV}{\ensuremath{v}\xspace}
\providecommand{\WindW}{\ensuremath{w}\xspace}
\providecommand{\WindSpeed}{\ensuremath{|\mathbf{u}|}\xspace}

% 3.6 扩散与传输系数
\providecommand{\DiffCoef}{\ensuremath{K}\xspace}
\providecommand{\DiffCoefX}{\ensuremath{K_x}\xspace}
\providecommand{\DiffCoefY}{\ensuremath{K_y}\xspace}
\providecommand{\DiffCoefZ}{\ensuremath{K_z}\xspace}

% =========================================================
% 4. 图结构与时空参数 (Graph & Spatiotemporal)
% =========================================================
% 4.1 图结构基础
\providecommand{\Graph}{\ensuremath{\mathcal{G}}\xspace}
\providecommand{\Nodes}{\ensuremath{\mathcal{V}}\xspace}
\providecommand{\Edges}{\ensuremath{\mathcal{E}}\xspace}
\providecommand{\NodeSet}{\ensuremath{V}\xspace}
\providecommand{\EdgeSet}{\ensuremath{E}\xspace}
\providecommand{\NumNodes}{\ensuremath{|V|}\xspace}
\providecommand{\NumEdges}{\ensuremath{|E|}\xspace}

% 4.2 图矩阵
\providecommand{\Adj}{\ensuremath{\mathbf{A}}\xspace}
\providecommand{\AdjNorm}{\ensuremath{\tilde{\mathbf{A}}}\xspace}
\providecommand{\Laplacian}{\ensuremath{\mathbf{L}}\xspace}
\providecommand{\LaplacianNorm}{\ensuremath{\tilde{\mathbf{L}}}\xspace}
\providecommand{\Degree}{\ensuremath{\mathbf{D}}\xspace}
\providecommand{\Identity}{\ensuremath{\mathbf{I}}\xspace}
\providecommand{\Neighbor}[1]{\ensuremath{\mathcal{N}(#1)}\xspace}

% 4.3 双图结构 (SPIN模型)
\providecommand{\AdjDiff}{\ensuremath{\tilde{\mathbf{A}}^{\mathcal{D}}}\xspace}
\providecommand{\AdjAdv}{\ensuremath{\tilde{\mathbf{A}}^{\mathcal{A}}}\xspace}
\providecommand{\GraphSpatial}{\ensuremath{\mathcal{G}^{\mathcal{S}}}\xspace}
\providecommand{\GraphAdvection}{\ensuremath{\mathcal{G}^{\mathcal{A}}}\xspace}
\providecommand{\GraphDiffusion}{\ensuremath{\mathcal{G}^{\mathcal{D}}}\xspace}

% 4.4 边特征
\providecommand{\EdgeWeight}{\ensuremath{w_{ij}}\xspace}
\providecommand{\EdgeFeature}{\ensuremath{\mathbf{e}_{ij}}\xspace}
\providecommand{\Distance}{\ensuremath{d_{ij}}\xspace}

% 4.5 时间参数
\providecommand{\HistLen}{\ensuremath{T'}\xspace}
\providecommand{\PredLen}{\ensuremath{T}\xspace}
\providecommand{\LeadTime}{\ensuremath{\tau}\xspace}
\providecommand{\TimeStep}{\ensuremath{\Delta t}\xspace}
\providecommand{\TimeIndex}{\ensuremath{t}\xspace}

% =========================================================
% 5. 损失函数与优化 (Loss Functions & Optimization)
% =========================================================
% 5.1 损失函数符号
\providecommand{\Loss}{\ensuremath{\mathcal{L}}\xspace}
\providecommand{\LossTotal}{\ensuremath{\mathcal{L}_{\mathrm{total}}}\xspace}
\providecommand{\LossSup}{\ensuremath{\mathcal{L}_{\mathrm{sup}}}\xspace}
\providecommand{\LossDIC}{\ensuremath{\mathcal{L}_{\mathrm{DIC}}}\xspace}
\providecommand{\LossAOD}{\ensuremath{\mathcal{L}_{\mathrm{AOD}}}\xspace}
\providecommand{\LossInfer}{\ensuremath{\mathcal{L}_{\mathrm{infer}}}\xspace}
\providecommand{\LossInit}{\ensuremath{\mathcal{L}_{\mathrm{init}}}\xspace}
\providecommand{\LossPred}{\ensuremath{\mathcal{L}_{\ell_1}}\xspace}
\providecommand{\LossLone}{\ensuremath{\mathcal{L}_{\ell_1}}\xspace}
\providecommand{\LossLtwo}{\ensuremath{\mathcal{L}_{\ell_2}}\xspace}
\providecommand{\LossMSE}{\ensuremath{\mathcal{L}_{\mathrm{MSE}}}\xspace}
\providecommand{\LossMAE}{\ensuremath{\mathcal{L}_{\mathrm{MAE}}}\xspace}
\providecommand{\LossSmooth}{\ensuremath{\mathcal{L}_{\mathrm{smooth}}}\xspace}
\providecommand{\LossPhys}{\ensuremath{\mathcal{L}_{\mathrm{phys}}}\xspace}

% 5.2 优化参数
\providecommand{\Params}{\ensuremath{\Theta}\xspace}
\providecommand{\LearnRate}{\ensuremath{\alpha}\xspace}
\providecommand{\RegWeight}{\ensuremath{\lambda}\xspace}
\providecommand{\BatchSize}{\ensuremath{B}\xspace}
\providecommand{\Epoch}{\ensuremath{E}\xspace}

% 5.3 神经网络层
\providecommand{\Linear}{\ensuremath{\mathrm{Linear}}\xspace}
\providecommand{\MLP}{\ensuremath{\mathrm{MLP}}\xspace}
\providecommand{\GRUcell}{\ensuremath{\mathrm{GRU}}\xspace}
\providecommand{\LSTMcell}{\ensuremath{\mathrm{LSTM}}\xspace}
\providecommand{\Softmax}{\ensuremath{\mathrm{softmax}}\xspace}
\providecommand{\Sigmoid}{\ensuremath{\sigma}\xspace}
\providecommand{\ReLU}{\ensuremath{\mathrm{ReLU}}\xspace}
\providecommand{\Tanh}{\ensuremath{\tanh}\xspace}

% =========================================================
% 6. 模型名称 (Model Names)
% =========================================================
% 6.1 本文提出的模型
\providecommand{\ModelPred}{\ensuremath{\mathrm{PM}_{2.5}}\text{-GNN}\xspace}
\providecommand{\ModelSurr}{PCDCNet\xspace}
\providecommand{\ModelInfer}{SPIN\xspace}
\providecommand{\ModelSim}{IGNN\xspace}
\providecommand{\ModelControl}{PCDCNet-IR\xspace}
\providecommand{\SystemName}{KnowAir\xspace}

% 6.2 基线模型
\providecommand{\GCLSTM}{GC-LSTM\xspace}
\providecommand{\STGCN}{STGCN\xspace}
\providecommand{\AirFormer}{AirFormer\xspace}
\providecommand{\IGNNK}{IGNNK\xspace}
\providecommand{\XGBoost}{XGBoost\xspace}
\providecommand{\LightGBM}{LightGBM\xspace}
\providecommand{\LSTM}{LSTM\xspace}
\providecommand{\GRU}{GRU\xspace}
\providecommand{\Transformer}{Transformer\xspace}
\providecommand{\TCN}{TCN\xspace}

% 6.3 气象AI模型
\providecommand{\Pangu}{Pangu-Weather\xspace}
\providecommand{\GraphCast}{GraphCast\xspace}
\providecommand{\FourCastNet}{FourCastNet\xspace}
\providecommand{\Aurora}{Aurora\xspace}

% =========================================================
% 7. 数据源与区域 (Data Sources & Regions)
% =========================================================
% 7.1 排放清单
\providecommand{\MEIC}{MEIC\xspace}
\providecommand{\DPEC}{DPEC\xspace}
\providecommand{\EDGAR}{EDGAR\xspace}
\providecommand{\ABaCAS}{ABaCAS\xspace}

% 7.2 气象数据
\providecommand{\ERAFive}{ERA5\xspace}
\providecommand{\GFS}{GFS\xspace}
\providecommand{\CMIP}{CMIP6\xspace}
\providecommand{\CAMS}{CAMS\xspace}
\providecommand{\ECMWF}{ECMWF\xspace}
\providecommand{\IFS}{IFS\xspace}

% 7.3 遥感数据
\providecommand{\Himawari}{Himawari-8\xspace}
\providecommand{\MODIS}{MODIS\xspace}
\providecommand{\VIIRS}{VIIRS\xspace}

% 7.4 数值模式
\providecommand{\CMAQ}{CMAQ\xspace}
\providecommand{\WRFChem}{WRF-Chem\xspace}
\providecommand{\WRF}{WRF\xspace}
\providecommand{\GEOSChem}{GEOS-Chem\xspace}
\providecommand{\CAMx}{CAMx\xspace}
\providecommand{\NAQPMS}{NAQPMS\xspace}

% 7.5 监测网络
\providecommand{\CNEMC}{CNEMC\xspace}

% 7.6 研究区域
\providecommand{\BTHSA}{BTHSA\xspace}
\providecommand{\YRD}{YRD\xspace}
\providecommand{\PRD}{PRD\xspace}
\providecommand{\NCUA}{NCUA\xspace}
\providecommand{\FWP}{FWP\xspace}

% 7.7 数据集名称
\providecommand{\KnowAirOne}{KnowAir-DS-V1\xspace}
\providecommand{\KnowAirTwo}{KnowAir-DS-V2\xspace}
\providecommand{\KnowAirDS}{KnowAir-DS\xspace}

% =========================================================
% 8. 评价指标 (Evaluation Metrics)
%    所有指标使用 \ensuremath 包裹
% =========================================================
% 8.1 回归指标
\providecommand{\RMSE}{\ensuremath{\mathrm{RMSE}}\xspace}
\providecommand{\MAE}{\ensuremath{\mathrm{MAE}}\xspace}
\providecommand{\MSE}{\ensuremath{\mathrm{MSE}}\xspace}
\providecommand{\MAPE}{\ensuremath{\mathrm{MAPE}}\xspace}
\providecommand{\Rsquare}{\ensuremath{R^{2}}\xspace}
\providecommand{\Corr}{\ensuremath{r}\xspace}
\providecommand{\CorrCoef}{\ensuremath{\rho}\xspace}

% 8.2 分类/检测指标
\providecommand{\CSI}{\ensuremath{\mathrm{CSI}}\xspace}
\providecommand{\POD}{\ensuremath{\mathrm{POD}}\xspace}
\providecommand{\FAR}{\ensuremath{\mathrm{FAR}}\xspace}
\providecommand{\ACC}{\ensuremath{\mathrm{ACC}}\xspace}
\providecommand{\Precision}{\ensuremath{\mathrm{Precision}}\xspace}
\providecommand{\Recall}{\ensuremath{\mathrm{Recall}}\xspace}
\providecommand{\Fone}{\ensuremath{F_{1}}\xspace}

% 8.3 环境评估指标
\providecommand{\NMB}{\ensuremath{\mathrm{NMB}}\xspace}
\providecommand{\NME}{\ensuremath{\mathrm{NME}}\xspace}
\providecommand{\MFB}{\ensuremath{\mathrm{MFB}}\xspace}
\providecommand{\MFE}{\ensuremath{\mathrm{MFE}}\xspace}
\providecommand{\IA}{\ensuremath{\mathrm{IA}}\xspace}
\providecommand{\IOA}{\ensuremath{\mathrm{IOA}}\xspace}
\providecommand{\MDAeight}{\ensuremath{\mathrm{MDA8}}\xspace}

% =========================================================
% 9. 数学算子与常用符号 (Math Operators)
% =========================================================
% 9.1 微分算子
\providecommand{\Grad}{\ensuremath{\nabla}\xspace}
\providecommand{\Div}{\ensuremath{\nabla \cdot}\xspace}
\providecommand{\Curl}{\ensuremath{\nabla \times}\xspace}
\providecommand{\Lapl}{\ensuremath{\nabla^2}\xspace}
\providecommand{\PartialT}{\ensuremath{\frac{\partial}{\partial t}}\xspace}
\providecommand{\PartialX}{\ensuremath{\frac{\partial}{\partial x}}\xspace}
\providecommand{\PartialY}{\ensuremath{\frac{\partial}{\partial y}}\xspace}
\providecommand{\PartialZ}{\ensuremath{\frac{\partial}{\partial z}}\xspace}

% 9.2 差分与增量
\providecommand{\Diff}{\ensuremath{\Delta}\xspace}
\providecommand{\DiffX}{\ensuremath{\Delta X}\xspace}
\providecommand{\DiffE}{\ensuremath{\Delta E}\xspace}
\providecommand{\DiffC}{\ensuremath{\Delta C}\xspace}

% 9.3 模型映射函数
\providecommand{\ModelFunc}{\ensuremath{\mathcal{F}}\xspace}
\providecommand{\EncFunc}{\ensuremath{\mathcal{E}}\xspace}
\providecommand{\DecFunc}{\ensuremath{\mathcal{D}}\xspace}

% 9.4 集合与空间符号
\providecommand{\Real}{\ensuremath{\mathbb{R}}\xspace}
\providecommand{\Integer}{\ensuremath{\mathbb{Z}}\xspace}
\providecommand{\Natural}{\ensuremath{\mathbb{N}}\xspace}
\providecommand{\Expect}{\ensuremath{\mathbb{E}}\xspace}
\providecommand{\Prob}{\ensuremath{\mathbb{P}}\xspace}

% 9.5 范数与距离
\providecommand{\Lnorm}[1]{\ensuremath{\|\cdot\|_{#1}}\xspace}
\providecommand{\LoneNorm}{\ensuremath{\|\cdot\|_{1}}\xspace}
\providecommand{\LtwoNorm}{\ensuremath{\|\cdot\|_{2}}\xspace}
\providecommand{\FrobNorm}{\ensuremath{\|\cdot\|_{F}}\xspace}

% 9.6 特殊标记 (需要 pifont 宏包)
\providecommand{\cmark}{\ding{51}}
\providecommand{\xmark}{\ding{55}}

% 9.7 常用缩写
\providecommand{\ie}{\textit{i.e.}\xspace}
\providecommand{\eg}{\textit{e.g.}\xspace}
\providecommand{\etc}{\textit{etc.}\xspace}
\providecommand{\etal}{\textit{et al.}\xspace}
\providecommand{\vs}{\textit{vs.}\xspace}
\providecommand{\wrt}{w.r.t.\xspace}
\providecommand{\iid}{\textit{i.i.d.}\xspace}

% =========================================================
% 10. 模块名称与方法缩写 (Module Abbreviations)
% =========================================================
% 10.1 PCDCNet架构模块
\providecommand{\LID}{LID\xspace}
\providecommand{\STD}{STD\xspace}
\providecommand{\TAD}{TAD\xspace}
\providecommand{\DIC}{DIC\xspace}

% 10.2 通用神经网络模块
\providecommand{\GNN}{GNN\xspace}
\providecommand{\GCN}{GCN\xspace}
\providecommand{\GAT}{GAT\xspace}
\providecommand{\RNN}{RNN\xspace}
\providecommand{\CNN}{CNN\xspace}
\providecommand{\MHA}{MHA\xspace}
\providecommand{\FFN}{FFN\xspace}

% 10.3 化学机制
\providecommand{\CBsix}{CB6\xspace}
\providecommand{\SAPRC}{SAPRC\xspace}
\providecommand{\ISORROPIA}{ISORROPIA\xspace}

% 10.4 数据同化方法
\providecommand{\ISAM}{ISAM\xspace}
\providecommand{\DDM}{DDM\xspace}
\providecommand{\PSAT}{PSAT\xspace}
\providecommand{\OSAT}{OSAT\xspace}

% =========================================================
% 11. 物理过程术语 (Physical Process Terms)
% =========================================================
% 11.1 大气过程
\providecommand{\Advection}{\ensuremath{\mathrm{Adv}}\xspace}
\providecommand{\Diffusion}{\ensuremath{\mathrm{Diff}}\xspace}
\providecommand{\Emission}{\ensuremath{\mathrm{Emis}}\xspace}
\providecommand{\Deposition}{\ensuremath{\mathrm{Dep}}\xspace}
\providecommand{\DryDep}{\ensuremath{\mathrm{Dry}}\xspace}
\providecommand{\WetDep}{\ensuremath{\mathrm{Wet}}\xspace}
\providecommand{\ChemReaction}{\ensuremath{\mathrm{Chem}}\xspace}

% 11.2 边界层参数
\providecommand{\PBL}{\ensuremath{\mathrm{PBL}}\xspace}
\providecommand{\PBLH}{\ensuremath{h_{\mathrm{PBL}}}\xspace}

% =========================================================
% 12. 情景与政策术语 (Scenarios & Policies)
% =========================================================
% 12.1 排放情景
\providecommand{\RCP}{\ensuremath{\mathrm{RCP}}\xspace}
\providecommand{\SSP}{\ensuremath{\mathrm{SSP}}\xspace}
\providecommand{\BAU}{\ensuremath{\mathrm{BAU}}\xspace}

% 12.2 政策目标
\providecommand{\CarbonPeak}{碳达峰\xspace}
\providecommand{\CarbonNeutral}{碳中和\xspace}

% =========================================================
% 13. 文献引用格式辅助
%     用于表格中的简化引用
% =========================================================
\providecommand{\citemark}[1]{\textsuperscript{#1}}

% =========================================================
% 文件结束
% =========================================================


\begin{document}

% 封面
\maketitle

% 使用授权的说明
% \copyrightpage
% 将签字扫描后授权文件 scan-copyright.pdf 替换原始页面
\copyrightpage[file=scan-copyright.pdf]


\frontmatter
% !TeX root = ../wangshuo_phdthesis.tex
% 中英文摘要和关键字

\begin{abstract}
\bnusetup{
keywords = {地球系统科学, 数据驱动建模, 大气污染智能预测, 物理启发深度学习, 时空图神经网络},
}

大气污染是当前全球最严峻的环境与公共卫生挑战之一,每年导致约700万人过早死亡。我国\PM 年均浓度从2013年的72~\ug 降至2023年的30~\ug{},但与世界卫生组织指导值(5~\ug{})仍有较大差距;\PM 持续下降而\ozone 浓度逐年攀升的分化态势,更给协同治理带来了新的科学难题。如何实现高精度的空气质量预测、如何在无观测区域推断污染浓度分布、如何高效模拟不同减排政策下的空气质量演变,已成为支撑精细化污染防控的三项关键科学问题。

从系统科学视角审视,大气污染具备复杂系统的典型特征:排放、气象与化学演化三大子系统紧密耦合,跨区域传输沿地形廊道形成有向时变网络,多源数据在时空结构与物理内涵上差异显著。化学传输模式虽具物理可解释性,却受困于计算代价高昂、对排放清单高度依赖等瓶颈;纯数据驱动方法虽计算高效,却缺乏物理规律的显式建模,在极端情景下泛化能力不足。如何融合两者优势、构建兼具可解释性与高效性的智能建模框架,是当前大气环境领域的前沿科学问题。

针对上述困境,本研究提出物理启发的图神经网络建模范式。该范式以对流--扩散方程为物理基础,以图神经网络为核心建模工具,通过图结构的构建与约束函数的设计,将对流--扩散方程等物理先验知识系统性地嵌入深度学习框架。围绕空气质量预测、无观测区域浓度推断与减排情景模拟三类核心任务,本文整合地面监测、卫星遥感、气象与排放清单四类数据源,发展了一套物理与数据深度融合的空气质量智能建模方法体系。

在空气质量预测方面,本文提出\ModelPred 与\ModelSurr 两个模型。\ModelPred 利用风速向量投影定义图边权重,在图神经网络框架下显式构建\cqt{上风向影响下风向}的定向传输关系,首次将大气污染物沿风场传播的物理规律融入图结构设计。\ModelSurr 从对流--扩散方程出发,设计三模块架构分别对应化学反应、平流扩散与沉降累积三个物理过程,实现过程解耦与联合建模;同时提出领域一致性约束(DIC),将质量守恒等物理规律作为约束损失函数嵌入训练过程,保障预测结果的物理一致性。在京津冀及长三角区域72小时\PM 与\ozone 联合预报中,相比iTransformer、GC-LSTM等主流时空预测模型,RMSE降低13\%--23\%,重污染过程的峰值捕捉能力显著提升。

在无观测区域推断方面,本文提出\ModelInfer 模型,旨在基于稀疏地面监测数据重建高分辨率全域污染场。该模型构建扩散--平流双图并行机制,在图神经网络中实现对流--扩散过程的解耦表达;提出\cqt{以卫星AOD空间梯度为约束而非输入}的融合策略,通过掩码机制规避遥感缺测影响,实现全天候连续制图。在京津冀地区30\%站点缺测条件下,相比IGNNK、STGCN等基准模型,MAE降低约25\%。

在假设情景模拟方面,本文构建\ModelSim 模型,首次将污染物排放作为模型的输入引入,使模型既能预测污染物浓度,也能同时实现不同排放分布下的污染情景模拟。相比传统化学传输模式数小时的计算代价,本模型将单情景推理压缩至秒级。基于中国双碳排放路径数据集(DPEC),揭示了2025--2050年碳中和路径下\PM 持续下降而\ozone 普遍上升的反向演变规律,为协同减排决策提供科学依据。

在工程应用方面,本文基于云原生架构部署\SystemName 系统,该系统已服务于国家重大活动空气质量保障任务,并在中国环境监测总站官方模型比对中取得综合评分中位数第一名,验证了上述方法的实用性与可靠性。

本研究的核心贡献在于:提出物理启发的图神经网络建模范式,将图神经网络的关系建模能力与大气污染传输的网络化特征相结合,发展了预测、推断、模拟三位一体的空气质量智能建模方法体系。主要创新包括:(1)在图结构中显式编码\cqt{上风向影响下风向}的定向传输机制,并提出领域一致性约束(DIC)将物理守恒律嵌入模型训练;(2)提出以AOD空间梯度为约束的融合策略实现全天候推断;(3)首次将排放清单作为可控输入纳入深度学习框架,赋能假设情景模拟。完成从理论方法到业务系统的全链条落地,推动大气污染建模从\cqt{能预测}向\cqt{能推断、能模拟}跃升,为空气质量精细化管理提供智能技术支撑。本研究所发展的\cqt{物理先验驱动图结构与约束学习}方法论,可推广至地球科学乃至系统科学中涉及多源数据融合与复杂时空动力学建模的广泛问题,具有普适的理论与方法价值。

\end{abstract}

\begin{abstract*}
\bnusetup{
keywords* = {Earth System Science, Data-Driven Modeling, Intelligent Air Pollution Prediction, Physics-Inspired Deep Learning, Spatiotemporal Graph Neural Network},
}

Air pollution represents one of the most pressing environmental and public health challenges worldwide, causing approximately 7 million premature deaths annually. China's national average \PM concentration has declined from 72~\ug in 2013 to 30~\ug in 2023, yet a considerable gap remains compared to the WHO guideline of 5~\ug. Meanwhile, the divergent trends of declining \PM and rising \ozone pose new scientific challenges for synergistic pollution control. How to achieve accurate air quality prediction, how to infer pollution concentration in unmonitored areas, and how to efficiently simulate air quality evolution under different emission reduction policies have become three critical scientific problems supporting refined pollution prevention and control.

From a systems science perspective, atmospheric pollution exhibits typical characteristics of a complex system: emissions, meteorology, and chemical evolution form tightly coupled subsystems; cross-regional transport follows directed, time-varying networks shaped by terrain corridors; and multi-source data differ substantially in spatiotemporal structure and physical interpretation. Traditional chemical transport models possess physical interpretability but are constrained by high computational costs and strong emission dependence. Pure data-driven approaches, while computationally efficient, lack explicit physical modeling and demonstrate insufficient generalization under extreme events. How to integrate the respective advantages of physical knowledge and data-driven methods to construct an intelligent modeling framework with both interpretability and efficiency represents a frontier scientific challenge in the atmospheric environment field.

To address these challenges, this dissertation proposes a physics-inspired graph neural network (GNN) modeling paradigm. Built on the advection-diffusion equation as the physical foundation and graph neural networks as the core modeling tool, this paradigm systematically embeds physical prior knowledge---such as the advection-diffusion equation---into the deep learning framework through physics-informed graph construction and constraint function design. Targeting three core tasks---air quality prediction, pollution inference in unmonitored areas, and emission scenario simulation---this work integrates ground monitoring, satellite remote sensing, meteorological data, and emission inventories, developing a methodology that deeply fuses physical knowledge with data-driven modeling for air quality research.

For air quality prediction, \ModelPred and \ModelSurr are proposed. \ModelPred utilizes wind velocity projections to define graph edge weights, explicitly constructing ``upwind-to-downwind'' directional transport relationships within the GNN framework, pioneering the incorporation of wind-driven pollutant transport physics into graph structure design. \ModelSurr, grounded in the advection-diffusion equation, designs a three-module architecture corresponding to chemical reactions, advection-diffusion, and deposition-accumulation processes respectively, achieving process decomposition with joint modeling. Domain-Informed Constraints (DIC) are proposed to embed mass conservation and other physical laws as constraint loss functions into the training process, ensuring physical consistency of predictions. For 72-hour joint \PM and \ozone forecasting in the BTH and YRD regions, \RMSE is reduced by 13\%--23\% compared to mainstream spatiotemporal models such as iTransformer and GC-LSTM, with notably improved peak capture capability during heavy pollution episodes.

For pollution inference in unmonitored areas, \ModelInfer is proposed to reconstruct high-resolution regional pollution fields from sparse ground monitoring data. The model constructs a diffusion-advection dual-graph parallel mechanism to achieve decoupled representation of advection-diffusion processes within graph neural networks, and introduces an ``AOD spatial gradients as constraints rather than inputs'' fusion strategy that leverages satellite remote sensing gradient information to guide spatial inference while using masking mechanisms to handle data gaps, enabling all-weather continuous mapping. Under 30\% station missing conditions in the BTH region, \MAE is reduced by approximately 25\% compared to baselines such as IGNNK and STGCN.

For hypothetical scenario simulation, \ModelSim introduces pollutant emissions as model inputs for the first time, enabling the model to both predict pollutant concentrations and simultaneously simulate pollution scenarios under different emission distributions. Compared to traditional chemical transport models requiring hours of computation, this model compresses single-scenario inference to seconds. Using the China DPEC dataset, the model reveals opposing evolution patterns under 2025--2050 carbon neutrality pathways where \PM continues declining while \ozone generally rises, providing scientific basis for synergistic emission reduction decisions.

For engineering applications, the cloud-native \SystemName system is deployed, which has served air quality assurance for major national events and achieved the highest median composite score in official model comparison evaluations organized by CNEMC, validating the practicality and reliability of the proposed methods.

The core contributions of this research are: proposing a physics-inspired graph neural network modeling paradigm that combines the relational modeling capability of GNNs with the networked nature of atmospheric pollutant transport, developing an integrated air quality intelligent modeling methodology encompassing prediction, inference, and simulation. Key innovations include: (1) explicitly encoding ``upwind-to-downwind'' directional transport mechanisms in graph structures and proposing Domain-Informed Constraints (DIC) to embed physical conservation laws into model training; (2) proposing an AOD spatial gradient-as-constraint fusion strategy for all-weather inference; (3) incorporating emission inventories as controllable inputs into the deep learning framework for the first time to enable hypothetical scenario simulation. Full-chain implementation from theoretical methods to operational systems is completed, advancing air pollution modeling from ``capable of prediction'' to ``capable of inference and simulation,'' providing intelligent technology support for refined air quality management. The ``physics-prior-driven graph structure and constraint learning'' methodology developed in this research can be extended to a broad range of problems in earth science and systems science involving multi-source data fusion and complex spatiotemporal dynamics modeling, offering generalizable theoretical and methodological value.

\end{abstract*}

% 目录
\tableofcontents

% 插图和附表清单
% \listoffiguresandtables  % 插图和附表清单
% 如图表较多,可以分别列出清单置于目录页之后。
\listoffigures*           % 插图清单
\listoftables*            % 附表清单
\listofalgorithms*        % 算法清单

% 符号对照表
% !TeX root = ../wangshuo_phdthesis.tex

\begin{denotation}[3cm]

% 1. 核心变量集合 (合并了具体的污染物、气象、排放子项,大幅减少行数)
\item[\(\mathbf{X}^t, \hat{\mathbf{X}}^t\)] 时间 \(t\) 的空气质量观测值与预测值集合(单位:\(\mu\)g/m\(^3\))。$\mathbf{X} \in \mathbb{R}^{N \times D_X}$,其中 $D_X$ 为污染物种类数,包含 PM\(_{2.5}\)、PM\(_{10}\)、O\(_3\)、NO\(_2\) 等。
\item[\(\mathbf{M}^t, \mathbf{E}^t\)] 时间 \(t\) 的气象变量集合与排放变量集合。
    \begin{itemize}
        \item \(\mathbf{M}^t\) 包含:气温 ($M_{\text{t2m}}$)、露点 ($M_{\text{d2m}}$)、边界层高度 ($M_{\text{blh}}$)、风速 ($M_{u}$, $M_{v}$) 等
        \item \(\mathbf{E}^t\) 包含:NO\(_x\) ($E_{\text{NO}_x}$)、VOC ($E_{\text{VOC}}$)、SO\(_2\) ($E_{\text{SO}_2}$) 等
    \end{itemize}
    
\item[\(\mathbf{X}_F^t, \mathbf{M}_F^t, \mathbf{E}_F^t\)] 对应上述变量的连续空间场(Field),表示在整个研究区域内的连续分布,用于网格化推断与模拟任务。
\item[\(\mathbf{X}^{\text{AOD}}\)] 遥感观测的气溶胶光学厚度(Aerosol Optical Depth),作为对污染物空间分布的间接约束。附带有效性掩码 $\boldsymbol{\Omega}^{\text{AOD}} \in \{0,1\}^{N}$,用于标记云层遮挡等无效观测区域。

% 2. 图结构与时空参数
\item[\(\mathcal{G}_t = (\mathcal{V}, \mathcal{E}_t)\)] 动态图结构,\(\mathcal{V}\) 为站点/城市节点集合,\(\mathcal{E}_t\) 为基于风场或距离构建的边集合。
\item[\(\mathbf{A}, \tilde{\mathbf{A}}\)] 图邻接矩阵与归一化后的邻接矩阵。
\item[\(\mathbf{L}, \tilde{\mathbf{L}}\)] 图拉普拉斯矩阵$\mathbf{L} = \mathbf{D} - \mathbf{A}$与归一化拉普拉斯矩阵,用于谱图卷积。
\item[\(\boldsymbol{\Lambda}, \mathbf{U}\)] 拉普拉斯矩阵的特征值对角矩阵与特征向量矩阵,满足$\mathbf{L} = \mathbf{U} \boldsymbol{\Lambda} \mathbf{U}^\top$。
\item[\(g_\theta(\cdot)\)] 可学习的频谱滤波器,作用于图的频谱域。
\item[\(K\)] 大气扩散系数,用于对流-扩散方程中的扩散项。
\item[\(t=0\)] 起报时刻(Forecast Initialization Time),即最后一个可获取空气质量观测数据的时间戳。\(t \leq 0\) 表示历史时刻,\(t>0\) 表示未来待预测时刻。
\item[\(T', T\)] 历史输入窗口长度与未来预测窗口长度。\(T'\) 个历史时间步用于捕捉时序演化规律,\(T\) 个未来时间步为预测目标。
\item[\(\tau\)] 预测步长(Lead Time),即从起报时刻到目标预测时刻的时间间隔。

% 3. 模型、物理算子与优化
\item[\(f, \mathcal{F}_\Theta\)] 真实的空气质量生成过程 \(f(\mathbf{M}, \mathbf{E})\) 与参数为 \(\Theta\) 的深度学习近似模型。
\item[\(C\)] 污染物浓度场的点值形式,用于对流-扩散方程表述。与向量形式$\mathbf{X}$的关系:$C_i = \mathbf{X}_i$。
\item[\(\mathbf{u}\)] 风速矢量场,用于计算平流传输。
\item[\(\mathbf{H}^{(l)}, \mathbf{h}_i\)] 第$l$层的节点隐表示,$\mathbf{H}^{(l)} \in \mathbb{R}^{N \times F}$,$\mathbf{h}_i$为节点$i$的隐向量。
\item[\(\mathcal{V}_{\text{obs}}, \mathcal{V}_{\text{target}}\)] 观测节点集与目标推断节点集,满足$\mathcal{V}_{\text{obs}} \cup \mathcal{V}_{\text{target}} = \mathcal{V}$。
\item[\(\mathcal{T}_{\text{hist}}, \mathcal{T}_{\text{future}}\)] 历史时间集与未来时间集,分别用于模型训练与推理/模拟。
\item[\(\nabla, \increment\)] 空间梯度算子(用于扩散/平流约束)与时间差分算子。
\item[\(\mathcal{L}_{\mathrm{total}}\)] 总损失函数,由监督损失与物理约束损失组成。
\item[\(\mathcal{L}_{\mathrm{Pred}}, \mathcal{L}_{\mathrm{DIC}}\)] 预测损失(L1误差)与领域一致性约束损失(第\ref{chap:prediction}章PCDCNet)。
\item[\(\mathcal{L}_{\mathrm{infer}}, \mathcal{L}_{\mathrm{init}}, \mathcal{L}_{\mathrm{AOD}}\)] 推断损失、初始化损失与AOD梯度约束损失(第\ref{chap:inference}章SPIN)。

% 4. 缩略语 (按字母顺序排列)
% 4. 通用缩略语 (Abbreviations)
\item[AOD] Aerosol Optical Depth (气溶胶光学厚度),在部分文献中亦称为 AOT (Aerosol Optical Thickness)。
\item[AQF] Air Quality Forecast (空气质量预报)。
\item[CMAQ] Community Multiscale Air Quality (社区多尺度空气质量模型),传统的数值化学传输模式。
\item[CTM] Chemical Transport Model (化学传输模式),基于大气物理化学机理方程求解的数值模型统称,如 CMAQ 和 WRF-Chem。
\item[CMIP6] Coupled Model Intercomparison Project Phase 6 (第六次国际耦合模式比较计划),本文使用其 SSP-RCP 情景下的气象数据进行未来模拟。
\item[MEIC] Multi-resolution Emission Inventory for China (中国多尺度排放清单模型)。
\item[DPEC] Dynamic Projection model for Emissions in China (中国未来排放动态评估模型),本文使用其与 CMIP6 耦合的未来排放清单数据。
\item[ECMWF / ERA5] European Centre for Medium-Range Weather Forecasts 及其第五代再分析数据。
\item[GFS] Global Forecast System (全球预报系统),由美国国家海洋和大气管理局 (NOAA) 运行的全球数值天气预报系统。\item[GNN / GCN] Graph Neural Network / Graph Convolutional Network (图神经网络/图卷积网络)。
\item[GRU / LSTM] Gated Recurrent Unit / Long Short-Term Memory (门控循环单元/长短期记忆网络),处理时间序列的 RNN 变体。
\item[PCDCNet] Physical-Chemical Dynamics and Constraints Network (物理-化学动力学约束网络),本文提出的正向预测模型。
\item[DIC] Domain-Informed Constraints (领域一致性约束),基于质量守恒原理设计的物理损失函数(第\ref{chap:prediction}章)。
\item[SPIN] Spatiotemporal Physics-Guided Inference Network (物理引导时空推断网络),本文提出的无缝推断模型。
\item[IGNN] Integrated Graph Neural Network (集成图神经网络),本文用于长期情景模拟的模型。
\item[KnowAir] 领域知识驱动的空气质量智能预报系统,涵盖数据融合、模型推理与可视化展示等模块。
\item[KnowAir-DS] KnowAir 配套数据集,包含站点观测、气象再分析及排放清单等多源数据。目前发布两个版本:KnowAir-DS-V1和KnowAir-DS-V2。
\item[SOTA] State-of-the-Art (当前最佳水平)。

\end{denotation}


% 正文部分
\mainmatter
%!TEX encoding = UTF-8 Unicode
%!TEX program = xelatex

% ============================================================
% 第一章:绪论
% 基于复杂系统数据驱动建模的大气污染研究
% ============================================================

\chapter{绪论}
\label{chap:introduction}

\section{研究背景及意义}
\label{sec:background}

\subsection{大气污染问题的全球性挑战}
\label{subsec:global_challenge}

当前,大气污染已成为影响人类生存与发展的重大环境议题。世界卫生组织(World Health Organization, WHO)发布的监测数据显示,全球约有99\%的居民所处的空气质量条件未能达到该组织制定的健康指导标准\footnote{\url{https://www.who.int/news/item/04-04-2022-billions-of-people-still-breathe-unhealthy-air-new-who-data}}。每年因空气污染引发的过早死亡人数高达约700万\citep{world2021global,murray2020global}。在众多大气污染因子中,细颗粒物(Fine Particulate Matter, \PM)与近地面臭氧(Ground-Level Ozone, \ozone)由于其对公众健康造成的严重威胁,始终是科学研究和环境治理的重点关注对象。\PM 的空气动力学直径不超过2.5微米,这种极微小的粒径特性使其能够突破人体呼吸系统的防护屏障,抵达肺泡深处,进而随血液循环对各器官产生影响。大量流行病学研究证实,\PM 暴露与心血管病变、呼吸道疾患以及肺部恶性肿瘤的发生存在显著正相关\citep{burnett2018global}。\ozone 具有强烈的氧化性,持续暴露于高浓度环境会削弱呼吸系统机能,同时对农业生产造成减产威胁\citep{nuvolone2018effects,mills2018ozone}。

作为全球最大的发展中经济体,中国在应对大气污染问题上的探索具有重要的示范意义。伴随着工业化与城镇化进程的快速推进,经济高速增长的同时也积累了突出的空气质量问题。面对这一挑战,中央政府实施了层层递进的治理举措。2013年,国务院颁布《大气污染防治行动计划》(\cqt{大气十条})\footnote{\url{https://www.gov.cn/zwgk/2013-09/12/content_2486773.htm}},首次设立\PM 浓度的刚性考核目标\citep{china2013action}。2018年,《打赢蓝天保卫战三年行动计划》\footnote{\url{https://www.gov.cn/zhengce/content/2018-07/03/content_5303158.htm}}进一步强化治理力度。2023年,《空气质量持续改善行动计划》\footnote{\url{https://www.gov.cn/zhengce/content/202312/content_6919000.htm}}将\PM 削减确立为核心主线,推进NOx与VOCs协同减排,力图实现减污降碳双重目标\citep{jiang2021government}。值得一提的是,清华大学研发的中国多尺度排放清单(Multi-resolution Emission Inventory for China, MEIC)在污染源头解析与减排成效评估中发挥了重要的数据支撑作用\citep{zheng2018trends,geng2024efficacy}。历经十余年坚持不懈的努力,全国\PM 年均浓度从2013年的72 $\mu$g/m$^3$\footnote{\url{https://www.mee.gov.cn/gkml/sthjbgw/qt/201403/t20140325_269648.htm}}大幅下降至2023年的30 $\mu$g/m$^3$\footnote{\url{https://www.mee.gov.cn/hjzl/sthjzk/zghjzkgb/202406/P020240604551536165161.pdf}},重污染天气频次明显减少,空气质量改善速度在全球范围内首屈一指\citep{zhang2019drivers,geng2024efficacy,zhao2024challenges,cn_lishaolin2023daqishitiao}。

尽管如此,当前空气质量距WHO于2021年修订的\PM 年均浓度指导值(5 \ug)尚有相当差距\citep{world2021global}。就区域分布而言,京津冀及其周边、长三角、汾渭平原等敏感区域在秋冬季节仍面临重污染天气的侵扰\citep{chen2020influence,cn_yanli2021quyu};就污染类型而言,\ozone 问题的严峻程度与日俱增,呈现出\PM 稳步下行而\ozone 逐年攀升的差异化走势\citep{wang2020contrasting,liu2023drivers}。\PM 与\ozone 的联合管控遭遇深层次的科学难题:二者拥有NOx和VOCs这两类共同的前驱物质,在实际减排过程中呈现所谓\cqt{跷跷板效应}。其主要机理在于:随着\PM 浓度下降,气溶胶对太阳辐射的散射和吸收作用减弱,到达近地面的光合有效辐射增强,加速了光化学反应进程,从而促进\ozone 的生成\citep{qu2023underlying,li2023spatiotemporal,cn_lihong2019pm25o3}。这种错综复杂的大气化学关联使得针对单一污染物的减排策略往往难以取得理想效果,迫切需要研发能够表征多污染物相互作用机制的新型建模方法。与此同时,随着背景浓度的持续改善,重污染事件逐渐从周期性常态转变为偶发性、突发性特征,对污染预测的精度和时效性提出了更高要求。

从宏观战略角度审视,大气污染治理与碳达峰碳中和目标深度交织、相互影响。化石能源燃烧过程既排放CO$_2$,也产生SO$_2$、NOx、PM等空气污染物,减污与降碳在实现路径和政策手段上存在广阔的协同空间\citep{shi2022co,li2024synergistic}。2022年6月,生态环境部会同六部门联合发布《减污降碳协同增效实施方案》\footnote{\url{https://www.mee.gov.cn/xxgk2018/xxgk/xxgk03/202206/t20220617_985879.html}},明确提出\cqt{系统提升环境治理综合效能,实现环境、气候、经济效益协同共赢}的总体要求,标志着我国生态文明建设迈入减污降碳一体推进的新阶段\citep{cn_guochangsheng2023xietong}。在\cqt{2030年前实现碳达峰、2060年前实现碳中和}的战略部署下\footnote{\url{https://www.gov.cn/zhengce/202511/content_7047492.htm}},能源结构和产业格局必将经历深刻重塑。不同碳减排路径下空气质量将如何演进、怎样达成\PM 与\ozone 的协同改善,既是学术研究的前沿命题,也是关系国家发展的重大决策问题\citep{cheng2021pathways,sun2024air}。

综合以上分析,大气污染与公众健康、区域环境质量密切相关。在污染浓度整体下降但复合污染特征日益凸显的背景下,研发高精度、实时响应的大气污染建模与预测方法,对于推进空气质量科学研究与环境管理实践具有重要的理论和应用价值。

\subsection{大气污染的复杂系统特性}
\label{subsec:complex_system}

运用系统科学(System Science)的分析视角,大气污染体系远非简单的线性叠加系统可以描述,而是具备\cqt{开放复杂巨系统}(Open Complex Giant System)的典型特征\citep{xuesen1993new}。按照钱学森先生对开放复杂巨系统的界定,大气污染系统在以下维度上呈现典型特征:\textbf{开放性}——系统与外界持续进行物质(污染物排放与沉降)、能量(太阳辐射与地表热通量)和信息(监测与预报反馈)的交换;\textbf{复杂性}——排放、气象、化学三大子系统紧密耦合,形成非线性反馈回路;\textbf{巨量性}——涉及数以千计的监测站点、数百种化学物种、以及多源异构的海量观测数据;\textbf{涌现性}——区域性重污染事件表现为城市群的同步响应,非单个节点行为的简单叠加\citep{cn_wangshang2023}。系统把握这些复杂性表征,是突破现有建模瓶颈、发展新一代智能预测模型的必要前提。

\subsubsection{多要素强耦合}

大气污染物在时间和空间维度上的分布并非孤立的物理化学现象,而是排放源(Emissions)、气象条件(Meteorology)与化学演化(Chemistry)三大子系统紧密交织、相互作用的结果,各过程之间形成了复杂的反馈回路\citep{jacob2000heterogeneous,seinfeld2016atmospheric}。

气象要素从根本上制约着污染物的输送、混合与清除效率。当出现静稳型天气时,边界层高度显著压低,仿佛在城市上空形成一个\cqt{穹顶},致使污染物急速累积\citep{zhong2018feedback}。与之相反,高浓度气溶胶通过散射与吸收入射太阳辐射(即气溶胶--辐射相互作用),打破地表能量收支平衡,进一步削弱边界层的发展能力,形成\cqt{污染积聚$\to$辐射衰减$\to$边界层收缩$\to$污染加重}的恶性循环链条\citep{huang2014high,wang2016persistent}。相关研究揭示,气象因子对中国\PM 浓度的年际波动具有相当强的解释力\citep{zhang2019drivers,chen2020influence}。

人为排放活动的时空分布特征同样受到气象场的显著调制。夏季高温热浪期间居民空调用电负荷骤增,生物源VOCs的自然释放也随之加快,两者叠加效应极易催生光化学烟雾事件\citep{ma2019substantial};冬季取暖时段排放强度大幅增加,若再遇上静稳天气,便构成\PM 重污染的主要触发条件\citep{chen2020influence}。上述多要素耦合特性决定了建模方法必须在统一框架内综合表征排放、气象与监测数据等多元信息,而不能孤立地处理任何单一因子。

\subsubsection{动力学的高度非线性}

大气化学反应过程呈现出极为显著的非线性动力学行为,系统输入端(排放)与输出端(浓度)之间非简单的比例对应关系\citep{seinfeld2016atmospheric}。

\ozone 生成速率对其前驱物NOx与VOCs的非线性响应是大气化学非线性现象的标志性案例。研究证实,\ozone 的净生成取决于NOx与VOCs的相对配比,其等浓度响应曲线呈现明显的\cqt{山脊}形态\citep{sillman1999relation}。在VOC敏感区(如城市核心地带,VOCs/NOx比值偏低的情形),削减NOx反而可能因滴定作用减弱而推高\ozone 浓度;而在NOx敏感区(如郊区或远郊),削减NOx则可有效压低\ozone 水平\citep{xing2020deep,liu2023drivers}。只有当减排力度足够大、驱使系统跨越\cqt{分水岭}之后,继续降低NOx排放才能实现\ozone 的持续下降\citep{huang2020large}。类似地,二次气溶胶(涵盖硫酸盐、硝酸盐、铵盐及二次有机气溶胶)的形成涉及气相--液相--固相三态转化,受相对湿度、气溶胶酸碱度、氧化剂浓度等多重因素的非线性调节\citep{guo2014elucidating,wang2016persistent}。这种非线性特质意味着,基于简单线性外推的减排策略可能适得其反。深度神经网络作为通用的非线性函数逼近器,为刻画排放--浓度之间错综复杂的响应关系开辟了新的技术途径。

\subsubsection{复杂网络拓扑与涌现性}

污染物的跨区域迁移并非各向均匀的扩散过程,而是沿着特定地形廊道和大气流场形成有向性、时变性、长程相关的复杂传输网络\citep{quan2020regional,chen2020influence}。以京津冀区域为例,研究表明北京市\PM 污染中有30--50\%的比重源自周边区域的输入贡献\citep{chang2018assessment}。在典型\cqt{偏南气流输送型}重污染过程中,污染气团从河北中南部沿太行山东麓向北移动,途经石家庄、保定直达北京,形成绵延数百公里的\cqt{污染传输走廊}\citep{yin2025regional,quan2020regional,cn_liulin2017wrfchem}。这种传输表现出强烈的方向性和定向性,且随着大气流场的变化而动态调整路径。

借助网络科学的概念框架,监测站点之间的空间关联可被抽象为非欧几里得图结构:节点对应各监测点位,边权重反映传输强弱,天然适合表达\cqt{空间邻近但因山脉阻隔而传输微弱}等复杂的空间关系\citep{wang2020pm2,wu2020comprehensive}。图神经网络通过在此类图结构上执行信息传播与聚合运算,为非欧式空间数据建模提供了理想的计算框架\citep{kipf2017semi,velivckovic2018graph}。区域性重污染往往表现为城市群的\cqt{同步响应}现象,呈现出典型的涌现性特征:单个城市的治理成效不仅取决于本地减排行动,更受制于整个区域传输网络的系统性行为\citep{wang2017higher},必须运用图论和网络科学方法进行整体建模与分析。

\subsubsection{多源异构数据特征}

现代大气环境监测体系依托多种观测技术手段,汇聚形成了丰富但高度异构的数据资源\citep{geng2021tracking}。地面监测网络(如中国国家空气质量监测网下辖约1618个国控站点\footnote{\url{https://www.mee.gov.cn/xxgk2018/xxgk/xxgk06/202509/W020250905372029416580.pdf}})可提供精度较高但空间分布稀疏的污染物浓度实测值;卫星遥感平台(如MODIS搭载的MAIAC算法\footnote{\url{https://modis-land.gsfc.nasa.gov/MAIAC.html}}、静止卫星Himawari-8反演的气溶胶光学厚度AOD\footnote{\url{https://www.eorc.jaxa.jp/ptree/index.html}})能够获取空间连续但易受云层遮挡干扰的大气成分信息\citep{xiao2017full,wei2021himawari}。气象再分析资料(如ERA5)以格点形式提供驱动场数据\citep{hersbach2020era5};排放清单(如MEIC)描绘污染物前驱体的时空分布状况,但存在较大不确定性且更新周期较长\citep{zheng2018trends}。上述数据源在空间结构、时间分辨率、质量控制水平和物理内涵等方面差异明显。如何将多源异质信息在统一的潜在空间中实现有效融合与优势互补,构成了数据驱动建模亟待攻克的核心难题。

综上所述,大气污染体系的多要素耦合、动力学非线性、网络结构复杂以及数据多源异构等特性,共同界定了建模方法必须正视的核心挑战。如何在统一的理论框架内有效表征上述复杂性,是大气污染预测与模拟研究的关键科学问题。

\subsection{传统数值模型的原理与局限}
\label{subsec:traditional_models}

化学传输模型(Chemical Transport Models, CTMs)是大气环境科学领域模拟污染物时空演变过程的经典工具,其代表性模式包括社区多尺度空气质量模型(Community Multiscale Air Quality, CMAQ)\citep{byun2006review}、天气研究与预报--化学耦合模式(Weather Research and Forecasting model coupled with Chemistry, WRF-Chem)\citep{grell2005fully}以及全球大气化学传输模型(GEOS-Chem)\citep{bey2001global}等。历经数十年的迭代与完善,CTMs已成为空气质量业务预报、污染源头解析以及减排方案评估的核心技术支撑,在环境管理实践中发挥着难以替代的重要作用。

CTMs的数学基础是对流--扩散方程(Advection-Diffusion Equation),该方程从质量守恒定律出发,刻画污染物在大气中伴随风场输送(平流)、湍流混合(扩散)、化学转化、源排放以及沉降清除等过程的动态平衡关系(详见第\ref{chap:methodology}章)。CTMs通过数值方法在三维计算网格上求解该方程,显式地模拟污染物的排放、传输、化学转化与沉降过程,因此具备较好的物理可解释性与过程一致性。

然而,这种机理驱动的数值模型在实际应用中遭遇多重瓶颈制约。

\textbf{计算资源消耗巨大}是CTMs面临的首要障碍。化学传输模型需要在三维网格上对数十种污染物同步进行时间积分,每个时间步长都需完成平流、扩散、化学反应等多个子模块的求解。以公式(\ref{eq:advection_diffusion})中的化学反应项为例,其计算涉及包含数百个反应方程的化学机制(如CB6、SAPRC-07),运算量极为可观\citep{appel2021community}。一次常规的CMAQ区域模拟(例如针对京津冀及周边区域,采用4 km水平分辨率,模拟时长一周)需要在高性能计算平台上运行数小时乃至数天,难以满足高频次、实时化业务预报的时效要求。

\textbf{对排放清单输入的高度依赖}构成另一重要制约因素。公式(\ref{eq:advection_diffusion})中的排放源项$S$是CTMs的关键驱动输入,其准确程度直接影响模拟质量。然而,排放清单的编制周期漫长(通常滞后1--2年甚至更久)、空间分辨率受限,且在时间分解和行业拆分上存在较大不确定性\citep{thongthammachart2021integrated}。例如,交通源排放的逐时变化受实时交通流量影响显著,但清单通常仅能提供月均或日均的时间分配因子\citep{inventory},难以精细刻画真实的排放动态。这导致模式输出与地面实测之间常存在系统性偏差,在排放快速变化的特殊时段(如春节假期、森林火灾、秸秆焚烧季节等)偏差尤为突出。

\textbf{参数化方案的不确定性}同样制约着模式的预报精度。CTMs需要为化学反应、气溶胶演化、干湿沉降等过程选定特定的参数化方案(如化学机制CB06或SAPRC07、气溶胶方案AERO7、干沉降方案M3Dry等),不同方案组合对模拟结果影响显著\citep{appel2021community}。然而,最优参数化方案的选取往往依赖专家经验,且难以针对特定区域或季节进行动态适配。这种\cqt{固定方程的正向模拟}范式,缺乏从观测数据中逆向学习、自适应优化参数的机制。

\textbf{初始场同化技术尚不成熟}限制了CTMs的短临预报能力。与气象预报领域发展完善的资料同化系统相比,大气化学模式的初始场同化技术仍处于发展阶段。模式难以高效利用实时监测数据对公式(\ref{eq:advection_diffusion})中的浓度场$C$进行状态订正,导致预报技巧随预报时效延长而快速衰减——通常预报72小时后与实测的相关性显著下降\citep{sun2021improvement}。

\textbf{空间分辨率与观测尺度的不匹配}制约了CTMs的精细化应用能力。受计算资源约束,区域CTMs的水平分辨率通常为4--12 km,而地面监测站点代表的是局地尺度(约100 m量级)的污染物浓度。网格平均值与站点观测之间的\cqt{代表性误差}(Representativeness Error)难以完全消除,在地形复杂或排放空间异质性强的区域尤为明显。

\textbf{情景模拟的\cqt{组合爆炸}困境}制约了CTMs在决策支持中的应用效能。当需要评估不同减排方案的空气质量改善效果时,CTMs需要为每种排放情景重新执行完整模拟。假设需要评估5种污染物(SO$_2$、NOx、VOCs、NH$_3$、PM)各3个减排水平(0\%、30\%、50\%)的组合效应,则需运行$3^5=243$次独立模拟,计算代价极其高昂。这种\cqt{组合爆炸}问题严重限制了CTMs在交互式减排方案设计和实时决策辅助中的应用潜力。

综合来看,传统机理驱动的数值模型虽然具备较强的物理可解释性与过程一致性,但\textbf{\cqt{可解释却难以实时预测、可模拟却难以反向学习}}的内在矛盾,限制了其在精细化、实时化环境管理中的应用空间。如何在保持物理可解释性的同时突破计算效率与数据同化的双重瓶颈,成为大气污染建模领域亟需攻克的关键科学问题。

\subsection{AI与大数据时代的机遇与挑战}
\label{subsec:ai_opportunity}

近年来,深度学习技术与大数据方法的蓬勃发展为突破上述困境开辟了新的可能性路径。2019年,Reichstein等学者在\textit{Nature}期刊发表了一篇里程碑式的综述文章,正式提出\cqt{AI for Earth System Science}的研究范式\citep{reichstein2019deep}。该综述指出,深度学习能够自动从数据中提取时空特征以增进过程理解,借助混合建模(Hybrid Modeling)策略将物理过程模型与数据驱动方法有机结合,有望在保持机理可解释性的同时实现计算效率提升与数据融合能力增强。

地球系统科学数据具备典型的\cqt{4V}特征\citep{reichstein2019deep}:\textbf{Volume}(体量庞大)——多源观测、数值模式与再分析资料持续积累,全球气象再分析数据集ERA5已累积超过PB量级的数据规模;\textbf{Velocity}(更新迅速)——实时监测与预报需求不断提升,空气质量监测数据以小时级频率持续更新;\textbf{Variety}(类型多元)——涵盖格点场、站点观测、卫星遥感、排放清单等多模态异构数据;\textbf{Veracity}(可信度差异)——不同来源数据的精度与可靠程度存在显著差异。传统数值模型难以在这种高维动态数据空间中实现实时求解与多源融合,而数据驱动方法能够在学习复杂非线性映射的同时,有效捕捉多要素间的时空耦合关系。

近年来,以GraphCast\citep{lam2023learning}、Pangu-Weather\citep{bi2023accurate}、FourCastNet\citep{pathak2022fourcastnet}和Aurora\citep{bodnar2025foundation}为代表的AI气象大模型在全球天气预报领域展示出巨大潜力,标志着从\cqt{求解方程}向\cqt{从数据学习}的范式转变。GraphCast采用图神经网络架构\citep{battaglia2018relational},在0.25°分辨率下预测227个大气变量,单次推理耗时不足60秒,在1380个验证指标中约90\%的指标上优于欧洲中期天气预报中心(ECMWF)的高分辨率业务预报系统HRES\citep{lam2023learning}。Pangu-Weather引入三维地球特定Transformer架构,实现了相比数值模式约10000倍的计算加速\citep{bi2023accurate}。GenCast采用扩散模型架构生成集合预报,在97.2\%的验证指标上超越ECMWF业务集合预报系统\citep{price2025probabilistic}。Aurora作为新一代地球系统基础模型,拥有13亿参数规模,首次将预测范围从天气拓展至大气化学与空气质量领域\citep{bodnar2025foundation}。

然而,将上述AI气象大模型迁移应用于大气污染领域时,仍面临若干关键瓶颈。

\textbf{观测网络的稀疏性与非结构化特征}。气象变量(如温度、气压、风场)具有较强的空间连续性与平滑性,可借助高密度格点化再分析数据(如ERA5)开展模型训练。然而,空气质量监测站点分布稀疏且呈不规则布局,难以直接套用基于卷积神经网络(CNN)的格点模型。这种非结构化的观测网络特性为图神经网络(GNN)在稀疏空间数据上的应用提供了天然场景,也是本文选择图神经网络作为核心架构的重要考量。

\textbf{对初始场与再分析数据的路径依赖}。现有AI气象模型通常以再分析数据(如ERA5、CAMS)作为初始场输入,但这些数据本身存在数小时至数天的生产延迟\citep{kashinath2021physics}。以Aurora为例,虽然其支持空气质量预测功能,但初始场来源于哥白尼大气监测服务(CAMS)的再分析产品,在中国区域存在明显的系统性偏差且时效滞后。更关键的是,这些模型在架构设计上并不支持同化实时地面站点观测,难以对突发污染事件作出快速响应——而本文提出的模型正是针对站点级实时预测与应急响应需求量身设计的。

\textbf{物理一致性与可解释性的缺失}。纯数据驱动模型可能学习到与物理规律相悖的模式\citep{karniadakis2021physics}。在质量守恒约束缺失的情况下,模型可能预测出污染物\cqt{凭空产生}或\cqt{异常消失}的非物理结果,尤其在训练数据分布之外的极端情景下问题更为突出。这与公式(\ref{eq:advection_diffusion})所描述的质量守恒定律形成鲜明对照。物理约束的缺失不仅影响预测精度,更从根本上削弱了模型结果的可信度与可解释性。

\textbf{污染演化机制建模不完整}。现有AI预测模型大多采用自回归模式——依据历史污染物浓度序列预测未来浓度,而未显式建模公式(\ref{eq:advection_diffusion})中\cqt{排放$\to$传输$\to$化学转化$\to$沉降}的完整污染演化链条。这意味着模型缺乏对排放源项$S$的显式响应能力:当排放发生变化(如减排政策实施、突发事故或春节假期排放骤变)时,自回归模型无法准确捕捉这种外部驱动的变化,也就难以支撑\cqt{如果减排50\%结果如何}这类假设情景推演与政策评估需求。

\textbf{任务范围的局限性}。现有AI模型主要聚焦于时序预测单一任务,而大气污染治理实践需要的是一套完整的决策支持能力体系,包括空间推断(评估未设站区域的污染水平)和情景模拟(评估不同减排方案的改善效果)。仅依靠时序预测功能难以满足精细化环境管理的多元需求。

上述挑战表明:大气污染建模需要一种\textbf{融合领域知识与深度学习}的新范式——既保持数据驱动方法的计算效率与强拟合能力,又借助物理约束确保结果的合理性与可解释性;既能实现高精度的实时预测,又能支撑情景模拟与决策优化。这正是本文研究工作的核心目标与出发点。

\subsection{大气污染复杂系统建模的关键问题}
\label{subsec:challenges}

基于以上分析,大气污染作为典型的开放复杂巨系统,其数据驱动建模面临三个方面的核心挑战,这些挑战贯穿于预测、推断与模拟三类核心任务,同时构成了本研究的科学问题。

\textbf{挑战一:物理约束的有效嵌入}。如第\ref{subsec:complex_system}节所述,污染演化过程受排放、气象、化学反应等多要素共同驱动,存在显著的非线性交互与反馈机制。传统CTMs通过求解公式(\ref{eq:advection_diffusion})显式刻画物理过程,但计算效率受限;纯数据驱动模型虽然高效,却缺乏机理约束,容易产生违背物理定律的预测结果。这一挑战在不同任务中的具体表现为:在预测任务中,模型需保证质量守恒与时空连续性(第\ref{chap:prediction}章);在推断任务中,模型需刻画风场驱动的各向异性传输机制(第\ref{chap:inference}章);在模拟任务中,模型需建立排放源强与污染浓度之间的物理响应关系(第\ref{chap:simulation}章)。如何在数据驱动框架下有效嵌入这些物理先验知识,实现\cqt{数据拟合能力}与\cqt{物理一致性}的平衡,是理论上的核心难点。

\textbf{挑战二:多源异构数据的融合}。大气污染建模涉及地面监测、卫星遥感、气象再分析、排放清单等多种数据源,这些数据在空间结构、时间分辨率、覆盖特性和物理意义上存在显著差异。这一挑战在不同任务中的具体表现为:在预测任务中,需要融合风场信息构建有向图结构以刻画跨区域传输(第\ref{chap:prediction}章);在推断任务中,需要处理卫星AOD的非随机缺失并将其转化为空间约束(第\ref{chap:inference}章);在模拟任务中,需要桥接历史排放清单(MEIC)与未来情景数据(DPEC)的差异(第\ref{chap:simulation}章)。如何在统一的潜在空间中融合这些异构数据、处理系统性缺失并保持物理一致性,是数据处理的关键瓶颈。

\textbf{挑战三:模型的泛化与外推能力}。多数深度学习模型采用转导式学习范式,在训练样本分布内表现良好,但泛化能力不足\citep{wu2021inductive}。这一挑战在不同任务中的具体表现为:在预测任务中,模型需同时预测具有复杂光化学耦合关系的\PM 与\ozone(第\ref{chap:prediction}章);在推断任务中,模型需泛化至训练集中未出现的新位置(第\ref{chap:inference}章);在模拟任务中,模型需外推至未来气候与排放情景(第\ref{chap:simulation}章)。如何突破固定图拓扑与历史分布的限制,构建具有强泛化能力的时空学习框架,是模型设计的核心难题。

\subsection{研究意义与价值}
\label{subsec:significance}

针对上述挑战,本研究以复杂系统数据驱动建模为核心思想,面向大气污染的多任务场景——预测、推断与模拟——构建物理约束的时空学习与推理框架。研究意义体现在以下四个方面。

\textbf{科学价值:数据--机理融合的建模新范式}。本研究提出\cqt{物理约束的数据驱动建模}范式,将第\ref{chap:methodology}章公式(\ref{eq:advection_diffusion})所描述的平流、扩散、化学反应等物理过程显式嵌入深度学习架构,在模型结构设计与损失函数约束两个环节实现数据与机理的有效融合。该框架揭示了在复杂系统建模中平衡\cqt{数据拟合}与\cqt{物理一致性}的可行路径,为地球系统科学中的时空过程建模提供了新的方法论支撑。

\textbf{国家战略价值:支撑\cqt{双碳}目标与协同治理}。研究成果直接服务于国家\cqt{双碳}战略目标与\PM--\ozone 协同治理的技术体系建设。通过构建排放--浓度响应的代理模型与可微分优化框架,能够在数秒内完成传统CTMs需要数小时的情景模拟,定量评估不同减排方案对空气质量的改善效果,为国家及地方污染防治政策的制定提供高效、科学的决策支持工具。

\textbf{社会应用价值:从科学建模到智能服务}。本研究实现了空气质量智能预报系统的工程化部署,支撑从实时预报、空间推断到智能减排决策的全链条应用。系统已在\cqt{彩云天气}等平台上线运行,日均服务数千万用户查询请求,为公众健康防护提供及时信息;同时支撑了上海进博会等国家重大活动的空气质量保障工作,展现了研究成果的实际应用价值。

\textbf{方法论普适性:可迁移的时空建模框架}。本研究提出的核心方法——时空图网络建模传输过程、物理约束嵌入损失函数、多源数据在隐空间对齐——具有广泛的普适性。其建模思想可迁移至气象预报、水文模拟、交通流预测、传染病传播等其他涉及时空演化的复杂系统问题,为相关领域的数据驱动建模提供方法借鉴与技术参考。


\section{国内外研究现状}
\label{sec:literature}

\subsection{传统数值建模研究进展}
\label{subsec:numerical_models}

化学传输模型(CTMs)长期以来作为大气污染数值模拟的传统核心工具,其发展历程可追溯至20世纪70年代。在全球尺度上,具有代表性的模式系统包括GEOS-Chem\citep{bey2001global}、MOZART\citep{emmons2010description}以及CAM-chem\citep{lamarque2012cam}等;在区域尺度上,主流应用模式包括CMAQ\citep{byun2006review}、WRF-Chem\citep{grell2005fully}和CAMx\citep{emery2024comprehensive}等。这些模式通过数值求解第\ref{chap:methodology}章公式(\ref{eq:advection_diffusion})所描述的对流--扩散方程,从物理化学机理角度刻画污染物的形成、输送、转化与清除全过程。

CMAQ是目前应用最为广泛的区域空气质量模式之一,由美国环保署(EPA)开发并维护,采用模块化架构设计,涵盖气相化学(CB6、SAPRC机制)、气溶胶热力学(ISORROPIA)、云化学过程、干湿沉降等完整的过程模块\citep{appel2021community}。WRF-Chem由美国国家大气研究中心(NCAR)开发,将气象模式WRF与化学模块紧密耦合,可实现气象--化学的双向反馈模拟\citep{grell2005fully}。GEOS-Chem由哈佛大学主导开发,采用全球--区域嵌套技术架构,适用于跨区域乃至全球尺度的污染物传输过程研究\citep{bey2001global}。

在国内研究方面,嵌套网格空气质量预报模式系统(NAQPMS)由中国科学院大气物理研究所自主开发,采用多尺度嵌套技术与数据同化模块,已在APEC会议、G20峰会等重大活动的空气质量保障工作中发挥关键作用\citep{cn_wangzifa2006naqpms,wang2014modeling,cn_chenxueshun2015naqpms},体现了我国在大气污染数值模拟领域已具备独立自主的创新能力。

近年来,CTMs在以下方向取得重要进展:模式嵌套技术使区域模式能够与全球模式实现有效衔接\citep{wang2015implementation};高分辨率模拟能力不断提升,部分研究已实现公里级城市空气质量精细化模拟\citep{sicard2021high};排放清单持续完善更新,中国多尺度排放清单(MEIC)\citep{zheng2018trends}为模式运行提供了更精细的排放输入支撑。卢亚灵等\citep{cn_luyaling2021}系统梳理了空气质量预测技术的演变历程,指出从统计方法到数值模式再到人工智能的发展趋势。然而,如第\ref{subsec:traditional_models}节所述,CTMs面临的计算成本高、排放依赖强、实时性差等固有瓶颈始终未能根本突破,这为数据驱动方法的兴起创造了契机。

\subsection{数据驱动的大气污染建模研究}
\label{subsec:ai_airquality}

数据驱动的大气污染预测方法经历了从统计模型到深度学习的演进历程\citep{reichstein2019deep,karniadakis2021physics},近年来已成为\PM 和\ozone 浓度预报的重要技术手段。本节系统梳理该领域的研究现状,重点关注深度学习时空建模、图神经网络以及物理约束神经网络等前沿方向,并指出现有方法存在的关键不足。

\subsubsection{统计与传统机器学习方法}

早期研究工作主要采用统计模型和传统机器学习方法建立气象因子与污染浓度之间的经验关系。多元线性回归(MLR)、自回归积分滑动平均模型(ARIMA)和支持向量回归(SVR)等方法实现简单、计算成本低且具有较好的可解释性,但难以捕捉复杂的非线性关系,在长期预测任务上性能欠佳\citep{goyal2006statistical,wong2021using,leong2020prediction}。XGBoost和LightGBM等梯度提升集成方法在表格数据上表现优异,Ma等人\citep{ma2020application}将XGBoost与WRF-Chem进行集成,通过机器学习后处理校正数值模式的系统性偏差;Thongthammachart等人\citep{thongthammachart2021integrated}采用LightGBM融合多源特征实现区域\PM 预测。陈镇等\citep{cn_chenzhen2024zhujiang}运用SVR、随机森林等方法建立珠三角臭氧预测模型;曲悦等\citep{cn_quyue2019}对比了BP神经网络、CNN和LSTM模型在北京\PM 预测中的性能差异。然而,这类方法依赖人工特征工程,虽能快速拟合局地污染规律,但难以自动捕捉跨区域的污染传输关系和复杂的时空依赖结构\citep{xiao2018ensemble,chen2018machine}。

\subsubsection{深度学习时序建模方法}

随着深度学习技术的快速发展,循环神经网络(RNN)及其变体被广泛应用于大气污染时序预测任务\citep{cn_zhuyanmin2020shendu}。长短期记忆网络(LSTM)是该领域应用最多的模型架构,通过门控机制捕捉长程时序依赖关系并有效缓解梯度消失问题\citep{li2017long}。Chang等人\citep{chang2020lstm}采用聚合LSTM架构融合多站点信息进行北京\PM 预测。尹文君等\citep{cn_yinwenjun2015dbn}提出基于深度信念网络(DBN)的预报模型,利用多层RBM结构自动学习特征表示。门控循环单元(GRU)通过简化门控结构减少参数量,在保持预测性能的同时提升计算效率\citep{faraji2022integrated}。然而,RNN类模型采用序贯计算方式,难以实现并行化处理,且主要关注时间维度,将各站点独立建模或简单拼接,忽视了站点间的空间关联特性。

Transformer架构凭借自注意力机制实现的长程依赖建模和全并行计算能力,克服了RNN的序贯计算瓶颈,已成为空气质量预测的重要范式\citep{vaswani2017attention}。Liang等人\citep{liang2023airformer}提出AirFormer模型,采用分层Transformer架构处理大规模监测网络的72小时预报任务,其创新之处包括底层确定性阶段的新型自注意力机制和顶层随机性阶段的隐变量建模,有效处理了监测网络中的多尺度时空依赖问题。Yi等人\citep{yi2018deep}提出Deep Distributed Fusion Network,通过分布式融合架构整合空气质量、气象和天气预报数据,实现了多城市PM$_{2.5}$浓度的联合预测,展现了多源数据融合在深度学习框架中的潜力。陶辰亮\citep{cn_taochenliang2024}构建了融合循环神经网络、Transformer和梯度提升树的集成学习模型,结合SHAP可解释方法揭示了\ozone-\NOx-VOCs-\PM 之间的非线性交互关系。

近年来,通用时序预测模型的研究进展为大气污染预测提供了新的技术支撑。iTransformer\citep{liuitransformer}通过将注意力机制应用于变量维度而非时间维度,更有效地捕捉多变量时序数据中的跨通道依赖关系;TimeXer\citep{wang2024timexer}引入外生变量建模机制,使模型能够显式利用外部协变量增强预测性能,这一设计与大气污染预测中融合气象驱动的需求高度契合。在大气污染的具体应用中,TFB\citep{qiu2024tfb}构建了包含大气污染在内的多领域时序预测基准,对现有深度学习模型进行了全面评估;DUET\citep{qiu2025duet}通过双通道编码器分别建模污染物浓度的趋势成分和季节成分,有效处理了空气质量数据的复杂周期性模式。这些工作表明,通用时序预测框架经过领域适配后,能够在大气污染预测任务中取得具有竞争力的性能。

\subsubsection{图神经网络时空建模方法}

近年来,图神经网络(GNN)因其对非欧几里得空间数据的天然适用性,成为时空预测领域的主流技术架构\citep{zhou2020graph}。这类模型通过图结构(监测站点网络)引入归纳偏置,使用图卷积或改进的Transformer等空间模块刻画监测站点间的关联,配合时序卷积网络或LSTM等时间模块模拟污染物浓度的时间演变\citep{wu2020comprehensive}。

\textbf{核心GNN架构}。Yu等人\citep{yu2018spatio}提出时空图卷积网络(STGCN),通过图卷积捕捉空间依赖、一维卷积捕捉时序演化,建立了时空图建模的基础框架。Li等人\citep{li2018diffusion}提出扩散卷积循环神经网络(DCRNN),将信息传播建模为有向图上的扩散过程,其双向随机游走机制与大气污染物的扩散传输过程具有天然的相似性——直接对应公式(\ref{eq:advection_diffusion})中的扩散项。Wu等人\citep{wu2019graph}提出Graph WaveNet,引入自适应邻接矩阵学习机制和膨胀卷积,无需预定义图结构即可端到端学习节点间的隐式依赖关系。

\textbf{领域知识增强的图神经网络}。Qi等人\citep{qi2019hybrid}提出GC-LSTM模型,将图卷积嵌入LSTM单元内部实现时空联合建模。Wang等人\citep{wang2021modeling}提出注意力时序图卷积网络(ATGCN),通过站点级注意力机制自适应编码多种时空依赖类型。Chen等人\citep{chen2023group}提出群组感知图神经网络(GAGNN),构建城市图与城市群图的层级结构,通过可微分分组网络发现地理距离远但污染相关的城市间潜在依赖。王凯等\citep{cn_wangkai2023gcnlstm}提出GCN-LSTM城市臭氧预测模型,利用图卷积捕捉空间传输特征,结合LSTM提取时间依赖。汪宇晖\citep{cn_wangyuhui2023stgnn}提出自适应时空聚焦图卷积网络模型,通过多目标进化算法优化图结构,并结合长短周期时序分析进行\PM 预测。

\textbf{动态与自适应图}。预定义的静态图结构难以适应大气污染的动态特性。Xu等人\citep{xu2023dynamic}提出动态图神经网络(DGN-AEA),通过学习边属性生成自适应双向动态图,集成风场数据作为有向动态连接。Teng等人\citep{teng202372,teng2024new}采用聚合邻域时空信息的混合图深度网络,通过多尺度特征融合实现72小时实时预报。这些工作解决了静态图结构无法捕捉气象条件变化对污染传输影响的局限。

然而,现有时空图神经网络模型的成功主要局限于相对小规模的数据集,在全国范围的大尺度数据上往往面临扩展性问题。大部分深度模型在全国性数据集上的应用受限于其高计算复杂度和内存开销。

\subsubsection{物理约束的深度学习方法}

除从数据中挖掘模式外,另一个重要的发展方向是将大气物理和化学机理融入数据驱动模型,形成物理约束的深度学习框架\citep{reichstein2019deep,raissi2019physics,cn_zhawenshu2022pinn}。

\textbf{物理引导的图神经网络}。Hettige等人\citep{hettigeairphynet}提出AirPhyNet,是户外空气质量领域首个物理引导深度学习模型。其架构包含三个核心模块:GRU编码器将PM$_{2.5}$浓度编码为初始状态;GNN微分方程网络由物理规律约束,从连续性方程推导出扩散-平流微分方程;解码器生成最终预测。该模型将质量守恒原理直接嵌入网络结构,在稀疏数据和突变场景下展现出更强的鲁棒性。

Li等人\citep{li2023improving}提出物理启发深度图学习方法,采用欧拉2D网格系统,通过连续性方程编码流体物理,采用雷诺分解处理平流和扩散项。混合GCN与全残差深度网络架构能够生成物理一致的时空趋势,在外推场景下表现出显著优势。

\textbf{物理约束损失函数}。Li等人\citep{li2025knowledge}的物理信息深度学习框架将平流-扩散方程编码为软约束,通过在损失函数中引入物理一致性惩罚项,引导模型学习符合大气动力学规律的表示。关键发现是物理约束从根本上改变了学习动态,有效降低了多污染物联合预测的系统性偏差。石佳超等\citep{cn_shijiachao2018cmaqnn}将CMAQ模型预测值与前馈神经网络结合,构建长三角PM$_{2.5}$浓度快速响应模型。张斌等\citep{cn_zhangbin2020jicheng}融合多种深度学习模型对CMAQ预报进行误差订正。黄丛吾等\citep{cn_huangcongwu2018mos}采用极端随机树方法优化WRF-CMAQ-MOS模型。

\textbf{双分支物理-数据融合架构}。Tian等人\citep{tianair}提出Air-DualODE,是开放系统下物理引导的双分支神经ODE模型。该模型包括两个并行的动态分支:物理分支直接求解开放体系的边界感知扩散-平流方程,捕获由物理定律主导的污染物时空传播动态;数据驱动分支学习物理分支未能解释的额外依赖关系。两分支的隐表示在时间维度上对齐,通过融合模块加权合成最终预测,实现了物理可解释性与数据拟合能力的平衡。

\subsubsection{气象预报融合的关键缺失}

值得特别指出的是,\textbf{多数现有数据驱动模型未能充分利用未来气象预报信息},这是制约其业务化应用的关键缺陷。

在传统数值空气质量模式中,气象预报是必不可少的驱动输入。以CMAQ(Community Multiscale Air Quality)模式为例,其运行依赖WRF(Weather Research and Forecasting)模式提供的气象场预报\citep{byun2006review}。关键气象变量包括:天气形势(海平面气压、位势高度)决定大气稳定度和传输模式;化学驱动因子(温度、相对湿度)影响化学反应速率和二次气溶胶形成;传输因子(风速/风向)控制污染物扩散和平流;垂直混合(边界层高度PBLH)决定垂直稀释能力。CMAQ、WRF-Chem、GEOS-Chem等化学传输模式在预报阶段\textbf{固有地使用未来气象预报}作为输入。

与数值模式形成鲜明对比的是,多数机器学习模型仅使用历史气象观测数据。在许多方法中,气象要素往往被视为可有可无的辅助输入,通过浅层神经网络编码后在预测末段简单拼接进污染物特征中。这一局限的根本原因包括:(1)\textbf{时序错配}——机器学习模型使用历史气象观测与历史污染物浓度配对训练,预测时无法获取未来气象条件;(2)\textbf{时序依赖问题}——T+24h的空气质量取决于T+24h的气象条件而非T-24h的条件;(3)\textbf{预报时效衰减}——研究表明,无适当约束时模型性能随预报时间恶化,超过24小时后历史模式预测能力显著下降。

2022--2025年间已有若干重要工作开始着手解决此问题。Qiu等人\citep{qiu2023regional}提出过程参数化神经网络(PPN)模型,采用非对称编码器-解码器结构,在解码阶段输入气象变量、排放和上一时步PM$_{2.5}$预报,类似于化学传输模式使用四维数据同化(FDDA)的方式。Ma等人\citep{ma2025causal}提出CauAir模型,采用因果注意力机制显式建模气象协变量与空气质量的因果关系,凸显了气象预测信息在数据驱动空气质量预报中的关键作用。

\subsubsection{现有方法的共性问题}

综合上述分析,现有数据驱动模型仍存在以下共性问题:

\begin{enumerate}
    \item \textbf{气象预报融合不足}:多数模型未能有效利用未来气象预报信息,仅依赖历史气象观测,导致预报时效受限、长时预测精度衰减;
    \item \textbf{物理规律建模缺失}:缺乏对公式(\ref{eq:advection_diffusion})所描述的平流-扩散过程的显式建模,导致在极端情景或分布外数据上泛化能力不足;
    \item \textbf{排放响应与情景模拟能力有限}:缺乏对排放输入$E$的显式建模,无法响应排放变化或支持未来情景模拟,限制了模型在政策评估中的应用;
    \item \textbf{多污染物协同建模不足}:\PM 与\ozone 的协同预测研究有限,未能充分考虑二者共享前体物(NO$_x$、VOCs)的光化学耦合关系。
\end{enumerate}

\subsubsection{本文前期工作基础}

针对上述问题,本文作者在前期研究中开展了系统性探索。在图神经网络时空建模方面,本文作者提出PM$_{2.5}$-GNN模型\citep{wang2020pm2},首次将气象领域知识(风向、风速)融入图神经网络的边特征构建,基于物理地形约束构建图结构(污染物传输限于300km内且无山脉阻隔的城市间),设计基于风场的平流系数边权重,通过风驱动的消息传递机制显式建模跨区域污染传输。同时,本文作者构建并公开了KnowAir-DS数据集(已更新至KnowAir-DS-V2\footnote{\url{https://zenodo.org/records/15614907}},2016--2023,含O$_3$)\citep{wang2020pm2},为该领域后续研究提供了重要的数据基础。

在物理约束建模方面,本文作者进一步提出PCDCNet模型\citep{wang2025pcdcnet},将其设计为数值模型的代理(surrogate),通过在深度学习架构中融合物理-化学动力学知识实现高效预测。该模型明确将排放源、气象影响以及其他领域知识以约束形式纳入网络,结合图神经网络的空间传输建模、循环网络的时间累积建模,并引入光化学反应等局地相互作用的表示增强模块,在72小时\PM 与\ozone 浓度预报上取得了当前最先进的性能。

本文后续章节将在上述前期工作基础上,系统阐述物理约束数据驱动建模的完整框架,通过在数据融合、模型结构和损失函数三个环节嵌入大气科学领域知识,进一步解决现有方法的局限性。

\subsection{大气污染空间推断研究}
\label{subsec:inference_review}

由于监测站点分布稀疏且主要集中于城市区域,如何推断无观测区域的污染物浓度是大气环境领域的重要研究问题。全球范围内大部分人口生活在缺乏地面监测的区域\citep{southerland2022global},这使得空间推断成为污染暴露评估和健康风险分析的关键技术环节。

\textbf{传统空间插值方法}。反距离加权(IDW)和普通克里金(Ordinary Kriging)是最常用的空间插值技术\citep{li2011review}。这些方法基于空间自相关性假设,仅利用几何距离信息构建权重矩阵,忽略了地形、气象、排放源分布等因素的影响,在复杂地形区域或站点稀疏区域精度较低。回归克里金(Regression Kriging)和土地利用回归模型(LUR)通过引入辅助变量部分缓解了这一问题\citep{hoek2008review},但仍难以捕捉污染物传输的动态时空特性。协克里金(Co-Kriging)方法利用多变量间的空间相关性进行联合估计,但对变量间关系的线性假设限制了其在复杂大气化学体系中的适用性。

\textbf{卫星遥感反演方法}。利用卫星观测的气溶胶光学厚度(AOD)推算地面PM$_{2.5}$是目前实现全覆盖制图的主流手段\citep{geng2021tracking,xiao2018ensemble,van2016global}。Van Donkelaar等人\citep{van2016global}结合MODIS、MISR和SeaWiFS多源卫星AOD与GEOS-Chem模拟,构建了长时间序列的全球PM$_{2.5}$年均数据集。Geng等人开发的TAP(Tracking Air Pollution)数据库\citep{geng2021tracking}通过多源数据融合方法整合卫星AOD、地面观测与CTM模拟,实现了中国区域近实时PM$_{2.5}$反演。Wei等人\citep{wei2021reconstructing}采用时空统计方法重建了2000年以来中国1公里分辨率的高质量PM$_{2.5}$数据集,后续工作进一步将该方法扩展至全球尺度\citep{wei2023first}。Ma等人\citep{ma2015satellite}采用时空加权回归结合随机森林方法,有效处理了AOD与PM$_{2.5}$关系的时空异质性。然而,AOD产品受到云层遮挡、地表高反射率、气溶胶垂直分布假设和夜间无法观测等因素影响,存在严重的非随机缺失问题\citep{just2020gradient}。现有的深度学习模型多将AOD作为强制输入特征,导致在阴雨天或夜间模型完全失效,严重制约了实时空气质量评估的可用性。

\textbf{深度学习空间估计方法}。近年来,深度学习方法被广泛应用于空气质量空间估计。Lee等人\citep{lee2021hourly}采用深度神经网络融合静止卫星GOCI影像与UM再分析数据,实现了逐小时地面PM$_{2.5}$估计。Di等人\citep{di2019ensemble}提出集成神经网络方法,结合梯度提升、随机森林和神经网络的互补优势进行空间估计。Park等人\citep{park2020estimating}采用卷积神经网络处理网格化的多源输入数据,利用空间卷积核捕捉局部空间模式。Wei等人\citep{wei2023first}采用时空加权人工智能方法,首次实现了全球日均1公里分辨率无缝隙PM$_{2.5}$制图,为大范围空气质量评估提供了重要数据支撑。然而,这些方法多采用欧几里得网格结构,难以适应监测网络的不规则分布和动态变化。毛文静等\citep{cn_maowenjing2022lianxu}基于多层LSTM迭代预测模型,建立中国PM$_{2.5}$时空预报系统,实现未来24小时连续空间覆盖预报。

\textbf{归纳式图神经网络}。图神经网络为非规则分布的监测网络提供了更自然的建模框架。Wu等人提出的IGNNK(Inductive Graph Neural Networks for Spatiotemporal Kriging)\citep{wu2021inductive}为时空推断提供了新思路,通过随机子图采样和动态节点掩码训练使模型具备对未见节点的归纳式泛化能力,突破了传统转导式方法仅能预测训练时已见节点的局限。后续工作Increase\citep{zheng2023increase}引入因果注意力机制增强空间依赖建模,Kits\citep{xu2025kits}结合知识蒸馏提升未监测位置的预测性能,SATCN\citep{wu2021spatial}采用自适应时空卷积网络处理动态图结构。Appleby等人\citep{appleby2020kriging}将图卷积网络应用于空间插值任务,验证了GNN在克里金问题上的有效性。然而,这些方法主要面向交通流场景设计,在大气污染推断中缺少对排放驱动的物理建模、气象条件的动态影响以及卫星数据的有效融合机制。此外,现有方法通常假设目标节点在推断时已知其位置特征,未能解决完全无先验信息区域的浓度估计问题。

针对上述问题,本文后续章节将提出SPIN(Spatial Pollution Inference Network)模型,通过物理引导的图传播机制和多源数据自适应融合策略,实现对无监测区域的鲁棒空间推断。

\subsection{大气污染情景模拟研究}
\label{subsec:simulation_review}

在环境治理决策中,不仅需要\cqt{预测未来会发生什么},更需要\cqt{推演假设情景}(如:减排50\%后空气质量如何变化?碳中和目标下的空气质量如何演变?)。这类假设情景推理问题对模型提出了更高要求。

\textbf{基于CTMs的情景分析}。目前的排放情景分析和源贡献解析主要依赖化学传输模式。例如,CMAQ-ISAM(Integrated Source Apportionment Method)可量化各类排放源对污染物浓度的贡献\citep{appel2021community},CMAQ-DDM(Decoupled Direct Method)可计算浓度对排放的敏感性系数$\partial C_i / \partial E_i$,CAMx-OSAT/PSAT支持O$_3$和颗粒物的源解析。这些方法基于严格的物理化学方程,结果具有较强的科学可解释性。Hu等人\citep{hu2023changing}利用GEOS-Chem伴随模型分析了长三角地区\PM 和\ozone 对排放源的响应变化。Wang等人\citep{wang2019responses}采用WRF-CMAQ分析了气象条件和排放变化对\PM 和\ozone 浓度的响应关系。然而,这些分析每次都需要重新运行完整模拟,单次年度模拟在高性能集群上需要数天至数周的计算时间,计算代价极高,难以支撑实时决策和大规模情景扫描。

\textbf{长期情景模拟}。在气候变化与空气质量交互的长时间尺度上,CMIP6(Coupled Model Intercomparison Project Phase 6)\footnote{\url{https://wcrp-cmip.org/cmip-phases/cmip6/}} 提供了未来气候情景的标准化框架\citep{eyring2016overview}。Turnock等人\citep{turnock2020historical}基于CMIP6多模式集合分析了未来全球空气质量变化趋势,指出强减排情景(SSP1-26)下全球O$_3$和PM$_{2.5}$浓度均有显著下降潜力。然而,全球排放情景对中国的刻画较为粗糙,难以准确反映中国自2013年\cqt{大气十条}实施以来的快速减排进程。Tong等人\citep{tong2020dynamic}开发了中国未来排放动态评估模型DPEC(Dynamic Projection model for Emissions in China)\footnote{\url{http://meicmodel.org.cn/?page_id=1922}},通过衔接GCAM-China能源系统模型与MEIC排放清单模型,构建了与CMIP6 SSP-RCP情景矩阵衔接的中国本地化排放情景数据集,能够动态模拟不同社会经济情景、气候目标约束和污染控制政策组合下的未来排放变化。Cheng等人\citep{cheng2021pathways}结合DPEC情景开展了碳达峰碳中和路径下的空气质量预测,验证表明DPEC情景能更准确地重现2015--2020年中国排放变化趋势。Liu等人\citep{liu2021wind}采用与本文IGNN相似的\cqt{历史校准-未来预测}范式,基于WRF-CMAQ模式评估了SSP情景下风沙对华北地区颗粒物的长期影响,发现高排放情景下2050年沙尘事件将加剧并部分抵消人为减排效益;但该研究仅针对PM$_{10}$单一污染物和2050年单一目标年份,历史模拟精度亦有限。然而,CTM的高计算成本限制了长期集合模拟的数量和分辨率,数据驱动的快速情景评估工具具有重要应用价值。

\subsection{研究现状总结与存在问题}
\label{subsec:summary}

综合国内外研究进展,当前大气污染数据驱动建模领域仍存在以下四方面突出问题。

\textbf{机理模型可解释但实时性与精度受限}。传统化学传输模型以公式(\ref{eq:advection_diffusion})等物理--化学方程为基础,具备较强可解释性,但高计算成本限制了其实时应用;对排放清单和边界条件的依赖过强,导致在快速变化或极端天气条件下预测精度显著下降。

\textbf{数据驱动模型高效但缺乏物理一致性}。现有的深度学习模型在计算效率上具有显著优势,但缺乏对公式(\ref{eq:advection_diffusion})所描述物理规律的显式建模,在极端情景、分布外数据、长时序预测等任务上泛化能力不足;大多数模型不包含排放输入,无法响应排放变化或支持政策评估。

\textbf{情景模拟计算成本高昂}。传统灵敏度分析与情景模拟需反复运行数值模式,计算量巨大,存在\cqt{组合爆炸}问题。现有AI代理模型多采用自回归预测范式,缺乏对排放变化的响应能力,难以支撑\cqt{如果减排50\%结果如何}这类假设情景推演需求。

\textbf{系统落地能力不足,科研原型与应用脱节}。多数研究停留在离线实验阶段,缺乏稳定的数据流管理与云端服务能力,尚未实现从模型开发到业务化运行的闭环。

针对上述不足,本研究提出一种面向复杂系统的数据驱动建模框架,在统一潜在空间中融合多源观测与物理约束,构建从预测、推断到模拟的多任务建模体系,并通过云原生架构实现空气质量智能预报系统的工程化落地。


\section{研究框架与创新点}
\label{sec:framework}

\subsection{核心研究问题与挑战}
\label{subsec:core_problems}

本论文围绕大气污染复杂系统的数据驱动建模,聚焦于\textbf{预测}、\textbf{推断}与\textbf{模拟}三个核心研究问题,呈现\cqt{点$\rightarrow$面}与\cqt{当前$\rightarrow$未来}的递进关系:

\textbf{研究问题一(Q1):大气污染时序预测问题}(第\ref{chap:prediction}章)。给定历史空气质量观测序列、气象场数据和排放清单信息,如何构建物理约束的时空图神经网络模型,实现\PM 与\ozone 浓度的高精度联合预报?\textbf{核心挑战}:(1)长程传输建模——污染物跨区域迁移受风场主导,传统各向同性图结构难以刻画\cqt{上风向影响下风向}的定向传输规律;(2)物理一致性保障——纯数据驱动模型可能产生违背质量守恒的非物理预测,尤其在极端情景下泛化能力不足。

\textbf{研究问题二(Q2):大气污染空间推断问题}(第\ref{chap:inference}章)。由于地面监测站点分布稀疏且主要集中于城市区域,如何融合卫星遥感、气象再分析和排放清单等多源异构数据,对未设监测站点的区域进行高分辨率空间推断?\textbf{核心挑战}:(1)遥感AOD大面积缺失——云层遮挡导致卫星观测缺失率高达60\%以上,传统方法难以有效利用;(2)归纳式泛化——现有模型多采用转导式学习,仅能处理固定图拓扑,无法泛化至未见站点。

\textbf{研究问题三(Q3):未来情景模拟问题}(第\ref{chap:simulation}章)。面向碳达峰碳中和目标,如何利用历史数据训练的模型迁移至未来排放情景,预测不同减排路径下的空气质量演变?\textbf{核心挑战}:(1)数值模式计算慢——传统CTMs单次区域年尺度模拟需数百核时,难以支撑多情景大规模探索;(2)数据驱动方法难以响应排放变化——现有深度学习模型不包含排放输入,无法回答\cqt{减排后空气质量如何变化}这一关键问题。

\textbf{三个问题的共性挑战}。尽管上述问题在形式上各有侧重,但存在深层共性:(1)\textbf{时空复杂性}——大气污染的演化涉及多尺度时空耦合,传统方法难以同时捕捉局地化学反应与跨区域传输;(2)\textbf{物理--数据鸿沟}——机理模型可解释但计算昂贵,数据驱动方法高效但缺乏物理一致性;(3)\textbf{多源异构数据融合}——地面监测、卫星遥感、气象再分析、排放清单等数据源在时空分辨率、覆盖范围、观测误差等方面差异显著。如何在统一框架内有效应对这些共性挑战,是本研究的核心科学问题。

\subsection{研究范式}
\label{subsec:paradigm}

针对上述共性挑战,本论文提出\textbf{\cqt{物理约束的数据驱动建模}}核心范式:\textbf{在数据驱动的深度学习框架中融合大气科学领域知识}。

深度学习与传统统计机器学习的本质区别在于\textbf{表示学习}(Representation Learning)能力\citep{goodfellow2016deep}。传统方法依赖人工设计的特征工程,而深度学习能够自动从原始数据中提取层次化的特征表示。本文的核心思想是将大气科学领域知识融入表示学习的全过程——不仅约束模型的输入输出映射,更重要的是约束模型学习到的\textbf{隐空间表示}本身,使其具备物理意义与一致性。

具体而言,本文构建了\cqt{编码$\rightarrow$隐空间动力学$\rightarrow$解码}的统一建模框架(如图\ref{fig:framework}所示):通过\textbf{编码层}将多源异构数据(污染物浓度、气象变量、排放数据)映射到统一隐空间,在\textbf{隐空间动力学层}中通过空间模块与时间模块的交替建模学习时空演化规律,最后由\textbf{解码层}生成预测结果。物理约束在三个层次嵌入:编码层通过\textbf{图结构设计}编码风场等物理先验,隐空间层通过\textbf{图算子设计}模拟扩散与平流过程,解码层通过\textbf{损失函数}嵌入质量守恒等物理约束。这种三层架构的技术细节将在第\ref{chap:methodology}章详细阐述。

\begin{figure}[htbp]
    \centering
    \includegraphics[width=\textwidth]{figures/chap01_thesis_framework.pdf}
    \caption{本文研究框架}
    \caption*{框架涵盖四个层次:第二章建立物理约束的时空图神经网络建模范式(理论基础);第三至五章分别针对预测、推断、模拟三类问题提出定制化模型,各问题间体现\cqt{时间$\rightarrow$空间}与\cqt{当前$\rightarrow$未来}的维度迁移;第六章实现工程落地。}
    \label{fig:framework}
\end{figure}

这种范式的核心优势在于:表示学习使模型能够在隐空间中捕捉观测数据背后的本质规律,而物理约束确保学习到的表示具有科学意义。本文在预测、推断、模拟等不同任务上的一致有效性,验证了物理约束表示学习范式的普适性。

\subsection{解决方法与创新点}
\label{subsec:contributions_chap1}

基于上述研究范式,针对三个核心问题分别提出创新性解决方案:

\textbf{创新点一:物理约束的时空图神经网络预测框架}(第\ref{chap:prediction}章)。提出PM$_{2.5}$-GNN与PCDCNet模型,实现\PM 与\ozone 的72小时协同预报。设计LID--STD--TAD三模块架构,分别对应第\ref{chap:methodology}章公式(\ref{eq:advection_diffusion})中的化学反应与排放项、平流与扩散项、沉降与累积项,实现过程解耦与联合建模;利用风速向量投影定义有向边权重,突破传统GNN各向同性图结构的局限;提出领域一致性约束(DIC),将质量守恒与时空连续性嵌入训练目标。

\textbf{创新点二:多源数据融合与归纳式空间推断方法}(第\ref{chap:inference}章)。提出SPIN模型,实现无监测区域的全域制图。设计\cqt{以AOD空间梯度为约束而非输入}的融合策略,通过掩码机制规避缺测影响,实现全天候连续制图;采用扩散--平流双图并行传播机制刻画物理传输过程;引入动态节点掩码训练,赋予模型对未见站点的归纳式泛化能力,突破转导式学习仅能处理固定图拓扑的局限。

\textbf{创新点三:排放响应的深度学习情景模拟框架}(第\ref{chap:simulation}章)。提出IGNN模型,首次将排放清单作为可控变量纳入深度学习框架,融合多尺度排放数据(MEIC历史清单与DPEC未来情景),实现碳中和路径下2025--2060年的长期情景预测,揭示\PM 与\ozone 反向演变趋势与气候惩罚效应,为协同减排决策提供科学依据。

\textbf{工程落地}(第\ref{chap:deployment}章)。基于云原生架构构建KnowAir大气污染智能预报系统,实现模型的工程化部署与实时业务化服务,完成从科研原型到落地应用的闭环。

上述三个研究问题及其解决方案构成了\textbf{\cqt{理论基础$\rightarrow$科学问题$\rightarrow$方法创新$\rightarrow$工程落地}}的完整研究体系,各章引言将结合具体问题示意图详细阐述研究动机与技术挑战。

\section{本文组织结构}
\label{sec:organization}

本论文共分七章,按照\cqt{问题定义$\rightarrow$方法论基础$\rightarrow$预测建模$\rightarrow$空间推断$\rightarrow$情景模拟$\rightarrow$系统部署$\rightarrow$总结展望}的逻辑组织。

\textbf{第一章~绪论}。介绍研究背景,阐述大气污染的复杂系统特征,综述国内外研究现状,提出研究框架与创新点。

\textbf{第二章~物理约束的时空图神经网络建模范式}。系统阐述本文的方法论基础,包括图神经网络基础(谱方法与消息传递两大范式)、图神经网络在地球科学中的应用、物理启发的领域知识(对流-扩散方程与质量守恒约束)、以及本文提出的\cqt{编码$\rightarrow$隐空间动力学$\rightarrow$解码}统一建模框架,为后续应用章节提供统一的理论框架。

\textbf{第三章~物理约束时空图神经网络的大气污染预测}。针对现有模型缺乏物理一致性的问题,提出PM$_{2.5}$-GNN和PCDCNet模型,验证\PM 和\ozone 的72小时联合预报性能。

\textbf{第四章~基于多源数据融合的大气污染空间推断}。针对监测站点稀疏导致的空间盲区问题,提出SPIN模型,介绍扩散--平流双图构建与AOD梯度约束机制,验证不同缺测比例下的推断性能。

\textbf{第五章~未来污染情景模拟}。针对长期情景模拟问题,提出IGNN模型,首次将排放清单作为可控变量纳入深度学习框架,介绍排放数据融合方法,预测碳中和路径下的空气质量演变。

\textbf{第六章~系统部署与落地应用}。介绍从科研模型到业务系统的工程化落地,阐述云原生架构设计与实时数据处理流程,展示典型事件预报效果。

\textbf{第七章~总结与展望}。总结研究工作与主要贡献,分析局限性,展望物理机制可学习表达、不确定性量化、生成式建模等未来方向。  % 绪论
% ============================================================
% 第二章 物理启发的时空图神经网络建模范式
% 基于复杂系统数据驱动建模的大气污染研究
% ============================================================

\chapter{物理启发的时空图神经网络建模范式}
\label{chap:methodology}

大气污染作为典型的开放复杂巨系统,其时空演化过程涉及排放、传输、化学反应与沉降等多个物理化学子过程的耦合作用。本章系统阐述物理启发的时空图神经网络建模范式,为后续预测(第\ref{chap:prediction}章)、推断(第\ref{chap:inference}章)和模拟(第\ref{chap:simulation}章)三类核心任务提供统一的理论框架与方法论基础。

% ------------------------------------------------------------
% 2.1 图神经网络基础
% ------------------------------------------------------------
\section{图神经网络基础}
\label{sec:gnn_basics}

% 从系统科学角度引入
从系统科学的角度来看,图神经网络(Graph Neural Network, GNN)可被视为一种学习系统动力学的通用框架\citep{battaglia2018relational}。复杂系统由相互作用的实体构成,其宏观行为涌现于微观个体间的局部交互。GNN的核心思想是:首先根据问题特点构建图结构(将实体建模为节点、交互关系建模为边),然后在图上运行神经网络,通过迭代执行消息传递与聚合操作学习实体间的交互规律。神经网络提供的可学习参数(如卷积核权重、注意力系数等)使模型能够从数据中自动拟合复杂的非线性映射关系,从而逼近系统的演化动力学。这种\cqt{从局部交互到全局涌现}的建模思想与大气污染系统的物理本质高度契合——污染物浓度的时空演化正是由无数局部的排放、传输、反应与沉降过程共同决定的。

\subsection{图的基本概念}
\label{subsec:graph_concept}

图(Graph)是描述实体及其关系的数学结构,形式化定义为$\mathcal{G} = (\mathcal{V}, \mathcal{E}, \mathbf{A})$,其中$\mathcal{V} = \{v_1, v_2, \ldots, v_N\}$为节点集合,$\mathcal{E} \subseteq \mathcal{V} \times \mathcal{V}$为边集合,$\mathbf{A} \in \mathbb{R}^{N \times N}$为邻接矩阵。每个节点$v_i$附带特征向量$\mathbf{x}_i \in \mathbb{R}^{F}$,边$e_{ij}$附带边特征$\mathbf{e}_{ij}$。在大气污染问题中,节点通常对应监测站点或城市,节点特征包含污染物浓度与气象变量,边特征可编码节点间的空间距离或风场信息。具体的图构建方法将在各应用章节(第\ref{chap:prediction}--\ref{chap:simulation}章)中根据任务特点详细阐述。

\subsection{图神经网络的两大范式}
\label{subsec:gnn_paradigms}

图神经网络的发展形成了两大技术路线:基于谱的方法和基于消息传递的方法。两者各有特点,本文的模型设计综合采用了这两种范式。

\subsubsection{谱方法:图卷积网络}

谱方法从图信号处理的角度出发,在图的频谱域定义卷积操作\citep{bruna2014spectral}。其核心思想是利用图拉普拉斯矩阵$\mathbf{L} = \mathbf{D} - \mathbf{A}$(式中$\mathbf{D}$为度矩阵,$\mathbf{A}$为邻接矩阵)的特征分解进行滤波。下面通过公式\eqref{eq:spectral_conv}--\eqref{eq:gcn}展示谱方法从理论形式到实用形式的演化过程。

对于图信号$\mathbf{x} \in \mathbb{R}^N$,谱卷积的理论形式定义为:
\begin{equation}
\mathbf{y} = \mathbf{U} g_\theta(\boldsymbol{\Lambda}) \mathbf{U}^\top \mathbf{x}
\label{eq:spectral_conv}
\end{equation}
\noindent 式中,$\mathbf{y}$为输出信号,$\mathbf{x}$为输入图信号,$\mathbf{L} = \mathbf{U} \boldsymbol{\Lambda} \mathbf{U}^\top$为拉普拉斯矩阵的特征分解($\mathbf{U}$为特征向量矩阵,$\boldsymbol{\Lambda}$为特征值对角矩阵),$g_\theta(\cdot)$为可学习的频谱滤波器。

然而,公式\eqref{eq:spectral_conv}中的全特征分解计算复杂度为$O(N^3)$,难以应用于大规模图。为此,ChebNet\citep{defferrard2016convolutional}采用切比雪夫多项式近似滤波器$g_\theta(\boldsymbol{\Lambda})$,将复杂度降至$O(|\mathcal{E}|)$($|\mathcal{E}|$为边数):
\begin{equation}
\mathbf{y} = \sum_{k=0}^{K_c-1} \theta_k T_k(\tilde{\mathbf{L}}) \mathbf{x}
\label{eq:chebynet}
\end{equation}
\noindent 式中$\theta_k$为可学习的多项式系数,$T_k(\cdot)$为$k$阶切比雪夫多项式,$\tilde{\mathbf{L}} = 2\mathbf{L}/\lambda_{\max} - \mathbf{I}$为归一化拉普拉斯矩阵,$K_c$控制滤波器的阶数(即感受野大小)。

图卷积网络(Graph Convolutional Network, GCN)\citep{kipf2017semi}进一步将ChebNet简化为一阶形式(取$K_c=1$),并引入归一化技巧以稳定训练:
\begin{equation}
\mathbf{H}^{(l+1)} = \sigma\left(\tilde{\mathbf{D}}^{-\frac{1}{2}} \tilde{\mathbf{A}} \tilde{\mathbf{D}}^{-\frac{1}{2}} \mathbf{H}^{(l)} \mathbf{W}^{(l)}\right)
\label{eq:gcn}
\end{equation}
\noindent 式中,$\mathbf{H}^{(l)}$为第$l$层的节点特征矩阵,$\mathbf{H}^{(l+1)}$为第$l+1$层的输出,$\tilde{\mathbf{A}} = \mathbf{A} + \mathbf{I}$为添加自环的邻接矩阵,$\tilde{\mathbf{D}}$为对应的度矩阵,$\mathbf{W}^{(l)}$为可学习参数矩阵,$\sigma(\cdot)$为非线性激活函数(如ReLU)。经过$L$层图卷积后,最终的节点表示$\mathbf{H}^{(L)}$可通过读出层(如线性映射$\mathbf{Y} = \mathbf{H}^{(L)}\mathbf{W}_{\text{out}}$)生成预测输出。

谱方法的优势在于具有坚实的数学基础,卷积操作在频谱域具有明确的物理意义(低频对应平滑信号,高频对应突变信号);其局限性在于基于邻接矩阵或拉普拉斯矩阵进行运算,难以直接输入边属性(Edge Attributes),但近年来边条件卷积等扩展方法已逐步克服这一限制。

\subsubsection{消息传递神经网络}

与谱方法从频谱域定义卷积不同,消息传递神经网络(MPNN)\citep{gilmer2017neural}从空间域出发,通过邻居间的显式信息交互更新节点表示:
\begin{equation}
\mathbf{h}_i^{(l+1)} = \phi\left(\mathbf{h}_i^{(l)}, \bigoplus_{j \in \mathcal{N}(i)} \psi\left(\mathbf{h}_i^{(l)}, \mathbf{h}_j^{(l)}, \mathbf{e}_{ji}\right)\right)
\label{eq:message_passing}
\end{equation}
\noindent 式中$\mathbf{h}_i^{(l)}$为节点$i$在第$l$层的隐藏表示,$\mathcal{N}(i)$为节点$i$的邻居集合,$\psi(\cdot)$为消息函数,$\bigoplus$为聚合操作(如求和、均值),$\phi(\cdot)$为更新函数。该范式的物理直觉是:节点从邻居收集\cqt{消息}并聚合,再结合自身状态生成新的表示。这与大气污染物的传输机制一致——站点的污染水平受本地排放与周边传输的共同作用。

消息传递方法的优势在于灵活性强,可自然处理有向图、动态图和边特征;局限性在于缺乏全局视野,多跳依赖需堆叠多层网络。

\subsubsection{两种范式的统一视角}

从数学角度,谱方法与消息传递方法可以统一为图滤波框架:两者本质上都是对邻居信息的加权聚合,区别在于权重的定义方式——谱方法通过频谱滤波器隐式定义,消息传递方法通过显式的消息函数定义。

从计算实现角度,两种范式具有不同的适用场景。谱方法基于邻接矩阵或拉普拉斯矩阵的稠密/稀疏矩阵运算,通过矩阵乘法一次性完成所有节点的信息聚合,计算效率高,适用于节点规模较小的图(如本文的城市级网络,$N \approx 10^2$)。消息传递方法则采用边索引(edge index)表示图结构,以$[2, |\mathcal{E}|]$的张量形式存储所有边的源节点-目标节点对,逐条边计算消息并聚合。这种实现方式对稀疏图的适应性更强:当图规模增大时(如$N \approx 10^5 \sim 10^6$),稠密邻接矩阵$\mathbf{A} \in \mathbb{R}^{N \times N}$可能无法装入显存,而边索引表示的空间复杂度仅为$O(|\mathcal{E}|)$,对于稀疏图($|\mathcal{E}| \ll N^2$)具有显著优势。

本文根据不同任务的特点灵活选择两种范式:第\ref{chap:prediction}章PM$_{2.5}$-GNN采用消息传递范式,利用其对边特征的灵活建模能力将风场编码为有向边权重;第\ref{chap:prediction}章PCDCNet采用基于拉普拉斯矩阵的一阶谱卷积,第\ref{chap:simulation}章IGNN采用ChebNet的多阶滤波特性捕捉多尺度空间依赖;第\ref{chap:inference}章SPIN则结合两种范式,分别建模扩散过程(谱卷积)和平流过程(有向消息传递)。

\subsection{时空图神经网络}
\label{subsec:stgnn}

时空图神经网络(Spatiotemporal Graph Neural Network, STGNN)在空间维度的图卷积基础上,进一步引入时间维度的建模能力,是处理时空序列预测问题的有效工具\citep{yu2018spatio,li2018diffusion}。

典型的STGNN架构包含交替堆叠的空间模块与时间模块。空间模块采用上述GCN(公式\eqref{eq:gcn})或MPNN(公式\eqref{eq:message_passing})捕捉节点间的空间依赖;时间模块沿时间轴捕捉序列依赖,常用膨胀卷积(Dilated Convolution):
\begin{equation}
\mathbf{Y}_t = \sum_{s=0}^{K_c-1} \mathbf{W}_s \star \mathbf{X}_{t-d \cdot s}
\label{eq:tcn}
\end{equation}
\noindent 式中$\mathbf{Y}_t$为时刻$t$的输出,$\mathbf{X}_{t-d \cdot s}$为时刻$t-d \cdot s$的输入,$\mathbf{W}_s$为卷积核权重,$d$为膨胀因子。膨胀因子$d$通常按指数递增设置,即$d \in \{1, 2, 4, \ldots, 2^{L-1}\}$,使得$L$层TCN以线性参数量获得$\mathcal{O}(2^L)$的感受野。

综上,完整的STGNN通过空间模块(公式\eqref{eq:gcn}或\eqref{eq:message_passing})与时间模块(公式\eqref{eq:tcn})的交替堆叠,实现时空联合建模。这与大气污染系统的物理特性高度吻合——空间上相邻节点通过传输相互影响,时间上当前状态依赖历史演化。

\subsection{图神经网络的训练}
\label{subsec:gnn_training}

图神经网络的训练遵循深度学习的标准优化范式——基于梯度的反向传播算法\citep{goodfellow2016deep}。训练目标是最小化损失函数$\mathcal{L}(\Theta) = \frac{1}{M} \sum_{i=1}^{M} \ell(f_\Theta(\mathcal{G}_i), \mathbf{y}_i)$,其中$M$为训练样本数,$f_\Theta$为参数$\Theta$的GNN模型,$\mathcal{G}_i$为第$i$个输入图,$\mathbf{y}_i$为对应的标签,$\ell(\cdot)$为任务相关的损失函数。反向传播算法通过链式法则高效计算梯度,并采用Adam等优化器\citep{kingma2015adam}更新参数$\Theta^{(t+1)} = \Theta^{(t)} - \eta \nabla_\Theta \mathcal{L}$,其中$\eta$为学习率,$\nabla_\Theta \mathcal{L}$为损失函数关于参数的梯度。从系统科学角度,经过充分训练后的图神经网络,其学习到的消息传递与聚合机制可视为对真实系统中数据传播动力学的模拟——节点间的信息流动对应物理空间中物质或信号的传输过程,而训练所得的边权重与聚合函数则编码了系统的传播规律。因此,一个训练良好的GNN本质上构成了目标系统动力学过程的数据驱动模拟器,能够在给定初始状态与边界条件下,复现系统的时空演化行为。


% ------------------------------------------------------------
% 2.2 图神经网络在地球科学中的应用
% ------------------------------------------------------------
\section{图神经网络在地球科学中的应用}
\label{sec:gnn_earth_science}

近年来,图神经网络在地球科学领域取得了突破性进展,涵盖气象预报与大气污染预测两大方向。本节综述该领域的代表性工作,提炼这类方法的共性特点,为本文的模型设计提供方法论基础。

\subsection{气象预报领域}

在气象预报领域,Google DeepMind的GraphCast\citep{lam2023learning}首次实现数据驱动模型在中期天气预报精度上超越欧洲中期天气预报中心(European Centre for Medium-Range Weather Forecasts, ECMWF)业务系统;NVIDIA的FourCastNet\citep{pathak2022fourcastnet}基于傅里叶神经算子在频谱域建模全球动力学。GraphCast的核心创新在于多尺度图结构——采用\cqt{编码器--处理器--解码器}架构,在细网格(ERA5再分析数据格点,约100万节点)与粗网格(准均匀球面网格,约4万节点)间建立跨尺度连接,实现计算效率(10天全球预报仅需1分钟)与长程依赖建模的平衡。这类基于规则网格构建的稠密图结构,节点按经纬度均匀分布,适用于全域场的连续建模。本文第\ref{chap:inference}章的网格推断任务采用了类似的稠密网格图结构,在规则格点上进行空间插值与推断。

\subsection{大气污染预测领域}

在大气污染预测领域,时空图神经网络(STGNN)已成为主流技术架构\citep{zhou2020graph,wu2020comprehensive}。与气象预报的规则网格不同,空气质量监测站点在空间上呈稀疏且不规则分布——站点数量有限(通常为数百至数千个)、站点位置受城市规划与地形约束,形成天然的稀疏图结构。如第\ref{chap:introduction}章所述,这类模型通过图结构(监测站点网络)引入归纳偏置,使用图卷积刻画站点间的空间关联,配合时序卷积或长短期记忆网络(Long Short-Term Memory, LSTM)建模浓度的时间演变。代表性工作包括:STGCN\citep{yu2018spatio}建立了时空图建模的基础框架;DCRNN\citep{li2018diffusion}将信息传播建模为有向图上的扩散过程,与大气污染物的扩散传输机制天然对应;Graph WaveNet\citep{wu2019graph}引入自适应邻接矩阵学习机制,无需预定义图结构即可端到端学习节点间的隐式依赖。本文除第\ref{chap:inference}章的网格推断任务外,第\ref{chap:prediction}章(预测)和第\ref{chap:simulation}章(模拟)均采用这种基于监测站点或城市的稀疏图结构。

在物理启发融合方向,AirPhyNet\citep{hettigeairphynet}将质量守恒原理嵌入网络结构,在稀疏数据和突变场景下展现出更强的鲁棒性;Air-DualODE\citep{tianair}采用双分支架构,物理分支求解边界感知的扩散-平流方程,数据驱动分支学习额外的依赖关系,实现物理可解释性与数据拟合能力的平衡。

\subsection{方法共性与统一框架}

上述工作尽管在具体实现上各有侧重,但可抽象为统一的时空图神经网络框架:
\begin{equation}
\mathbf{H}^{(l+1)} = f_{\text{temporal}}\left(f_{\text{spatial}}\left(\mathbf{H}^{(l)}, \mathcal{G}\right)\right)
\label{eq:stgnn_general}
\end{equation}
\noindent 式中,$\mathbf{H}^{(l)} \in \mathbb{R}^{N \times T \times F}$为第$l$层的节点时空表示($N$为节点数,$T$为时间步数,$F$为特征维度),$f_{\text{spatial}}(\cdot)$为空间模块(GCN/ChebNet/MPNN),$f_{\text{temporal}}(\cdot)$为时间模块,如时间卷积网络(Temporal Convolutional Network, TCN)、门控循环单元(Gated Recurrent Unit, GRU)或Transformer,$\mathcal{G}$为图结构。

这类方法具有以下共性特点:

(1)图结构编码空间依赖。通过邻接矩阵$\mathbf{A}$编码节点间的空间关系,可基于地理距离、风场、相关性等多种先验构建。图结构的设计直接影响模型捕捉空间传输模式的能力。

(2)时空交替建模。空间模块与时间模块交替堆叠,分别捕捉空间依赖与时序演化,最终实现时空联合建模。典型架构为$L$层时空块的串联:$[\text{S-Conv} \rightarrow \text{T-Conv}]^L$。

(3)编码-处理-解码架构。遵循\cqt{编码器--处理器--解码器}的通用范式:编码器将原始输入映射到隐空间,处理器在隐空间中执行时空演化,解码器将隐表示映射回目标空间。

(4)数据驱动的算子学习。核心思想是学习从输入场到输出场的映射算子,而非显式求解偏微分方程。这种范式避免了数值离散化的稳定性限制,同时通过大规模数据驱动实现高精度预测。

基于上述共性认识,本文在后续章节中将针对大气污染问题的特点进行定制化设计:通过风场驱动的有向图建模平流传输方向性(第\ref{chap:prediction}章),通过扩散--平流双图融合两类物理传输过程(第\ref{chap:inference}章),通过非自回归映射消除长期模拟的误差累积(第\ref{chap:simulation}章)。


% ------------------------------------------------------------
% 2.3 大气污染的物理基础
% ------------------------------------------------------------
\section{大气污染的物理基础}
\label{sec:atmospheric_dynamics}

纯数据驱动的图神经网络虽然具备强大的表示学习能力,但在地球科学领域面临两个核心挑战:一是数据稀疏性导致的过拟合风险,二是预测结果可能违背物理规律(如质量守恒)。将领域知识融入深度学习模型——即物理启发的机器学习(Physics-Informed Machine Learning)——是应对这些挑战的有效途径\citep{karniadakis2021physics,reichstein2019deep}。

本节的核心问题是:大气污染建模的物理基础是什么?如何将这些物理知识转化为神经网络的归纳偏置(Inductive Bias,即学习算法从有限数据泛化时所依赖的先验假设\citep{battaglia2018relational})?

在大气污染物理领域,存在两个核心的约束方程:\textbf{对流-扩散方程}(公式\eqref{eq:advection_diffusion})描述污染物浓度场的时空演化规律,\textbf{质量守恒方程}(公式\eqref{eq:mass_conservation})作为其积分形式,约束封闭系统内污染物总量的收支平衡。化学传输模式(如CMAQ、WRF-Chem)通过数值方法直接求解这些方程来模拟污染物演化。本文的策略有所不同:不直接求解方程,而是将这两个核心约束\textbf{隐式地融入图神经网络的架构与训练过程}中——对流-扩散的动力学通过隐空间的消息传递机制体现,质量守恒约束则在解码到物理空间后,以损失函数的形式与数据拟合项并列作用。

\subsection{对流-扩散方程}
\label{subsec:advection_diffusion}

大气污染物浓度场的时空演化遵循对流-扩散方程,这是化学传输模式求解的基本控制方程:
\begin{equation}
\frac{\partial C}{\partial t} = \underbrace{-\mathbf{u} \cdot \nabla C}_{\text{平流项}} + \underbrace{\nabla \cdot (K \nabla C)}_{\text{扩散项}} + \underbrace{R}_{\text{化学反应}} + \underbrace{S}_{\text{排放源}} - \underbrace{D}_{\text{沉降汇}}
\label{eq:advection_diffusion}
\end{equation}
\noindent 式中$C$为污染物浓度,$\mathbf{u}$为风速矢量,$K$为扩散系数,$R$、$S$、$D$分别为化学反应项、排放源项和沉降汇项。该方程的各项在本文的GNN设计中均有对应实现(见表\ref{tab:pde_gnn_mapping}):

\begin{table}[htbp]
    \centering
    \caption{对流-扩散方程各项与GNN设计的对应关系}
    \caption*{对流-扩散方程的各物理项在本文提出的图神经网络模型中均有对应的结构设计,实现了物理机理与数据驱动的深度融合。}
    \label{tab:pde_gnn_mapping}
    \begin{tabular}{@{}lll@{}}
        \toprule
        方程项 & 物理含义 & GNN实现(本文章节) \\
        \midrule
        平流项$-\mathbf{u} \cdot \nabla C$ & 风场驱动的定向迁移 & 有向图消息传递(第\ref{chap:prediction}、\ref{chap:inference}章) \\
        扩散项$\nabla \cdot (K \nabla C)$ & 湍流混合的各向同性扩散 & 对称邻接矩阵/扩散核(第\ref{chap:inference}章) \\
        化学反应项$R$ & 光化学反应生成与消耗 & 节点级非线性变换(第\ref{chap:prediction}章LID模块) \\
        排放源项$S$ & 污染物排放速率 & 显式输入特征(第\ref{chap:prediction}--\ref{chap:simulation}章) \\
        沉降汇项$D$ & 干湿沉降移除过程 & 时间维度衰减(第\ref{chap:prediction}章TAD模块) \\
        \bottomrule
    \end{tabular}
\end{table}

\subsection{质量守恒约束}
\label{subsec:physics_constraints}

对流-扩散方程\eqref{eq:advection_diffusion}的积分形式即为质量守恒定律。在封闭系统内,污染物总量的变化应等于源汇之差:
\begin{equation}
\frac{\partial}{\partial t} \int_{\Omega} C \, d\Omega = \int_{\Omega} (S - D + R) \, d\Omega - \oint_{\partial\Omega} \mathbf{F} \cdot \mathbf{n} \, dS
\label{eq:mass_conservation}
\end{equation}
\noindent 式中$\Omega$为控制区域,$\partial\Omega$为区域边界,$\mathbf{F}$为通量,$\mathbf{n}$为边界法向量。

综上,公式\eqref{eq:advection_diffusion}(对流-扩散方程)与公式\eqref{eq:mass_conservation}(质量守恒方程)是大气污染物理的两个核心约束。在本文的图神经网络建模中,这两个约束被隐式地融入网络的不同层次:

对流-扩散动力学$\rightarrow$隐空间消息传递。公式\eqref{eq:advection_diffusion}描述的平流与扩散过程,通过GNN在隐空间(Latent Space)中的消息传递机制体现。具体而言,编码器(Encoder)将观测数据从物理空间映射到隐空间表示;在隐空间中,图卷积的消息传递模拟污染物在节点间的传输——平流项对应有向图上的定向传递,扩散项对应无向图上的各向同性扩散。

质量守恒$\rightarrow$物理空间损失约束。公式\eqref{eq:mass_conservation}的守恒约束则在物理空间中施加。解码器(Decoder)将隐空间表示映射回物理空间的浓度预测后,质量守恒以损失函数的形式约束模型输出——该物理约束项与数据拟合损失并列,共同引导模型学习符合物理规律的时空表示。这种设计使得守恒约束直接作用于可观测的物理量,而非抽象的隐层表示,确保约束的物理可解释性。

下一节将具体阐述物理知识融入GNN的三种实现途径。

\subsection{物理知识融入图神经网络的途径}
\label{subsec:physics_guidance}

基于上述物理方程,本文将物理知识融入GNN的途径归纳为以下三个层面:

(1)图网络设计。通过邻接矩阵$\mathbf{A}$将物理传输机制编码为图的拓扑结构。例如,基于风场构建有向边以模拟平流项的方向性(第\ref{chap:prediction}章PM$_{2.5}$-GNN),设计扩散核与平流核分别模拟公式\eqref{eq:advection_diffusion}中的扩散项与平流项(第\ref{chap:inference}章SPIN)。

(2)网络架构设计。将对流-扩散方程的物理过程解耦为独立的网络模块。例如,第\ref{chap:prediction}章PCDCNet的LID--STD--TAD三模块分别对应化学反应与排放项、平流与扩散项、沉降与累积项。

(3)损失函数设计。通过物理约束项软约束模型输出满足物理规律:
\begin{equation}
\mathcal{L} = \mathcal{L}_{\text{data}} + \lambda \mathcal{L}_{\text{physics}}
\label{eq:physics_loss}
\end{equation}
\noindent 式中$\mathcal{L}_{\text{data}}$为数据拟合项(如均方误差),衡量模型预测与观测数据之间的偏差;$\mathcal{L}_{\text{physics}}$为物理约束项,将领域物理知识以软约束的形式嵌入优化目标;$\lambda$为平衡系数,控制数据拟合与物理一致性之间的权衡。物理约束项$\mathcal{L}_{\text{physics}}$的具体形式因任务而异:第\ref{chap:prediction}章设计了领域知识约束(DIC)损失,基于公式\eqref{eq:mass_conservation}的质量守恒原理,约束相邻时间步之间空间传输贡献的时间连续性与空间一致性;第\ref{chap:inference}章则利用卫星遥感反演的AOD(气溶胶光学厚度)数据构建梯度约束,引导模型学习与遥感观测一致的空间分布模式。


% ------------------------------------------------------------
% 2.4 本文研究范式
% ------------------------------------------------------------
\section{本文研究范式}
\label{sec:our_paradigm}

前述各节介绍了图神经网络的基本原理(第\ref{sec:gnn_basics}节)、其在地球科学中的前沿应用(第\ref{sec:gnn_earth_science}节)以及大气污染动力学的物理基础(第\ref{sec:atmospheric_dynamics}节)。本节在此基础上,阐述\textbf{本文提出的物理启发时空图神经网络研究范式},为后续三个应用章节奠定方法论框架。

\subsection{核心思想:物理启发的时空图神经网络}
\label{subsec:core_idea}

本文提出的研究范式以时空图神经网络为核心建模工具,通过在其架构中嵌入大气科学的领域知识,实现\cqt{数据拟合}与\cqt{物理一致}的双重目标。选择GNN源于其与大气污染系统的天然契合——监测站点构成图的节点,站点间的空间关联构成图的边,污染物的时空演化可建模为图上的信号传播过程。根据融入方式的不同,物理知识可在数据、模型、损失三个环节嵌入\citep{willard2022integrating}:数据环节通过图结构编码物理先验(如风场有向图),模型环节通过网络架构嵌入物理结构(如平流/扩散解耦),损失环节通过物理惩罚项约束模型输出(如质量守恒)。

\subsection{统一建模框架}
\label{subsec:unified_framework}

基于上述理论基础,本文构建了面向预测、推断与模拟三类核心任务的统一建模框架,如图\ref{fig:unified_framework}所示。该框架将物理启发的时空图神经网络抽象为参数化映射$\mathcal{F}_\Theta$,通过端到端的梯度优化从数据中学习模型参数$\Theta$。尽管三类任务在具体形式上有所差异,但均可在统一的\cqt{编码$\rightarrow$隐空间动力学$\rightarrow$解码}架构下表达。

\begin{figure}[htbp]
    \centering
    \includegraphics[width=0.95\textwidth]{figures/chap02_unified_framework.pdf}
    \caption{物理启发的时空图神经网络统一框架}
    \caption*{蓝色模块为深度学习流水线(Pipeline),橙色标注为本文的核心创新——在编码层、隐空间动力学、解码层三个层次融入先验知识。模型通过损失函数$\mathcal{L}$计算预测结果$\hat{\mathbf{X}}$与观测数据的差异,并通过反向传播更新参数$\Theta$,实现数据驱动的端到端优化。}
    \label{fig:unified_framework}
\end{figure}

如图\ref{fig:unified_framework}所示,该框架包含三个核心模块,每个模块均可融入物理启发的设计:

编码层(Encoder)——将多源异构输入映射到高维隐空间。输入包括污染物历史浓度$\mathbf{X}$、气象变量$\mathbf{M}$(风速、风向、温度、湿度等)以及排放数据$\mathbf{E}$。编码层负责特征提取、异构数据融合与图结构嵌入。如图中橙色标注所示,物理启发可通过图网络设计实现——例如基于风场信息构建有向图,将大气传输的方向性先验编码到图的拓扑结构中。

隐空间动力学(Latent Dynamics)——在隐空间中学习污染物的时空演化规律,是模型的核心计算模块。如图所示,该模块包含空间模块与时间模块的交互建模:空间模块采用消息传递机制(Message Passing)或图卷积网络(GCN/ChebNet)捕获站点间的空间依赖;时间模块采用循环网络(LSTM/GRU)或时序卷积(TCN)建模时间演化。物理启发可通过图算子设计实现——例如设计扩散核(对称,模拟湍流扩散)与平流核(非对称,模拟风场输送),使网络的信息传播机制与物理传输过程相对应。

解码层(Decoder)——将隐空间表示映射回目标空间,输出预测、推断或模拟结果$\hat{\mathbf{X}}$。物理约束可通过损失函数嵌入——在数据拟合损失之外添加物理惩罚项(如基于质量守恒的DIC约束、基于遥感反演的AOD梯度约束),引导模型学习符合物理规律的时空表示。

图\ref{fig:unified_framework}右侧展示了模型的训练机制:损失函数$\mathcal{L}$综合数据拟合项与物理约束项,通过比较模型输出$\hat{\mathbf{X}}$与观测数据(Ground Truth)计算误差,并通过反向传播算法计算梯度、更新模型参数$\Theta$。这种端到端的梯度优化机制是本文方法区别于数值模式的关键——数值模式通过求解偏微分方程获得解析解,而本文方法通过数据驱动的参数学习自适应拟合观测数据,同时通过物理约束保证结果的物理一致性。

基于该统一框架,本文针对三类任务设计了具体模型:

预测任务(第\ref{chap:prediction}章):给定历史观测、未来气象和排放数据,预测未来污染物浓度。本文提出PM$_{2.5}$-GNN与PCDCNet模型,通过DIC损失显式约束时空连续性。

推断任务(第\ref{chap:inference}章):给定稀疏监测站点观测和辅助信息(AOD、气象),推断全域连续浓度场。本文提出SPIN模型,通过基于物理启发的扩散核与平流核设计增强空间插值能力。

模拟任务(第\ref{chap:simulation}章):给定未来气象情景和假设排放情景,模拟对应的浓度响应。本文提出IGNN模型,通过大规模数据驱动隐式学习排放-浓度响应关系。

三类任务共享相同的\cqt{编码$\rightarrow$隐空间$\rightarrow$解码}框架(图\ref{fig:unified_framework}蓝色模块),但物理启发的嵌入方式因任务特点而异(图\ref{fig:unified_framework}橙色标注)。这种灵活的物理启发机制使得统一框架能够适应不同任务的特定需求,同时保持方法论上的一致性。

\subsection{与现有方法的对比}
\label{subsec:comparison}

表\ref{tab:paradigm_comparison}对比了本文研究范式与现有方法的主要区别。

\begin{table}[htbp]
    \centering
    \caption{本文研究范式与现有方法的对比}
    \caption*{本文提出的融合物理启发的图神经网络建模范式兼具数值模式的物理基础和数据驱动方法的计算效率,在泛化能力和可解释性之间实现平衡。}
    \label{tab:paradigm_comparison}
    \begin{tabular}{@{}lp{2.8cm}p{2.8cm}p{3cm}@{}}
        \toprule
        \textbf{维度} & \textbf{数值模式} & \textbf{纯数据驱动} & \textbf{本文范式} \\
        \midrule
        物理基础 & 显式PDE求解 & 无 & 隐式物理启发 \\
        计算效率 & 低(小时级) & 高(秒级) & 高(秒级) \\
        数据需求 & 初边值条件 & 大量历史数据 & 中等+物理先验 \\
        泛化能力 & 强(物理外推) & 弱(分布内) & 中等(物理启发) \\
        可解释性 & 强 & 弱 & 中等 \\
        \bottomrule
    \end{tabular}
\end{table}

数值模式(如CMAQ、WRF-Chem)基于第一性原理求解偏微分方程,具有完备的物理机理,但计算代价高昂;纯数据驱动方法(如标准LSTM)计算高效,但缺乏物理启发,泛化能力有限。本文提出的融合物理启发的图神经网络建模范式取两者之长——保持深度学习的计算效率,同时通过多层次物理启发增强模型的物理一致性与泛化能力。


% ------------------------------------------------------------
% 2.5 本章小结
% ------------------------------------------------------------
\section{本章小结}
\label{sec:method_summary}

本章系统阐述了物理启发的时空图神经网络建模范式,为后续三个应用章节奠定理论基础。首先介绍了图神经网络的基础理论(谱方法与消息传递范式)及其在地球科学中的前沿应用;然后阐述了对流-扩散方程的物理含义与质量守恒约束,并从图结构、网络架构、损失函数和输入特征四个层面说明了物理知识融入GNN的具体途径;最后提出了\cqt{融合物理启发的图神经网络建模}研究范式,构建了预测、推断、模拟三类任务的统一框架(图\ref{fig:unified_framework})。本章所建立的方法论框架为后续章节的具体模型设计提供了统一的理论视角。
  % 物理约束的时空图神经网络建模范式(理论基础)
% ============================================================
% 第三章 物理约束时空图神经网络的大气污染预测
% 基于复杂系统数据驱动建模的大气污染研究
% ============================================================

\chapter{物理约束时空图神经网络的大气污染预测}
\label{chap:prediction}

空气质量预报对于政府应急决策、企业生产调度和公众健康防护具有重要意义。本章基于第\ref{chap:methodology}章所建立的物理约束时空图神经网络建模范式,针对多站点多污染物协同预测问题,提出PM$_{2.5}$-GNN与PCDCNet两个递进式模型,实现高精度的72小时\PM 与\ozone 联合预报。

% ------------------------------------------------------------
% 3.1 引言
% ------------------------------------------------------------
\section{引言}
\label{sec:pred_intro}

空气质量预报(Air Quality Forecasting, AQF)旨在预判未来24至72小时的污染物浓度变化,为重污染天气预警、应急响应决策以及公众健康防护提供科学依据。正如第\ref{chap:introduction}章所述,大气污染作为一个复杂系统演化过程(详见第\ref{subsec:complex_system}节),其动态行为受排放源、气象条件以及物理化学过程的多重交互影响,呈现显著的非线性特征和时空异质性。大气污染时序预测问题可表述为:给定历史空气质量观测序列、气象场数据和排放清单信息,预测未来多时步的污染物浓度。

核心挑战与现有方法局限。实现高精度的空气质量预报面临以下三方面挑战:

(1)物理一致性保障。纯数据驱动模型可能产生违背质量守恒等物理定律的预测结果,尤其在长时序预测和极端污染事件中表现不稳定。现有方法中,数值化学传输模式(如CMAQ、WRF-Chem)虽具完备的物理化学方程组,但高昂的计算代价制约了其实时应用;早期深度学习方法(如LSTM、CNN)虽计算高效,却将物理过程视为\cqt{黑盒},缺乏显式物理约束。如何将公式(\ref{eq:advection_diffusion})中的物理约束有效嵌入深度学习框架,是保障预测结果物理合理性的关键。

(2)风场驱动的跨区域传输建模。污染物的跨区域迁移受风场主导,呈现强烈的方向性和时变性。然而,现有时空图神经网络多采用基于欧氏距离的静态图结构,未能有效融合风场信息,无法刻画污染物\cqt{上风向影响下风向}的定向传输规律。如何将气象数据中的风场信息有效融入图神经网络的边特征构建,实现物理驱动的有向图结构动态更新,是多源数据融合的核心问题。

(3)多污染物协同预测。\PM 与\ozone 共享NOx和VOCs等前体物,二者在光化学过程中存在复杂的非线性耦合关系——在特定条件下呈现\cqt{跷跷板效应}。然而,多数现有模型仅针对\PM 单一污染物,未能建模\PM 与\ozone 之间的光化学耦合关系。如何在统一的网络架构内同时建模一次污染物与二次污染物的生成与消耗过程,考验模型的泛化与表征能力。

针对上述挑战,本章提出两个递进式模型:PM$_{2.5}$-GNN首次将风场信息融入图神经网络的边特征构建,实现对定向传输的显式建模;PCDCNet在此基础上进一步引入物理化学动力学约束,通过LID--STD--TAD模块化架构实现过程解耦与联合建模。


% ------------------------------------------------------------
% 3.2 问题定义
% ------------------------------------------------------------
\section{问题定义}
\label{sec:pred_problem}

本章聚焦的核心科学问题是:基于历史空气质量观测、气象条件和排放数据,预测未来$T$个时间步的多污染物浓度分布。图\ref{fig:prediction_problem}展示了该问题的整体框架。

\begin{figure}[htbp]
    \centering
    \includegraphics[width=\textwidth]{figures/chap03_prediction_problem.pdf}
    \caption{大气污染时序预测问题示意图}
    \caption*{输入:历史空气质量观测$\mathbf{X}_{-T'+1:0}$(浅蓝色背景)、气象场$\mathbf{M}_{-T'+1:T}$、排放数据$\mathbf{E}_{-T'+1:T}$;输出:未来$T$个时间步的\PM 与\ozone 浓度预测$\hat{\mathbf{X}}_{1:T}$(浅粉色背景);模型:基于空间距离构建图结构$\mathcal{G}$,通过PM$_{2.5}$-GNN/PCDCNet建模节点间的时空相关性。}
    \label{fig:prediction_problem}
\end{figure}

如图所示,大气污染时序预测问题的输入-输出结构如下:输入包括三类数据——历史空气质量观测序列$\mathbf{X}_{-T'+1:0}$(图中左侧浅蓝色背景区域,从$t=-T'+1$至$t=0$共$T'$个时间步)、气象场数据$\mathbf{M}$(包括风速风向、温度、湿度、边界层高度等)以及排放清单$\mathbf{E}$;输出为未来$T$个时间步的\PM 与\ozone 浓度预测序列$\hat{\mathbf{X}}_{1:T}$(图中右侧浅粉色背景区域,从$t=1$至$t=T$共$T$个时间步)。图中蓝色圆点代表监测站点/城市节点,节点间的连线表示基于空间距离构建的图结构。

时间坐标定义。在空气质量预报问题中,我们将起报时刻(Forecast Initialization Time)定义为$t=0$,它是最后一个可获取空气质量观测数据的时间戳。以此为参照,$t \leq 0$表示历史时刻($t=0, -1, -2, \ldots, -T'+1$共$T'$个时间步),$t>0$表示未来时刻($t=1, 2, \ldots, T$共$T$个待预测时间步)。历史窗口$T'$用于捕捉污染物浓度的时序演化规律,预测窗口$T$决定了预报的时间范围(Lead Time),在此期间模型需要未来的气象预报和排放数据作为驱动输入。时间分辨率与$t$的具体对应关系详见第\ref{subsec:knowair_dataset}节的数据集描述。

形式化地,设研究区域包含$N$个城市或监测站点,构成图结构$\mathcal{G} = (\mathcal{V}, \mathcal{E})$,其中$\mathcal{V}$为节点集合($|\mathcal{V}|=N$),$\mathcal{E}$为边集合。预测问题可形式化表述为:

\begin{equation}
\hat{\mathbf{X}}_{1:T} = \mathcal{F}_\Theta\left(\mathbf{X}_{-T'+1:0}, \mathbf{M}_{-T'+1:T}, \mathbf{E}_{-T'+1:T}, \mathcal{G} \right)
\label{eq:pred_problem}
\end{equation}

\noindent 式中,$\mathbf{X} \in \mathbb{R}^{N \times D_X}$表示空气质量观测($D_X$为污染物种类数,本文主要关注\PM 和\ozone),$\mathbf{M} \in \mathbb{R}^{N \times D_M}$表示气象变量(包括风速、风向、温度、湿度、边界层高度等),$\mathbf{E} \in \mathbb{R}^{N \times D_E}$表示排放数据(包括\NOx、VOC、SO$_2$等),$T'$为历史窗口长度,$\Theta$为模型参数。

有别于传统的自回归预测(仅依赖历史浓度),本文的问题设定显式地将未来气象预报和未来排放作为输入变量纳入模型框架。这种设计思路与数值模式(如CMAQ)的输入范式保持一致,使得模型能够对排放变化作出响应,从而支撑后续的情景模拟应用。不同预测模型的输入输出范式对比将在第\ref{sec:pcdcnet}节详细介绍。


% ------------------------------------------------------------
% 3.3 数据与预处理
% ------------------------------------------------------------
\section{数据与预处理}
\label{sec:pred_data}

本节介绍用于模型训练与评估的多源数据集。如第\ref{chap:methodology}章所述,大气污染演化遵循对流-扩散方程(公式\eqref{eq:advection_diffusion}),涉及平流传输、扩散混合、化学反应、排放源和沉降汇等过程。相应地,本章的数据融合需涵盖三类核心变量:空气质量观测$\mathbf{X}$(对应浓度场$C$)、气象驱动$\mathbf{M}$(对应风场$\mathbf{u}$和扩散系数$K$)、排放清单$\mathbf{E}$(对应源项$S$)。

\subsection{研究区域与空气质量数据($\mathbf{X}$)}
\label{subsec:aq_data}

空气质量数据来源于中国国家环境监测总站(CNEMC)\footnote{\url{https://www.cnemc.cn/}}。研究覆盖两个重点区域:

京津冀及周边地区(BTHSA):涵盖\cqt{2+26}城市群,共152个国控监测站点,面积约430,000 km$^2$。该区域冬季重污染频发,是检验模型极端情景预测能力的理想场所。

长江三角洲地区(YRD):涵盖上海、江苏、浙江等省市,共约175个站点,面积约270,000 km$^2$。该区域\ozone 污染问题突出,是检验多污染物协同预测能力的典型区域。

监测变量包括\PM 和\ozone 等六种常规污染物,时间分辨率为1小时(2016-2023年)。数据经质量控制处理:异常值(\PM $>$1000 $\mu$g/m$^3$)标记为缺失,短期缺失($<$6小时)线性插值,长期缺失采用历史同期均值填补。

\subsection{气象数据($\mathbf{M}$)}
\label{subsec:meteo_data}

气象数据服务于对流-扩散方程中的平流项和扩散项建模。采用两个互补数据源:

离线训练:ERA5再分析数据集\citep{hersbach2020era5},空间分辨率0.25°×0.25°。

在线部署:GFS预报产品,每日更新4次,提供未来15天气象预报。

选取的变量包括:风速分量($M_u$, $M_v$)用于平流传输建模,边界层高度($M_{\text{blh}}$)用于垂直扩散表征,温度、湿度、降水、辐射等用于化学反应和沉降过程建模。这些变量与第\ref{chap:methodology}章表\ref{tab:pde_gnn_mapping}中的物理过程一一对应。

\subsection{排放清单数据($\mathbf{E}$)}
\label{subsec:emis_data}

排放数据对应对流-扩散方程中的源项$S$,是本文方法区别于传统自回归模型的关键输入。采用清华大学MEIC清单\citep{zheng2018trends}\footnote{\url{http://meicmodel.org.cn/}},包含\NOx、VOC、SO$_2$、NH$_3$、PM$_{2.5}$等物种的月均排放量,空间分辨率0.25°×0.25°。参照\cite{inventory}的方法,通过时间分配因子将月均排放降尺度至小时分辨率,考虑日变化(交通早晚高峰)、周变化和季节变化。表\ref{tab:emission_variables}列出了排放清单中包含的主要变量及其物理意义。

\begin{table}[htbp]
    \centering
    \caption{排放清单数据($\mathbf{E}$)中包含的变量}
    \caption*{排放向量$\mathbf{E}_i^t \in \mathbb{R}^{5}$包含5种主要污染物的排放强度,用于表征节点$i$在时刻$t$的人为排放特征。}
    \label{tab:emission_variables}
    \begin{tabular}{@{}lllp{5cm}@{}}
        \toprule
        \textbf{变量名称} & \textbf{符号} & \textbf{单位} & \textbf{物理意义} \\
        \midrule
        氮氧化物 & $E_{\text{NO}_x}$ & ton/h & \ozone 的关键前驱物,参与光化学反应 \\
        挥发性有机物 & $E_{\text{VOC}}$ & ton/h & \ozone 的关键前驱物,与\NOx 协同生成\ozone \\
        二氧化硫 & $E_{\text{SO}_2}$ & ton/h & 氧化生成硫酸盐,贡献于二次\PM \\
        氨 & $E_{\text{NH}_3}$ & ton/h & 与酸性气体中和生成铵盐气溶胶 \\
        一次\PM & $E_{\text{PM}_{2.5}}$ & ton/h & 直接排放的颗粒物 \\
        \bottomrule
    \end{tabular}
\end{table}

\subsection{图结构构建($\mathcal{G}$)}
\label{subsec:graph_construction}

基于第\ref{chap:methodology}章的消息传递范式,构建空间图$\mathcal{G} = (\mathcal{V}, \mathcal{E})$。两节点$i$、$j$间建立边当且仅当:(1)地理距离$d_{ij} < 300$km;(2)无高于1200m的山脉阻隔。

具体地,设节点$i$的地理位置为$\rho_i = (\text{lat}_i, \text{lon}_i)$,表示其纬度和经度坐标。节点$i$与$j$之间的邻接关系定义为:
\begin{equation}
    A_{ij} = H(d_\theta - d_{ij}) \cdot H(m_\theta - m_{ij})
\label{eq:adjacency}
\end{equation}
\noindent 其中$H(x)$为Heaviside阶跃函数:
\begin{equation}
    H(x) = \begin{cases} 1, & x > 0 \\ 0, & x \leq 0 \end{cases}
\label{eq:heaviside}
\end{equation}

\noindent $d_{ij}$为节点$i$与$j$之间的地理距离,采用L2范数计算:
\begin{equation}
    d_{ij} = \|\rho_i - \rho_j\|_2
\label{eq:distance}
\end{equation}

\noindent $m_{ij}$为两节点连线路径上的相对山脉高度,定义为路径上最高点与两端点中较高者的海拔差:
\begin{equation}
    m_{ij} = \sup_{\lambda \in (0,1)} \left\{ h\left(\lambda\rho_i + (1-\lambda)\rho_j\right) \right\} - \max\left\{h(\rho_i), h(\rho_j)\right\}
\label{eq:mountain}
\end{equation}
\noindent 其中$h(\cdot)$为地形高程函数,$\lambda\rho_i + (1-\lambda)\rho_j$表示两城市连线上的插值点。该定义的物理含义是:只有当路径上存在显著高于两端城市的山脉时,$m_{ij}$才会取较大正值,从而阻断连边。

阈值参数设置为$d_\theta = 300$km和$m_\theta = 1200$m(约为边界层平均高度)。图\ref{fig:multi-image}展示了基于公式\eqref{eq:adjacency}构建的两个研究区域的城市网络结构。

\begin{figure}[htbp]
  \centering
  \subcaptionbox{京津冀及周边地区(BTHSA)城市网络\label{fig:graph-bthsa}}
    {\includegraphics[width=0.48\linewidth]{figures/chap03_graph_bthsa.png}}
  \hfill
  \subcaptionbox{长江三角洲地区(YRD)城市网络\label{fig:graph-yrd}}
    {\includegraphics[width=0.48\linewidth]{figures/chap03_graph_yrd.png}}
  \caption{研究区域的城市网络图结构}
  \caption*{节点代表城市/站点,边代表潜在污染传输通道。边的建立基于公式\eqref{eq:adjacency},综合考虑地理距离($<$300km)和地形阻隔(山脉海拔差$<$1200m)。}
  \label{fig:multi-image}
\end{figure}


\subsection{KnowAir-DS数据集}
\label{subsec:knowair_dataset}

基于上述多源数据,我们构建并公开发布了\textbf{KnowAir-DS}系列数据集\citep{wang2020pm2}:

KnowAir-DS-V1(2015-2018)\footnote{\url{https://github.com/shuowang-ai/PM2.5-GNN}}:覆盖184个城市,包含\PM 和8种气象变量,时间分辨率3小时,用于PM$_{2.5}$-GNN评估。图\ref{fig:knowair_v1_study_area}展示了KnowAir-DS-V1的研究区域及基于公式\eqref{eq:adjacency}构建的城市网络图结构。

\begin{figure}[htbp]
  \centering
  \includegraphics[width=0.7\linewidth]{figures/chap03_knowair_study_area.jpg}
  \caption{KnowAir-DS-V1数据集研究区域与城市网络}
  \caption*{左图为研究区域地理范围,覆盖中国中东部184个城市;右图为基于公式\eqref{eq:adjacency}构建的城市网络图$\mathcal{G}$,节点代表城市,边代表满足距离阈值($d < 300$km)和地形约束的潜在污染传输通道。}
  \label{fig:knowair_v1_study_area}
\end{figure}

KnowAir-DS-V2(2016-2023)\footnote{\url{https://zenodo.org/records/15614907}}:覆盖BTHSA和YRD共327个站点,包含2种污染物、8种气象变量和6种排放变量,时间分辨率1小时,总计70,128小时记录,用于PCDCNet评估。数据划分:训练集(2016-2019)、验证集(2020-2021)、测试集(2022-2023)。

两版本虽空间尺度不同,但均基于第\ref{chap:methodology}章的物理约束时空图神经网络范式,核心组件(平流系数、消息传递、物理约束)具有跨尺度迁移性。


% ------------------------------------------------------------
% 3.4 方法一:基于风场驱动图网络的PM2.5传输建模(PM2.5-GNN)
% ------------------------------------------------------------
\section{PM$_{2.5}$-GNN:风场驱动的图网络模型}
\label{sec:pm25gnn}

细颗粒物(\PM)具有显著的跨区域传输特性。研究表明,在季风气候条件下,污染物可在72小时内传输数百公里\citep{pongkiatkul2007assessment,wang2015long,hao2019transport}。\PM 的演化过程呈现典型的双重动力学特性:一方面,污染物沿风向从上游城市传输至下游城市;另一方面,在局地发生扩散累积。这种\cqt{传输-扩散}的双重动力学特性,要求预测模型必须具备捕捉长距离、有向时空依赖的能力。

现有的时空图神经网络(如GC-LSTM\citep{qi2019hybrid})通常基于地理距离构建无向图,假设节点间的影响是相互对称的。然而,大气中的平流输送(Advection)具有强烈的方向性:上游城市对下游城市的影响显著,反之则不然。针对这一问题,我们提出了PM$_{2.5}$-GNN\citep{wang2020pm2}模型,通过物理启发式的图结构设计显式编码风场信息,实现对\PM 传输过程的精确建模。


\subsection{图结构设计}
\label{subsec:pm25gnn_graph}

为了显式建模风场驱动的传输过程,我们设计了包含节点属性、边属性和有向邻接矩阵的动态知识图谱。

(1)节点属性。除了历史\PM 浓度外,我们将边界层高度、相对湿度、风速u/v分量、气温、总降水量、地表气压、K指数等8个气象因子作为节点特征,这些变量与\PM 的局地累积和垂直扩散密切相关。节点$i$在时刻$t$的特征向量定义为:
\begin{equation}
    \boldsymbol{\xi}_i^t = [\hat{\mathbf{X}}_i^{t-1}, \mathbf{M}_i^t]
\end{equation}
\noindent 式中$\hat{\mathbf{X}}_i^{t-1}$为上一时刻的\PM 浓度预测值(或观测值),$\mathbf{M}_i^t \in \mathbb{R}^{8}$为当前时刻的气象特征向量。

(2)边属性与平流系数。对于边特征,我们创新性地引入了平流系数(Advection Coefficient),将风场对污染传输的影响显式编码到图结构中。如图\ref{fig:advection_coeff}所示,设源节点$j$指向目标节点$i$,源节点的风速为$|v|$,两节点间距离为$d$,源节点风向$\beta$与两节点连线方向$\gamma$的夹角为$\alpha = |\gamma - \beta|$,则平流系数$S$定义为:

\begin{equation}
    S_{j \to i} = \text{ReLU}\left(\frac{|v|}{d} \cos(\alpha)\right)
\label{eq:advection_coeff}
\end{equation}

\begin{figure}[htbp]
  \centering
  \includegraphics[width=0.4\linewidth]{figures/chap03_advection_coeff.png}
  \caption{平流系数的计算示意图}
  \caption*{源节点$j$的风速为$|v|$,风向为$\beta$;$\gamma$为$j$指向$i$的连线方向;$\alpha$为二者夹角;$d$为两节点距离。当风从$j$吹向$i$时($\cos\alpha > 0$),平流系数为正。}
  \label{fig:advection_coeff}
\end{figure}

利用ReLU函数,我们确保了仅当风从$j$吹向$i$时(即$\cos\alpha > 0$),边权重才为正,从而在物理上强制了传输的方向性约束。这一设计使得模型能够理解\cqt{上风向影响下风向}的定向传输规律。公式的物理含义是:传输强度与源节点风速成正比,与传输距离成反比,且仅在风向与连线方向一致时有效。

表\ref{tab:pm25gnn_edge_attributes}列出了边属性中包含的所有变量。除平流系数外,还包括原始的风速、距离、角度信息,为模型提供了冗余特征以增强学习能力。

\begin{table}[htbp]
    \centering
    \caption{边属性($\mathbf{A}$)中包含的变量}
    \caption*{边属性向量$\mathbf{A}_{j \to i} \in \mathbb{R}^{5}$包含5个特征,用于刻画从源节点$j$到目标节点$i$的传输特性。}
    \label{tab:pm25gnn_edge_attributes}
    \begin{tabular}{@{}lllp{4.5cm}@{}}
        \toprule
        \textbf{变量名称} & \textbf{符号} & \textbf{单位} & \textbf{说明} \\
        \midrule
        源节点风速 & $A_{|v|}$ & km/h & 驱动传输的动力 \\
        节点间距离 & $A_{d}$ & km & 传输的空间尺度 \\
        源节点风向 & $A_{\beta}$ & ($^{\circ}$) & 气象学风向定义 \\
        连线方向 & $A_{\gamma}$ & ($^{\circ}$) & 从源节点指向目标节点 \\
        平流系数 & $A_{S}$ & - & 公式\eqref{eq:advection_coeff}计算 \\
        \bottomrule
    \end{tabular}
\end{table}


\subsection{模型架构与计算步骤}
\label{subsec:pm25gnn_model}

在构建的有向图上,我们采用消息传递神经网络(Message Passing Neural Network, MPNN)框架\citep{gilmer2017neural,battaglia2018relational}来模拟污染流的输送过程。模型架构如图\ref{fig:pm25gnn_model}所示,由知识增强的图神经网络(GNN)和时空门控循环单元(GRU)两个核心组件构成。

\begin{figure}[htbp]
  \centering
  \includegraphics[width=0.8\linewidth]{figures/chap03_pm25gnn_arch.png}
  \caption{PM$_{2.5}$-GNN模型架构示意图}
  \caption*{模型由三个核心组件构成:(1)知识增强的消息传递模块,利用风驱边权重计算邻居节点的污染传输通量;(2)时空GRU单元,融合空间聚合信息与历史隐状态;(3)输出层,预测下一时刻的\PM 浓度。橙色箭头表示输入流(import),蓝色箭头表示输出流(export)。}
  \label{fig:pm25gnn_model}
\end{figure}

节点$i$在时刻$t$的状态更新过程包含三个步骤:

步骤1:消息生成。根据源节点$j$和目标节点$i$的状态,以及有向边属性$\mathbf{A}_{j \to i}^t$,计算从$j$到$i$的传输消息:
\begin{equation}
    e_{j \to i}^t = \Psi([\boldsymbol{\xi}_j^t, \boldsymbol{\xi}_i^t, \mathbf{A}_{j \to i}^t])
\end{equation}
\noindent 式中$\Psi$为可学习的消息函数,由两层MLP实现。边属性$\mathbf{A}_{j \to i}^t$包含了平流系数等物理信息,使消息生成过程具有物理意义。

步骤2:净通量聚合。与传统GNN仅聚合输入消息不同,我们计算每个节点的净通量——即输入流与输出流之差:
\begin{equation}
    \zeta_i^t = \Phi\left(\sum_{j \in \mathcal{N}(i)} (e_{j \to i}^t - e_{i \to j}^t)\right)
\label{eq:message_agg}
\end{equation}
\noindent 式中$e_{j \to i}^t$代表从邻居$j$流入节点$i$的污染物(输入流),$e_{i \to j}^t$代表从节点$i$流出到邻居$j$的污染物(输出流),二者之差表示该节点的净通量积累。$\Phi$为单层MLP聚合函数。

这种\cqt{输入减输出}的设计具有明确的物理意义:根据质量守恒定律,节点$i$的浓度变化应等于净输入通量。通过显式建模输入与输出的差值,模型能够更准确地刻画污染物的\cqt{收支平衡}。

步骤3:时空状态更新。将聚合的空间信息$\zeta_i^t$与节点特征$\boldsymbol{\xi}_i^t$拼接后,输入到GRU单元中,与上一时刻的隐状态融合,完成时空状态的更新:
\begin{equation}
\begin{aligned}
x_i^t &= [\boldsymbol{\xi}_i^t, \zeta_i^t] \\
z_i^t &= \sigma(W_z \cdot [h_i^{t-1}, x_i^t]) \\
r_i^t &= \sigma(W_r \cdot [h_i^{t-1}, x_i^t]) \\
\tilde{h}_i^t &= \tanh(W \cdot [r_i^t \odot h_i^{t-1}, x_i^t]) \\
h_i^t &= (1 - z_i^t) \odot h_i^{t-1} + z_i^t \odot \tilde{h}_i^t
\end{aligned}
\label{eq:grucell}
\end{equation}
\noindent 式中$W_z$、$W_r$、$W$为可学习参数,$\sigma$为sigmoid激活函数,$\odot$为逐元素乘法。GRU的门控机制使模型能够自适应地选择保留历史信息还是更新新信息,有效捕捉污染物的长时序累积与衰减过程。

最后,通过输出层预测\PM 浓度:
\begin{equation}
    \hat{\mathbf{X}}_i^t = \Omega(h_i^t)
\end{equation}
\noindent 式中$\Omega$为单层MLP。

算法\ref{alg:pm25_gnn}给出了PM$_{2.5}$-GNN的完整训练与推理流程。

\begin{algorithm}[htbp]
\caption{PM$_{2.5}$-GNN 模型训练与推理流程}
\caption*{\textbf{PM$_{2.5}$-GNN}:算法包含初始化、自回归预测循环、MSE损失计算和参数更新四个阶段。预测循环中依次进行节点特征构建、净通量聚合、GRU状态更新和浓度预测。}
\label{alg:pm25_gnn}
\begin{algorithmic}[1]
    \REQUIRE 初始观测值 $\mathbf{X}_{0}$,未来$T$步的节点属性 $\{\mathbf{M}_t\}_{t=1}^T$ 和边属性 $\{\mathbf{A}_t\}_{t=1}^T$,图结构 $\mathcal{G}=(\mathcal{V}, \mathcal{E})$,已初始化参数 $\Theta$,学习率 $\alpha$
    \ENSURE 优化后的参数 $\Theta$(训练),预测序列 $\{\hat{\mathbf{X}}_t\}_{t=1}^T$(推理)

    \STATE 初始化所有节点的GRU隐状态 $\{h_i^0\}_{i \in \mathcal{V}} \leftarrow \mathbf{0}$
    \STATE 初始化预测序列 $\hat{\mathbf{X}}_0 \leftarrow \mathbf{X}_0$
    
    \FOR{$t \leftarrow 1$ to $T$}
        \FOR{每个节点 $i \in \mathcal{V}$}
            \STATE $\boldsymbol{\xi}_i^t \leftarrow [\hat{\mathbf{X}}_i^{t-1}, \mathbf{M}_i^t]$ \COMMENT{构建节点输入特征}
            \STATE $\zeta_i^t \leftarrow \Phi(\sum_{j\in \mathcal{N}(i)}(\Psi([\boldsymbol{\xi}_j^t, \boldsymbol{\xi}_i^t, \mathbf{A}_{j\to i}^t]) - \Psi([\boldsymbol{\xi}_i^t, \boldsymbol{\xi}_j^t, \mathbf{A}_{i\to j}^t])))$ \COMMENT{净通量聚合}
            \STATE $h_i^t \leftarrow \text{GRUcell}([\boldsymbol{\xi}_i^t, \zeta_i^t], h_i^{t-1})$ \COMMENT{时空状态更新}
            \STATE $\hat{\mathbf{X}}_i^t \leftarrow \Omega(h_i^t)$ \COMMENT{浓度预测}
        \ENDFOR
    \ENDFOR

    \STATE $\mathcal{L} \leftarrow \frac{1}{T} \sum_{t=1}^{T}\frac{1}{N} \sum_{i=1}^{N}(\hat{\mathbf{X}}_i^t - \mathbf{X}_i^t)^{2}$ \COMMENT{MSE损失}
    \STATE $\Theta \leftarrow \Theta - \alpha \frac{\partial \mathcal{L}}{\partial \Theta}$
    \STATE \RETURN $\Theta$(训练)\textbf{或} $\{\hat{\mathbf{X}}_t\}_{t=1}^T$(推理)
\end{algorithmic}
\end{algorithm}


\subsection{局限性分析}
\label{subsec:pm25gnn_limitation}

PM$_{2.5}$-GNN在KnowAir-DS数据集上的实验表明,相比于GC-LSTM等基线模型,其在72小时\PM 长时序预测中取得了显著优势(详见\ref{subsec:pm25gnn_results}节),特别是能够准确捕捉风向下游城市的污染峰值滞后现象。然而,随着环境治理需求的不断深化,PM$_{2.5}$-GNN逐渐显露出若干关键局限:

(1)污染物覆盖单一。PM$_{2.5}$-GNN仅针对\PM 建模,未覆盖\ozone 等与\PM 存在复杂耦合关系的重要污染物。而在实际治理中,\PM 与\ozone 的\cqt{跷跷板效应}日益凸显——\PM 浓度下降导致气溶胶对太阳辐射的散射减弱,近地面辐射增强反而加速光化学反应、促进\ozone 生成\citep{qu2023underlying}。单一污染物的预测已无法满足协同治理的需求。

(2)缺乏化学机制约束。模型侧重于物理传输过程(平流、扩散),缺乏对光化学反应(如VOC和\NOx 在太阳辐射下生成\ozone)的显式物理-化学约束。这导致模型在\ozone 预测任务上表现不佳,且在极端情景下可能产生违背化学机理的预测结果。

(3)无法响应排放变化。模型仅利用历史浓度和气象作为输入,未纳入排放清单(Emissions)。这使得模型无法回答\cqt{如果减排50\%,空气质量会如何变化}这一关键的政策制定问题,也无法支持后续的情景模拟应用(第\ref{chap:simulation}章)。

(4)物理约束缺失。模型训练仅优化预测损失,未引入质量守恒等物理约束。在长时序自回归预测中,误差可能累积发散,产生非物理结果。

鉴于上述局限,我们需要构建一个更全面、更具物理可解释性的模型体系。下一节将介绍针对这些问题的解决方案——PCDCNet模型。


% ------------------------------------------------------------
% 3.5 方法二:物理化学动力学约束网络(PCDCNet)
% ------------------------------------------------------------
\section{PCDCNet:物理化学动力学约束模型}
\label{sec:pcdcnet}

为解决PM$_{2.5}$-GNN的上述局限,我们又提出了\textbf{PCDCNet(Physical-Chemical Dynamics and Constraints Network)}\citep{wang2025pcdcnet}。该模型定位为数值模式(如CMAQ)的\textbf{深度学习代理模型(Surrogate Model)},在继承PM$_{2.5}$-GNN风场驱动图结构设计的基础上,进行了以下关键升级:

\begin{itemize}
    \item 多污染物联合预测:同时预报\PM 与\ozone 浓度,并建模其协同效应;
    \item 排放响应建模:融合排放清单作为动态输入,实现对排放变化的显式响应;
    \item 物理化学动力学建模:设计LID--STD--TAD三模块架构,分别建模局地化学反应、空间传输和时间累积过程;
    \item 领域知识约束:引入DIC约束,将质量守恒嵌入训练目标。
\end{itemize}


\subsection{设计理念}
\label{subsec:pcdcnet_design}

传统的AI预测模型通常采用自回归形式$\hat{\mathbf{X}}_{t+1} = f(\mathbf{X}_{1:t}, \mathbf{M}_{1:t})$,仅依赖历史观测和气象数据。这种范式存在两个根本性问题:(1)无法利用未来气象预报信息,导致预报时效受限;(2)无法响应排放变化,无法支持情景模拟。

PCDCNet旨在模拟数值模式(如CMAQ)的演化过程,其输入输出范式对齐了数值模式的设计:
\begin{equation}
    \hat{\mathbf{X}}_{t+1} = \mathcal{F}_\Theta\left(\mathbf{X}_t, \mathbf{M}_{t+1}, \mathbf{E}_{t+1} \right)
\label{eq:pcdcnet_formulation}
\end{equation}

即利用当前的污染状态$\mathbf{X}_t$、未来的气象条件$\mathbf{M}_{t+1}$和未来的排放强度$\mathbf{E}_{t+1}$来推演下一时刻的浓度。这种设计使得模型具备了对排放变化的敏感性,从而支持:

(1)排放情景模拟——给定不同排放路径(如减排30\%、50\%、70\%),预测空气质量的演变趋势(详见第\ref{chap:simulation}章)。

\begin{figure}[htbp]
  \centering
  \includegraphics[width=0.9\linewidth]{figures/chap03_io_paradigm.pdf}
  \caption{不同预测模型的输入输出范式对比}
  \caption*{$\mathbf{X}_t$为空气质量观测值,$\mathbf{M}_t$为气象变量,$\mathbf{E}_t$为排放变量,$\hat{\mathbf{X}}_t$为模型预测输出。XGBoost/LightGBM等机器学习方法对每个时间步独立拟合气象、排放到空气质量的映射关系,缺乏时序建模能力;AirPhyNet基于Neural ODE虽可使用历史气象与排放,但采用一次性输出的自回归范式,无法在预测阶段逐步融入未来气象预报和排放数据;iTransformer/TimeXer具备长程时序建模能力,但同样无法利用未来外源变量驱动预测。本文PCDCNet采用外驱动的状态空间模型范式——当前状态$\hat{\mathbf{X}}_t$由前一状态$\hat{\mathbf{X}}_{t-1}$与外部驱动($\mathbf{M}_t$、$\mathbf{E}_t$)共同决定,通过隐状态$\mathbf{H}_t$逐步传递时序信息,在每个预测时间步融合未来气象预报和排放数据,对齐数值模式CMAQ的代理模型范式。}
  \label{fig:input_output_paradigm}
\end{figure}

图\ref{fig:input_output_paradigm}与表\ref{tab:methods_comparison_vertical}对比了不同预测模型的输入输出范式。可以看到,PCDCNet是唯一同时融合历史观测、未来气象预报和排放数据的模型,与数值模式CMAQ的范式最为接近。

\subsection{模型总体架构}
\label{subsec:pcdcnet_architecture}

PCDCNet遵循第\ref{chap:methodology}章图\ref{fig:unified_framework}所示的\cqt{编码$\rightarrow$隐空间动力学$\rightarrow$解码}统一框架。如图\ref{fig:overall_framework}所示,模型通过三个专门设计的动力学模块来解耦大气过程,分别对应大气污染演化的三类核心物理机制:

\begin{figure}[htbp]
  \centering
  \includegraphics[width=\linewidth]{figures/chap03_pcdcnet_arch.pdf}
  \caption{PCDCNet模型总体架构}
  \caption*{模型输入为上一时刻污染物浓度$\mathbf{X}_{t-1}$、当前气象$\mathbf{M}_t$和排放$\mathbf{E}_t$,经嵌入层(Embed)映射为隐状态$\mathbf{H}_t$。三个核心动力学模块对应大气连续性方程$\frac{\partial C}{\partial t} + \nabla \cdot (\mathbf{u}C) = S - D$的不同物理过程:LID(局地交互动力学,MLP)输出$\mathbf{H}_t^R$,对应方程右端的源汇项$S-D$,建模本地化学反应与排放响应;STD(空间传输动力学,GNN消息传递)输出$\mathbf{H}_t^S$,对应平流-扩散项$\nabla \cdot (\mathbf{u}C)$,建模风驱空间传输;TAD(时间积累动力学,GRUcell)输出$\mathbf{H}_t^T$,对应时间演化项$\frac{\partial C}{\partial t}$,建模长时序累积效应。主路径通过读出层(Readout)生成浓度预测$\hat{\mathbf{X}}_t$;STD模块额外设置独立的读出层,显式提取空间传输贡献$\nabla\hat{\mathbf{X}}_t^S$,与上一时刻的$\nabla\hat{\mathbf{X}}_{t-1}^S$共同送入DIC(领域知识约束)模块,通过质量守恒约束确保传输过程的物理一致性。图中左上角的浓度-时间曲线示意了残差预测$\Delta\mathbf{X}$的含义。}
  \label{fig:overall_framework}
\end{figure}

如图\ref{fig:overall_framework}右上角所示,大气连续性方程$\frac{\partial C}{\partial t} + \nabla \cdot (\mathbf{u}C) = S - D$描述了浓度场$C$的时空演化,三个动力学模块分别对应该方程的不同物理项:

LID(Local Interaction Dynamics):对应方程右端的源汇项$S - D$,建模发生在局地的排放源$S$(一次排放与光化学生成)和沉降汇$D$(干/湿沉降),这些过程不涉及站点间的空间传输;

STD(Spatial Transport Dynamics):对应方程左端的平流-扩散项$\nabla \cdot (\mathbf{u}C)$,通过图神经网络的消息传递与邻域聚合机制建模污染物在风场驱动下的空间传输过程;

TAD(Temporal Accumulation Dynamics):对应方程左端的时间演化项$\frac{\partial C}{\partial t}$,通过GRU的门控记忆机制捕捉污染物浓度的长时序累积与衰减规律。

此外,如图\ref{fig:overall_framework}所示,STD模块设置了独立的读出层(Readout),将隐状态$\mathbf{H}_t^S$显式映射为空间传输贡献$\nabla\hat{\mathbf{X}}_t^S$。该分支与上一时刻的$\nabla\hat{\mathbf{X}}_{t-1}^S$共同送入DIC(领域知识约束)模块,通过质量守恒原则约束传输过程的物理一致性(详见\ref{subsec:pcdcnet_training}节)。这种\cqt{过程解耦+联合建模}的设计使得每个模块具有明确的物理对应,既保留了深度学习的灵活拟合能力,又增强了模型的可解释性。


\subsubsection{嵌入层:多源数据融合}
在进入三个动力学模块之前,PCDCNet首先通过嵌入层(Embed)将多源异构数据映射到统一的隐空间。设时刻$t$的输入包括上一时刻的污染物浓度$\hat{\mathbf{X}}_{t-1}$、当前时刻的气象变量$\mathbf{M}_t$和排放数据$\mathbf{E}_t$,嵌入过程为:

\begin{equation}
    \mathbf{H}_t = \text{Linear}([\hat{\mathbf{X}}_{t-1}, \mathbf{M}_t, \mathbf{E}_t])
\end{equation}

\noindent 式中$[\cdot, \cdot, \cdot]$表示特征拼接,$\text{Linear}$为线性投影层。这一步骤将维度不同的污染物、气象、排放数据统一投影到$d$维隐空间中,为后续的动力学建模提供统一的特征表示。

在训练阶段,当$t < 1$(历史编码阶段)时,$\hat{\mathbf{X}}_{t-1}$使用真实观测值$\mathbf{X}_{t-1}$;当$t \geq 1$(预测阶段)时,使用上一时刻的预测值。这种设计使模型能够在训练时学习真实的\cqt{状态-演化}映射,在推理时进行自回归预测。


\subsubsection{局地交互动力学模块(LID)}
局地交互动力学(Local Interaction Dynamics, LID)模块对应大气连续性方程右端的源汇项$S - D$,模拟发生在网格或站点内部的排放生成($S$)与化学转化过程。

在大气化学中,源项$S$包括一次排放(直接排放的颗粒物)和二次生成——\ozone 的生成依赖于\NOx 和VOC在太阳辐射下的光化学反应链,其速率受温度、辐射强度等因素的非线性调制;\PM 则受到一次排放与气态前驱物二次转化的双重贡献。汇项$D$对应干沉降与湿沉降过程。这些过程发生在每个站点局地,不涉及站点间的空间传输。

LID模块通过多层感知机(MLP)学习这种复杂的非线性响应:
\begin{equation}
    \mathbf{H}_t^R = \text{MLP}(\text{RMSNorm}(\mathbf{H}_t))
\end{equation}
\begin{equation}
    \mathbf{H}_t = \mathbf{H}_t + \mathbf{H}_t^R
\end{equation}

\noindent 式中$\mathbf{H}_t^R$表示LID模块的输出($R$代表Reaction,化学反应),MLP包含Linear层、RMSNorm归一化\citep{zhang2019root}、SiLU激活函数和Dropout正则化。残差连接确保梯度稳定传播。

通过将排放$\mathbf{E}_t$显式作为输入,LID模块能够学习:(1)\NOx 与VOC在不同气象条件(光照$rad$、温度$T$)下生成\ozone 的复杂非线性函数;(2)一次\PM 的直接排放贡献;(3)气态前驱物(SO$_2$、\NOx、NH$_3$)向二次气溶胶转化的响应关系。


\subsubsection{空间传输动力学模块(STD)}
空间传输动力学(Spatial Transport Dynamics, STD)模块对应大气连续性方程左端的平流-扩散项$\nabla \cdot (\mathbf{u}C)$,继承并升级了PM$_{2.5}$-GNN的思想,利用图神经网络的消息传递与邻域聚合机制模拟污染物在风场驱动下的平流与扩散传输。

基于城市网络图$\mathcal{G}$,STD模块通过图卷积操作聚合邻居节点的信息:
\begin{equation}
    \mathbf{H}_t^S = \text{Linear}\left(\tilde{\mathbf{L}}\mathbf{H}_t\right) = \text{Linear}\left((\mathbf{I} - \mathbf{D}^{-1/2}\mathbf{A}\mathbf{D}^{-1/2})\mathbf{H}_t\right)
\label{eq:std_gcn}
\end{equation}
\begin{equation}
    \mathbf{H}_t = \mathbf{H}_t + \mathbf{H}_t^S
\end{equation}

\noindent 式中$\mathbf{H}_t^S$表示STD模块的输出($S$代表Spatial,空间传输),$\tilde{\mathbf{L}}$为归一化图拉普拉斯矩阵,$\mathbf{A}$为邻接矩阵,$\mathbf{D}$为度矩阵。这种图卷积操作在物理上对应于大气扩散方程的空间离散化\citep{kipf2017semi,li2023improving}——拉普拉斯算子正是扩散方程中的核心算子。

与PM$_{2.5}$-GNN相比,STD模块的主要升级在于:(1)采用归一化拉普拉斯而非自定义的消息传递,提高了计算效率和数值稳定性;(2)引入了领域知识约束(DIC),确保传输过程满足质量守恒原则(详见\ref{subsec:pcdcnet_training}节)。


\subsubsection{时间积累动力学模块(TAD)}
时间积累动力学(Temporal Accumulation Dynamics, TAD)模块对应大气连续性方程左端的时间演化项$\frac{\partial C}{\partial t}$,利用门控循环单元(GRU)捕捉污染物浓度的长时序累积与衰减效应:
\begin{equation}
    \mathbf{H}_t^T = \text{GRUcell}(\mathbf{H}_t, \mathbf{H}_{t-1}^T)
\end{equation}
\begin{equation}
    \mathbf{H}_t = \mathbf{H}_t + \mathbf{H}_t^T
\end{equation}

\noindent 式中$\mathbf{H}_t^T$表示TAD模块的输出($T$代表Temporal,时间累积),$\mathbf{H}_{t-1}^T$为上一时刻的GRU隐状态。TAD模块整合了LID产生的化学变化和STD产生的物理传输,通过GRU的门控机制自适应地决定信息的保留与遗忘,从而捕捉:

(1)污染累积:在静稳天气下,边界层高度降低,污染物难以垂直扩散,导致浓度逐日累积;

(2)沉降衰减:污染物通过干沉降(重力沉降、湍流扩散)和湿沉降(降水冲刷)从大气中去除;

(3)长时序依赖:捕捉跨越多个时间步的污染演化规律,如周期性变化(日变化、周变化)和趋势变化。


\subsubsection{读出层与残差预测}
如图\ref{fig:overall_framework}所示,PCDCNet设置了两条读出路径。主读出层采用残差预测方式生成浓度输出,即预测浓度的变化量$\Delta\hat{\mathbf{X}}_t$而非绝对值:
\begin{equation}
    \Delta \hat{\mathbf{X}}_t = \text{Linear}(\mathbf{H}_t)
\end{equation}
\begin{equation}
    \hat{\mathbf{X}}_t = \hat{\mathbf{X}}_{t-1} + \Delta \hat{\mathbf{X}}_t
\label{eq:residual_pred}
\end{equation}

\noindent 这种残差形式与大气连续性方程中$\frac{\partial C}{\partial t}$的物理意义对应——浓度的变化通常是渐进的,残差预测有助于稳定长时序预测、减少非物理突变。

此外,STD模块设置了独立的读出层,从$\mathbf{H}_t^S$中显式提取空间传输贡献$\nabla\hat{\mathbf{X}}_t^S$(公式\eqref{eq:transport_contribution}),用于后续的DIC约束。这一设计使模型能够将浓度变化解耦为\cqt{传输引起的变化}与\cqt{局地源汇引起的变化},从而对传输过程施加独立的物理约束。


\subsection{损失函数与训练策略}
\label{subsec:pcdcnet_training}

损失函数的设计是PCDCNet实现物理一致性预测的核心。如第\ref{chap:methodology}章所述,大气污染物的时空演化遵循对流-扩散方程(公式\eqref{eq:advection_diffusion}),其中隐含着质量守恒原则——污染物不会凭空产生或消失,只能通过排放源产生、通过沉降去除或在空间中传输。为了将这一物理先验嵌入深度学习模型,我们设计了包含预测损失和领域知识约束(Domain-Informed Constraints, DIC)的复合损失函数。

\subsubsection{预测损失}

预测损失衡量模型输出与真实观测之间的差异。我们采用L1损失(平均绝对误差,MAE)作为预测损失函数:
\begin{equation}
    \mathcal{L}_{\mathrm{Pred}} = \frac{1}{NT} \sum_{n=1}^{N} \sum_{t=1}^{T} \|\hat{\mathbf{X}}_n^t - \mathbf{X}_n^t\|_1
\label{eq:pred_loss}
\end{equation}
\noindent 式中$N$为站点数量,$T$为预测时间步数,$\hat{\mathbf{X}}_n^t$和$\mathbf{X}_n^t$分别为站点$n$在时刻$t$的预测浓度和真实浓度。

选择L1损失而非L2损失(MSE)的原因在于:(1)L1损失对异常值更为鲁棒,而空气质量数据中常存在极端污染事件;(2)L1损失产生的梯度在误差较小时不会趋近于零,有助于模型在收敛后期继续优化。

\subsubsection{领域知识约束(DIC)}

如图\ref{fig:overall_framework}右上角的DIC模块所示,领域知识约束直接源于大气连续性方程$\frac{\partial C}{\partial t} + \nabla \cdot (\mathbf{u}C) = S - D$的质量守恒含义:在不考虑源汇项$S-D$的条件下,仅由空间传输$\nabla \cdot (\mathbf{u}C)$引起的浓度变化应满足全局守恒——从一个节点流出的污染物必然流入另一个节点,不会凭空产生或消失。

为此,我们通过STD模块的独立读出层,从隐状态$\mathbf{H}_t^S$中显式提取空间传输贡献:
\begin{equation}
    \nabla \hat{\mathbf{X}}_t^S = \text{Linear}(\mathbf{H}_t^S)
\label{eq:transport_contribution}
\end{equation}
\noindent 式中$\nabla \hat{\mathbf{X}}_t^S$仅编码由平流-扩散项$\nabla \cdot (\mathbf{u}C)$引起的浓度变化,而LID模块学习到的源汇项$S-D$贡献已通过$\mathbf{H}_t^R$独立表示。正是这种过程解耦,使得我们能够单独对传输过程施加质量守恒约束,而不影响LID模块对排放与化学反应的自由拟合。基于$\nabla \hat{\mathbf{X}}_t^S$,DIC约束包含空间守恒和时间守恒两个维度。

空间质量守恒约束。根据第\ref{chap:methodology}章的质量守恒方程(公式\eqref{eq:mass_conservation}),在不考虑化学源汇的情况下,通过扩散和平流传输的污染物在全局范围内应满足质量守恒。具体地,在任一时刻,每个节点通过传输过程\cqt{流出}的污染物总量应等于其邻居节点\cqt{流入}的总量:
\begin{equation}
    \sum_{v' \in \mathcal{N}(v)} \nabla \hat{X}_{v \to v'}^S = 0, \quad \forall v \in \mathcal{V}
\label{eq:spatial_conservation}
\end{equation}
\noindent 式中$\nabla \hat{X}_{v \to v'}^S$表示从节点$v$传输到邻居节点$v'$的污染物通量。该约束确保传输过程是\cqt{零和博弈}——一个节点的流出必然对应其他节点的流入,防止模型在传输建模中凭空\cqt{创造}或\cqt{消灭}污染物。

时间质量守恒约束。在封闭系统中,全局污染物总量在时间上应保持稳定(不考虑排放和沉降):
\begin{equation}
    \frac{d}{dt} \sum_{v \in \mathcal{V}} \nabla \hat{X}_{v}^S = 0
\label{eq:temporal_conservation}
\end{equation}
该约束确保STD模块学习到的传输模式在时间维度上也满足守恒性,防止长时序预测中出现累积性的质量漂移。

综合空间和时间两个维度的守恒约束,DIC损失函数定义为:
\begin{equation}
    \mathcal{L}_{\mathrm{DIC}} = \underbrace{\frac{1}{|\mathcal{V}|} \sum_{v} \left| \sum_{v' \in \mathcal{N}(v)} \nabla \hat{X}_{v \to v'}^S \right|}_{\text{空间守恒项}} + \underbrace{\frac{1}{T} \sum_{t} \left| \frac{d \nabla \hat{\mathbf{X}}_t^S}{dt} \right|}_{\text{时间守恒项}}
\label{eq:dic_loss}
\end{equation}
\noindent 第一项惩罚每个节点的净传输通量偏离零的程度,确保空间守恒;第二项惩罚全局传输贡献随时间的变化率,确保时间守恒。实际计算中,时间导数通过差分近似:$\frac{d \nabla \hat{\mathbf{X}}_t^S}{dt} \approx \nabla \hat{\mathbf{X}}_t^S - \nabla \hat{\mathbf{X}}_{t-1}^S$。

\subsubsection{总损失函数}

模型的总损失函数由预测损失与DIC约束损失加权组合:
\begin{equation}
    \mathcal{L} = \mathcal{L}_{\mathrm{Pred}} + \lambda \mathcal{L}_{\mathrm{DIC}}
\label{eq:total_loss}
\end{equation}
\noindent 式中$\lambda$为约束权重超参数,用于平衡预测精度与物理一致性。

DIC约束的引入体现了\cqt{物理约束数据驱动建模}的核心思想:通过软约束方式将领域知识嵌入损失函数,引导模型在优化预测精度的同时学习物理一致的传输模式。与硬约束(如在网络结构中强制守恒)相比,软约束具有更好的灵活性,允许模型在物理近似与数据拟合之间自适应平衡。实验表明(见\ref{subsec:pcdcnet_ablation}节),$\lambda$在$10^{-3}$至$10^{-2}$区间时模型性能最优——过小的$\lambda$无法有效约束传输行为,过大的$\lambda$则会牺牲预测精度。

\subsubsection{训练策略}

PCDCNet采用以下训练策略:
\begin{itemize}
    \item 优化器:Adam优化器\citep{kingma2015adam},初始学习率$1 \times 10^{-4}$;
    \item 学习率调度:采用ReduceLROnPlateau策略,当验证集MAE连续10个epoch不下降时,学习率衰减为原来的0.5倍;
    \item 早停机制:当验证集MAE连续20个epoch不下降时,停止训练,防止过拟合;
    \item 混合精度训练:使用FP16混合精度训练,在保持数值稳定性的同时提升计算效率;
    \item 批量大小:32个样本/批次,每个样本包含完整的时间序列(历史+预测)。
\end{itemize}

算法\ref{alg:algorithm}给出了PCDCNet的完整训练与推理流程。

\begin{algorithm}[htbp]
\caption{PCDCNet 模型训练与推理流程}
\caption*{\textbf{PCDCNet}:算法包含历史编码和预测两个阶段。每个时间步依次执行:嵌入层融合多源数据、LID模块处理局地化学反应、STD模块处理空间传输、TAD模块处理时间累积、残差预测生成浓度。训练时同时优化预测损失和DIC约束损失。}
\label{alg:algorithm}
\begin{algorithmic}[1]
    \REQUIRE 训练数据集 $\mathcal{D} = \{D_n\}_{n=1}^N$,初始化参数 $\Theta$,学习率 $\alpha$,DIC权重 $\lambda$
    \ENSURE 优化后的参数 $\Theta$(训练),预测序列 $\{\hat{\mathbf{X}}_t\}_{t=1}^T$(推理)

    \FOR{每个样本 $D_k = (\mathbf{X}, \mathbf{M}, \mathbf{E}) \in \mathcal{D}$}
        \STATE 初始化 $\mathbf{H}_{-T'+1}^T \leftarrow \mathbf{0}$
        \FOR{$t \leftarrow -T'+2$ to $T$}
            \IF{$t < 1$}
                \STATE $\mathbf{H}_t \leftarrow \text{Linear}([\mathbf{X}_{t-1}, \mathbf{M}_t, \mathbf{E}_t])$ \COMMENT{历史编码阶段:使用真实观测}
            \ELSE
                \STATE $\mathbf{H}_t \leftarrow \text{Linear}([\hat{\mathbf{X}}_{t-1}, \mathbf{M}_t, \mathbf{E}_t])$ \COMMENT{预测阶段:使用预测值}
            \ENDIF

            \STATE $\mathbf{H}_t^R \leftarrow \text{MLP}(\mathbf{H}_t); \quad \mathbf{H}_t \mathrel{+}= \mathbf{H}_t^R$ \COMMENT{LID模块:局地化学反应}
            \STATE $\mathbf{H}_t^S \leftarrow \text{GraphConv}(\mathbf{H}_t, \mathcal{G}); \quad \mathbf{H}_t \mathrel{+}= \mathbf{H}_t^S$ \COMMENT{STD模块:空间传输}
            \STATE $\mathbf{H}_t^T \leftarrow \text{GRUcell}(\mathbf{H}_t, \mathbf{H}_{t-1}^T); \quad \mathbf{H}_t \mathrel{+}= \mathbf{H}_t^T$ \COMMENT{TAD模块:时间累积}
            \STATE $\Delta \hat{\mathbf{X}}_{t} \leftarrow \text{Linear}(\mathbf{H}_t); \quad \hat{\mathbf{X}}_{t} \leftarrow \hat{\mathbf{X}}_{t-1} + \Delta \hat{\mathbf{X}}_{t}$ \COMMENT{残差预测}

            \IF{$t \geq 1$}
                \STATE 存储 $\hat{\mathbf{X}}_t$
                \STATE $\nabla \hat{\mathbf{X}}_t^S \leftarrow \text{Linear}(\mathbf{H}_t^S)$ \COMMENT{提取传输贡献}
                \STATE $\mathcal{L}_{\mathrm{DIC}} \mathrel{+}= \text{DIC}(\nabla \hat{\mathbf{X}}_{t-1}^S, \nabla \hat{\mathbf{X}}_t^S)$ \COMMENT{累加DIC损失}
            \ENDIF
        \ENDFOR
    \ENDFOR

    \STATE $\mathcal{L} \leftarrow \mathcal{L}_{\mathrm{Pred}} + \lambda \mathcal{L}_{\mathrm{DIC}}$
    \STATE $\Theta \leftarrow \Theta - \alpha \frac{\partial \mathcal{L}}{\partial \Theta}$
    \STATE \RETURN $\Theta$(训练)\textbf{或} $\{\hat{\mathbf{X}}_t\}_{t=1}^T$(推理)
\end{algorithmic}
\end{algorithm}


% ------------------------------------------------------------
% 3.6 实验与结果
% ------------------------------------------------------------
\section{实验与结果}
\label{sec:pred_experiment}

本节通过系统的实验验证PM$_{2.5}$-GNN和PCDCNet的有效性。实验按照递进关系组织:首先在KnowAir-DS数据集上评估PM$_{2.5}$-GNN的\PM 单污染物预测能力(\ref{subsec:pm25gnn_results}节);随后在KnowAir-DS-V2数据集上评估PCDCNet的多污染物联合预测能力(\ref{subsec:pcdcnet_main_results}节);最后通过消融实验深入探讨模型各组件的贡献(\ref{subsec:pcdcnet_ablation}节)。


\subsection{实验设置}
\label{subsec:exp_setup}

(1)数据集划分。本章实验使用两个数据集:

KnowAir-DS-V1(2015-2018):用于评估PM$_{2.5}$-GNN的\PM 单指标预测能力。数据按时间划分为三个子数据集,用于评估不同场景下的模型性能:

\begin{itemize}
    \item 全年评估(Dataset 1):训练集2015/1/1--2016/12/31,验证集2017年,测试集2018年,评估模型在一般场景下的预测能力;
    \item 采暖季评估(Dataset 2):仅使用11月至次年2月的数据,评估模型在重污染高发期的预测能力;
    \item 滚动预测(Dataset 3):训练集2016/9/1--2016/11/30,验证集2016/12,测试集2017/1,模拟在线系统的滚动预测场景。
\end{itemize}

KnowAir-DS-V2(2016-2023):用于评估PCDCNet的多污染物预测能力。训练集2016-2019年,验证集2020-2021年,测试集2022-2023年。表\ref{tab:pm25gnn_dataset_split}汇总了数据集划分方案。

\begin{table}[htbp]
    \centering
    \caption{数据集划分方案}
    \caption*{KnowAir-DS-V1数据集用于PM$_{2.5}$-GNN评估,包含全年、采暖季和滚动预测三种场景;KnowAir-DS-V2数据集用于PCDCNet多污染物评估。}
    \label{tab:pm25gnn_dataset_split}
    \begin{tabular}{@{}llccc@{}}
        \toprule
        \textbf{数据集} & \textbf{实验场景} & \textbf{训练集} & \textbf{验证集} & \textbf{测试集} \\
        \midrule
        \multirow{3}{*}{KnowAir-DS-V1} & 全年评估 & 2015/1--2016/12 & 2017年 & 2018年 \\
        & 采暖季评估 & 2015/11--2016/2 & 2016/11--2017/2 & 2017/11--2018/2 \\
        & 滚动预测 & 2016/9--2016/11 & 2016/12 & 2017/1 \\
        \midrule
        KnowAir-DS-V2 & 多污染物评估 & 2016--2019 & 2020--2021 & 2022--2023 \\
        \bottomrule
    \end{tabular}
\end{table}

(2)评价指标。采用以下指标评估模型性能\citep{evaluation}:

\begin{itemize}
    \item RMSE(均方根误差):$\text{RMSE} = \sqrt{\frac{1}{n}\sum_{i=1}^{n}(y_i - \hat{y}_i)^2}$,对大误差敏感;
    \item MAE(平均绝对误差):$\text{MAE} = \frac{1}{n}\sum_{i=1}^{n}|y_i - \hat{y}_i|$,反映平均预测偏差;
    \item CSI(临界成功指数):$\text{CSI} = \frac{\text{hits}}{\text{hits} + \text{misses} + \text{false alarms}}$,综合评估预警能力;
    \item POD(命中率):$\text{POD} = \frac{\text{hits}}{\text{hits} + \text{misses}}$,反映污染事件的捕捉能力;
    \item FAR(虚警率):$\text{FAR} = \frac{\text{false alarms}}{\text{hits} + \text{false alarms}}$,反映误报情况。
\end{itemize}

其中,CSI、POD、FAR的计算以75 $\mu$g/m$^3$为阈值(中国《环境空气质量标准》良好等级分界点\citep{evaluation}),将预测和观测二值化为\cqt{污染/非污染}后统计。

(3)基线方法。将模型与以下方法进行对比:

\begin{itemize}
    \item 机器学习基线:MLP、XGBoost\citep{chen2016xgboost}、LightGBM\citep{ke2017lightgbm};
    \item 通用时序模型:LSTM\citep{hochreiter1997long}、GRU\citep{cho2014learning}、iTransformer\citep{liuitransformer}、TimeXer\citep{wang2024timexer};
    \item 空气质量专用模型:GC-LSTM\citep{qi2019hybrid}、AirPhyNet\citep{hettigeairphynet};
    \item 消融变体:nodesFC-GRU(用全连接层替代GNN的消融模型,用于验证图结构的必要性)。
\end{itemize}

表\ref{tab:methods_comparison_vertical}对比了各模型的原生能力特征。可以看到,PCDCNet是唯一同时具备时序建模、多变量预测、外源变量融合、空间建模和物理约束的方法,与数值模式CMAQ的能力对齐。

\begin{table}[htbp]
\centering
\caption{基线模型原生能力对比}
\caption*{各列分别代表:AQF--是否为空气质量预报专用模型;Temp--是否建模时间依赖性;MultiV--是否支持多变量预测;Exog--是否能融合未来外源变量;Spat--是否建模空间相关性;Phy--是否融合物理约束。PCDCNet是唯一具备全部能力的方法。}
\label{tab:methods_comparison_vertical}
\begin{tabular}{@{}llcccccc@{}}
\toprule
\textbf{模型类别} & 模型 & \textbf{AQF} & \textbf{Temp} & \textbf{MultiV} & \textbf{Exog} & \textbf{Spat} & \textbf{Phy} \\
\midrule
\multirow{2}{*}{机器学习基线} & XGBoost & \xmark & \xmark & \xmark & \cmark & \xmark & \xmark \\
& LightGBM & \xmark & \xmark & \xmark & \cmark & \xmark & \xmark \\
\midrule
\multirow{2}{*}{通用时序模型} & iTransformer & \xmark & \cmark & \cmark & \xmark & \xmark & \xmark \\
& TimeXer & \xmark & \cmark & \cmark & \xmark & \xmark & \xmark \\
\midrule
\multirow{3}{*}{空气质量模型} & GC-LSTM & \cmark & \cmark & \xmark & \xmark & \cmark & \xmark \\
& PM$_{2.5}$-GNN & \cmark & \cmark & \xmark & \cmark & \cmark & \xmark \\
& AirPhyNet & \cmark & \cmark & \xmark & \xmark & \cmark & \cmark \\
\midrule
\multirow{2}{*}{本文方法} & CMAQ (参考) & \cmark & \cmark & \cmark & \cmark & \cmark & \cmark \\
& \textbf{PCDCNet} & \cmark & \cmark & \cmark & \cmark & \cmark & \cmark \\
\bottomrule
\end{tabular}
\end{table}

(4)实现细节。模型基于PyTorch实现,图计算使用PyTorch Geometric(PyG)库\citep{fey2019fast}。训练在NVIDIA RTX 4070 SUPER GPU上进行。主要超参数设置:隐层维度$d=32$,历史窗口$T'=24$(72小时),预测窗口$T=24$(72小时),学习率$10^{-4}$,批大小32,训练轮数100,早停patience为10。优化器使用Adam,学习率调度使用ReduceLROnPlateau。


\subsection{PM$_{2.5}$-GNN实验结果}
\label{subsec:pm25gnn_results}

表\ref{tab:pm25gnn_overall_performance}展示了PM$_{2.5}$-GNN与基线模型在KnowAir-DS-V1数据集三个子数据集上的整体性能对比。每个指标报告10次实验的均值和标准差。

\begin{table*}[htbp]
    \centering
    \caption{PM$_{2.5}$-GNN与基线模型在KnowAir-DS-V1数据集上的性能对比}
    \caption*{报告的指标为72小时预测的平均值,包括RMSE、MAE(单位:$\mu$g/m$^3$)和预警指标CSI、POD、FAR(单位:\%)。每个指标报告10次实验的均值$\pm$标准差,最优结果已加粗显示。}
    \label{tab:pm25gnn_overall_performance}
    \resizebox{\textwidth}{!}{%
    \begin{tabular}{@{}clcccccc@{}}
        \toprule
        \textbf{数据集} & \textbf{指标} & \textbf{MLP} & \textbf{LSTM} & \textbf{GRU} & \textbf{GC-LSTM} & \textbf{nodesFC-GRU} & \textbf{PM$_{2.5}$-GNN} \\
        \midrule
        \multirow{5}{*}{全年} 
        & RMSE ($\mu$g/m$^3$) & $22.98\pm0.98$ & $21.07\pm0.38$ & $21.13\pm0.37$ & $20.90\pm0.40$ & $20.28\pm0.29$ & \textbf{20.16}$\pm$\textbf{0.48} \\
        & MAE ($\mu$g/m$^3$) & $18.37\pm0.94$ & $16.68\pm0.39$ & $16.77\pm0.37$ & $16.53\pm0.41$ & $15.98\pm0.30$ & \textbf{15.91}$\pm$\textbf{0.49} \\
        & CSI (\%) & $40.77\pm2.69$ & $44.87\pm1.09$ & $44.71\pm0.99$ & $45.64\pm1.10$ & $47.61\pm0.92$ & \textbf{47.91}$\pm$\textbf{1.65} \\
        & POD (\%) & $51.43\pm5.68$ & $56.43\pm2.43$ & $56.17\pm2.45$ & $57.98\pm2.51$ & $59.79\pm2.11$ & \textbf{60.33}$\pm$\textbf{3.42} \\
        & FAR (\%) & $32.80\pm4.29$ & $31.21\pm1.68$ & $31.16\pm1.80$ & $31.65\pm1.73$ & $29.87\pm1.43$ & \textbf{29.83}$\pm$\textbf{2.36} \\
        \midrule
        \multirow{5}{*}{采暖季} 
        & RMSE ($\mu$g/m$^3$) & $35.55\pm2.76$ & $33.53\pm1.04$ & $33.09\pm1.00$ & $33.20\pm1.23$ & $33.07\pm1.03$ & \textbf{32.11}$\pm$\textbf{1.47} \\
        & MAE ($\mu$g/m$^3$) & $28.67\pm2.52$ & $26.90\pm1.04$ & $26.54\pm0.97$ & $26.57\pm1.22$ & $26.40\pm0.97$ & \textbf{25.68}$\pm$\textbf{1.42} \\
        & CSI (\%) & $45.52\pm5.49$ & $49.75\pm2.09$ & $49.83\pm1.79$ & $50.13\pm2.50$ & $48.79\pm1.38$ & \textbf{51.35}$\pm$\textbf{2.53} \\
        & POD (\%) & $60.85\pm9.17$ & $64.94\pm3.30$ & $64.58\pm3.03$ & $64.54\pm3.49$ & $61.29\pm2.07$ & \textbf{66.24}$\pm$\textbf{4.56} \\
        & FAR (\%) & $34.56\pm6.21$ & $31.88\pm2.28$ & $31.31\pm2.44$ & $30.73\pm2.80$ & \textbf{29.37}$\pm$\textbf{2.60} & $30.11\pm3.67$ \\
        \midrule
        \multirow{5}{*}{滚动预测} 
        & RMSE ($\mu$g/m$^3$) & $50.70\pm4.57$ & $46.19\pm2.04$ & $46.06\pm2.03$ & $45.71\pm2.38$ & $47.97\pm1.67$ & \textbf{44.36}$\pm$\textbf{2.85} \\
        & MAE ($\mu$g/m$^3$) & $41.89\pm4.22$ & $37.97\pm1.94$ & $37.94\pm1.92$ & $37.46\pm2.29$ & $39.03\pm1.65$ & \textbf{36.32}$\pm$\textbf{2.81} \\
        & CSI (\%) & $52.44\pm3.81$ & $58.85\pm2.36$ & $59.16\pm1.87$ & $58.98\pm2.47$ & $58.84\pm1.60$ & \textbf{60.57}$\pm$\textbf{2.78} \\
        & POD (\%) & $74.16\pm7.25$ & $81.03\pm3.14$ & $83.32\pm1.95$ & $81.92\pm2.91$ & $79.40\pm1.71$ & \textbf{83.94}$\pm$\textbf{3.34} \\
        & FAR (\%) & $35.25\pm5.32$ & $31.71\pm2.38$ & $32.86\pm2.37$ & $32.18\pm2.36$ & \textbf{30.51}$\pm$\textbf{2.28} & $31.37\pm3.63$ \\
        \bottomrule
    \end{tabular}}
\end{table*}

从表\ref{tab:pm25gnn_overall_performance}可以得出以下结论:

(1)PM$_{2.5}$-GNN在所有三个数据集上均取得最优或接近最优的性能,验证了风场驱动图结构设计的有效性。相比于仅使用时序信息的LSTM/GRU,PM$_{2.5}$-GNN的RMSE降低约4-6\%;相比于使用无向图的GC-LSTM,RMSE降低约2-3\%。

(2)在采暖季(重污染高发期)场景下,PM$_{2.5}$-GNN的优势更为明显。CSI指标从GC-LSTM的50.13\%提升至51.35\%,POD从64.54\%提升至66.24\%。这表明风场驱动的有向图设计有助于捕捉重污染事件中的跨区域传输,提高预警准确率。

(3)在滚动预测场景下,PM$_{2.5}$-GNN展现出最强的泛化能力。该场景使用最近3个月数据预测下个月,最接近实际业务系统的运行方式。PM$_{2.5}$-GNN的RMSE(44.36)显著低于其他方法,表明其在数据分布变化时仍能保持稳定性能。

(4)nodesFC-GRU的过拟合问题。nodesFC-GRU用全连接层替代GNN,虽然训练损失最低,但测试性能较差,表现出明显的过拟合。这说明GNN的归纳偏置(仅聚合邻居信息)有助于防止过拟合,提高泛化能力。

表\ref{tab:pm25gnn_ablation_study}展示了PM$_{2.5}$-GNN的消融实验结果,验证了边界层高度特征和净通量(export项)设计的贡献。

\begin{table}[htbp]
    \centering
    \caption{PM$_{2.5}$-GNN消融实验结果}
    \caption*{\cqt{无blh特征}表示从节点属性中移除边界层高度;\cqt{无export项}表示在消息聚合中仅使用输入流,不减去输出流。结果表明两项设计均对模型性能有显著贡献。}
    \label{tab:pm25gnn_ablation_study}
    \begin{tabular}{@{}clccc@{}}
        \toprule
        \textbf{数据集} & \textbf{指标} & \textbf{完整模型} & \textbf{无blh特征} & \textbf{无export项} \\
        \midrule
        \multirow{3}{*}{全年} & RMSE & \textbf{20.16}$\pm$\textbf{0.48} & $20.46\pm0.43$ & $20.98\pm0.33$ \\
        & MAE & \textbf{15.91}$\pm$\textbf{0.49} & $16.12\pm0.44$ & $16.67\pm0.35$ \\
        & CSI & \textbf{47.91}$\pm$\textbf{1.65\%} & $46.70\pm1.48\%$ & $45.41\pm1.17\%$ \\
        \midrule
        \multirow{3}{*}{采暖季} & RMSE & \textbf{32.11}$\pm$\textbf{1.47} & $33.25\pm1.65$ & $32.70\pm1.31$ \\
        & MAE & \textbf{25.68}$\pm$\textbf{1.42} & $26.67\pm1.59$ & $26.16\pm1.27$ \\
        & CSI & \textbf{51.35}$\pm$\textbf{2.53\%} & $49.42\pm2.90\%$ & $50.41\pm2.43\%$ \\
        \midrule
        \multirow{3}{*}{滚动预测} & RMSE & \textbf{44.36}$\pm$\textbf{2.85} & $46.12\pm3.38$ & $44.80\pm2.59$ \\
        & MAE & \textbf{36.32}$\pm$\textbf{2.81} & $38.04\pm3.38$ & $36.78\pm2.53$ \\
        & CSI & \textbf{60.57}$\pm$\textbf{2.78\%} & $58.72\pm3.15\%$ & $60.12\pm2.38\%$ \\
        \bottomrule
    \end{tabular}
\end{table}

消融实验表明:

(1)边界层高度是关键特征。移除blh后,所有数据集上的性能均显著下降,RMSE上升1-4\%。这验证了边界层高度与\PM 浓度的强负相关性\citep{su2018relationships,luan2018quantifying}——低边界层导致污染物难以垂直扩散,在近地面累积。

(2)净通量设计有效。移除export项(即仅聚合输入流)后,性能有所下降,尤其在全年评估数据集上RMSE上升4\%。这验证了\cqt{输入减输出}设计的物理合理性——它更准确地刻画了节点的\cqt{收支平衡}。


\subsection{PCDCNet主实验结果}
\label{subsec:pcdcnet_main_results}

表\ref{tab:performance_vertical}展示了PCDCNet与基线模型在BTHSA和YRD两个区域72小时预报任务中的性能对比。报告的指标为所有时间步的平均RMSE和MAE。

\begin{table*}[htbp]
\centering
\caption{PCDCNet与基线模型在72小时预报任务中的性能对比}
\caption*{报告的指标为所有时间步的平均RMSE和MAE(单位:$\mu$g/m$^3$)。对比方法涵盖机器学习基线、通用时序模型和空气质量专用模型。最优结果已加粗,次优结果已加下划线。}
\label{tab:performance_vertical}
\begin{tabular}{@{}llcccc@{}}
\toprule
\multirow{2}{*}{\textbf{区域}} & \multirow{2}{*}{模型} & \multicolumn{2}{c}{\textbf{RMSE} $\downarrow$} & \multicolumn{2}{c}{\textbf{MAE} $\downarrow$} \\
\cmidrule(lr){3-4} \cmidrule(lr){5-6}
& & \textbf{PM$_{2.5}$} & \textbf{O$_3$} & \textbf{PM$_{2.5}$} & \textbf{O$_3$} \\
\midrule
\multirow{8}{*}{BTHSA} & XGBoost & 35.47 & \underline{27.53} & 26.05 & \underline{20.68} \\
& LightGBM & 35.48 & 27.97 & 26.07 & 21.05 \\
& GC-LSTM & 32.95 & 35.12 & 23.09 & 26.82 \\
& PM$_{2.5}$-GNN & 31.51 & 31.85 & 23.50 & 22.90 \\
& iTransformer & \underline{31.17} & 29.90 & 20.72 & 22.58 \\
& AirPhyNet & 50.49 & 68.04 & 40.26 & 51.53 \\
& TimeXer & 31.51 & 30.07 & \underline{20.60} & 22.87 \\
\cmidrule(lr){2-6}
& \textbf{PCDCNet (本文)} & \textbf{24.13} & \textbf{22.45} & \textbf{15.46} & \textbf{16.73} \\
\midrule
\multirow{8}{*}{YRD} & XGBoost & 22.07 & \underline{27.71} & 16.64 & \underline{20.99} \\
& LightGBM & 22.06 & 28.07 & 16.74 & 21.33 \\
& GC-LSTM & 18.77 & 34.65 & 13.57 & 26.31 \\
& PM$_{2.5}$-GNN & 18.17 & 31.46 & 13.24 & 24.05 \\
& iTransformer & 17.92 & 30.53 & \underline{12.86} & 23.28 \\
& AirPhyNet & 28.53 & 56.96 & 22.90 & 42.76 \\
& TimeXer & \underline{17.89} & 30.47 & \underline{12.86} & 23.28 \\
\cmidrule(lr){2-6}
& \textbf{PCDCNet (本文)} & \textbf{15.55} & \textbf{23.03} & \textbf{10.97} & \textbf{17.27} \\
\bottomrule
\end{tabular}
\end{table*}

从表\ref{tab:performance_vertical}可以得出以下结论:

(1)PCDCNet在所有指标上均取得最优性能。在BTHSA区域,\PM 的RMSE从次优方法iTransformer的31.17降至24.13,降幅达\textbf{22.6\%};\ozone 的RMSE从次优方法XGBoost的27.53降至22.45,降幅达\textbf{18.5\%}。在YRD区域,\PM 的RMSE降低\textbf{13.1\%},\ozone 的RMSE降低\textbf{16.9\%}。

(2)PM$_{2.5}$-GNN在\ozone 预测上表现不佳。PM$_{2.5}$-GNN的\ozone RMSE(BTHSA: 31.85, YRD: 31.46)显著高于PCDCNet,验证了我们在\ref{subsec:pm25gnn_limitation}节分析的局限性——缺乏化学机制约束和排放输入导致其难以捕捉\ozone 的光化学生成过程。

(3)通用时序模型在\ozone 预测上优于专用空气质量模型。iTransformer和TimeXer在\ozone 预测上的表现优于GC-LSTM和PM$_{2.5}$-GNN,但仍显著落后于PCDCNet。这表明:(a)\ozone 预测需要捕捉复杂的非线性依赖,Transformer架构在此有一定优势;(b)仅依靠通用时序建模能力仍不足够,需要融合排放和物理约束。

(4)AirPhyNet性能较差的原因分析。AirPhyNet虽然引入了基于Neural ODE的物理约束,但其设计假设较强(仅使用风场和历史浓度),缺乏排放输入和多气象变量融合。在本文的多污染物、长时序预测任务中,AirPhyNet出现了严重的误差发散问题(\PM RMSE达50.49,\ozone RMSE达68.04)。

(5)XGBoost在\ozone 预测上的相对优势。XGBoost在\ozone 预测上取得次优性能,这是因为它直接利用了排放数据和未来气象预报作为输入(表\ref{tab:methods_comparison_vertical}中的Exog能力)。然而,由于缺乏时序建模和空间建模能力,其\PM 预测性能较差。

图\ref{fig:leadtime}展示了不同预测时效下PCDCNet与基线模型的性能对比。

\begin{figure}[htbp]
  \centering
  \includegraphics[width=\linewidth]{figures/chap03_pcdcnet_leadtime.png}
  \caption{不同预测时效下各模型的MAE变化曲线}
  \caption*{左图为BTHSA区域的\PM 预测,右图为YRD区域的\ozone 预测。PCDCNet在所有时效上均保持最优性能,且误差随时效增长的速率最为平缓,表明DIC约束有效抑制了长时序预测中的误差累积。}
  \label{fig:leadtime}
\end{figure}

从图中可以观察到:

(1)PCDCNet在所有时效上均保持最优。从6小时到72小时,PCDCNet的MAE曲线始终位于最下方,且与其他方法的差距随时效增加而扩大。

(2)误差增长速率差异显著。GC-LSTM和AirPhyNet的误差随时效快速增长,在48小时后出现加速发散;而PCDCNet的误差增长近似线性,表明DIC约束有效抑制了误差累积。

(3)Transformer模型在中长时效表现稳定。iTransformer和TimeXer的误差增长速率介于GC-LSTM和PCDCNet之间,验证了自注意力机制在长程依赖建模中的优势。


\subsection{消融实验}
\label{subsec:pcdcnet_ablation}

为验证各模块和设计选择的贡献,我们设计了系统的消融实验。

(1)核心模块消融。分别移除LID、STD、TAD模块,评估其对预测性能的影响。图\ref{fig:ablation}展示了消融实验结果。

\begin{figure}[htbp]
  \centering
  \includegraphics[width=0.8\linewidth]{figures/chap03_ablation.png}
  \caption{PCDCNet消融实验结果}
  \caption*{左图为\PM 预测性能,右图为\ozone 预测性能。w/o Emission表示移除排放数据输入;w/o DIC表示移除领域知识约束;w/o LID/STD/TAD分别表示移除对应的动力学模块。移除任何模块均导致性能下降。}
  \label{fig:ablation}
\end{figure}

消融实验的关键发现包括:

排放数据对\ozone 预测至关重要:移除排放输入(w/o Emission)后,\ozone 的MAE上升约\textbf{15\%},而\PM 的影响相对较小(上升约5\%)。这验证了排放清单对于捕捉二次污染物(\ozone)生成的关键作用——\ozone 的生成强烈依赖于前驱物\NOx 和VOC的排放量,而\PM 中有相当比例是一次排放,对排放变化的响应相对直接。

DIC约束提升长时序稳定性:移除DIC约束(w/o DIC)后,模型在短时效($<$24小时)的性能变化不大,但在72小时预测末期的误差显著增大。这证明物理约束的主要作用是抑制长时序预测中的误差累积,确保预测结果的物理合理性。

三个动力学模块缺一不可:LID、STD、TAD分别对应化学生成、空间传输、时间累积三类物理过程,移除任一模块都会导致性能下降。其中,移除TAD模块的影响最大(\PM MAE上升约10\%),表明时序累积建模对于捕捉污染物浓度的演化趋势至关重要。

(2)排放物种消融。进一步分析不同排放物种对预测性能的贡献。图\ref{fig:emis_ablation}展示了分别移除\NOx、VOC、SO$_2$等排放变量后的性能变化。

\begin{figure}[htbp]
  \centering
  \includegraphics[width=0.8\linewidth]{figures/chap03_emission_ablation.png}
  \caption{不同排放物种对预测性能的贡献分析}
  \caption*{分别移除\NOx、VOC、SO$_2$等排放变量后,评估对\PM 和\ozone 预测的影响。结果表明,\NOx 和VOC对\ozone 预测贡献最大,SO$_2$对\PM 预测贡献显著。}
  \label{fig:emis_ablation}
\end{figure}

排放物种消融的关键发现包括:

\NOx 和VOC是\ozone 预测的关键排放:移除\NOx 后\ozone MAE上升约8\%,移除VOC后上升约6\%。这与\ozone 的光化学生成机理一致——\NOx 和VOC是\ozone 的两个主要前驱物。

SO$_2$对\PM 预测贡献显著:移除SO$_2$后\PM MAE上升约5\%,这是因为SO$_2$氧化生成的硫酸盐是\PM 的重要组分。

NH$_3$的影响相对较小:移除NH$_3$后性能变化不大,可能是因为在当前模型分辨率下,氨盐生成的非线性过程难以被充分捕捉。

(3)隐层维度敏感性分析。评估不同隐层维度$d$对模型性能和计算效率的影响。图\ref{fig:ablation_sensitivity}左图展示了隐层维度为16、32、64时的性能对比。

\begin{figure}[htbp]
  \centering
  \includegraphics[width=0.8\linewidth]{figures/chap03_ablation.png}
  \caption{参数敏感性分析}
  \caption*{左图:隐层维度敏感性分析,$d=32$取得最优性能,在模型容量与泛化能力之间达到平衡;右图:核心模块消融实验结果。}
  \label{fig:ablation_sensitivity}
\end{figure}

结果表明:$d=16$时模型容量不足,欠拟合导致性能下降;$d=64$时参数量增加但性能未见提升,存在轻微过拟合风险;$d=32$在模型容量与泛化能力之间达到最优平衡。


% DIC物理约束损失效果分析
如第\ref{chap:methodology}章所述,物理约束数据驱动建模的核心在于将物理先验知识融入损失函数,总损失函数形如$\mathcal{L} = \mathcal{L}_{\text{data}} + \lambda \mathcal{L}_{\text{physics}}$(公式\ref{eq:physics_loss})。本节通过训练过程中数据损失与物理约束损失的变化曲线,深入分析DIC约束的作用机制,验证物理约束融合策略的有效性。

图\ref{fig:loss}展示了不同约束权重$\lambda$下模型训练过程中预测损失$\mathcal{L}_{\text{data}}$与DIC物理约束损失$\mathcal{L}_{\text{physics}}$的变化曲线。

\begin{figure}[htbp]
  \centering
  \includegraphics[width=\linewidth]{figures/chap03_loss_curves.png}
  \caption{数据损失与物理约束损失的训练曲线分析}
  \caption*{上排:不同$\lambda$值下的训练/验证预测损失$\mathcal{L}_{\text{data}}$;下排:对应的DIC物理约束损失$\mathcal{L}_{\text{physics}}$(空间约束、时间约束、总约束)。当$\lambda=0$时,物理约束损失虽未被直接优化,但仍呈下降趋势,表明STD模块具有隐式学习质量守恒的能力;当$\lambda>0$时,物理约束损失下降更快且收敛值更低,验证了显式物理约束的有效性。}
  \label{fig:loss}
\end{figure}

基于损失曲线的分析揭示了物理约束融合的核心机制:

(1)图神经网络具有隐式学习物理规律的能力。即使$\lambda=0$(不使用显式物理约束),DIC物理约束损失在训练过程中也呈下降趋势。这表明图卷积操作本身具有一定的\cqt{守恒倾向}——由归一化拉普拉斯算子定义的消息传递机制在数学上与对流-扩散方程(公式\ref{eq:advection_diffusion})相关联,使得网络能够隐式捕捉物理传输规律。

(2)显式物理约束显著提升物理一致性。当$\lambda>0$时,物理约束损失的收敛值显著降低(从$\lambda=0$时的约0.15降至$\lambda=10$时的约0.05),表明显式约束有效引导模型学习更符合质量守恒(公式\ref{eq:mass_conservation})的时空传输模式。这验证了第\ref{chap:methodology}章提出的\cqt{通过损失函数嵌入物理约束}策略的有效性。

(3)物理约束权重$\lambda$需要精细调节。$\lambda$过大时(如$\lambda=100$),物理约束过强会限制模型的数据拟合能力,导致预测损失上升。最优$\lambda$值在$10^{-3}$至$10^{-2}$区间,此时模型在数据拟合$\mathcal{L}_{\text{data}}$与物理一致性$\mathcal{L}_{\text{physics}}$之间达到帕累托最优平衡。

(4)物理约束具有正则化效果。对比训练损失和验证损失可以发现:$\lambda=0$时存在明显的过拟合现象(训练损失低但验证损失相对较高);$\lambda>0$时过拟合得到缓解,验证损失与训练损失的差距缩小。这表明物理约束不仅提升了模型的物理可解释性,还通过引入物理先验知识有效约束了模型的假设空间,增强了泛化能力。



% ------------------------------------------------------------
% 3.7 本章小结
% ------------------------------------------------------------
\section{本章小结}
\label{sec:pred_summary}

本章聚焦于大气污染的时空预测问题,完成了从单一物理传输建模到多过程物理化学耦合建模的跨越。通过PM$_{2.5}$-GNN和PCDCNet两个模型的递进式研究,系统探索了如何在深度学习框架中融合大气科学领域知识。主要贡献总结如下:

(1)提出了基于风场驱动图网络的PM$_{2.5}$-GNN模型。通过在图神经网络中显式编码风场信息,设计了平流系数公式(公式\eqref{eq:advection_coeff})和净通量聚合机制(公式\eqref{eq:message_agg}),有效解决了\PM 的跨区域定向传输预测难题。实验表明,PM$_{2.5}$-GNN相比GC-LSTM等基线模型,在72小时长时序预测和重污染事件预警方面具有显著优势,验证了风场驱动有向图设计的有效性。

(2)针对PM$_{2.5}$-GNN的局限性,提出了PCDCNet模型。作为数值模式CMAQ的深度学习代理模型,PCDCNet具备以下创新特征:

\begin{itemize}
    \item 排放响应建模:将排放清单作为动态输入变量纳入模型框架,实现对排放变化的显式响应能力,为后续的情景模拟应用奠定基础;
    \item LID--STD--TAD三模块架构:分别对应公式\eqref{eq:advection_diffusion}中的化学反应与排放项(R、E)、平流与扩散项、沉降项与时间累积(D),实现\cqt{过程解耦+联合建模},增强了模型的物理可解释性;
    \item 领域知识约束(DIC):将质量守恒原则嵌入损失函数,引导模型学习物理一致的传输模式,有效抑制长时序预测中的误差累积。
\end{itemize}

(3)验证了物理知识约束在深度学习中的有效性。系统的消融实验和约束效果分析表明:(a)在损失函数中引入质量守恒等物理约束,能显著提升模型的泛化能力和长时序稳定性;(b)排放数据对于\ozone 等二次污染物的预测至关重要;(c)DIC约束具有正则化效果,有助于缓解过拟合。

(4)构建并公开了KnowAir-DS系列数据集。KnowAir-DS-V1以城市为节点、3小时为时间分辨率,覆盖更大空间范围(184个城市),用于验证PM$_{2.5}$-GNN的区域传输建模能力;KnowAir-DS-V2以观测站点为节点、1小时为时间分辨率,空间精度更高,用于验证PCDCNet的精细化预报能力。尽管两个数据集的空间尺度和时间分辨率不同,PM$_{2.5}$-GNN和PCDCNet采用的核心方法论——物理约束的时空图神经网络——保持一致,体现了数据驱动方法在不同尺度间迁移的灵活性。

在京津冀和长三角区域的72小时多污染物预报任务中,PCDCNet相较现有最优方法,\PM 预测的RMSE降低\textbf{13-23\%},\ozone 预测的RMSE降低\textbf{17-19\%}。这一性能提升验证了\cqt{物理约束数据驱动建模}范式的有效性。

从方法论角度,本章的核心贡献在于:将第\ref{chap:methodology}章公式\eqref{eq:advection_diffusion}所描述的大气物理化学过程(平流、扩散、化学反应、排放、沉降)显式地映射到深度学习架构设计中,构建了\cqt{Embed$\to$LID$\to$STD$\to$TAD$\to$Readout}的完整建模框架。这种\cqt{观测空间$\leftrightarrow$隐空间$\leftrightarrow$观测空间}的表示学习范式,不仅约束了模型的输入输出映射,更重要的是约束了模型学习到的隐空间表示本身,使其具备物理意义与一致性。

PCDCNet的成功构建,不仅实现了高精度的实时预报,更为后续章节开展基于多源数据融合的\textbf{空间推断}(第\ref{chap:inference}章)和基于排放清单的\textbf{情景模拟}(第\ref{chap:simulation}章)奠定了坚实的建模基础。在第\ref{chap:deployment}章中,PCDCNet将作为核心预报引擎部署到KnowAir智能预报系统中实现业务化运行,并与\CMAQ、WRF-Chem、NAQPMS等数值模式进行直接对比评估,以验证其在真实业务场景中的实战表现。  % 物理约束时空图神经网络的大气污染预测
% ============================================================
% 第四章 基于多源数据融合的大气污染空间推断
% 基于复杂系统数据驱动建模的大气污染研究
% ============================================================

\chapter{基于多源数据融合的大气污染空间推断}
\label{chap:inference}

上一章解决了已知监测站点的时间预测问题,本章则聚焦于更具挑战性的空间推断问题——如何基于稀疏的地面监测和多源遥感数据,重建全域高分辨率的污染物浓度场。本章提出SPIN模型,通过物理启发的归纳式图神经网络实现对无监测区域的精确推断。

% ------------------------------------------------------------
% 4.1 引言
% ------------------------------------------------------------
\section{引言}
\label{sec:infer_intro}

正如第\ref{chap:introduction}章所述,大气污染对公众健康构成严重威胁,而精准的健康风险评估依赖于高时空分辨率的污染物分布数据。在环境流行病学研究中,暴露评估的准确性直接决定了健康效应估计的可靠性;在区域联防联控决策中,精细化的污染分布图是识别重点管控区域的基础依据;在公众健康服务中,无缝覆盖的空气质量信息能够帮助敏感人群规避高暴露风险区域。

然而,地面监测站点的覆盖范围有限,难以满足上述精细化空间分析的需求。以中国为例,尽管已建成全球规模最大的空气质量监测网络,但监测站点主要分布在城市建成区,广大农村地区、山区和边远地区仍存在大面积监测盲区。这种空间覆盖的不均衡性导致了\cqt{城市-农村}污染暴露评估的系统性偏差,构成当前环境健康研究的主要瓶颈。

空间推断任务的目标是:基于稀疏、不规则分布的地面监测站点数据,结合多源异构信息(气象场、排放清单、卫星遥感),对空间中任意未观测位置的污染物浓度进行精确推断。与第\ref{chap:prediction}章的预测任务不同,推断任务解决的是\cqt{未知位置的当前状态估计}问题,两者共享相同的物理基础——对流--扩散方程,但推断任务需要解决空间泛化问题。

核心挑战与现有方法局限。实现高精度的空间推断面临以下三方面挑战:

(1)监测网络稀疏与空间泛化。现有监测网络(约1618个国控站点)虽然规模庞大,但超过80\%的人口仍生活在缺乏直接监测的区域\citep{southerland2022global,wei2023first}。监测站点主要集中在城市建成区,广大农村地区、山区和边远地区存在大面积监测盲区。传统空间插值方法(如克里金、反距离加权)假设空间平稳性,难以处理站点分布不均匀、地形与气象条件差异显著的复杂场景。如何基于稀疏的已知观测,对空间中任意未观测位置进行精确推断,并具备归纳式泛化能力——即推广至训练时未见的位置——是首要挑战。

(2)遥感约束数据的大面积缺失。卫星遥感的气溶胶光学厚度(AOD)与\PM 具有明确的物理关联(柱积分气溶胶光学特性),可为无观测区域提供空间结构线索。然而,卫星AOD受云层遮挡影响严重,在重污染高发的冬季缺失率可超过70\%。现有方法多将AOD作为模型的强制输入特征,在AOD缺失时推断质量急剧下降。如何有效利用AOD的空间分布信息作为约束,同时在AOD不可用时自动回退至物理驱动模式以实现全天候推断,是数据融合的核心挑战。

(3)大气动力过程的显式建模。空间推断并非静态的空间插值——污染物浓度场受对流--扩散方程(公式\ref{eq:advection_diffusion})支配,呈现显著的时间演化特征。风场驱动的平流传输使得上风向站点的历史观测对下风向无观测区域具有重要参考价值。因此,推断模型需要以历史时间窗口(而非单一时刻快照)作为输入,通过时序特征提取捕捉动力过程的时空耦合效应。如何将物理传输机制显式嵌入推断框架,是保证推断结果物理合理性的关键挑战。

针对上述挑战,本章提出SPIN(Spatiotemporal Physics-Guided Inference Network)模型,通过四阶段推断流程系统性地加以解决:动态节点掩码赋予模型归纳式泛化能力(应对挑战一);时间卷积网络从历史时间窗口编码本地物理特征(应对挑战三);物理启发的扩散--平流双核空间传播将观测信息传递至无观测区域(应对挑战一与三);掩码AOD梯度约束将遥感数据作为空间结构约束而非强制输入,实现全天候推断(应对挑战二)。


% ------------------------------------------------------------
% 4.2 问题定义
% ------------------------------------------------------------
\section{问题定义}
\label{sec:infer_problem}

本章研究的核心问题可以形式化定义为归纳式时空克里金(Inductive Spatiotemporal Kriging)——利用稀疏的监测数据和辅助物理场,对空间中任意未观测位置的\PM 浓度进行推断。本章以\PM 为研究对象,原因在于:(1)\PM 是当前中国空气质量达标的主要制约因子,具有最迫切的精细化空间评估需求;(2)\PM 与卫星AOD具有明确的物理关联(柱积分气溶胶),使得遥感约束机制在物理上自洽。图\ref{fig:inference_problem}展示了该问题的整体框架。

\begin{figure}[htbp]
    \centering
    \includegraphics[width=\textwidth]{figures/chap04_inference_problem.pdf}
    \caption{大气污染空间推断问题示意图}
    \caption*{输入包括观测站点的历史时序、全域气象场、排放场和卫星AOD数据;输出对应两个子任务——站点推断(图$\mathcal{G}^{\text{s}}$,蓝色曲线)和网格推断(图$\mathcal{G}^{\text{g}}$,红色热力图)。两个子任务采用相同的SPIN架构但各自独立训练。图中黑色实心圆点为观测站点,空心圆点为待推断的目标站点,斜线阴影区域表示AOD缺失区域,虚线圆圈示意图神经网络的多跳感受野。详细说明见正文。}
    \label{fig:inference_problem}
\end{figure}

如图所示,空间推断问题包含\textbf{两个子任务},它们共享相同的输入但产生不同粒度的输出。\textbf{输入}(两个子任务共享)包括四类数据:观测站点的历史时序数据$\mathbf{X}^{\text{obs}}_{t-T'+1:t}$(图中黑色圆点表示的观测节点)、全域气象场$\mathbf{M}^{F}_{t-T'+1:t}$(包括风速风向、温度、湿度、边界层高度等)、排放场$\mathbf{E}^{F}_{t-T'+1:t}$以及卫星遥感AOD数据$\mathbf{X}^{\text{AOD}}_{t-T'+1:t}$(图中斜线区域表示缺失)。\textbf{输出}对应两个不同的子任务:(1)\textbf{站点推断}——在站点图$\mathcal{G}^{\text{s}}$上推断目标站点$\mathcal{V}_{\text{target}}$的\PM 浓度时序$\hat{\mathbf{X}}^{\text{target}}_{t-T'+1:t}$(图中蓝色曲线),解决离散站点间的空间插值问题;(2)\textbf{网格推断}——在网格图$\mathcal{G}^{\text{g}}$上推断全域所有$N^{\text{g}}$个$0.25^{\circ}$网格像素的\PM 浓度分布$\hat{\mathbf{X}}^{F}_t \in \mathbb{R}^{N^{\text{g}}}$(图中红色热力图),将离散站点观测扩展为覆盖整个研究区域的连续浓度场。两个子任务虽然输出粒度不同(离散站点 vs.\,全域网格),但通过统一的SPIN架构加以解决。

与第\ref{chap:prediction}章的\textbf{时间维度预测}不同,本章聚焦于\textbf{空间维度推断}。类似于第\ref{chap:prediction}章,模型同样采用历史时间窗口作为输入:给定任意时刻$t$之前$T'$个时间步($t-T'+1, \ldots, t$)观测站点的历史数据,推断该时刻$t$未观测位置的浓度值。时间窗口的引入使模型能够捕捉污染物的时序演化特征,从而提升空间推断的准确性。这种设计使得模型能够同时服务于历史重建(用于暴露评估)和实时监测(用于填补监测盲区)两类应用场景。

该问题包含两个层次化的子任务,分别对应两套不同粒度的图结构:

(1)\textbf{站点推断(Station Inference)}。在站点图$\mathcal{G}^{\text{s}} = (\mathcal{V}^{\text{s}}, \mathcal{E}^{\text{s}})$上进行,其中$\mathcal{V}^{\text{s}}$为监测站点节点集合($|\mathcal{V}^{\text{s}}| = N^{\text{s}} = 152$,如图~\ref{fig:study_area}(b)所示),$\mathcal{E}^{\text{s}}$为站点间的边集合。推断保留测试集(Held-out)中未观测站点的时序数据,用于评估模型的插值泛化能力。这一任务模拟的场景是:当新建一个监测站点时,如何基于周边已有站点的观测数据估算其污染水平。

(2)\textbf{网格推断(Grid Inference)}。在网格图$\mathcal{G}^{\text{g}} = (\mathcal{V}^{\text{g}}, \mathcal{E}^{\text{g}})$上进行,其中$\mathcal{V}^{\text{g}}$为$0.25^{\circ}$网格像素节点集合($|\mathcal{V}^{\text{g}}| = N^{\text{g}}$,如图~\ref{fig:study_area}(d)所示),$\mathcal{E}^{\text{g}}$为网格像素间的边集合。对区域内所有网格点进行推断,生成连续的高分辨率污染场$\hat{\mathbf{X}}^{F}_t$,以消除监测盲区。这一任务的应用场景包括:生成空间连续的污染暴露评估图、识别缺乏监测的污染热点、支撑精细化的区域联防联控决策。

需要强调的是,虽然两个子任务使用不同粒度的图$\mathcal{G}^{\text{s}}$和$\mathcal{G}^{\text{g}}$,但它们采用\textbf{完全相同的模型架构}和\textbf{图核构建方法}(扩散核公式\ref{eq:diffusion_adjacency}--\ref{eq:diffusion_normalized}与平流核公式\ref{eq:advection_adjacency}--\ref{eq:advection_normalized}),仅在输入节点集合、训练数据与损失函数上有所区别(网格推断额外引入AOD空间约束损失),且各自独立训练。由于节点集合不同,两套图上所得的邻接矩阵维度不同:在$\mathcal{G}^{\text{s}}$上$\tilde{A}^{\mathcal{D}}, \tilde{A}^{\mathcal{A}} \in \mathbb{R}^{N^{\text{s}} \times N^{\text{s}}}$,在$\mathcal{G}^{\text{g}}$上$\tilde{A}^{\mathcal{D}}, \tilde{A}^{\mathcal{A}} \in \mathbb{R}^{N^{\text{g}} \times N^{\text{g}}}$。这种设计体现了SPIN的核心思想:同一物理传输架构可在不同空间尺度上复用。

表\ref{tab:task_comparison}对比了两个子任务的主要特征与差异。

\begin{table}[htbp]
    \centering
    \caption{站点推断与网格推断任务对比}
    \caption*{两个子任务采用相同的SPIN架构和图核构建方法,但在节点集合、输出粒度和损失函数上有所区别。}
    \label{tab:task_comparison}
    \begin{tabular}{@{}lll@{}}
        \toprule
        特征 & 站点推断 & 网格推断 \\
        \midrule
        图结构 & $\mathcal{G}^{\text{s}}$ & $\mathcal{G}^{\text{g}}$ \\
        节点数 & $N^{\text{s}}=152$ & $N^{\text{g}}$(全域网格) \\
        输出类型 & 时序浓度$\hat{\mathbf{X}}^{\text{target}}_{t-T'+1:t}$ & 空间场$\hat{\mathbf{X}}^{F}_t$ \\
        应用场景 & 离散站点插值 & 连续空间制图 \\
        AOD约束 & 无 & 有($\mathcal{L}_{\mathrm{AOD}}$) \\
        参数 & $\Theta^{\text{s}}$ & $\Theta^{\text{g}}$ \\
        \bottomrule
    \end{tabular}
\end{table}

形式化地,以$\mathcal{G} = (\mathcal{V}, \mathcal{E})$统一表示两类图结构($\mathcal{G} \in \{\mathcal{G}^{\text{s}}, \mathcal{G}^{\text{g}}\}$),其中$\mathcal{V}$为节点集合($|\mathcal{V}| = N$),$\mathcal{E}$为边集合。我们将节点动态划分为观测节点集$\mathcal{V}_{\text{obs}}$和目标推断节点集$\mathcal{V}_{\text{target}}$。给定观测节点过去$T'$个时间步的历史污染数据$\mathbf{X}^{\text{obs}}_{t-T'+1:t}$,以及全域气象场$\mathbf{M}^{F}_{t-T'+1:t}$和排放场$\mathbf{E}^{F}_{t-T'+1:t}$,两个子任务分别学习映射$\mathcal{F}^{\text{s}}_{\Theta^{\text{s}}}$与$\mathcal{F}^{\text{g}}_{\Theta^{\text{g}}}$:
\begin{align}
    \text{站点推断:} \quad \hat{\mathbf{X}}^{\text{target}}_{t-T'+1:t} &= \mathcal{F}^{\text{s}}_{\Theta^{\text{s}}}(\mathbf{X}^{\text{obs}}_{t-T'+1:t}, \mathbf{M}^{\text{s}}_{t-T'+1:t}, \mathbf{E}^{\text{s}}_{t-T'+1:t}, \mathcal{G}^{\text{s}}) \label{eq:infer_problem} \\
    \text{网格推断:} \quad \hat{\mathbf{X}}^{F}_t &= \mathcal{F}^{\text{g}}_{\Theta^{\text{g}}}(\mathbf{X}^{\text{obs}}_{t-T'+1:t}, \mathbf{M}^{F}_{t-T'+1:t}, \mathbf{E}^{F}_{t-T'+1:t}, \mathcal{G}^{\text{g}}) \label{eq:infer_problem_grid}
\end{align}
\noindent 其中站点级气象与排放数据$\mathbf{M}^{\text{s}}, \mathbf{E}^{\text{s}}$通过对全域网格场$\mathbf{M}^{F}, \mathbf{E}^{F}$按各站点经纬度进行空间索引(Spatial Indexing)得到,网格推断则直接使用完整的全域场$\mathbf{M}^{F}, \mathbf{E}^{F}$。
\noindent 式中$\mathcal{F}^{\text{s}}_{\Theta^{\text{s}}}$与$\mathcal{F}^{\text{g}}_{\Theta^{\text{g}}}$具有相同的网络架构(即四阶段流水线),但各自独立训练,$\Theta^{\text{s}}$与$\Theta^{\text{g}}$分别为两套独立的模型参数。两者的核心差异在于:网格推断额外引入了AOD空间约束损失$\mathcal{L}_{\mathrm{AOD}}$以保证无观测区域推断结果的物理合理性。如图~\ref{fig:inference_problem}所示:站点推断(公式\ref{eq:infer_problem})在站点图$\mathcal{G}^{\text{s}}$上推断目标站点$\mathcal{V}_{\text{target}}$在$T'$个时间步上的浓度时序$\hat{\mathbf{X}}^{\text{target}}_{t-T'+1:t}$(图中蓝色曲线);网格推断(公式\ref{eq:infer_problem_grid})在网格图$\mathcal{G}^{\text{g}}$上推断全域所有$N^{\text{g}}$个网格像素在时刻$t$的浓度分布$\hat{\mathbf{X}}^{F}_t \in \mathbb{R}^{N^{\text{g}}}$(图中红色热力图)。

与第\ref{chap:prediction}章的预测问题(公式\eqref{eq:pred_problem})相比,本章的关键区别在于:目标节点$\mathcal{V}_{\text{target}}$本身没有任何历史观测数据,模型必须完全依靠空间传播和环境上下文来进行推断。这一区别决定了本章需要采用归纳式学习范式,而非第\ref{chap:prediction}章的直推式范式。

在网格推断任务中,一个关键挑战是如何保证广阔无观测区域内推断结果的物理合理性。虽然卫星AOD能提供空间指导,但存在严重的缺失问题。因此,本框架将AOD数据$\mathbf{X}^{\text{AOD}}$作为训练时的空间结构约束(Spatial Structural Constraint),而非直接输入,从根本上解决了传统方法对卫星数据的过度依赖问题。


% ------------------------------------------------------------
% 4.3 研究区域与多源数据
% ------------------------------------------------------------
\section{研究区域与多源数据}
\label{sec:infer_data}

高质量、多源异构数据的融合是实现精准时空推断的基础。本节将详细阐述研究区域的选取依据,并系统介绍构建\cqt{地面--气象--排放--遥感}多模态数据集的过程。

\subsection{研究区域概况}
\label{subsec:study_area}

本研究选取京津冀及周边地区作为核心研究区域,共包含152个国控监测站点(区域定义详见第\ref{subsec:aq_data}节)。该区域覆盖了\cqt{2+26}城市大气污染传输通道\citep{lu2021estimation},总面积约43万平方公里,是我国大气污染防治的重点区域,如图~\ref{fig:study_area}所示。

选择该区域作为研究试验台(Testbed)主要基于以下考量:

(1)地形的高度异质性。该区域西倚太行山脉,北枕燕山山脉,东临渤海,呈半封闭的盆地状地形。正如第\ref{subsec:complex_system}节所述,这种复杂的地貌特征导致了独特的中尺度气象模式,如山前阻挡效应和海陆风环流,极易造成污染物的局地累积与回流\citep{zhong2018feedback}。太行山脉对西部气流的阻挡作用尤为显著,导致山前地带(石家庄、邯郸等城市)成为污染累积的高发区\citep{quan2020regional}。

(2)污染梯度的显著性。区域内包含了北京、天津等超大城市,同时也分布着大量重工业基地(如唐山、邯郸的钢铁产业)和广阔的农业农村地区。这种城乡二元结构和密集的工业布局,使得该区域内的污染物浓度在空间上呈现出极陡峭的梯度变化(Sharp Gradients)\citep{chen2020influence}。研究表明,相邻城市之间的\PM 浓度差异可达50 $\mu$g/m$^3$以上,对推断模型的空间解析能力提出了极大挑战。

(3)传输通道的典型性。\cqt{南风输送型}重污染过程在该区域频繁发生:污染气团从河北中南部沿太行山东麓北上,经石家庄、保定至北京,形成跨越数百公里的\cqt{污染传输带}\citep{yin2025regional}。这种定向传输特征为验证平流核的有效性提供了理想的实验条件。

\begin{figure}[htbp]
    \centering
    \includegraphics[width=0.90\textwidth]{figures/chap04_study_area.pdf}
    \caption{京津冀及周边地区研究区域与任务设置}
    \caption*{(a) 区域内的\cqt{2+26}城市分布;(b) 站点图$\mathcal{G}^{\text{s}}$($N^{\text{s}} = 152$个节点):基于地理邻近性($<$200 km)构建的站点级空间图结构(邻接矩阵$A^{\text{s}} \in \mathbb{R}^{N^{\text{s}} \times N^{\text{s}}}$),节点为监测站点,连边表示地理距离阈值内的空间邻接关系,用于问题一(站点推断);(c) $N^{\text{s}}$个地面监测站点的稀疏分布;(d) 网格图$\mathcal{G}^{\text{g}}$($N^{\text{g}}$个节点):用于网格推断的$0.25^{\circ}$标准网格(邻接矩阵$A^{\text{g}} \in \mathbb{R}^{N^{\text{g}} \times N^{\text{g}}}$),图构建规则与(b)相同,但节点为网格像素,用于问题二(网格推断)。两套图$\mathcal{G}^{\text{s}}$与$\mathcal{G}^{\text{g}}$采用相同的扩散核$\tilde{A}^{\mathcal{D}}$和平流核$\tilde{A}^{\mathcal{A}}$的构建方法,但各自独立训练(参数不共享),仅节点集合与矩阵维度不同。}
    \label{fig:study_area}
\end{figure}


\subsection{多源异构数据集构建}
\label{subsec:multimodal_data}

为了全面捕捉\PM 的时空演变规律,我们构建了一个包含地面观测、气象再分析、排放清单和卫星遥感的多源异构数据集。所有数据的时间跨度为2018年1月1日至2023年12月31日。为了支持网格推断任务,所有数据均经过时空对齐处理,统一为小时级时间分辨率,并插值到$0.25^{\circ} \times 0.25^{\circ}$的标准网格上。各类数据的详细规格如表~\ref{table:data_summary}所示。

\begin{table}[htbp]
\centering
\caption{京津冀及周边地区多源数据集汇总}
\caption*{数据集涵盖地面空气质量监测(CNEMC)、气象再分析(ERA5)、排放清单(MEIC)和卫星遥感(Himawari-8)四类数据源,时间跨度为2018--2023年。}
\label{table:data_summary}
\small
\begin{tabular}{@{}llllcc@{}}
\toprule
\textbf{类型} & \textbf{变量} & \textbf{符号} & \textbf{单位} & \textbf{分辨率} & \textbf{来源} \\
\midrule
\textbf{目标($\mathbf{X}$)} & \PM & $X_{\text{PM}_{2.5}}$ & $\mu$g/m$^3$ & 1h & CNEMC \\
\midrule
\multirow{5}{*}{\textbf{气象($\mathbf{M}$)}}
 & 气温/露点 & $M_{\text{t2m}}, M_{\text{d2m}}$ & K & 1h & \multirow{5}{*}{ERA5} \\
 & 湿度/气压 & $M_{\text{rh}}, M_{\text{sp}}$ & \%, Pa & 1h & \\
 & 降水/边界层 & $M_{\text{tp}}, M_{\text{blh}}$ & m & 1h & \\
 & 风速 & $M_{u}, M_{v}$ & m/s & 1h & \\
 & 短波辐射 & $M_{\text{rad}}$ & W/m$^2$ & 1h & \\
\midrule
\multirow{2}{*}{\textbf{排放($\mathbf{E}$)}}
 & 气态前驱物 & $E_{\text{NO}_x}$等 & ton & 月$\to$1h & \multirow{2}{*}{MEIC} \\
 & 颗粒物 & $E_{\text{PM}_{2.5}}$等 & ton & 月$\to$1h & \\
\midrule
\textbf{辅助} & AOD & $\mathbf{X}^{\text{AOD}}$ & - & 1h & Himawari-8 \\
\bottomrule
\end{tabular}
\end{table}


\subsubsection{地面空气质量监测数据($\mathbf{X}$)}

地面观测数据(记为$\mathbf{X} \in \mathbb{R}^{N \times 1}$,本章以\PM 为研究对象)是模型训练和验证的\cqt{真值(Ground Truth)}。我们从中国环境监测总站(CNEMC)获取了该区域内152个国家级空气质量自动监测站点的实时小时浓度数据。为了保证数据质量,执行了严格的质量控制流程:剔除因设备故障导致的连续缺失值超过24小时的记录,去除极值异常点(如$\PM > 1000$ $\mu$g/m$^3$),并对短期缺失($<$6小时)采用线性插值填补。最终保留的数据集涵盖了城市中心、近郊及部分背景站点,构成了稀疏但高精度的观测网络。

如图~\ref{fig:study_area}(c)所示,监测站点的空间分布呈现明显的城市偏向性——站点密集于北京、天津等大城市,而广大农村地区几乎没有覆盖。这种分布不均匀性正是本章研究的出发点:如何利用城市区域的稠密观测来推断农村地区的污染水平。


\subsubsection{气象驱动数据($\mathbf{M}$)}

气象条件(记为$\mathbf{M} \in \mathbb{R}^{N \times D_M}$)是决定污染物扩散、输送和沉降的关键动力学因素。本研究采用欧洲中期天气预报中心(ECMWF)发布的ERA5再分析资料\citep{hersbach2020era5}。相比于传统的地面气象站点观测,ERA5提供了全覆盖的格点化数据,能够更好地描述区域气象场的空间连续性。

选取的气象变量与第\ref{chap:prediction}章一致,涵盖了影响\PM 演化的关键物理过程:风场向量($u_{100}, v_{100}$)直接决定污染物的水平平流输送方向与强度;边界层高度(BLH)决定污染物垂直扩散的混合层体积;降水(TP)与相对湿度(RH)反映湿沉降清除机制和吸湿性增长过程\citep{tai2010correlations}。

值得注意的是,对于推断任务,气象数据在所有节点(包括目标节点)上都是可用的——这与目标节点缺乏历史污染观测形成鲜明对比。这一特点为模型提供了重要的物理上下文信息,使得即使在完全无观测的位置,模型也能基于当地的气象条件进行合理推断。


\subsubsection{排放清单数据($\mathbf{E}$)}

人为排放(记为$\mathbf{E} \in \mathbb{R}^{N \times D_E}$)是大气污染的物质基础。本研究使用了清华大学开发的中国多尺度排放清单模型(MEIC)\citep{zheng2018trends,li2017anthropogenic}。该清单提供了包括\NOx($E_{\text{NO}_x}$)、VOC($E_{\text{VOC}}$)、\SOtwo($E_{\text{SO}_2}$)、\NHthree($E_{\text{NH}_3}$)以及一次\PM($E_{\text{PM}_{2.5}}$)等主要污染物的月度网格化排放数据。

由于月度数据的时效性不足以支撑小时级推断,采用与第\ref{chap:prediction}章相同的时间分配方法\citep{inventory},将月度排放总量降尺度为小时分辨率(详见第\ref{subsec:emis_data}节)。

排放数据在推断任务中的角色与预测任务类似:通过提供污染物的\cqt{源强}信息,帮助模型学习\cqt{高排放区域污染水平通常较高}的关联规律。这一信息对于推断缺乏监测的工业区域尤为重要。


\subsubsection{卫星遥感AOD数据($\mathbf{X}^{\text{AOD}}$)}
\label{subsubsec:aod_data}

卫星遥感数据(记为$\mathbf{X}^{\text{AOD}} \in \mathbb{R}^{N}$,附带有效性掩码$\boldsymbol{\omega}^{\text{AOD}} \in \{0,1\}^{N}$)是弥补地面监测空间盲区的关键信息源。本研究采用日本气象厅(JMA)发射的新一代地球静止气象卫星葵花8号(Himawari-8)搭载的高级葵花成像仪(AHI)所反演的气溶胶光学厚度产品\citep{bessho2016introduction}。

相比于MODIS等极轨卫星每天仅能提供1--2次观测,作为静止卫星的Himawari-8具有极高的时间分辨率,能够对同一区域进行连续观测(全盘扫描周期为10分钟)\citep{wei2021himawari}。本研究使用的是L2级气溶胶产品,其能够反映整层大气柱内的气溶胶消光能力,与近地面颗粒物浓度具有较强的正相关性,能够清晰地展示污染气团的空间形态和传输路径。

然而,如第\ref{sec:infer_intro}节所述,AOD数据存在以下局限性:(1)云层遮挡导致非随机缺失——在AHI的反演算法中,被判定为有云的像素将被直接标记为无效值,夜间也完全不可用;(2)地表反射率干扰——在冬季的北方地区,地表积雪的高反射率会与大气信号混淆;(3)垂直分布解耦——当存在高空传输层时,高AOD值并不一定代表近地面高污染。

鉴于上述特性,若将AOD作为模型的输入特征,会导致模型在云覆盖区域因特征缺失而瘫痪。因此,本研究采取了全新的策略:不将AOD作为输入,而是将其作为训练时的空间结构约束。我们对原始AOD数据进行了严格的质量控制,剔除置信度较低的像素,并生成对应的有效性掩码矩阵$\boldsymbol{\omega}^{\text{AOD}}$,使得损失函数能够自动忽略无效区域。

\subsection{图结构构建($\mathcal{G}^{\text{s}}$与$\mathcal{G}^{\text{g}}$)}
\label{subsec:spin_graph}

在进入推断流水线之前,需要为两个子任务分别构建空间传播的基础图结构。两套图采用相同的连边规则,仅节点集合不同。

\textbf{站点图$\mathcal{G}^{\text{s}} = (\mathcal{V}^{\text{s}}, \mathcal{E}^{\text{s}})$}。如图~\ref{fig:study_area}(b)所示,节点集$\mathcal{V}^{\text{s}}$为$N^{\text{s}} = 152$个国控监测站点,连边基于站点间的测地距离构建(邻接矩阵$A^{\text{s}} \in \mathbb{R}^{N^{\text{s}} \times N^{\text{s}}}$),权重反映站点间的空间关联强度。

\textbf{网格图$\mathcal{G}^{\text{g}} = (\mathcal{V}^{\text{g}}, \mathcal{E}^{\text{g}})$}。节点集$\mathcal{V}^{\text{g}}$为覆盖研究区域的$N^{\text{g}}$个$0.25^{\circ}$网格像素,连边规则与站点图完全相同(邻接矩阵$A^{\text{g}} \in \mathbb{R}^{N^{\text{g}} \times N^{\text{g}}}$)。为了在网格图上提供训练监督信号,需要将图~\ref{fig:study_area}(c)中的站点观测映射到$0.25^{\circ}$网格上:将每个站点的坐标匹配到最近的网格像素,若同一像素内落入多个站点则取算术平均。如图~\ref{fig:study_area}(d)所示,蓝色区域为映射后拥有观测值的网格像素,其余大面积区域为无观测像素,直观展示了地面监测网络在网格尺度上的极端稀疏性——这正是网格推断任务需要克服的核心困难。

两套图的邻接矩阵均基于地理学第一定律\cqt{相近事物更相关}(Tobler's First Law of Geography),按节点间的测地距离定义:
\begin{equation}
    A_{ij} = \begin{cases}
        \exp\left(-\frac{d_{ij}^2}{2\sigma_d^2}\right), & \text{if } d_{ij} < d_{\text{th}} \\
        0, & \text{otherwise}
    \end{cases}
\label{eq:diffusion_adjacency}
\end{equation}
\noindent 式中$d_{ij}$为节点$i$和$j$之间的测地距离,$\sigma_d$为距离衰减系数(默认50 km),$d_{\text{th}}$为截断阈值(默认200 km)。公式\eqref{eq:diffusion_adjacency}确定了图的拓扑结构,即哪些节点对$(i,j)$之间存在边($A_{ij}>0$)。在此基础拓扑上,第\ref{sec:spin_model}节的阶段三将为每一条边赋予两类物理含义不同的传播权重——扩散核$\tilde{A}^{\mathcal{D}}_{ij}$对$A_{ij}$作对称归一化,建模各向同性的浓度梯度扩散;平流核$\tilde{A}^{\mathcal{A}}_{ij}$根据实时风场重新计算边权,建模各向异性的定向平流输送(第\ref{subsec:stage3_propagation}节)。两类核共享同一拓扑连接($d_{ij} < d_{\text{th}}$),但对边权施加不同的物理变换,在图传播时每个节点同时沿两条通道聚合邻居信息。


% ------------------------------------------------------------
% 4.4 数据融合与推断流程
% ------------------------------------------------------------
\section{数据融合与推断流程}
\label{sec:spin_model}

针对第\ref{sec:infer_intro}节提出的两个核心问题,本节详细阐述SPIN(Spatiotemporal Physics-Guided Inference Network)模型的设计思路与数据融合流程。该模型的关键不在于单一网络架构的创新,而在于如何将多源异构数据通过物理合理的方式融合到统一的推断框架中。具体而言,SPIN需要融合四类信息:(1)地面观测数据——提供稀疏但精确的\cqt{锚点}浓度值;(2)气象与排放数据——为每个空间位置(包括无观测区域)提供本地物理环境背景;(3)图结构与物理传播机制——将扩散与平流的大气传输规律编码为节点间的信息传递路径,实现从有观测区域向无观测区域的物理合理外推;(4)卫星遥感AOD数据——提供覆盖面广但存在非随机缺失的空间结构约束。这四类数据在精度、覆盖率和可靠性上各有优劣,SPIN的核心设计思想是让它们在推断流程中各司其职、互为补充。

该模型遵循第\ref{subsec:paradigm}节提出的\cqt{物理启发的数据驱动建模}范式,与第\ref{chap:prediction}章的PCDCNet形成互补关系:PCDCNet通过LID--STD--TAD三模块解耦时间维度的化学反应、空间传输和累积过程,用于预测已知站点的未来状态;而SPIN通过扩散--平流双核解耦空间维度的各向同性与各向异性传输过程,用于推断未知位置的当前状态。

基于第\ref{subsec:spin_graph}节构建的站点图$\mathcal{G}^{\text{s}}$与网格图$\mathcal{G}^{\text{g}}$,SPIN的推断流程由四阶段流水线组成。两个子任务$\mathcal{F}^{\text{s}}_{\Theta^{\text{s}}}$与$\mathcal{F}^{\text{g}}_{\Theta^{\text{g}}}$采用完全相同的四阶段架构,但各自独立训练,仅在输入节点集合、训练数据与损失函数上有所区别。

如图~\ref{fig:model_architecture}所示,四个阶段依次为:阶段一,通过动态节点掩码模拟数据缺失场景,赋予模型归纳式泛化能力(第\ref{subsec:stage1_masking}节);阶段二,利用时间卷积网络编码本地物理特征,为所有节点提供初始浓度估计(第\ref{subsec:stage2_encoding}节);阶段三,在基础图上构建扩散核$\tilde{A}^{\mathcal{D}}$与平流核$\tilde{A}^{\mathcal{A}}$,通过物理启发的双流传播机制实现空间信息交互(第\ref{subsec:stage3_propagation}节);阶段四,施加多层级损失约束,其中网格推断$\mathcal{F}^{\text{g}}_{\Theta^{\text{g}}}$额外引入AOD梯度约束(第\ref{subsec:stage4_constraints}节)。

\begin{figure}[htbp]
    \centering
    \includegraphics[width=1.0\linewidth]{figures/chap04_spin_architecture.pdf}
    \caption{SPIN模型总体架构图}
    \caption*{上半部分为前向推断流程(Input $\to$ Stage 1 $\to$ Stage 2 $\to$ Stage 3 $\to$ Output),下半部分为Stage 4的多层级损失约束(橙色虚线表示梯度回传路径)。\textbf{输入}(左侧)包含气象场$\mathbf{M}^{F}_{t-T'+1:t}$、排放场$\mathbf{E}^{F}_{t-T'+1:t}$与观测站点历史数据$\mathbf{X}^{\text{obs}}_{t-T'+1:t}$三类数据源。\textbf{阶段一}(动态节点掩码):在图$\mathcal{G}$上对站点进行随机划分,生成观测节点集$\mathcal{V}_{\text{obs}}$与目标节点集$\mathcal{V}_{\text{target}}$。\textbf{阶段二}(本地物理特征编码):将多源输入经嵌入层送入时间卷积网络(TCN),为每个节点生成初始隐表示$\mathbf{H}^{(0)}$;同时通过Readout分支输出初始猜测$\hat{\mathbf{X}}^{\text{init}}$,用于$\mathcal{L}_{\mathrm{init}}$约束。\textbf{阶段三}(物理启发的双图空间传播):在时空图神经网络(ST-GNN)中,利用扩散核$\tilde{A}^{\mathcal{D}}$(各向同性,基于距离)与平流核$\tilde{A}^{\mathcal{A}}$(各向异性,基于风场)双通道聚合邻居信息,经门控融合更新节点表示。\textbf{输出}(右侧):站点推断产生目标站点时序$\hat{\mathbf{X}}^{\text{target}}_{t-T'+1:t}$,网格推断产生全域浓度场$\hat{\mathbf{X}}^{F}_t$。\textbf{阶段四}(多层级约束与AOD融合,下半部分):$\mathcal{L}_{\mathrm{init}}$约束初始猜测与网格真值$\mathbf{X}^{F}_t$的一致性;$\mathcal{L}_{\mathrm{infer}}$约束推断输出与站点/网格真值的匹配;$\mathcal{L}_{\mathrm{AOD}}$利用卫星遥感AOD数据对网格推断施加空间梯度约束。站点推断$\mathcal{F}^{\text{s}}_{\Theta^{\text{s}}}$与网格推断$\mathcal{F}^{\text{g}}_{\Theta^{\text{g}}}$采用完全相同的四阶段架构,但各自独立训练。}
    \label{fig:model_architecture}
\end{figure}


\subsection{阶段一:归纳式训练策略——动态节点掩码}
\label{subsec:stage1_masking}

本阶段为站点推断和网格推断提供统一的训练基础。为了使模型具备对任意未观测节点(无论是监测站点还是网格像素)的推断能力,采用了动态节点掩码(Dynamic Node Masking)策略\citep{wu2021inductive}。与传统的直推式(Transductive)方法不同,本策略不依赖固定的图结构,而是通过模拟数据缺失场景,强迫模型学习通用的时空插值规律。

记所有拥有真实观测数据的监测站点全集为$\mathcal{V}_{\text{stations}}$(即图~\ref{fig:study_area}(a)(c)中展示的152个国家级监测站点)。归纳式训练的核心思想是:在每个训练批次$(b)$中,将$\mathcal{V}_{\text{stations}}$随机划分为两个互补子集——观测节点集$\mathcal{V}_{\text{obs}}^{(b)}$和目标节点集$\mathcal{V}_{\text{target}}^{(b)}$:
\begin{equation}
\mathcal{V}_{\text{obs}}^{(b)} = \text{RandomSample}(\mathcal{V}_{\text{stations}}, 1 - r_{\text{mask}}), \quad \mathcal{V}_{\text{target}}^{(b)} = \mathcal{V}_{\text{stations}} \setminus \mathcal{V}_{\text{obs}}^{(b)}
\label{eq:inductive_sampling}
\end{equation}
\noindent 式中$r_{\text{mask}}$为目标节点比例(如30\%、50\%),$(b)$表示第$b$个批次。$\mathcal{V}_{\text{obs}}^{(b)}$中的节点扮演问题定义(公式\ref{eq:infer_problem})中$\mathcal{V}_{\text{obs}}$的角色(图~\ref{fig:inference_problem}中黑色实心圆点),模型可获取其完整的历史污染数据;$\mathcal{V}_{\text{target}}^{(b)}$中的节点扮演$\mathcal{V}_{\text{target}}$的角色(图~\ref{fig:inference_problem}中空心圆点),其历史污染数据被遮蔽,模型需要对其进行推断。该划分过程对应算法\ref{alg:spin}第3--4行。

在训练过程中,$\mathcal{V}_{\text{target}}^{(b)}$中的节点仅能获取:(1)本地的气象与排放特征;(2)$\mathcal{V}_{\text{obs}}^{(b)}$节点经图传播传递过来的时空上下文信息。这种机制模拟了真实的推断场景——即推断无历史数据的盲点。更重要的是,由于每个批次的划分是随机生成的,模型在训练过程中会\cqt{见到}同一站点在不同批次中分别扮演观测节点和目标节点的角色。这种随机性有效防止了模型过拟合于特定站点的ID特征,迫使模型学习\cqt{基于环境背景和邻域关系进行推断}的通用物理规律。


\subsection{阶段二:本地物理特征编码}
\label{subsec:stage2_encoding}

本阶段为站点推断和网格推断提供统一的特征编码,为所有节点(包括阶段一中被掩码的目标节点,无论是站点还是网格像素)提供基于本地物理条件的初始浓度估计。

% 时序特征编码器
在大气系统中,某一点的污染物浓度不仅受周边传输影响,更取决于本地的排放强度和气象条件(如湿度影响吸湿性增长,边界层高度影响垂直稀释)\citep{tai2010correlations}。为了捕捉这些本地过程的时间依赖性,我们采用时间卷积网络(Temporal Convolutional Network, TCN)作为特征提取器。

令$\mathbf{x}_i^{in} = [\mathbf{M}_i, \mathbf{E}_i] \in \mathbb{R}^{T' \times F_{in}}$表示节点$i$在过去$T'$个时间步内的输入特征序列,其中包含了气象变量$\mathbf{M}$和排放变量$\mathbf{E}$。TCN将该序列映射为高维潜在表示$\mathbf{h}_i^{(0)}$:
\begin{equation}
    \mathbf{h}_i^{(0)} = \text{TCN}(\mathbf{x}_i^{in})
\label{eq:tcn_encoding}
\end{equation}

相较于循环神经网络(RNNs,如LSTM/GRU),TCN在本任务中具有显著优势:通过膨胀卷积(Dilated Convolution),TCN能够以指数级扩大的感受野捕捉长周期的累积效应(例如过去24小时的排放累积);同时支持并行处理整个时间序列,极大提升了大规模网格推断时的计算效率\citep{bai2018empirical}。

TCN由多层膨胀卷积组成。第$l$层的输出为:
\begin{equation}
\mathbf{h}^{(l)} = \text{ReLU}\left(\text{Conv1D}(\mathbf{h}^{(l-1)}, \mathbf{W}^{(l)}, \text{dilation}=2^l) + \mathbf{b}^{(l)}\right)
\label{eq:tcn_layer}
\end{equation}
\noindent 式中膨胀因子$2^l$使得$L$层TCN的感受野达到$2^L$个时间步。本研究采用4层膨胀卷积(膨胀因子$[1, 2, 4, 8]$),感受野覆盖16小时,足以捕捉日变化特征。

TCN的输出$\mathbf{h}_i^{(0)}$实际上为每个节点提供了一个基于本地物理条件的\cqt{初始猜测(Initial Guess)}。这一设计是实现归纳式推断的关键:目标节点虽然缺乏历史污染观测,但其气象和排放数据是完整可用的。

对于观测节点$v \in \mathcal{V}_{\text{obs}}$,将观测值$X_v$与时序表征拼接:
\begin{equation}
\mathbf{h}_v^{(0)} = [\mathbf{h}_v^{\text{temp}}, X_v] \in \mathbb{R}^{d_h+1}
\label{eq:obs_init}
\end{equation}
\noindent 对于目标节点$v \in \mathcal{V}_{\text{target}}$,仅使用时序表征作为初始化:
\begin{equation}
\mathbf{h}_v^{(0)} = [\mathbf{h}_v^{\text{temp}}, 0] \in \mathbb{R}^{d_h+1}
\label{eq:target_init}
\end{equation}

这种差异化的初始化策略体现了归纳式学习的核心思想:观测节点携带真实的污染信息,而目标节点仅携带环境背景信息,后者需要通过图传播从前者\cqt{借用}污染信息。


\subsection{阶段三:物理启发的双图空间传播}
\label{subsec:stage3_propagation}

经过阶段二的编码后,观测节点携带真实的污染信息,而目标节点(站点或网格像素)仅携带环境背景信息。本阶段在第\ref{subsec:spin_graph}节构建的基础图结构上,进一步定义两类具有不同物理意义的传播核——扩散核与平流核,分别对应大气污染物传输的两种基本机制(各向同性扩散与各向异性平流),从而在图神经网络中显式建模对流--扩散方程(公式\ref{eq:advection_diffusion})的物理过程。两个子任务$\mathcal{F}^{\text{s}}_{\Theta^{\text{s}}}$与$\mathcal{F}^{\text{g}}_{\Theta^{\text{g}}}$采用相同的核\textbf{构建方法}和\textbf{网络层设计}(包括传播权重$\mathbf{W}^{\mathcal{D}}$、$\mathbf{W}^{\mathcal{A}}$和门控参数$\mathbf{W}_g$的结构),但各自独立训练,参数不共享;由于节点集合不同,所得核矩阵维度分别为$N^{\text{s}} \times N^{\text{s}}$和$N^{\text{g}} \times N^{\text{g}}$。


\subsubsection{扩散核}

扩散核建模污染物在浓度梯度驱动下产生的各向同性随机运动。基于第\ref{subsec:spin_graph}节的距离邻接矩阵$A$(公式\ref{eq:diffusion_adjacency}),对其进行对称归一化,构造扩散算子$\tilde{A}^{\mathcal{D}}$:
\begin{equation}
    \tilde{A}^{\mathcal{D}} = \mathbf{D}^{-\frac{1}{2}} A \mathbf{D}^{-\frac{1}{2}}
\label{eq:diffusion_normalized}
\end{equation}
\noindent 式中$\mathbf{D}$是$A$的度矩阵。该对称归一化对应于图拉普拉斯算子的平滑作用\citep{kipf2017semi},使得节点能够聚合来自所有方向邻居的信息,模拟了公式(\ref{eq:advection_diffusion})中扩散项$K_h\left(\frac{\partial^2 C}{\partial x^2} + \frac{\partial^2 C}{\partial y^2}\right)$所描述的大气湍流导致的浓度均一化过程。扩散核为静态核,在模型整个生命周期内保持不变。


\subsubsection{平流核}

传统的空间图往往只考虑距离,隐含了各向同性假设。然而在大气环境中,风场主导的平流输送具有强烈的各向异性——上游节点对下游节点的影响远大于反向影响,对应公式(\ref{eq:advection_diffusion})中的平流项$-\left(u\frac{\partial C}{\partial x} + v\frac{\partial C}{\partial y}\right)$。

为此,本研究构建了平流核,这是一个随风场实时更新的动态有向传播机制。如图~\ref{fig:advection_kernel}所示,从节点$j$到节点$i$的平流权重定义为:
\begin{equation}
    A_{ij}^{\mathcal{A}} = \begin{cases}
        \max\left( \frac{|\mathbf{u}_j|}{d_{ij}} \cdot \cos(\alpha_{ji}), 0 \right) \cdot \exp\left(-\frac{d_{ij}}{L_s}\right), & \text{if } d_{ij} < d_{\text{th}} \\
        0, & \text{otherwise}
    \end{cases}
\label{eq:advection_adjacency}
\end{equation}
\noindent 式中$|\mathbf{u}_j|$是源节点$j$处的风速大小,$\alpha_{ji}$是风向向量$\mathbf{u}_j$与边向量$\mathbf{e}_{ji}$(从$j$指向$i$)之间的夹角,$L_s$为特征传输长度尺度(默认100 km)。$\max(\cdot, 0)$函数确保了权重的非负性和方向性:只有当风从$j$吹向$i$(即$\cos(\alpha_{ji}) > 0$)时,边才存在。这一设计与第\ref{chap:prediction}章PM$_{2.5}$-GNN中的平流系数(公式\eqref{eq:advection_coeff})异曲同工,都显式编码了\cqt{上游影响下游}的定向传输机制。

归一化采用行归一化(出度归一化):
\begin{equation}
\tilde{A}^{\mathcal{A}} = \mathbf{D}_{\text{out}}^{-1} A^{\mathcal{A}}
\label{eq:advection_normalized}
\end{equation}
\noindent 式中$D_{\text{out},ii} = \sum_j A_{ij}^{\mathcal{A}}$。行归一化确保每个节点的出流总和为1,符合质量守恒的物理约束——这与第\ref{chap:prediction}章\ref{subsec:pcdcnet_training}节讨论的DIC约束在理念上一致。

\begin{figure}[htbp]
    \centering
    \includegraphics[width=0.5\textwidth]{figures/chap04_advection_kernel.pdf}
    \caption{平流核(Advection Kernel)构建示意图}
    \caption*{以节点$i$为中心展示平流权重$A^{\mathcal{A}}_{ij}$的计算过程。棕色箭头为节点$i$处的风速矢量(风速$|\mathbf{v}|$、风向角$\beta$),蓝色箭头为从节点$j$(节点1)到节点$i$的边向量(方位角$\gamma$、距离$d$),$\alpha = |\beta - \gamma|$为风向与边方向的夹角。红色实心箭头表示节点1因$\cos\alpha > 0$(位于上风方向)而获得较大的平流权重$A^{\mathcal{A}}_{ij}$;红色虚线箭头表示节点2、3因与风向夹角较大而权重较小或为零。该机制显式编码了\cqt{上游影响下游}的定向传输规律。}
    \label{fig:advection_kernel}
\end{figure}


\subsubsection{双流传播层}

基于上述扩散核$\tilde{A}^{\mathcal{D}}$与平流核$\tilde{A}^{\mathcal{A}}$,我们设计了双流时空图神经网络层。对于节点$v$的特征表示$\mathbf{h}_v$,双图传播的更新规则为:

扩散传播(各向同性,无方向性):
\begin{equation}
\mathbf{m}_v^{\mathcal{D}} = \sum_{u \in \mathcal{N}(v)} \tilde{A}_{uv}^{\mathcal{D}} \cdot \mathbf{W}^{\mathcal{D}} \mathbf{h}_u^{(l)}
\label{eq:diffusion_message}
\end{equation}

平流传播(各向异性,沿风向):
\begin{equation}
\mathbf{m}_v^{\mathcal{A}} = \sum_{u \in \mathcal{N}(v)} \tilde{A}_{uv}^{\mathcal{A}} \cdot \mathbf{W}^{\mathcal{A}} \mathbf{h}_u^{(l)}
\label{eq:advection_message}
\end{equation}
\noindent 式中$\mathbf{W}^{\mathcal{D}}, \mathbf{W}^{\mathcal{A}} \in \mathbb{R}^{d_h \times d_h}$为可学习的权重矩阵,$l$为层索引。

两类消息通过门控机制融合,允许模型自适应平衡扩散和平流的相对贡献:
\begin{equation}
\mathbf{g}_v = \sigma\left(\mathbf{W}_g [\mathbf{m}_v^{\mathcal{D}}, \mathbf{m}_v^{\mathcal{A}}] + \mathbf{b}_g\right)
\label{eq:gate_weight}
\end{equation}
\begin{equation}
\mathbf{m}_v = \mathbf{g}_v \odot \mathbf{m}_v^{\mathcal{D}} + (1 - \mathbf{g}_v) \odot \mathbf{m}_v^{\mathcal{A}}
\label{eq:gate_fusion}
\end{equation}
\noindent 式中$\sigma$为Sigmoid函数,$\odot$为逐元素乘法,$\mathbf{W}_g \in \mathbb{R}^{d_h \times 2d_h}$为门控权重。具体而言,$\mathbf{g}_v \in \mathbb{R}^{d_h}$为门控权重向量(对应算法\ref{alg:spin}第15行的$\mathbf{G}$),其每个分量取值于$[0,1]$,逐维度控制扩散消息与平流消息的混合比例;$\mathbf{m}_v \in \mathbb{R}^{d_h}$为融合后的空间消息(对应算法\ref{alg:spin}第16行的$\Delta\mathbf{H}$),综合了来自两类图结构的邻域信息,随后通过残差连接注入节点表示。

门控机制的物理意义在于:在静风条件下,扩散过程主导污染物的空间分布,模型应侧重扩散核($\mathbf{g}_v \to 1$);在强风条件下,平流输送成为主导机制,模型应侧重平流核($\mathbf{g}_v \to 0$)。这种自适应的权重分配使得模型能够根据局地气象条件动态调整传播策略。

融合消息$\mathbf{m}_v$通过残差连接更新节点特征(对应算法\ref{alg:spin}第17行):
\begin{equation}
\mathbf{h}_v^{(l+1)} = \text{ReLU}\left(\mathbf{h}_v^{(l)} + \mathbf{m}_v\right)
\label{eq:residual_update}
\end{equation}

\noindent 残差连接确保了信息的有效传递和梯度的稳定流动,同时允许模型在必要时\cqt{跳过}某些传播层,直接利用初始特征。经过$L$层传播后,每个节点的最终表示$\mathbf{h}_v^{(L)}$已融合了TCN提取的本地物理特征(阶段二)与GNN从邻域聚合的外部污染信息(阶段三),随后送入阶段四的读出层生成浓度推断。




\subsection{阶段四:多层级约束与AOD融合}
\label{subsec:stage4_constraints}

本阶段是站点推断(图$\mathcal{G}^{\text{s}}$)与网格推断(图$\mathcal{G}^{\text{g}}$)的主要差异所在。经过阶段三的空间传播后,每个节点的潜在表示$\mathbf{H}_i^{(L)}$已深度融合了本地物理特征和邻域污染信息。两个子任务均采用相同设计的读出层、推断损失$\mathcal{L}_{\mathrm{infer}}$与初始化损失$\mathcal{L}_{\mathrm{init}}$,但由于独立训练,各自拥有独立的读出层参数。核心差异在于:AOD梯度损失$\mathcal{L}_{\mathrm{AOD}}$仅在网格推断$\mathcal{F}^{\text{g}}_{\Theta^{\text{g}}}$中启用,通过将卫星数据作为空间结构约束而非强制输入,解决非随机缺失遥感数据的有效融合问题。

\subsubsection{读出层}

经过$L$层物理启发的空间传播后,由于目标节点本身没有任何历史观测数据,无法采用自回归方式。因此设计了一个节点共享的读出层(Readout Layer),由多层感知机构成,直接将高维潜在特征映射为标量的\PM 浓度推断值(对应算法\ref{alg:spin}第21行及图\ref{fig:model_architecture}右侧):
\begin{equation}
    \hat{X}_v = \mathbf{W}_{\text{out}} \mathbf{h}_v^{(L)} + b_{\text{out}}
\label{eq:decoder_stage4}
\end{equation}

\noindent 式中$\mathbf{h}_v^{(L)} \in \mathbb{R}^{d_h}$为节点$v$经过阶段三传播后的最终潜在表示,$\mathbf{W}_{\text{out}} \in \mathbb{R}^{1 \times d_h}$和$b_{\text{out}} \in \mathbb{R}$为共享的读出参数,$\hat{X}_v \in \mathbb{R}$为节点$v$的\PM 浓度推断标量值。将所有节点的推断值堆叠即得到全局推断向量$\hat{\mathbf{X}} = [\hat{X}_1, \hat{X}_2, \ldots, \hat{X}_N]^\top \in \mathbb{R}^{N}$,对应问题定义(公式\ref{eq:infer_problem}--\ref{eq:infer_problem_grid})中的$\hat{\mathbf{X}}^{\text{target}}$或$\hat{\mathbf{X}}^{F}$。该输出直接用于后续三部分损失函数的计算:对目标节点计算推断损失$\mathcal{L}_{\mathrm{infer}}$(公式\ref{eq:infer_loss}),对观测节点计算初始化损失$\mathcal{L}_{\mathrm{init}}$(公式\ref{eq:init_loss}),在网格推断中还通过相邻节点的$\hat{X}_v$差值计算AOD梯度损失$\mathcal{L}_{\mathrm{AOD}}$(公式\ref{eq:aod_loss})。

共享参数的设计使得模型学习到的是一个通用的\cqt{特征到浓度}映射函数,而非针对特定站点的过拟合表示,确保了模型能够完全基于物理环境特征来填补数据盲区,实现真正的归纳式推断。

\subsubsection{复合损失函数设计}

为了在数据驱动的基础上引入物理约束,特别是解决卫星AOD数据\cqt{虽有空间结构但存在缺失与偏差}的难题,设计了一个包含三部分的复合损失函数:
\begin{equation}
    \mathcal{L} = \mathcal{L}_{\mathrm{infer}} + \lambda_1 \mathcal{L}_{\mathrm{init}} + \lambda_2 \mathcal{L}_{\mathrm{AOD}}
\label{eq:total_loss_chap3}
\end{equation}
\noindent 式中$\lambda_1$和$\lambda_2$为平衡系数。三项损失各司其职:$\mathcal{L}_{\mathrm{infer}}$为主损失,约束经图传播后目标节点的最终推断精度;$\mathcal{L}_{\mathrm{init}}$为辅助损失,约束TCN编码器在图传播之前即能给出合理的本地初始估计;$\mathcal{L}_{\mathrm{AOD}}$为空间结构约束,仅在网格推断中启用,利用卫星AOD的空间梯度引导预测场的形态。下面分别介绍各项的技术设计。


\subsubsection{推断损失(\LossInfer)}

这是模型的主要优化目标。计算经过图网络传播后的最终预测值$\hat{\mathbf{X}}$与被掩码目标节点的真实观测值之间的L1误差:
\begin{equation}
    \mathcal{L}_{\mathrm{infer}} = \frac{1}{|\mathcal{V}_{\text{target}}|} \sum_{v \in \mathcal{V}_{\text{target}}} |\hat{X}_v - X_v|
\label{eq:infer_loss}
\end{equation}

该损失函数直接驱动模型学习如何综合利用本地环境特征和邻居传播信息来填补数据盲区。采用$L_1$损失(MAE)而非$L_2$损失(MSE),以降低对异常值的敏感性——这在大气污染数据中尤为重要,因为极端污染事件虽然是重要的预测目标,但不应过度主导训练过程。


\subsubsection{初始化损失(\LossInit)}

为了确保时序特征编码器(TCN)提取的特征$\mathbf{H}^{(0)}$具有明确的物理意义,引入了辅助监督信号。要求TCN的直接输出(在进入图网络传播之前)也能对观测节点给出一个合理的\cqt{初始提案(Initial Proposal)}:
\begin{equation}
    \mathcal{L}_{\mathrm{init}} = \frac{1}{|\mathcal{V}_{\text{obs}}|} \sum_{v \in \mathcal{V}_{\text{obs}}} |\hat{X}_v^{\text{init}} - X_v|
\label{eq:init_loss}
\end{equation}
\noindent 式中$\hat{X}_v^{\text{init}} = \text{MLP}(\mathbf{h}_v^{(0)})$为节点$v$在阶段二(第\ref{subsec:stage2_encoding}节)中由TCN编码得到的潜在表示$\mathbf{h}_v^{(0)}$经一个独立的MLP映射后产生的标量浓度估计值,代表了\textbf{仅基于本地气象与排放信息、尚未经过任何图传播}的初始浓度判断。换言之,$\hat{X}_v^{\text{init}}$是阶段二的直接输出,而$\hat{X}_v$(公式\ref{eq:decoder_stage4})是经阶段三图传播后的最终输出,两者分别由$\mathcal{L}_{\mathrm{init}}$和$\mathcal{L}_{\mathrm{infer}}$约束。

该损失的设计动机在于:在推断阶段,目标节点$\mathcal{V}_{\text{target}}$没有历史观测数据,其初始表示$\mathbf{h}_v^{(0)}$完全来自本地气象与排放特征(公式\ref{eq:target_init})。如果TCN的编码缺乏物理意义,目标节点的初始估计将毫无信息量,后续的图传播将难以有效修正。因此,$\mathcal{L}_{\mathrm{init}}$通过在观测节点上施加监督,迫使TCN在没有任何邻居信息的情况下,仅凭本地条件就能给出一个尽可能接近真实值的基准估计。这样,目标节点即使缺乏观测数据,其基于相同TCN编码的初始估计也具有物理合理性,为后续的图传播提供了稳健的起始状态。

从物理角度理解,$\mathcal{L}_{\mathrm{init}}$迫使TCN学习\cqt{局地生成(Local Production)}的映射关系——即在排放强度高、边界层低、静风的条件下,本地污染水平应该较高。这一关系是公式(\ref{eq:advection_diffusion})中源项$S$和化学反应项$R$的数据驱动近似。


\subsubsection{掩码AOD空间梯度损失(\LossAOD)}

这是本研究解决AOD数据缺陷的核心创新。针对AOD数据存在的非随机缺失(云遮挡)和数值偏差(反演误差),不再将其作为输入,而是将其作为空间结构约束。

具体而言,约束模型预测场$\hat{\mathbf{X}}$的空间梯度(Spatial Gradient)与卫星AOD场$\mathbf{X}^{\text{AOD}}$的空间梯度保持一致。定义边$(i, j)$上的预测梯度和AOD梯度:
\begin{equation}
\nabla_{ij}(\hat{\mathbf{X}}) = \hat{X}_j - \hat{X}_i, \quad \nabla_{ij}(\mathbf{X}^{\text{AOD}}) = X_j^{\text{AOD}} - X_i^{\text{AOD}}
\label{eq:gradient_definition}
\end{equation}

为了处理缺失值,利用节点级有效性掩码$\boldsymbol{\omega}^{\text{AOD}} \in \{0,1\}^{N}$(第\ref{sec:infer_data}节),其中$\omega_i^{\text{AOD}} = 1$表示节点$i$处有有效AOD观测,$\omega_i^{\text{AOD}} = 0$表示该节点因云遮挡而缺失。在此基础上定义边级掩码$\Omega_{ij}^{\text{AOD}} = \omega_i^{\text{AOD}} \cdot \omega_j^{\text{AOD}}$,当且仅当边$(i,j)$两端节点均有有效AOD数据时$\Omega_{ij}^{\text{AOD}} = 1$:
\begin{equation}
    \mathcal{L}_{\mathrm{AOD}} = \sum_{t=1}^{T'} \sum_{(i,j) \in \mathcal{E}} \left| \nabla_{ij}(\hat{\mathbf{X}}_t) - \nabla_{ij}(\mathbf{X}^{\text{AOD}}_t) \right| \cdot \Omega_{ij}^{\text{AOD}}
\label{eq:aod_loss}
\end{equation}
\noindent 式中边级掩码由节点级掩码的乘积导出:
\begin{equation}
\Omega_{ij}^{\text{AOD}} = \omega_i^{\text{AOD}} \cdot \omega_j^{\text{AOD}}
\label{eq:aod_mask}
\end{equation}

\begin{figure}[htbp]
    \centering
    \includegraphics[width=0.75\textwidth]{figures/chap04_aod_mask_loss.pdf}
    \caption{网格图上的掩码AOD梯度约束示意图}
    \caption*{左图为模型预测浓度场在网格图上沿边$(i,j)$的空间梯度$\nabla_{ij}(\hat{\mathbf{X}}) = \hat{X}_j - \hat{X}_i$;右图为对应的卫星AOD场梯度$\nabla_{ij}(\mathbf{X}^{\text{AOD}})$,其中斜线阴影区域($\boldsymbol{\omega}^{\text{AOD}} = 0$)表示AOD因云层遮挡而缺失的网格节点。损失函数$\mathcal{L}_{\mathrm{AOD}}$(公式\ref{eq:aod_loss})仅在两端点均有有效AOD观测($\Omega^{\text{AOD}}_{ij}=1$)的边上计算,从而避免对缺失区域的错误约束。}
    \label{fig:aod_mask_loss}
\end{figure}

如图~\ref{fig:aod_mask_loss}所示,这一设计的物理直觉在于:

(1)关注\cqt{形}而非\cqt{值}。通过约束梯度,模型学习的是污染物的空间分布形态(如羽流的形状、扩散的边界),而不受AOD绝对数值偏差的影响。这实现了\cqt{数值解耦}——即使AOD系统性地高估或低估了地面浓度,只要其空间相对分布是正确的,模型仍能从中受益。

(2)鲁棒性。掩码机制$\boldsymbol{\omega}^{\text{AOD}}$使得损失函数自动忽略云层遮挡区域,避免了对缺失值的错误强行拟合。模型不会因为AOD在某些位置缺失而产生异常梯度。

(3)全天候适用。在夜间或云覆盖时,$\mathcal{L}_{\mathrm{AOD}}$约束自动失效(因为$\boldsymbol{\omega}^{\text{AOD}}$全为零),模型平滑回退到由物理核(扩散--平流)驱动的推断模式,仅凭气象场和地面站点生成了物理上合理的平滑场。这种\cqt{自适应回退}特性是本方法相比传统AOD输入方法的核心优势。

算法\ref{alg:spin}给出了SPIN的完整训练与推理流程。

\begin{algorithm}[htbp]
\caption{SPIN 模型训练与推理流程}
\caption*{\textbf{SPIN}:算法对应第\ref{sec:spin_model}节所述的四阶段数据融合与推断流程——阶段一:动态节点掩码(第\ref{subsec:stage1_masking}节),阶段二:TCN时序编码(第\ref{subsec:stage2_encoding}节),阶段三:双图传播(第\ref{subsec:stage3_propagation}节),阶段四:复合损失优化(第\ref{subsec:stage4_constraints}节)。}
\label{alg:spin}
\begin{algorithmic}[1]
    \REQUIRE 训练数据集$\mathcal{D}$,初始化参数$\Theta$,学习率$\eta$,掩码比例$r_{\text{mask}}$
    \ENSURE 优化后的参数$\Theta$(训练),推断结果$\{\hat{\mathbf{X}}\}$(推理)

    \FOR{每个样本$D_k = (\mathbf{X}, \mathbf{M}, \mathbf{E}, \mathbf{X}^{\text{AOD}}) \in \mathcal{D}$}
        \STATE \textbf{阶段1:动态节点掩码(第\ref{subsec:stage1_masking}节)}
        \STATE $\mathcal{V}_{\text{target}}^{(b)} \leftarrow \text{RandomSample}(\mathcal{V}_{\text{stations}},\; r_{\text{mask}})$ \COMMENT{随机采样目标节点}
        \STATE $\mathcal{V}_{\text{obs}}^{(b)} \leftarrow \mathcal{V}_{\text{stations}} \setminus \mathcal{V}_{\text{target}}^{(b)}$ \COMMENT{剩余为观测节点}

        \STATE \textbf{阶段2:时序特征编码(第\ref{subsec:stage2_encoding}节)}
        \FOR{每个节点$v \in \mathcal{V}$}
            \STATE $\mathbf{h}_v^{\text{temp}} \leftarrow \text{TCN}([\mathbf{M}_v, \mathbf{E}_v])$ \COMMENT{本地时序编码}
        \ENDFOR
        \STATE $\mathbf{h}_v^{(0)} \leftarrow [\mathbf{h}_v^{\text{temp}}, X_v]$($v \in \mathcal{V}_{\text{obs}}^{(b)}$);$\mathbf{h}_v^{(0)} \leftarrow [\mathbf{h}_v^{\text{temp}}, 0]$($v \in \mathcal{V}_{\text{target}}^{(b)}$) \COMMENT{差异化初始化}

        \STATE \textbf{阶段3:物理启发的图传播(第\ref{subsec:stage3_propagation}节)}
        \STATE 根据当前时刻风场更新平流图$\tilde{\mathbf{A}}^{\mathcal{A}}$
        \FOR{$l \leftarrow 1$ to $L$}
            \STATE $\mathbf{H}^{\mathcal{D}} \leftarrow \tilde{\mathbf{A}}^{\mathcal{D}} \mathbf{W}^{\mathcal{D}} \mathbf{H}^{(l-1)}$ \COMMENT{扩散消息(公式\ref{eq:diffusion_message})}
            \STATE $\mathbf{H}^{\mathcal{A}} \leftarrow \tilde{\mathbf{A}}^{\mathcal{A}} \mathbf{W}^{\mathcal{A}} \mathbf{H}^{(l-1)}$ \COMMENT{平流消息(公式\ref{eq:advection_message})}
            \STATE $\mathbf{G} \leftarrow \sigma(\mathbf{W}_g [\mathbf{H}^{\mathcal{D}}, \mathbf{H}^{\mathcal{A}}])$ \COMMENT{门控权重$g_v$(公式\ref{eq:gate_weight})}
            \STATE $\Delta\mathbf{H} \leftarrow \mathbf{G} \odot \mathbf{H}^{\mathcal{D}} + (1-\mathbf{G}) \odot \mathbf{H}^{\mathcal{A}}$ \COMMENT{融合消息$m_v$(公式\ref{eq:gate_fusion})}
            \STATE $\mathbf{H}^{(l)} \leftarrow \text{ReLU}(\mathbf{H}^{(l-1)} + \Delta\mathbf{H})$ \COMMENT{残差更新(公式\ref{eq:residual_update})}
        \ENDFOR

        \STATE \textbf{阶段4:读出与损失计算(第\ref{subsec:stage4_constraints}节)}
        \STATE $\hat{\mathbf{X}} \leftarrow \mathbf{W}_{\text{out}} \mathbf{H}^{(L)} + \mathbf{b}_{\text{out}}$ \COMMENT{读出层$\hat{X}_v$(公式\ref{eq:decoder_stage4})}
        \STATE $\hat{\mathbf{X}}^{\text{init}} \leftarrow \text{MLP}(\mathbf{H}^{(0)})$ \COMMENT{初始估计$\hat{X}_v^{\text{init}}$}
        \STATE $\mathcal{L}_{\mathrm{infer}} \leftarrow \frac{1}{|\mathcal{V}_{\text{target}}^{(b)}|}\sum_{v \in \mathcal{V}_{\text{target}}^{(b)}} |\hat{X}_v - X_v|$ \COMMENT{公式\ref{eq:infer_loss}}
        \STATE $\mathcal{L}_{\mathrm{init}} \leftarrow \frac{1}{|\mathcal{V}_{\text{obs}}^{(b)}|}\sum_{v \in \mathcal{V}_{\text{obs}}^{(b)}} |\hat{X}_v^{\text{init}} - X_v|$ \COMMENT{公式\ref{eq:init_loss}}
        \STATE $\mathcal{L}_{\mathrm{AOD}} \leftarrow \sum_{(i,j) \in \mathcal{E}} |\nabla_{ij}(\hat{\mathbf{X}}) - \nabla_{ij}(\mathbf{X}^{\text{AOD}})| \cdot \Omega_{ij}^{\text{AOD}}$ \COMMENT{公式\ref{eq:aod_loss}}
        \STATE $\mathcal{L} \leftarrow \mathcal{L}_{\mathrm{infer}} + \lambda_1 \mathcal{L}_{\mathrm{init}} + \lambda_2 \mathcal{L}_{\mathrm{AOD}}$ \COMMENT{公式\ref{eq:total_loss_chap3}}
    \ENDFOR

    \STATE $\Theta \leftarrow \Theta - \eta \nabla_{\Theta} \mathcal{L}$
    \RETURN $\Theta$(训练)\textbf{或} $\{\hat{\mathbf{X}}\}$(推理)
\end{algorithmic}
\end{algorithm}


% ------------------------------------------------------------
% 4.6 实验与结果
% ------------------------------------------------------------
\section{实验与结果}
\label{sec:infer_experiment}

为了全面验证SPIN的归纳泛化能力和物理一致性,实验评估围绕第\ref{sec:infer_problem}节定义的两个子任务展开:(1)\textbf{站点推断}——在保留测试站点上建立模型的定量优势(第\ref{subsec:station_inference}节),并通过极端数据稀缺(50\%站点缺失)条件下的案例分析深入剖析模型行为(第\ref{subsec:robustness_analysis}节);(2)\textbf{网格推断}——展示物理核与AOD约束如何协同生成连续、高保真的污染地图(第\ref{subsec:grid_inference}节)。


\subsection{实验设置}
\label{subsec:infer_exp_setup}

(1)数据集划分。为了严格防止时间维度的数据泄露,本研究采用按年份划分数据集:2018--2021年为训练集,2022年为验证集,2023年为测试集。此外,为了评估模型对季节性污染特征的适应能力,定义了两个季节性子集:

\begin{itemize}
    \item 冬季(供暖季):11月至次年3月,特征为高排放、静稳天气多发,\PM 污染严重;
    \item 夏季:5月至9月,特征为强对流天气频繁、大气扩散条件较好,\PM 浓度相对较低。
\end{itemize}

对于站点推断任务,将152个监测站点按7:3的比例划分为训练集(106个站点)和完全未观测的测试集(46个站点)。测试集中的站点在整个训练过程中从未被模型\cqt{看到}。

(2)评价指标。采用以下指标评估模型性能:

\begin{itemize}
    \item MAE(平均绝对误差):$\text{MAE} = \frac{1}{n}\sum_{i=1}^{n}|y_i - \hat{y}_i|$,反映平均推断偏差;
    \item RMSE(均方根误差):$\text{RMSE} = \sqrt{\frac{1}{n}\sum_{i=1}^{n}(y_i - \hat{y}_i)^2}$,对大误差敏感;
    \item $R^2$(决定系数):反映推断值与真值的相关性,$R^2 = 1 - \frac{\sum(y_i-\hat{y}_i)^2}{\sum(y_i-\bar{y})^2}$。
\end{itemize}

对于网格推断任务,额外计算空间相关系数(Spatial Correlation)以评估场的结构一致性。

(3)基线方法。将SPIN与三类主流方法进行了对比:

\begin{itemize}
    \item 特征工程模型:XGBoost\citep{chen2016xgboost}、MLP,代表传统机器学习方法;
    \item 时序深度学习模型:LSTM\citep{hochreiter1997long}、GRU\citep{cho2014learning},代表纯时序建模方法;
    \item 时空图神经网络模型:STGCN\citep{yu2018spatio}、IGNNK\citep{wu2021inductive},代表当前最先进的空间插值方法。
\end{itemize}

(4)实现细节。TCN采用4层膨胀卷积,膨胀因子为$[1, 2, 4, 8]$,卷积核大小为3。双图GNN采用2层消息传递。隐藏维度$d_h = 64$。优化器为Adam,初始学习率$10^{-3}$,采用余弦退火调度。批大小32,训练200个epoch,早停patience为20。损失权重$\lambda_1 = 0.5$,$\lambda_2 = 0.1$(基于验证集调优)。实验在NVIDIA A100 GPU上进行,单次训练约4小时。


\subsection{推断精度评估}
\label{subsec:station_inference}

针对子任务(1)站点推断,首先在保留的30\%测试站点(完全未参与训练的\cqt{盲点})上评估模型的推断精度。表~\ref{table:performance_comparison}展示了各模型在不同季节下的平均绝对误差(MAE)。

\begin{table}[htbp]
\centering
\caption{未观测站点(30\% Hold-out)的推断性能对比}
\caption*{评价指标为MAE($\mu$g/m$^3$),对比方法涵盖特征工程类(XGBoost、MLP)、时序类(LSTM、GRU)、时空图网络类(STGCN、IGNNK)以及本文提出的物理启发类方法SPIN。}
\label{table:performance_comparison}
\resizebox{0.9\textwidth}{!}{%
\begin{tabular}{@{}ll|c|c|c@{}}
\toprule
\multirow{2}{*}{\textbf{模型类别}} & \multirow{2}{*}{\textbf{方法}} & \textbf{全年(All Year)} & \textbf{冬季(Winter)} & \textbf{夏季(Summer)} \\
 &  & MAE$\downarrow$ & MAE$\downarrow$ & MAE$\downarrow$ \\ \midrule
\multirow{2}{*}{特征工程类} & XGBoost & 24.21 & 35.97 & 15.49 \\
 & MLP & 25.71 & 38.67 & 15.30 \\ \midrule
\multirow{2}{*}{时序类} & LSTM & 24.65 & 34.96 & 14.49 \\
 & GRU & 22.79 & 34.17 & 15.00 \\ \midrule
\multirow{2}{*}{时空图网络类} & STGCN & 17.84 & 23.43 & 11.23 \\
 & IGNNK & 12.73 & 16.29 & 8.53 \\ \midrule
\textbf{物理启发类} & \textbf{SPIN}(本文) & \textbf{9.52} & \textbf{15.09} & \textbf{7.65} \\ \bottomrule
\end{tabular}%
}
\end{table}

实验结果表明:

(1)总体性能优势。SPIN取得了全场最优的MAE(9.52 $\mu$g/m$^3$),相比目前最先进的归纳式基线IGNNK(12.73 $\mu$g/m$^3$),误差降低了25.2\%。这有力地证明了显式建模物理传输过程比纯数据驱动的统计插值更具优势。

(2)冬季鲁棒性。在污染最严重、气象条件最复杂的冬季,SPIN仍保持最优性能(MAE 15.09),相比IGNNK(16.29)降低7.4\%,相比STGCN(23.43)降低35.6\%。冬季的绝对误差高于夏季,反映了该季节污染过程本身的高变异性。STGCN等传统模型依赖固定的静态图结构,隐含了各向同性假设,难以适应冬季季风带来的强烈定向输送;而SPIN的动态平流核能够根据实时风场调整权重,从而精准捕捉了跨区域的污染传输——这与第\ref{chap:prediction}章PM$_{2.5}$-GNN的发现一致。

(3)基线缺陷分析。特征类和时序类模型由于完全忽略了空间相关性,误差普遍较高($>$20 $\mu$g/m$^3$)。IGNNK虽然引入了K-近邻图进行归纳推理,但其基于欧氏距离的构图缺乏物理方向性(各向同性假设),无法区分上游和下游的影响。SPIN通过扩散--平流双核的设计,显式解耦了这两种物理过程,实现了更准确的空间依赖建模。


\subsection{鲁棒性分析}
\label{subsec:robustness_analysis}

继续围绕子任务(1)站点推断,通过具体站点的案例分析深入剖析模型行为。为了测试模型在数据极度稀缺条件下的边界能力,将站点掩码比例提升至50\%,即仅利用一半的站点来推断另一半。

(1)区域泛化能力。图~\ref{fig:bj_group}展示了京津冀核心城市群(北京、天津、石家庄)中三个代表性未观测站点的推断结果。即便缺失了一半的监测网络,模型依然能够高精度地($R^2 > 0.85$)复现污染物的日变化和综观趋势。特别是对于2020年1月底的区域性重污染过程,模型准确捕捉到了污染的累积(Accumulation)和消散(Dissipation)阶段,证明物理核成功学到了区域传输的宏观规律。

\begin{figure}[htbp]
    \centering
    \includegraphics[width=\textwidth]{figures/chap04_station_beijing.png}\\[0.35em]
    \includegraphics[width=\textwidth]{figures/chap04_station_tianjin.png}\\[0.35em]
    \includegraphics[width=\textwidth]{figures/chap04_station_shijiazhuang.png}
    \caption{京津冀核心城市群的推断表现}
    \caption*{从上至下分别为北京定陵站、天津前进道站、石家庄人民会堂站。即便在50\%站点不可见的情况下,模型依然保持了极高的相关性($R^2 > 0.85$),准确捕捉了冬季污染过程的动态演变。}
    \label{fig:bj_group}
\end{figure}

(2)推断性能的上下界分析。图~\ref{fig:rmse_extremes}展示了推断性能的两个极端案例。上图(商丘粮食局站)展示了在密集邻居支持下的完美重建($R^2=0.96$)——密集的连接使得扩散核能够有效聚合梯度信息,实现高保真重建;下图(许昌监测站)展示了在复杂局部动态下的较大误差($R^2=0.644$),这可能是由于微尺度排放突变被邻居平均所平滑。然而,模型仍正确把握了污染的起止时间和峰值位置,证明平流核成功捕捉了宏观传输模式。

\begin{figure}[htbp]
    \centering
    \includegraphics[width=\textwidth]{figures/chap04_station_shangqiu.png}\\[0.35em]
    \includegraphics[width=\textwidth]{figures/chap04_station_xuchang.png}
    \caption{推断性能的上下界分析}
    \caption*{上图(商丘)展示了在密集邻居支持下的完美重建($R^2=0.96$);下图(许昌)展示了在复杂局部动态下的误差,但模型仍正确把握了污染起止时间和趋势方向。}
    \label{fig:rmse_extremes}
\end{figure}

(3)极端稀疏与物理保真。图~\ref{fig:sparse_neighbors}聚焦于一个极端的边缘站点——张家口1057A站。该站点位于西北山区边缘,上游几乎没有传感器支持。在此极端条件下,定量指标有所下降($R^2=0.132$),这是由于高频局部波动的丢失。然而,深入分析发现一个重要的物理特性:模型并未产生幻觉(Hallucination)或输出零值,而是成功恢复了正确的低频背景趋势(20--50 $\mu$g/m$^3$)。

这验证了扩散核在数据真空中起到了\cqt{保守低通滤波器}的作用——在缺乏信息时,模型倾向于输出物理上合理的背景值,而非随机噪声。这种\cqt{宁可平滑,绝不谬误}的特性对于环境决策支持系统至关重要:即使在最不利的数据条件下,模型也能提供一个合理的基线估计。

\begin{figure}[htbp]
    \centering
    \includegraphics[width=\textwidth]{figures/chap04_station_edge.png}\\[0.35em]
    \includegraphics[width=\textwidth]{figures/chap04_station_zhangjiakou.png}
    \caption{边缘稀疏区域的物理保真性}
    \caption*{上图展示了该站点在监测网络中的边缘位置(上游几乎无传感器覆盖);下图(张家口人民公园站)展示了在极度缺乏上游信息时,物理启发机制确保了模型输出合理的背景趋势,而非随机噪声或幻觉值。}
    \label{fig:sparse_neighbors}
\end{figure}


\subsection{网格推断分析}
\label{subsec:grid_inference}

针对子任务(2)网格推断,本节评估模型生成连续高分辨率污染场的能力。如第\ref{sec:infer_problem}节所定义,网格推断将研究区域离散为$0.25^{\circ} \times 0.25^{\circ}$的规则网格(如图~\ref{fig:study_area}(d)所示),所有网格像素构成目标节点集$\mathcal{V}_{\text{target}}$,所有152个监测站点构成观测节点集$\mathcal{V}_{\text{obs}}$。网格推断与站点推断采用相同的模型架构设计与物理图核构建方法(阶段一至三,第\ref{sec:spin_model}节),其归纳式训练策略使模型能够直接泛化至训练中从未出现过的网格节点——每个网格节点的初始表示由本地气象与排放特征经TCN编码生成(阶段二),随后通过扩散图与平流图从周围观测站点聚合污染信息(阶段三)。两类任务的主要差异在于阶段四:网格推断在训练时额外启用AOD梯度损失$\mathcal{L}_{\mathrm{AOD}}$(公式\ref{eq:aod_loss}),利用卫星观测的空间梯度约束预测场的形态。需要强调的是,AOD仅作为训练时的空间结构约束,而非推理时的直接输入特征,因此模型在推理阶段不依赖卫星数据的可用性。

图~\ref{fig:grid_inference}通过四个典型案例,深入剖析了上述机制的实际效果——SPIN如何利用物理核与AOD梯度损失来应对卫星数据的各类缺陷。图中的AOD面板作为外部参考,用于验证模型是否成功内化了污染的空间结构模式。

\begin{figure}[htbp]
    \centering
    \begin{subfigure}{\textwidth}
        \centering
        \includegraphics[width=0.9\textwidth]{figures/chap04_grid_ideal.png}
        \caption{案例(a):理想一致性——结构迁移}
    \end{subfigure}
    \begin{subfigure}{\textwidth}
        \centering
        \includegraphics[width=0.9\textwidth]{figures/chap04_grid_missing.png}
        \caption{案例(b):完全缺失——物理回退}
    \end{subfigure}
    \begin{subfigure}{\textwidth}
        \centering
        \includegraphics[width=0.9\textwidth]{figures/chap04_grid_conflict.png}
        \caption{案例(c):结构冲突——鲁棒性}
    \end{subfigure}
    \begin{subfigure}{\textwidth}
        \centering
        \includegraphics[width=0.9\textwidth]{figures/chap04_grid_calibration.png}
        \caption{案例(d):幅度偏差——自动校准}
    \end{subfigure}
    \caption{网格推断与AOD机制分析}
    \caption*{左列:模型推断的\PM 场;中列:地面站点真值;右列:卫星AOD参考(仅作训练约束)。(a) 理想情况下的结构迁移;(b) AOD缺失时的物理回退;(c) 抗干扰鲁棒性;(d) 幅度偏差的自动校准。}
    \label{fig:grid_inference}
\end{figure}

案例(a) 理想一致性(结构迁移)。当AOD数据完整且与地面观测一致时,模型通过梯度约束,成功将AOD的羽流细节\cqt{迁移}到了\PM 推断场中,实现了平滑且精细的插值。图中清晰地展示了沿太行山脉分布的高浓度污染带,与地形阻挡效应高度吻合。这验证了$\mathcal{L}_{\mathrm{AOD}}$约束的有效性:当卫星数据可用且可靠时,模型能够充分利用其空间结构信息。

案例(b) 完全缺失(物理回退)。在夜间或全云覆盖导致AOD完全缺失的场景下,AOD损失项自动失效($\boldsymbol{\omega}^{\text{AOD}}$全为零)。此时,模型并未瘫痪,而是稳健地回退(Fallback)到由物理核(平流与扩散)驱动的推断模式,仅凭气象场和地面站点生成了物理上合理的平滑场。这证明了模型具备全天候运行能力——这是相比传统AOD输入方法的核心优势。

案例(c) 结构冲突(鲁棒性)。当AOD出现虚假高值或纹理(可能由地表反射率误差引起)时,模型优先遵循地面观测的强约束($\mathcal{L}_{\mathrm{infer}}$),未被卫星伪影误导,避免了虚假热点的产生。这种\cqt{地面为主、卫星为辅}的层级约束结构确保了推断结果的可靠性。

案例(d) 幅度偏差(自动校准)。当AOD整体高估(颜色过深)时,模型仅提取了其\cqt{形状(梯度)}信息,而利用地面站点校准了\cqt{数值(幅度)}。这证明了基于梯度的损失函数成功实现了形状与数值的解耦——即使AOD存在系统性偏差,模型仍能从中提取有价值的空间结构信息。


% ------------------------------------------------------------
% 4.7 本章小结
% ------------------------------------------------------------
\section{本章小结}
\label{sec:infer_summary}

针对区域大气污染监测中的两个核心问题——稀疏监测网络下的空间泛化推断(问题一)和非随机缺失遥感数据的有效融合(问题二),本章提出了基于物理启发的归纳式图推断网络(SPIN),通过四阶段数据融合与推断流程(第\ref{sec:spin_model}节)系统性地加以解决。站点推断$\mathcal{F}^{\text{s}}_{\Theta^{\text{s}}}$与网格推断$\mathcal{F}^{\text{g}}_{\Theta^{\text{g}}}$采用相同的四阶段架构与图核构建方法,但各自独立训练、参数不共享,仅在输入节点集合与损失函数上有所区别(网格推断额外引入AOD梯度约束)。实验评估围绕第\ref{sec:infer_problem}节定义的两个子任务展开:站点推断实验验证了模型的空间泛化精度与鲁棒性,网格推断实验展示了物理核与AOD约束协同生成全域污染地图的能力。主要贡献总结如下:

(1)\textbf{构建了物理嵌入的双流图网络架构。}通过显式设计扩散核与平流核,模型成功解耦了浓度梯度驱动的各向同性扩散(公式\eqref{eq:diffusion_normalized})与风场驱动的各向异性平流输送(公式\eqref{eq:advection_adjacency})两类物理过程,分别对应公式(\ref{eq:advection_diffusion})中的扩散项与平流项。平流核根据实时风场动态调整边权重,显式编码\cqt{上游影响下游}的定向传输规律;扩散核模拟浓度梯度驱动的均一化过程,在数据稀疏区域起到\cqt{保守滤波}作用。这种设计与第\ref{chap:prediction}章PM$_{2.5}$-GNN和PCDCNet的物理启发思想一脉相承。

(2)\textbf{实现了归纳式时空推断能力。}通过动态掩码训练策略(公式\eqref{eq:inductive_sampling}),模型摆脱了对固定图结构的依赖,具备了对任意空间位置进行\cqt{虚拟传感}的能力。站点推断实验证明,该模型在未观测站点上的推断精度显著优于现有最优方法(MAE降低25.2\%),特别是在冬季复杂气象条件下表现出极强的鲁棒性。

(3)\textbf{提出了全天候制图的AOD融合范式。}创新性地提出\cqt{掩码AOD梯度损失}(公式\eqref{eq:aod_loss}),实现了从\cqt{依赖卫星输入}到\cqt{利用卫星约束}的范式转换。网格推断实验表明,这一机制不仅有效利用了晴空下的卫星空间信息,更在卫星数据缺失(云/夜)时,保证了系统能够平滑回退到物理驱动模式,从而生成了时空连续、物理一致的高分辨率污染地图。

从方法论角度,本章将第\ref{subsec:paradigm}节提出的\cqt{物理启发数据驱动建模}范式从时间维度(第\ref{chap:prediction}章的预测)扩展到空间维度(推断),通过在图神经网络中显式嵌入扩散--平流双核物理机制并采用归纳式学习策略,使模型从\cqt{监测什么推断什么}的直推式范式跨越到\cqt{学习物理规律、推断任意位置}的归纳式范式。SPIN有效地充当了\cqt{虚拟传感器}的角色:在站点推断中为未受监控的郊区和农村地区提供可靠的逐小时浓度估计(MAE降低25.2\%),在网格推断中生成$0.25^{\circ}$分辨率的全天候污染场,可用于识别监测盲区中的污染热点与评估环境暴露公平性。本章构建的高分辨率污染场也为后续第\ref{chap:simulation}章开展精细化的情景模拟奠定了空间数据基础。  % 基于多源数据融合的大气污染空间推断
% ============================================================
% 第五章 未来污染情景模拟
% 基于复杂系统数据驱动建模的大气污染研究
% ============================================================

\chapter{未来污染情景模拟}
\label{chap:simulation}

前两章分别解决了时间预测和空间推断问题,本章则聚焦于更具前瞻性的情景模拟问题——在碳中和战略背景下,未来数十年的空气质量将如何演变?本章提出IGNN(Integrated Graph Neural Network,集成图神经网络)模型,将排放清单作为可控变量显式纳入深度学习框架,实现多情景下\PM 与\ozone 的长期演变模拟。

% ------------------------------------------------------------
% 5.1 引言
% ------------------------------------------------------------
\section{引言}
\label{sec:sim_intro}

正如第\ref{chap:introduction}章所述,大气污染治理与碳达峰碳中和目标深度交织。在全球气候变化与可持续发展的双重压力下,大气污染治理已不再是孤立的环境问题,而是与国家能源战略、经济结构转型和全球气候承诺紧密交织的复杂系统性挑战。对于中国而言,\cqt{双碳}目标的提出不仅是一场深刻的能源与经济革命,也为空气质量的根本性改善带来了历史性机遇。

从科学问题角度,准确预估未来空气质量演变对于制定长期环境规划具有重要的决策支撑价值。一方面,政策制定者需要了解不同减排路径下的空气质量响应,以权衡经济成本与环境效益;另一方面,公众健康部门需要预判未来污染暴露水平的变化趋势,以前瞻性地配置医疗卫生资源。这类\cqt{如果-那么}式的情景分析需求,对传统预测模型提出了新的挑战——不仅要具备准确的预测能力,更要能够响应假设性的排放变化。

本章研究的核心问题是:面向碳达峰碳中和目标,如何利用历史数据训练的模型迁移至未来排放情景,预测不同减排路径下的空气质量演变?本章的技术路线是:首先利用历史数据(2014-2020年)训练IGNN模型,学习排放-气象-浓度之间的响应关系——其中历史排放数据来自MEIC清单,历史气象数据来自ERA5再分析资料,历史浓度观测来自国家环境监测网络;随后将训练好的模型应用于未来情景数据(2025-2050年),其中未来排放来自DPEC(Dynamic Projection of Emissions in China,中国排放动态预测)模型的六种减排策略,未来气象来自CMIP6(Coupled Model Intercomparison Project Phase 6,第六次耦合模式比较计划)气候模式输出,从而实现跨时段的迁移模拟。温室气体与大气污染物在很大程度上同根同源,特别是在\PM 和\ozone 这两种关键污染物的协同控制上,其复杂的非线性关系构成了决策过程中的核心科学难题。

本章需解决的关键技术难题包括以下三个方面:

(1)排放-浓度响应关系的物理建模。情景模拟的核心是建立排放源强与污染物浓度之间的响应映射,这要求模型能够刻画公式(\ref{eq:advection_diffusion})中排放项$S$与浓度场$\mathbf{X}$之间的非线性响应关系。现有自回归模型依赖历史浓度序列进行外推,缺乏对排放驱动的显式表征,无法响应\cqt{如果减排50\%结果如何}这类假设情景。如何设计\cqt{一对一映射}架构,使模型仅依赖气象和排放输入直接预测浓度,是物理启发嵌入的核心问题。

(2)历史与未来数据源的融合。历史训练数据基于MEIC排放清单,而未来情景数据来源于DPEC动态评估模型,两者在空间分辨率、行业分类和时间覆盖上存在差异\citep{zheng2018trends,cheng2021pathways}。此外,未来气象条件来自CMIP6耦合模式输出,与历史ERA5再分析数据在统计特性上也存在偏差。如何建立跨数据源的有效融合机制,确保模型在域迁移(Domain Shift)条件下仍能保持稳定性能,是多源数据融合的核心问题。

(3)未来情景外推能力。模拟任务要求模型外推至训练数据覆盖范围之外的未来情景——包括2025-2050年的气候变化路径和多种减排策略组合。模型需在气象条件和排放强度均超出历史分布的条件下保持预测稳定性,这对泛化能力提出了极高要求。如何避免在长期模拟中的误差累积,同时保持对极端情景的物理合理响应,是突破模型泛化瓶颈的核心问题。

针对上述挑战,现有方法存在以下不足:(1)在计算效率方面,化学传输模式(如CMAQ(Community Multiscale Air Quality,社区多尺度空气质量模型)、WRF-Chem(Weather Research and Forecasting model coupled with Chemistry,气象-化学耦合模型))虽具有完备的物理化学方程组,但单次区域年尺度模拟需要消耗数百至数千核时的超算资源,难以支撑多情景、多路径的探索性模拟需求。(2)在排放响应建模方面,现有数据驱动方法多采用自回归模式,将历史浓度作为输入进行时序外推,无法响应\cqt{如果排放减少50\%结果如何}这类假设情景,缺乏对排放-浓度响应关系的显式表征。(3)在误差控制方面,自回归模式在长期模拟(如25年尺度)中面临严重的误差累积问题,预测偏差随时间步逐层放大。与第\ref{chap:prediction}章和第\ref{chap:inference}章不同,本章的核心挑战不在于如何建模污染物的时空传输过程,而在于如何建立排放源强与浓度场之间的直接响应映射,从而支撑长期情景模拟。

针对上述挑战,本章提出IGNN(Integrated Graph Neural Network,集成图神经网络)模型,将排放清单作为可控变量显式纳入深度学习框架,通过\cqt{一对一映射}策略消除自回归误差累积,构建计算高效且精度可靠的代理模型。


% ------------------------------------------------------------
% 5.2 问题定义
% ------------------------------------------------------------
\section{问题定义}
\label{sec:sim_problem}

本章聚焦的核心科学问题是:\textbf{未来数十年的空气质量将如何演变?}与第\ref{chap:prediction}章的时间预测问题和第\ref{chap:inference}章的空间推断问题不同,本章研究的是一类\textbf{情景响应问题(Scenario Response Problem)}——给定不同的减排政策路径,预测2025-2050年间\PM 与\ozone 的长期演变趋势,从而支撑\cqt{假设-验证}式的政策评估。图\ref{fig:simulation_problem}展示了该问题的整体框架。

\begin{figure}[htbp]
    \centering
    \includegraphics[width=\textwidth]{figures/chap05_simulation_problem.pdf}
    \caption{未来情景模拟问题示意图}
    \caption*{输入:未来气象场$\mathbf{M}^{\text{future}}_{t-T'+1:t}$(CMIP6)和排放场$\mathbf{E}^{\text{future}}_{t-T'+1:t}$(DPEC);输出:2025-2050年的\PM 和\ozone 浓度模拟$\hat{\mathbf{X}}^{\text{future}}_t$;模型:IGNN($\mathcal{F}_\Theta$)通过历史数据(ERA5气象、MEIC排放、CNEMC观测,2014-2020年)训练得到(图中上半部分,详见第\ref{subsec:train_infer_paradigm}节),学习排放-气象-浓度之间的响应关系后迁移至未来情景(图中下半部分)。}
    \label{fig:simulation_problem}
\end{figure}

本章的时间维度与前两章存在本质区别:第\ref{chap:prediction}章关注小时-日尺度的短期预报(起报后72小时内),第\ref{chap:inference}章关注当前时刻的空间推断,而本章关注年-十年尺度的长期趋势模拟(2025-2050年)。

形式化地,给定研究区域的图结构$\mathcal{G} = (\mathcal{V}, \mathcal{E})$(式中$\mathcal{V}$为$N$个城市节点,$\mathcal{E}$为空间邻接边),未来时刻$t$之前$T'$个时间步的气象场$\mathbf{M}^{\text{future}}_{t-T'+1:t}$和排放数据$\mathbf{E}^{\text{future}}_{t-T'+1:t}$,目标是利用预训练的映射函数$\mathcal{F}_{\Theta^*}$模拟该时刻的污染物浓度场:
\begin{equation}
\hat{\mathbf{X}}^{\text{future}}_{t} = \mathcal{F}_{\Theta^*}\left(\mathbf{M}^{\text{future}}_{t-T'+1:t}, \mathbf{E}^{\text{future}}_{t-T'+1:t}, \mathcal{G}\right)
\label{eq:forward_simulation}
\end{equation}
\noindent 式中,$\mathbf{M}^{\text{future}}_t \in \mathbb{R}^{N \times D_M}$表示第$t$时刻的未来气象变量(CMIP6,包括温度、风速、辐射等),$\mathbf{E}^{\text{future}}_t \in \mathbb{R}^{N \times D_E}$表示第$t$时刻的未来排放数据(DPEC,包括\NOx、VOC、SO$_2$、\PM 一次排放等),$\hat{\mathbf{X}}^{\text{future}}_t \in \mathbb{R}^{N \times 2}$表示模拟的\PM 和\ozone 浓度,$\Theta^*$为通过历史数据预训练得到的固定模型参数。

与第\ref{chap:prediction}章的自回归预测范式不同,\textbf{本章的模型输入不包含污染物浓度观测,仅依赖气象和排放驱动}。具体而言,模型将一个长度为$T'$的时间窗口(从$t-T'+1$到$t$时刻)内的气象场$\mathbf{M}^{\text{future}}$和排放数据$\mathbf{E}^{\text{future}}$,映射为该窗口末端$t$时刻的污染物浓度$\hat{\mathbf{X}}^{\text{future}}_t$。由于每个时刻的模拟仅依赖当前窗口内的气象和排放输入,而不依赖前一时刻的模拟结果,因此避免了长期积分中的误差累积问题,同时使模型能够直接响应任意假设的排放情景,支撑\cqt{如果减排50\%结果如何}这类假设情景分析。

如图\ref{fig:simulation_problem}所示,模型$\mathcal{F}_{\Theta^*}$通过历史数据(ERA5气象、MEIC排放、CNEMC观测,2014-2020年)预训练得到,学习排放-气象-浓度之间的响应关系(详见第\ref{subsec:train_infer_paradigm}节);在推理阶段,以未来气象$\mathbf{M}^{\text{future}}_{t-T'+1:t}$(CMIP6)和排放$\mathbf{E}^{\text{future}}_{t-T'+1:t}$(DPEC)为输入,模拟2025-2050年的空气质量$\hat{\mathbf{X}}^{\text{future}}_t$。


% ------------------------------------------------------------
% 5.3 研究区域与数据
% ------------------------------------------------------------
\section{研究区域与数据}
\label{sec:sim_data}

\subsection{研究区域}
\label{subsec:study_region}

本章选取京津冀及周边地区(简称\cqt{2+26}城市)作为研究区域,涵盖北京、天津及周边26个城市,共计28个主要城市。该区域自1980年代以来经历了快速城市化进程\citep{bai2014society},工业生产和能源消费量巨大,自2010年以来频繁遭受严重的大气污染事件,尤其是冬季\citep{an2019severe}。选择该区域的理由包括:(1)该区域是中国大气污染防治的重点区域,具有典型性和代表性;(2)监测网络密集,数据质量较高;(3)区域内城市间存在显著的污染物传输关系,适合图神经网络建模。

值得说明的是,本章的图网络构建粒度与前述章节存在显著差异。第\ref{chap:prediction}章(PCDCNet)以中东部184个城市或京津冀--长三角327个监测站点为节点构建大尺度预测网络;第\ref{chap:inference}章(SPIN)虽然也聚焦于京津冀及周边的\cqt{2+26}城市区域(图\ref{fig:study_area}(a)展示了该区域的城市分布),但其图节点为该经纬度范围内全部152个国控监测站点,用于站点级和网格级的空间推断。与之不同,本章以28个城市作为图节点,将各城市内多个站点的观测数据聚合为城市级均值,构建城市尺度的时空图网络。采用城市级粒度的原因在于:情景模拟任务的核心驱动变量——排放清单(MEIC/DPEC)——本身以城市或区县为统计单元编制,城市级建模能够与排放数据的空间分辨率自然对齐,确保排放变化信号被模型有效接收和响应。

图\ref{fig:bthsa_network}展示了研究区域内28个城市的空间分布及其网络连接结构。网络的构建基于城市间的空间邻近性,距离阈值设为200 km。

\begin{figure}[htbp]
  \centering
  \includegraphics[width=0.7\linewidth]{figures/chap05_city_network.png}
  \caption{京津冀及周边28城市的空间分布与网络结构}
  \caption*{圆点表示城市位置,连线表示基于空间邻近性构建的网络边。北京、天津、邢台、太原、淄博、郑州六个城市为后续重点分析的代表性城市。}
  \label{fig:bthsa_network}
\end{figure}


\subsection{历史数据}
\label{subsec:historical_data}

为训练和验证IGNN模型,本研究使用了2014-2020年的多源历史数据。

空气质量观测数据($\mathbf{X}$):目标变量$\mathbf{X} \in \mathbb{R}^{N \times 2}$来源于中国国家环境监测总站(CNEMC)\footnote{\url{http://www.cnemc.cn/}},包括28个城市103个监测站点的\PM($X_{\text{PM}_{2.5}}$)和\ozone($X_{\text{O}_3}$)逐小时浓度数据。数据经过质量控制后,聚合为3小时时间分辨率,与气象数据对齐。

气象再分析数据($\mathbf{M}$):气象驱动$\mathbf{M} \in \mathbb{R}^{N \times D_M}$($D_M=7$)采用欧洲中期天气预报中心(ECMWF)发布的ERA5再分析数据集\citep{hersbach2020era5}。已有研究证实,ERA5在华北地区具有较高的模拟精度\citep{jiang2021evaluation}。选取的气象变量包括:2米气温($M_{\text{t2m}}$)、100米风速与风向($M_{\text{spd}}$, $M_{\text{dir}}$)、降水($M_{\text{tp}}$)、地表向下短波辐射($M_{\text{rsds}}$)、2米相对湿度($M_{\text{rh}}$)和地表气压($M_{\text{sp}}$)。空间分辨率为0.25°×0.25°,时间分辨率为3小时。

排放清单数据($\mathbf{E}$):排放驱动$\mathbf{E} \in \mathbb{R}^{N \times D_E}$($D_E=5$)采用清华大学多尺度排放清单模型(MEIC)\citep{zheng2018trends}提供的历史排放数据。排放物种包括:PM$_{2.5}$($E_{\text{PM}_{2.5}}$)、PM$_{10}$($E_{\text{PM}_{10}}$)、SO$_2$($E_{\text{SO}_2}$)、\NOx($E_{\text{NO}_x}$)和VOC($E_{\text{VOC}}$)。原始月度排放数据通过ISAT时间降尺度模型\citep{wang2021measure}转换为3小时分辨率。

表\ref{tab:sim_data_summary}汇总了本研究使用的所有数据变量。

\begin{table}[htbp]
    \centering
    \caption{模型输入输出变量汇总}
    \caption*{涵盖污染物观测、排放清单和气象驱动三类数据,历史数据用于模型训练验证,未来数据用于情景模拟。}
    \label{tab:sim_data_summary}
    \begin{tabular}{@{}lllccc@{}}
        \toprule
        \textbf{数据类型} & \textbf{变量} & \textbf{符号} & \textbf{单位} & \textbf{时间分辨率} & \textbf{数据来源} \\
        \midrule
        \multirow{2}{*}{污染物($\mathbf{X}$)} & PM$_{2.5}$ & $X_{\text{PM}_{2.5}}$ & $\mu$g m$^{-3}$ & 3小时 & CNEMC \\
        & O$_3$ & $X_{\text{O}_3}$ & $\mu$g m$^{-3}$ & 3小时 & CNEMC \\
        \midrule
        \multirow{5}{*}{排放($\mathbf{E}$)} & PM$_{2.5}$ & $E_{\text{PM}_{2.5}}$ & ton & 3小时 & MEIC/DPEC \\
        & PM$_{10}$ & $E_{\text{PM}_{10}}$ & ton & 3小时 & MEIC/DPEC \\
        & SO$_2$ & $E_{\text{SO}_2}$ & ton & 3小时 & MEIC/DPEC \\
        & \NOx & $E_{\text{NO}_x}$ & ton & 3小时 & MEIC/DPEC \\
        & VOC & $E_{\text{VOC}}$ & ton & 3小时 & MEIC/DPEC \\
        \midrule
        \multirow{7}{*}{气象($\mathbf{M}$)} & 2米气温 & $M_{\text{t2m}}$ & K & 3小时 & ERA5/CMIP6 \\
        & 100米风向 & $M_{\text{dir}}$ & ° & 3小时 & ERA5/CMIP6 \\
        & 100米风速 & $M_{\text{spd}}$ & m s$^{-1}$ & 3小时 & ERA5/CMIP6 \\
        & 降水 & $M_{\text{tp}}$ & mm & 3小时 & ERA5/CMIP6 \\
        & 短波辐射 & $M_{\text{rsds}}$ & W m$^{-2}$ & 3小时 & ERA5/CMIP6 \\
        & 2米相对湿度 & $M_{\text{rh}}$ & \% & 3小时 & ERA5/CMIP6 \\
        & 地表气压 & $M_{\text{sp}}$ & Pa & 3小时 & ERA5/CMIP6 \\
        \bottomrule
    \end{tabular}
\end{table}


\subsection{未来情景数据}
\label{subsec:future_data}

为驱动IGNN模型进行2025-2050年的未来情景模拟,本研究构建了两类未来驱动数据。

未来气象数据:选取高碳排放情景(RCP8.5,Representative Concentration Pathway 8.5,即典型浓度路径8.5)作为未来气候变化的背景。具体采用CMIP6\citep{eyring2016overview}高分辨率模式比较项目中的CAS FGOALS-f3-L模型输出\citep{bao2020cas}。该模型是唯一提供3小时高时间分辨率输出的气候模式,空间分辨率为25 km,能够为IGNN模型提供与历史数据一致的气象驱动场。数据通过地球系统网格联盟(ESGF)节点获取\footnote{\url{https://esgf-node.llnl.gov/search/cmip6/}}。

未来排放数据:采用MEIC团队开发的未来排放动态预测模型(DPEC)数据\citep{tong2020dynamic,cheng2021pathways}\footnote{\url{http://meicmodel.org.cn/?page_id=1901&lang=en}}。DPEC的核心优势在于其创新的构建方法:将\cqt{自下而上}的精细化技术分析与\cqt{自上而下}的宏观情景驱动相结合,内部集成了源自MEIC清单的700多种中国本土排放源及其详细的技术演变过程,并与全球综合评估模型GCAM-China无缝对接。

DPEC设计了六种不同力度的污染控制策略,依据减排力度从弱到强排列如下:

\begin{enumerate}
    \item 基准情景(Baseline):延续当前政策,环境控制保持在2015年水平,不施加额外控制措施;
    \item 当前目标情景(Current Goals):假设中国实现国家自主贡献承诺和国家\PM 空气质量标准(35 $\mu$g m$^{-3}$)至2030年;
    \item NDC目标情景(NDC Goals):在当前目标情景的能源和社会经济发展路径基础上,到2050年在所有部门全面部署最佳可用末端控制技术;
    \item 2°C温控目标情景(2D Goals):与NDC目标情景采用相同的末端控制技术,但实施更严格的2°C一致性气候政策;
    \item 碳中和目标情景(Neutral Goals):以实现中国碳中和承诺和WHO \PM 指南(10 $\mu$g m$^{-3}$)为目标,追求2060年长期空气质量改善;
    \item 1.5°C温控目标情景(1.5D Goals):最严格的减排情景,实施1.5°C一致性气候政策。
\end{enumerate}

月度排放数据采用与第\ref{chap:prediction}章相同的时间分配方法降尺度至3小时分辨率(详见第\ref{subsec:emis_data}节),与气象数据共同作为IGNN模型的输入。


% ------------------------------------------------------------
% 5.4 IGNN模型架构与验证
% ------------------------------------------------------------
\section{IGNN模型架构与验证}
\label{sec:ignn_model}

\subsection{模型架构}
\label{subsec:ignn_arch}

IGNN模型遵循第\ref{chap:methodology}章图\ref{fig:unified_framework}所示的\cqt{编码$\rightarrow$隐空间动力学$\rightarrow$解码}统一框架,将研究区域内的城市监测站点抽象为一个图结构$\mathcal{G} = (\mathcal{V}, \mathcal{E}, \mathbf{A})$,式中$\mathcal{V}$为节点集合(28个城市),$\mathcal{E}$为边集合(基于空间邻近性定义),$\mathbf{A} \in \mathbb{R}^{N \times N}$为邻接矩阵。模型通过图卷积网络(GCN)捕捉污染物在城市间的空间输送与扩散过程,同时采用时间卷积网络(TCN)提取气象条件和排放源强度随时间变化的动态特征。

如图\ref{fig:ignn_arch}所示,IGNN模型采用一对一映射的输入输出范式:

\begin{equation}
\hat{\mathbf{X}}_{t} = \mathcal{F}_{\text{IGNN}}\left(\mathbf{M}_{t-T'+1:t}, \mathbf{E}_{t-T'+1:t}\right)
\label{eq:ignn_formulation}
\end{equation}

\noindent 式中$\mathbf{M}_{t-T'+1:t}$表示过去$T'$个时间步的气象信息(包含温度、风速、风向、降水、辐射、湿度、气压7个变量),$\mathbf{E}_{t-T'+1:t}$表示对应时段的排放数据(包含\PM、PM$_{10}$、SO$_2$、\NOx、VOC共5个物种);输出$\hat{\mathbf{X}}_t \in \mathbb{R}^{N \times 2}$为时刻$t$的\PM 和\ozone 浓度。该设计避免了将历史污染物浓度作为输入所带来的误差累积问题\citep{qi2019hybrid}。

\begin{figure}[htbp]
  \centering
  \includegraphics[width=\linewidth]{figures/chap05_ignn_arch.pdf}
  \caption{IGNN模型架构框架图}
  \caption*{模型以气象数据$\mathbf{M}$和排放数据$\mathbf{E}$的时间序列为输入,通过堆叠的时空建模模块(包含通道注意力、GCN空间聚合、时间卷积和残差连接)提取时空特征,最终经MLP输出\PM 和\ozone 浓度预测$\hat{\mathbf{X}}$。}
  \label{fig:ignn_arch}
\end{figure}

谱图卷积:模型通过拉普拉斯矩阵将输入$(M, E)$变换到傅里叶空间\citep{kipf2017semi}:

\begin{equation}
\mathcal{G}_{\theta} * \mathcal{G}(M, E) = U\mathcal{G}_{\theta}(\Lambda)U^{T}(M, E)
\label{eq:ignn_spectral_conv}
\end{equation}

\noindent 式中$L = I_{N} - D^{-1/2}AD^{-1/2} = U\Lambda U^{T}$为归一化图拉普拉斯矩阵。为降低计算复杂度,采用切比雪夫多项式近似\citep{kipf2017semi}:

\begin{equation}
\mathcal{G}_{\theta}(L) = \sum_{k=0}^{K_c-1}\theta_{k}T_{k}(\tilde{L})
\label{eq:chebyshev}
\end{equation}

\noindent 式中$\tilde{L} = \frac{2}{\lambda_{\max}}L - I_{N}$,$K_c$为切比雪夫多项式的阶数。通过这种方式,第$k$阶邻域的气象和排放信息被有效聚合。

时间卷积:采用膨胀卷积从气象和排放时间序列中提取时间特征。完整的前向传播过程可表述为:

\begin{equation}
\mathcal{F} = \text{ReLU}\left(\varphi * \left(\text{ReLU}\left(\mathcal{G}_{\theta}(L)(M, E)\right)\right)\right)
\label{eq:temporal_conv}
\end{equation}

\noindent 式中$\varphi$为卷积参数,$*$为卷积操作。


\subsection{训练与推理范式}
\label{subsec:train_infer_paradigm}

IGNN模型采用两阶段范式实现从历史学习到未来模拟的知识迁移。如图\ref{fig:simulation_problem}所示,该范式的核心是建立从驱动数据(气象$\mathbf{M}$、排放$\mathbf{E}$)到响应变量(污染物浓度$\mathbf{X}$)的映射关系$\mathcal{F}_\Theta$。训练阶段利用历史观测数据学习这一映射,推理阶段将学习到的映射应用于未来情景数据,生成不同减排策略下的空气质量模拟结果。

\textbf{训练阶段}($t \in \mathcal{T}_{\text{hist}}$)。训练阶段的目标是从历史数据中学习排放-气象-浓度之间的非线性响应关系。如图\ref{fig:simulation_problem}上半部分所示,模型的输入为长度$T'$的时间窗口内的历史气象驱动$\mathbf{M}^{\text{hist}}_{t-T'+1:t}$和排放驱动$\mathbf{E}^{\text{hist}}_{t-T'+1:t}$(数据来源与变量定义详见第\ref{subsec:historical_data}节)。

模型的监督信号$\mathbf{X}^{\text{hist}}_{t}$来自CNEMC国家监测站点的实测浓度数据。训练过程中,时间索引$t$遍历整个历史时段$\mathcal{T}_{\text{hist}}$(2014-2020年的每一个3小时时间步)。对于每个$t$,模型接收其前$T'$个时间步的气象和排放数据$(\mathbf{M}^{\text{hist}}_{t-T'+1:t}, \mathbf{E}^{\text{hist}}_{t-T'+1:t})$作为输入,经由图神经网络$\mathcal{F}_\Theta$处理后,输出该时刻的\PM 和\ozone 浓度预测$\hat{\mathbf{X}}^{\text{hist}}_t$,并与真实观测$\mathbf{X}^{\text{hist}}_t$计算损失进行梯度更新。

训练阶段的核心约束是:
\begin{equation}
\min_{\Theta} \sum_{t \in \mathcal{T}_{\text{hist}}} \mathcal{L}\left(\hat{\mathbf{X}}^{\text{hist}}_t, \mathbf{X}^{\text{hist}}_t\right) = \min_{\Theta} \sum_{t \in \mathcal{T}_{\text{hist}}} \mathcal{L}\left(\mathcal{F}_\Theta\left(\mathbf{M}^{\text{hist}}_{t-T'+1:t}, \mathbf{E}^{\text{hist}}_{t-T'+1:t}, \mathcal{G}\right), \mathbf{X}^{\text{hist}}_t\right)
\label{eq:training_objective}
\end{equation}
\noindent 式中$\mathcal{T}_{\text{hist}}$为历史训练时段(2014-2020年),$\mathcal{L}$为均方误差损失函数,$\mathcal{G}$为图\ref{fig:simulation_problem}所示的城市网络结构。与第\ref{chap:prediction}章和第\ref{chap:inference}章不同,IGNN采用纯监督损失而未引入显式物理约束项,原因在于:(1)长期模拟的关键挑战是误差累积而非单步物理一致性,而非自回归架构已从根本上解决了这一问题;(2)模型通过大量历史数据隐式学习了公式\eqref{eq:advection_diffusion}所描述的物理响应关系;(3)城市级别的空间聚合平滑了局部物理约束的必要性。通过这种窗口到点的映射方式,模型学习到了在给定气象条件和排放强度下,大气系统如何响应并产生特定的污染物浓度。

\textbf{推理阶段}($t \in \mathcal{T}_{\text{future}}$)。推理阶段的目标是将训练好的模型迁移至未来情景,生成2025-2050年的空气质量模拟结果。如图\ref{fig:simulation_problem}下半部分所示,模型结构$\mathcal{F}$和参数$\Theta^*$保持固定(图中雪花图标表示参数冻结),仅将输入数据替换为未来气象驱动$\mathbf{M}^{\text{future}}_{t-T'+1:t}$和排放驱动$\mathbf{E}^{\text{future}}_{t-T'+1:t}$(数据来源与情景设置详见第\ref{subsec:future_data}节)。

推理过程中,时间索引$t$遍历整个未来时段$\mathcal{T}_{\text{future}}$(2025-2050年的每一个3小时时间步)。对于每个$t$,模型独立计算该时刻的浓度输出$\hat{\mathbf{X}}^{\text{future}}_t$:
\begin{equation}
\hat{\mathbf{X}}^{\text{future}}_t = \mathcal{F}_{\Theta^*}\left(\mathbf{M}^{\text{future}}_{t-T'+1:t}, \mathbf{E}^{\text{future}}_{t-T'+1:t}, \mathcal{G}\right), \quad t \in \mathcal{T}_{\text{future}}
\label{eq:inference}
\end{equation}
\noindent 式中$\Theta^*$为训练后的固定参数,$\mathcal{T}_{\text{future}}$为推理时段(2025-2050年)。

这种训练-推理范式具有以下关键优势:

(1)消除误差累积。如图\ref{fig:simulation_problem}所示,IGNN的每个时刻$t$的预测仅依赖当前窗口内的输入($\mathbf{M}^{\text{hist/future}}_{t-T'+1:t}$和$\mathbf{E}^{\text{hist/future}}_{t-T'+1:t}$),而不依赖前一时刻的预测结果$\hat{\mathbf{X}}_{t-1}$。这意味着即使某一时刻的预测存在偏差,该偏差不会传递到后续时刻。在25年尺度的长期模拟中,这一特性至关重要——自回归模型的预测误差会随时间步指数级放大,而IGNN的误差始终保持在单步水平。

(2)支撑假设情景分析。由于模型输入不包含历史浓度,仅依赖气象和排放驱动,因此可以直接响应\cqt{如果排放减少50\%结果如何}这类假设性问题。只需将$\mathbf{E}^{\text{future}}$替换为假设的排放情景,即可获得对应的浓度响应,这为政策评估提供了\cqt{数字沙盘}工具。

(3)实现跨域迁移。训练阶段学习的排放-气象-浓度响应关系具有物理普适性(公式\ref{eq:advection_diffusion}所描述的平流-扩散-化学反应机制),因此可以迁移至未来情景。尽管未来的气象条件$\mathbf{M}^{\text{future}}$(CMIP6)和排放强度$\mathbf{E}^{\text{future}}$(DPEC)与历史数据$\mathbf{M}^{\text{hist}}$(ERA5)/$\mathbf{E}^{\text{hist}}$(MEIC)存在域偏移,但模型捕捉到的基本物理关系仍然适用。图\ref{fig:simulation_problem}中从训练到推理的迁移(Transfer箭头)正是基于这一物理普适性假设。


\subsection{模型训练与验证}
\label{subsec:ignn_validation}

IGNN模型使用2014-2020年的历史数据进行训练与验证,数据按时间顺序划分为训练集(2014-2018年)、验证集(2019年)和测试集(2020年)。模型基于PyTorch实现。主要超参数设置:时间窗口$T'=24$(72小时),切比雪夫多项式阶数$K=3$,隐藏维度$d=32$,学习率$10^{-4}$,批大小32,优化器使用Adam。训练在NVIDIA Tesla K80 GPU上进行,共训练90轮,总耗时约3小时。训练后的模型文件大小仅119 KB,便于部署与快速推理。

将IGNN与以下方法进行了对比:
\begin{itemize}
    \item 梯度提升模型:XGBoost\citep{chen2016xgboost}、LightGBM\citep{ke2017lightgbm},代表传统特征工程方法;
    \item 时空图神经网络模型:STGCN\citep{yu2018spatio},代表深度学习时空建模方法。
\end{itemize}

表\ref{tab:ignn_vs_ml}展示了IGNN与上述方法的性能对比。采用以下指标评估模型性能\citep{evaluation}:

\begin{itemize}
    \item MAE(平均绝对误差):$\text{MAE} = \frac{1}{n}\sum_{i=1}^{n}|\hat{y}_i - y_i|$,反映平均模拟偏差;
    \item IA(一致性指数):$\text{IA} = 1 - \frac{\sum_{i=1}^{n}(\hat{y}_i - y_i)^2}{\sum_{i=1}^{n}(|\hat{y}_i - \bar{y}| + |y_i - \bar{y}|)^2}$,综合反映模拟值与观测值的一致程度;
    \item $r$(相关系数):反映模拟值与观测值的线性相关程度;
    \item NMB(归一化平均偏差):$\text{NMB} = \frac{\sum_{i=1}^{n}(\hat{y}_i - y_i)}{\sum_{i=1}^{n} y_i} \times 100\%$,反映系统性高估或低估;
    \item NME(归一化平均误差):$\text{NME} = \frac{\sum_{i=1}^{n}|\hat{y}_i - y_i|}{\sum_{i=1}^{n} y_i} \times 100\%$,反映整体误差水平。
\end{itemize}

\begin{table}[htbp]
    \centering
    \caption{IGNN与机器学习方法在2014-2020年模拟性能对比}
    \caption*{MAE单位为$\mu$g m$^{-3}$,IA和$r$为无量纲,NMB和NME单位为\%。最优结果已加粗,'*'表示$r$在$p < 0.05$水平显著。}
    \label{tab:ignn_vs_ml}
    \begin{tabular}{@{}lccccc|ccccc@{}}
        \toprule
        & \multicolumn{5}{c}{\textbf{PM$_{2.5}$}} & \multicolumn{5}{c}{\textbf{O$_3$}} \\
        \cmidrule(lr){2-6} \cmidrule(lr){7-11}
        \textbf{模型} & \textbf{MAE} & \textbf{IA} & \textbf{$r$} & \textbf{NMB} & \textbf{NME} & \textbf{MAE} & \textbf{IA} & \textbf{$r$} & \textbf{NMB} & \textbf{NME} \\
        \midrule
        XGBoost   & 48.79 & 0.69 & 0.42* & 56.71 & 65.20 & 27.59 & 0.89 & 0.86* & -20.65 & 32.21 \\
        LightGBM  & 48.75 & 0.67 & 0.36* & 45.09 & 71.34 & 29.13 & 0.86 & 0.88* & -20.21 & 32.77 \\
        STGCN     & 35.59 & 0.74 & 0.65* & 25.12 & 55.39 & 21.63 & 0.91 & 0.86* & -17.54 & 28.77 \\
        \textbf{IGNN} & \textbf{29.64} & \textbf{0.79} & \textbf{0.71*} & \textbf{13.42} & \textbf{45.26} & \textbf{19.54} & \textbf{0.93} & \textbf{0.88*} & \textbf{-10.70} & \textbf{28.17} \\
        \bottomrule
    \end{tabular}
\end{table}

从表中可以看出,IGNN在几乎所有指标上取得最优或并列最优性能。对于\PM,IGNN的IA达到0.79,MAE为29.64 $\mu$g m$^{-3}$,显著优于XGBoost(IA=0.69, MAE=48.79)和STGCN(IA=0.74, MAE=35.59)。对于\ozone,IGNN同样表现最优(IA=0.93, MAE=19.54)。值得注意的是,所有方法对\ozone 的模拟性能均优于\PM,这可能与\ozone 浓度的日变化规律更为规则有关。

图\ref{fig:ignn_scatter}展示了2019年北京站点IGNN模拟值与观测值的散点对比。

\begin{figure}[htbp]
  \centering
  \includegraphics[width=\linewidth]{figures/chap05_ignn_scatter.pdf}
  \caption{IGNN模拟值与观测值散点图}
  \caption*{(a)\PM($R^2=0.42$,MAE$=29.2\ \mu$g/m$^3$)和(b)\ozone($R^2=0.78$,MAE$=18.5\ \mu$g/m$^3$)的浓度对比。颜色表示数据点密度,红色实线为线性回归拟合线,黑色虚线为1:1参考线。}
  \label{fig:ignn_scatter}
\end{figure}


\subsection{与物理化学模型的对比}
\label{subsec:ctm_comparison}

为进一步验证IGNN模型的可靠性,将其与传统物理化学模型(CMAQ V5.3和WRF-Chem v3.9.1,均采用27-9-3km嵌套网格配置)进行了对比。选取2019年1月的\PM 重污染事件和2019年7月的\ozone 重污染事件作为典型案例,图\ref{fig:ignn_vs_ctm}和表\ref{tab:ignn_vs_ctm_metrics}分别展示了模拟对比结果和性能指标。

\begin{figure}[htbp]
  \centering
  \includegraphics[width=\linewidth]{figures/chap05_ignn_vs_ctm.pdf}
  \caption{IGNN与物理化学模型对典型污染过程的模拟对比}
  \caption*{(a)2019年1月北京\PM 重污染事件;(b)2019年7月北京\ozone 重污染事件。IGNN在捕捉浓度日变化和峰值特征方面表现优于CMAQ和WRF-Chem。}
  \label{fig:ignn_vs_ctm}
\end{figure}

\begin{table}[htbp]
    \centering
    \caption{IGNN与物理化学模型性能对比}
    \caption*{基于2019年1月\PM 事件和2019年7月\ozone 事件的模拟结果。}
    \label{tab:ignn_vs_ctm_metrics}
    \begin{tabular}{@{}lccc|ccc@{}}
        \toprule
        & \multicolumn{3}{c}{\textbf{PM$_{2.5}$}} & \multicolumn{3}{c}{\textbf{O$_3$}} \\
        \cmidrule(lr){2-4} \cmidrule(lr){5-7}
        \textbf{模型} & \textbf{$r$} & \textbf{IA} & \textbf{MAE} & \textbf{$r$} & \textbf{IA} & \textbf{MAE} \\
        \midrule
        CMAQ     & 0.38 & 0.57 & 29.19 & 0.57 & 0.65 & 39.50 \\
        WRF-Chem & 0.36 & 0.55 & 30.82 & 0.47 & 0.60 & 51.18 \\
        \textbf{IGNN} & \textbf{0.84} & \textbf{0.91} & \textbf{15.44} & \textbf{0.90} & \textbf{0.94} & \textbf{12.36} \\
        \bottomrule
    \end{tabular}
\end{table}

结果表明:(1)IGNN在\PM 和\ozone 模拟上均优于CMAQ和WRF-Chem(PM$_{2.5}$:$r$=0.84 vs 0.38;O$_3$:$r$=0.90 vs 0.57);(2)物理化学模型在高浓度时段存在较大偏差,这与排放清单的不确定性、边界条件设定及参数化方案的局限性有关;(3)IGNN作为数据驱动方法,能够直接从观测数据中学习复杂的非线性响应关系,在典型污染事件的模拟中表现更优。


% ------------------------------------------------------------
% 5.4 未来情景模拟结果与分析
% ------------------------------------------------------------
\section{模拟结果与分析}
\label{sec:sim_results}

\subsection{浓度演变趋势}
\label{subsec:regional_trends}

将2025-2050年的未来气象与多策略排放数据输入已训练好的IGNN模型,得到了\cqt{2+26}城市未来空气质量的整体演变趋势。

模拟结果(如图\ref{fig:future_average}和表\ref{tab:future_trends_bthsa}所示)揭示了一个核心且值得高度关注的现象:\textbf{\PM 与\ozone 的浓度变化呈现出显著的相反趋势}。在所有六种排放策略下,\cqt{2+26}城市的年平均\PM 浓度均表现出显著的下降趋势,年均变化(AAC,Annual Average Change)在$-0.14$至$-0.37$ $\mu$g m$^{-3}$之间($p < 0.05$)。这表明,即使在全球持续高碳排放的背景下,只要实施既定的或更强化的污染物控制政策,\PM 治理仍能取得积极成效。

然而,与\PM 的下降形成鲜明对比的是,年平均\ozone 浓度在除基准情景外的所有策略下都呈现出显著的上升趋势,AAC在$+0.07$至$+0.22$ $\mu$g m$^{-3}$之间($p < 0.05$)。这一结果预示着,未来\ozone 污染可能会取代\PM,成为该区域面临的主要大气环境挑战。

\begin{figure}[htbp]
  \centering
  \includegraphics[width=\linewidth]{figures/chap05_future_average.pdf}
  \caption{不同排放策略下\cqt{2+26}城市未来\PM 与\ozone 浓度变化趋势}
  \caption*{(a)\PM 浓度在所有策略下均呈下降趋势;(b)\ozone 浓度除基准情景外均呈上升趋势。实线表示年均浓度变化。六种策略包括:Current Goals(当前目标)、2D Goals(2度目标)、Baseline(基准情景)、1.5D Goals(1.5度目标)、NDC Goals(NDC目标)和Neutral Goals(碳中和目标)。}
  \label{fig:future_average}
\end{figure}

\begin{table}[htbp]
    \centering
    \caption{2025-2050年\cqt{2+26}城市模拟污染物浓度变化的线性拟合参数}
    \caption*{斜率单位为$\mu$g m$^{-3}$ (10 year)$^{-1}$,AAC(年均变化)单位为$\mu$g m$^{-3}$ year$^{-1}$,截距单位为$\mu$g m$^{-3}$。斜率在$p < 0.05$水平显著的已加粗。六种策略按减排力度从弱到强排列。}
    \label{tab:future_trends_bthsa}
    \begin{tabular}{@{}lcccccc@{}}
        \toprule
        & \multicolumn{3}{c}{\textbf{PM$_{2.5}$}} & \multicolumn{3}{c}{\textbf{O$_3$}} \\
        \cmidrule(lr){2-4} \cmidrule(lr){5-7}
        \textbf{排放策略} & \textbf{斜率} & \textbf{截距} & \textbf{AAC} & \textbf{斜率} & \textbf{截距} & \textbf{AAC} \\
        \midrule
        Baseline(基准) & \textbf{$-1.44$} & 53.87 & $-0.14$ & \textbf{$-0.49$} & 61.96 & $-0.05$ \\
        Current Goals(当前目标) & \textbf{$-2.98$} & 38.90 & $-0.29$ & \textbf{0.69} & 67.88 & 0.07 \\
        NDC Goals(NDC目标) & \textbf{$-2.89$} & 37.92 & $-0.29$ & \textbf{1.04} & 68.39 & 0.10 \\
        2D Goals(2°C目标) & \textbf{$-3.54$} & 32.65 & $-0.35$ & \textbf{1.86} & 71.34 & 0.19 \\
        Neutral Goals(碳中和) & \textbf{$-3.67$} & 31.83 & $-0.37$ & \textbf{2.07} & 71.81 & 0.21 \\
        1.5D Goals(1.5°C目标) & \textbf{$-3.41$} & 29.81 & $-0.34$ & \textbf{2.22} & 72.43 & 0.22 \\
        \bottomrule
    \end{tabular}
\end{table}

这一看似矛盾的结果,实际上揭示了大气化学过程中的\textbf{\cqt{气候惩罚}(Climate Penalty)}效应。全球变暖本身将直接导致空气污染问题(尤其是\ozone 污染)的恶化。这一效应主要通过以下两个途径实现:

(1)加速化学反应:\ozone 生成的光化学反应速率对温度高度敏感,在其他条件不变的情况下,温度越高,反应速率越快,\ozone 的生成量就越大\citep{cao2020future}。

(2)不利气象条件:气候变化预计将导致极端天气事件的频率和强度增加,例如更频繁、更持久的热浪和大气静稳事件\citep{cai2017weather},而这些天气条件恰恰是导致污染物累积和光化学烟雾爆发的最主要气象诱因。

此外,\PM 浓度的下降通过减少对太阳辐射的散射,增强了到达地表的短波辐射强度,进一步加速了\ozone 的光化学生成\citep{li2019two}。


\subsection{城市差异性分析}
\label{subsec:city_diff}

为探究区域内部的差异性,进一步分析了\cqt{当前目标情景}下六个代表性城市(北京、天津、邢台、太原、淄博、郑州)的模拟结果。

图\ref{fig:future_city}和表\ref{tab:future_trends_cities}展示了六城市的污染物浓度变化趋势。

\begin{figure}[htbp]
  \centering
  \includegraphics[width=\linewidth]{figures/chap05_future_city.pdf}
  \caption{当前目标策略下六城市未来\PM 与\ozone 浓度变化趋势}
  \caption*{(a)六城市\PM 浓度均呈下降趋势,太原降幅最大;(b)\ozone 浓度在多数城市呈上升趋势。六城市包括北京、天津、邢台、太原、淄博和郑州。}
  \label{fig:future_city}
\end{figure}

\begin{table}[htbp]
    \centering
    \caption{2025-2050年六城市模拟污染物浓度变化的线性拟合参数}
    \caption*{斜率和年均变化(AAC)单位为$\mu$g m$^{-3}$ year$^{-1}$,截距单位为$\mu$g m$^{-3}$。斜率在$p < 0.05$水平显著的已加粗。}
    \label{tab:future_trends_cities}
    \begin{tabular}{@{}lcccccc@{}}
        \toprule
        & \multicolumn{3}{c}{\textbf{PM$_{2.5}$}} & \multicolumn{3}{c}{\textbf{O$_3$}} \\
        \cmidrule(lr){2-4} \cmidrule(lr){5-7}
        \textbf{城市} & \textbf{斜率} & \textbf{截距} & \textbf{AAC} & \textbf{斜率} & \textbf{截距} & \textbf{AAC} \\
        \midrule
        北京   & \textbf{$-0.10$} & 30.72 & $-0.09$ & \textbf{$-0.14$} & 76.69 & $-0.14$ \\
        天津   & \textbf{$-0.49$} & 47.44 & $-0.47$ & \textbf{0.13} & 58.34 & 0.12 \\
        邢台   & \textbf{$-0.72$} & 42.22 & $-0.69$ & \textbf{0.20} & 71.81 & 0.19 \\
        太原   & \textbf{$-1.39$} & 88.00 & $-1.33$ & \textbf{$-0.10$} & 83.50 & $-0.09$ \\
        淄博   & \textbf{$-0.63$} & 37.19 & $-0.60$ & \textbf{0.22} & 74.56 & 0.21 \\
        郑州   & \textbf{$-0.56$} & 32.50 & $-0.54$ & \textbf{0.16} & 67.31 & 0.16 \\
        \bottomrule
    \end{tabular}
\end{table}

结果显示,\PM 的下降趋势在各城市间存在显著差异。大部分城市的下降速率较为平缓($-0.09$至$-0.69$ $\mu$g m$^{-3}$ year$^{-1}$),而太原的下降趋势最为显著($-1.33$ $\mu$g m$^{-3}$ year$^{-1}$),同时其\PM 和\ozone 的模拟浓度也最高。这是因为太原拥有大量的煤炭工业,排放控制策略对煤炭使用的大幅削减导致了当地\PM 水平的快速下降\citep{cheng2021pathways}。

在\ozone 变化方面,值得注意的是,北京和太原的\ozone 浓度呈现微弱的下降趋势,这与区域总体上升趋势形成对比,反映了不同城市间光化学环境的差异性。


\subsection{重污染日频率演变规律}
\label{subsec:heavy_days}

除了平均浓度的变化,污染事件的发生频率是衡量空气质量的另一重要维度。基于历史数据(2014-2020年),定义日最大污染物浓度超过第75百分位数的日期为\cqt{重污染日},阈值分别为\PM $>$ 88 $\mu$g m$^{-3}$和\ozone $>$ 92 $\mu$g m$^{-3}$。图\ref{fig:future_heavy_days}和表\ref{tab:future_heavy_days}展示了未来重污染日频率的模拟结果。

\begin{figure}[htbp]
  \centering
  \includegraphics[width=\linewidth]{figures/chap05_future_heavy_days.pdf}
  \caption{不同排放策略下未来重污染天数变化预测}
  \caption*{(a)\PM 重污染天数在所有策略下持续下降,在部分严格策略下可被完全消除;(b)\ozone 重污染天数在多数策略下呈增加趋势。}
  \label{fig:future_heavy_days}
\end{figure}

\begin{table}[htbp]
    \centering
    \caption{2025-2050年\cqt{2+26}城市重污染日变化的线性拟合参数}
    \caption*{AAC(年均变化)单位为天/年。六种策略按减排力度从弱到强排列。}
    \label{tab:future_heavy_days}
    \begin{tabular}{@{}lc|c@{}}
        \toprule
        \textbf{排放策略} & \textbf{PM$_{2.5}$重污染日 AAC} & \textbf{O$_3$重污染日 AAC} \\
        \midrule
        Baseline(基准) & $-0.54$ & $-0.24$ \\
        Current Goals(当前目标) & $-0.66$ & $0.12$ \\
        NDC Goals(NDC目标) & $-0.62$ & $0.04$ \\
        2D Goals(2°C目标) & $-0.58$ & $0.76$ \\
        Neutral Goals(碳中和) & $-0.48$ & $0.86$ \\
        1.5D Goals(1.5°C目标) & $-0.32$ & $0.88$ \\
        \bottomrule
    \end{tabular}
\end{table}

结果与平均浓度的变化趋势高度一致。在大多数策略下,\PM 重污染日的年发生天数呈现出显著的下降趋势。尤为鼓舞的是,在三种最严格的减排策略下(1.5°C目标、2°C目标和碳中和目标),\PM 重污染事件预计将在2030年前后被基本消除。

与此同时,\ozone 重污染日的年发生天数则普遍表现为增加趋势($+0.04$至$+0.88$天/年),这再次印证了未来\ozone 污染风险的严峻性。


% ------------------------------------------------------------
% 5.6 本章小结
% ------------------------------------------------------------
\section{本章小结}
\label{sec:sim_summary}

本章依托综合图神经网络(IGNN)模型,系统开展了未来污染情景模拟研究。主要贡献与发现如下:

(1)\textbf{构建了高效的未来空气质量代理模型。}IGNN模型采用\cqt{一对一映射}策略避免误差累积,在模拟历史\PM 与\ozone 浓度变化方面显著优于XGBoost、LightGBM和STGCN等方法(\PM:IA从0.67--0.74提升至0.79;\ozone:IA从0.86--0.91提升至0.93),且在典型污染事件模拟中优于CMAQ和WRF-Chem物理化学模型,计算效率大幅提升,为大规模多情景探索提供了可能。

(2)\textbf{揭示了\PM 与\ozone 的反向演变趋势与协同治理需求。}在所有排放策略下,\cqt{2+26}城市未来\PM 浓度将持续下降(年均$-0.14$至$-0.37$ $\mu$g m$^{-3}$),而\ozone 浓度普遍上升(年均$+0.07$至$+0.22$ $\mu$g m$^{-3}$)。这一\cqt{气候惩罚}效应表明,由于\PM 和\ozone 共享\NOx 等前驱物且生成机制相互影响,单一污染物控制策略可能引发负面效果,必须根据区域化学敏感性科学确定协同减排比例。该发现也与国家\cqt{双碳}目标高度契合——能源结构转型在减少CO$_2$排放的同时将天然带来空气质量改善的协同效益。

(3)\textbf{量化了不同城市的差异化响应。}北京和太原的\ozone 浓度呈微弱下降趋势,可能与其处于VOC控制区有关,为因地制宜制定减排策略提供了科学依据。在三种最严格的减排策略下,\PM 重污染事件预计将在2030年前后被基本消除。

本章研究仍存在局限:所选变量可能未涵盖所有关键大气过程,IGNN作为数据驱动方法在捕捉物理机制方面存在不足。未来可从构建物理启发混合模型、推广至其他城市群验证可迁移性、开发高分辨率预报接口等方向持续改进。  % 未来污染情景模拟
% ============================================================
% 第六章 系统部署与落地应用
% 基于复杂系统数据驱动建模的大气污染研究
% ============================================================

\chapter{系统部署与落地应用}
\label{chap:deployment}

前述章节从预测、推断和模拟三个方面系统构建了物理约束的时空图神经网络方法体系,本章的核心任务在于将这些研究成果从实验室原型转化为能够经受真实世界考验的业务化系统,实现从\cqt{科学建模}到\cqt{工程服务}的跨越。本章将详细阐述系统架构设计、业务化验证过程以及商业化落地应用,展示物理约束深度学习方法在实际环境管理中的应用价值。

% ------------------------------------------------------------
% 6.1 引言
% ------------------------------------------------------------
\section{引言}
\label{sec:deploy_intro}

尽管深度学习模型在离线测试中展现出优异的预测性能,但从研究原型到实际业务系统的转化仍面临诸多挑战。传统数值模式系统(如\CMAQ+WRF)虽然具备物理可解释性,但存在更新频率低、计算资源需求高、运维成本大等瓶颈,难以满足日益增长的实时空气质量服务需求。如何构建一套低成本、高效率、可扩展的智能预报系统,实现模型的自动化训练、部署与持续迭代,已成为将AI技术真正服务于环境管理的关键工程问题。

本章面临的核心挑战涵盖三个方面。其一为\textbf{系统工程化与实时服务化},即如何将复杂的时空图神经网络模型从研究原型转化为可持续运行的业务系统,实现稳定、高频、可维护的预测服务。其二为\textbf{数据实时性与多源同化},即如何融合实时气象预报、空气质量观测与排放估计等多源数据,达成小时级更新与快速响应。其三为\textbf{可扩展性与智能化运维},即如何构建具备自动更新、免维护、云端调度能力的智能系统架构,降低传统数值模式系统的高运维成本。


% ------------------------------------------------------------
% 6.2 系统总体架构与自动化流程
% ------------------------------------------------------------
\section{系统架构设计}
\label{sec:system_arch}

为实现系统的可扩展性与高可用性,本研究采用基于云全托管服务的微服务架构。该架构将复杂的大气污染预报任务拆分为多个独立、解耦的服务,并通过容器化技术部署在Kubernetes集群中,实现了\cqt{零运维}和按需付费的成本效益。


\subsection{微服务架构设计}
\label{subsec:microservice}

本系统的整体设计遵循关注点分离(Separation of Concerns)的核心原则,将系统的核心组件解耦为五大模块:数据接入服务、模型训练流水线(离线)、模型推理服务(在线)、API网关以及配套的数据库和可观测性体系。这种微服务化的架构设计确保了数据流、训练流和推理流的高效协同,并允许各组件独立扩展和维护。

\begin{figure}[htbp]
  \centering
  \includegraphics[width=\linewidth]{figures/chap06_system_arch.png}
  \caption{KnowAir系统全链路架构图}
  \caption*{该架构包含两条核心数据流:数据采集链路(左侧,负责多源数据的周期性采集与预处理)和API服务链路(右侧,处理用户实时请求)。系统部署于云端Kubernetes集群,通过WAF、SLB和API网关实现安全访问控制与负载均衡。}
  \label{fig:system_architecture}
\end{figure}

如图\ref{fig:system_architecture}所示,该架构包含两条异步且独立的核心数据流。\textbf{数据采集链路}是一条持续运行的后台数据流水线。部署在容器集群中的\cqt{数据接入服务}作为一个独立的微服务,会周期性地从各类外部数据源——涵盖全球天气预报系统(GFS、ERA5)、排放清单数据(MEIC、DPEC)以及国家空气质量监测网的实时观测数据——主动拉取原始数据。这些异构数据经过标准化的清洗和预处理后,被分别存入关系型数据库用于结构化存储,以及对象存储服务用于非结构化数据归档。该链路的设计确保了核心系统与外部数据源的解耦,为模型训练和推理提供了稳定、可靠的数据基础。

\textbf{API服务链路}处理所有来自用户的实时数据请求。当C端/B端用户或环境监管部门等客户端发起数据请求时,请求首先经过网络与网关层,该层由Web应用防火墙(WAF)、服务器负载均衡(SLB)和API网关组成,构成系统安全与流量管理的第一道防线,负责流量清洗、DDoS防护、SSL卸载以及统一的API入口管理。随后,合法请求被路由至核心业务系统中的\cqt{模型推理服务},该服务从高速缓存和数据库中加载所需的实时和历史特征数据,调用部署好的KnowAir模型进行即时计算,生成的预报结果沿逆向路径返回客户端。

\begin{table}[htbp]
    \centering
    \caption{KnowAir系统核心组件与技术选型}
    \caption*{系统采用四层架构设计,分别涵盖数据存储、计算推理、网关安全和监控运维等核心功能,各层组件通过标准化接口实现松耦合集成。}
    \label{tab:system_components}
    \begin{tabular}{@{}llll@{}}
        \toprule
        \textbf{组件类别} & \textbf{组件名称} & \textbf{技术选型} & \textbf{主要功能} \\
        \midrule
        数据层 & 结构化存储 & RDS & 气象、观测数据存储 \\
               & 对象存储 & OSS & 模型文件、原始数据归档 \\
               & 缓存服务 & Redis & 热点数据加速访问 \\
        \midrule
        计算层 & 容器编排 & Kubernetes & 服务部署与弹性伸缩 \\
               & 模型推理 & PyTorch Serving & 在线预测服务 \\
               & 定时任务 & CronJob & 周期性训练触发 \\
        \midrule
        网关层 & 负载均衡 & SLB & 流量分发与高可用 \\
               & API网关 & API Gateway & 认证、限流、路由 \\
               & 安全防护 & WAF & DDoS防护、流量清洗 \\
        \midrule
        监控层 & 日志服务 & SLS & 日志采集与分析 \\
               & 性能监控 & Prometheus + Grafana & 指标监控与告警 \\
        \bottomrule
    \end{tabular}
\end{table}

表\ref{tab:system_components}总结了系统各核心组件的功能与技术选型。整个系统采用业界成熟的云原生技术栈,确保了高可用性、可扩展性和运维便捷性。


\subsection{MLOps自动化训练流水线}
\label{subsec:mlops}

为实现模型从开发到生产的快速、可靠流转,本研究借鉴并实施了MLOps(Machine Learning Operations)的核心思想,构建了一套覆盖模型全生命周期的自动化工作流。

\begin{figure}[htbp]
  \centering
  \includegraphics[width=\linewidth]{figures/chap06_train_deploy.png}
  \caption{模型离线训练与自动化部署流水线}
  \caption*{上半部分展示自动化训练阶段:CronJob定时触发训练任务,容器实例拉取最新数据执行PyTorch训练,达标模型被版本化存入模型仓库。下半部分展示自动化部署阶段:GitOps流水线同步配置变更,ArgoCD检测到更新后自动执行滚动部署至生产环境。}
  \label{fig:mlops_pipeline}
\end{figure}

\subsubsection{自动化模型训练与版本化}

为确保模型能够持续从最新数据中学习,以对抗模型漂移(Model Drift)并适应不断变化的大气环境,本研究设计并实现了一套自动化的离线训练流水线,如图\ref{fig:mlops_pipeline}上半部分所示。

该流水线由部署在容器集群中的调度层(Scheduler)发起,其核心是一个Kubernetes CronJob定时任务控制器,根据预设时间表定时触发训练任务。任务触发后,训练执行层(Executor)立即启动专用训练Pod,该Pod首先连接到数据存储层,从数据库和对象存储中拉取涵盖最新周期的全量训练数据。数据准备就绪后,Pod执行核心的PyTorch模型训练过程,可利用GPU资源加速以缩短训练时间。

训练完成后,系统对新生成的模型进行严格的评估与验证,通过在预留验证集上计算关键性能指标(RMSE、MAE等)判断新模型性能是否优于当前生产环境模型。若性能达标,新模型被视为不可变(immutable)模型制品,赋予唯一版本号,安全存入作为模型仓库的对象存储服务;若性能不达标,则本次训练流程结束,生产环境模型保持不变。

\subsubsection{基于GitOps的声明式持续部署}

为实现高效、安全且可靠的软件交付,本系统采用业界先进的GitOps作为核心部署方法论,如图\ref{fig:mlops_pipeline}下半部分所示。GitOps的核心思想是将Git仓库作为系统期望状态的唯一真实来源(Single Source of Truth),并采用代码与配置分离的最佳实践。

整个流程始于开发者:当开发者将新的业务代码推送到代码仓库(AppRepo)后,自动触发CI/CD流水线。该流水线执行代码编译、单元测试、安全扫描等自动化步骤,并将通过测试的代码构建成不可变的Docker镜像,推送到镜像仓库。CI/CD流水线的最后一步自动修改独立的配置仓库(ConfigRepo),更新对应服务的Kubernetes部署文件,将镜像标签更改为新版本。

此时,部署在生产环境容器集群中的GitOps控制器(ArgoCD)发挥作用。ArgoCD作为持续运行的操作员(Operator),实时监控配置仓库状态。检测到部署文件变更后,它立即将Git仓库中定义的\cqt{期望状态}与集群中的\cqt{实际状态}进行比对,并自动执行\cqt{协调}(Reconciliation)操作,通过滚动更新策略将新版本服务平滑部署上线。


\subsection{系统性能与成本分析}
\label{subsec:performance}

\begin{figure}[htbp]
  \centering
  \includegraphics[width=\linewidth]{figures/chap06_api_performance.jpeg}
  \caption{API服务性能监控面板}
  \caption*{左侧纵轴显示QPS(每秒查询数),右侧纵轴显示响应延迟(毫秒)。系统在高峰期保持稳定的低延迟响应(P99延迟$<$200ms),验证了微服务架构的高可用性。}
  \label{fig:api_performance}
\end{figure}

图\ref{fig:api_performance}展示了系统在实际运行中的API性能表现。通过上述GitOps流程,实现了从代码提交到生产部署的全程自动化,所有变更均有记录、可追溯、可快速回滚,极大提升了软件交付的效率和系统稳定性。

\begin{table}[htbp]
    \centering
    \caption{KnowAir系统与传统数值模式系统的成本对比}
    \caption*{对比维度涵盖硬件配置、计算效率、更新频率、运维模式和扩展能力等关键指标,充分展示了AI驱动预报系统相较传统数值模式的显著优势。}
    \label{tab:cost_comparison}
    \begin{tabular}{@{}lcc@{}}
        \toprule
        \textbf{对比维度} & \textbf{\CMAQ+WRF} & \textbf{KnowAir} \\
        \midrule
        硬件配置 & 高性能集群(数十节点) & 4核12G云服务器 \\
        全国72h预报时间 & 2--4小时 & 3分钟 \\
        更新频率 & 每日1次 & 每小时1次 \\
        运维团队 & 专业团队驻场 & 全自动化运维 \\
        月均成本 & 数万元 & 数百元 \\
        扩展性 & 需硬件扩容 & 弹性伸缩 \\
        \bottomrule
    \end{tabular}
\end{table}

表\ref{tab:cost_comparison}对比了本系统与传统\CMAQ+WRF数值模式系统的资源需求与运维成本。值得强调的是,相较于传统数值模式系统需要专业运维团队长期驻场维护,本系统仅需一台普通配置的云服务器,即可在数分钟内完成全国范围的多日预报,展现了显著的计算效率和经济性优势。这一特性使得空气质量智能预报服务能够以极低的成本推广至各级环境监管部门和公众服务平台。


% ------------------------------------------------------------
% 6.3 业务化验证与应用落地
% ------------------------------------------------------------
\section{业务化验证与落地应用}
\label{sec:validation}

一个模型的真正价值需要在多样化、高标准的真实场景中得以检验,并最终转化为服务于社会的实际应用。本研究的模型系统(KnowAir,核心为PCDCNet模型)已在多个国家级重大活动保障和官方模型比对测试中接受严格检验,并成功实现商业化落地,服务于数千万用户和众多头部企业。


\subsection{在线预报性能监测}
\label{subsec:online_monitoring}

为全面评估模型的实战表现,建立了覆盖全国主要城市的在线预报性能持续监测体系,对AQI、\PM、\ozone 等关键指标进行实时跟踪与评估。

\begin{figure}[htbp]
  \centering
  \subcaptionbox{北京市AQI预报性能\label{fig:online-aqi-beijing}}
    {\includegraphics[width=0.9\linewidth]{figures/chap06_aqi_beijing.png}} \\[1ex]
  \subcaptionbox{上海市AQI预报性能\label{fig:online-aqi-shanghai}}
    {\includegraphics[width=0.9\linewidth]{figures/chap06_aqi_shanghai.png}} \\[1ex]
  \subcaptionbox{石家庄市AQI预报性能\label{fig:online-aqi-shijiazhuang}}
    {\includegraphics[width=0.9\linewidth]{figures/chap06_aqi_shijiazhuang.png}} \\[1ex]
  \subcaptionbox{郑州市AQI预报性能\label{fig:online-aqi-zhengzhou}}
    {\includegraphics[width=0.9\linewidth]{figures/chap06_aqi_zhengzhou.png}} 
  \caption{代表性城市AQI预报性能时序对比}
  \caption*{各子图展示了不同预报时效(3h、6h、12h、24h、48h、72h)下的预测值与实测值(AQI\_OBS)对比。模型在短期预报(3--12h)中表现出色,随预报时效延长误差逐渐增加但仍保持良好的趋势把握能力。}
  \label{fig:online-aqi-metrics}
\end{figure}

图\ref{fig:online-aqi-metrics}展示了北京、上海、石家庄、郑州四个代表性城市的AQI预报性能。从图中可以观察到:短期预报与实测值高度吻合,中期预报仍能较好把握污染变化趋势,即使在较长预报时效下,模型依然能够捕捉主要污染过程的峰值时间与量级。

\begin{figure}[htbp]
  \centering
  \subcaptionbox{北京市\ozone 预报性能\label{fig:online-o3-beijing}}
    {\includegraphics[width=0.9\linewidth]{figures/chap06_o3_beijing.jpeg}} \\[1ex]
  \subcaptionbox{广州市\ozone 预报性能\label{fig:online-o3-guangzhou}}
    {\includegraphics[width=0.9\linewidth]{figures/chap06_o3_guangzhou.jpeg}} \\[1ex]
  \subcaptionbox{上海市\ozone 预报性能\label{fig:online-o3-shanghai}}
    {\includegraphics[width=0.9\linewidth]{figures/chap06_o3_shanghai.jpeg}} \\[1ex]
  \subcaptionbox{苏州市\ozone 预报性能\label{fig:online-o3-suzhou}}
    {\includegraphics[width=0.9\linewidth]{figures/chap06_o3_suzhou.jpeg}} 
  \caption{代表性城市\ozone 浓度预报性能时序对比}
  \caption*{臭氧具有显著的日周期变化特征——白天光化学反应活跃导致浓度升高,夜间则显著下降。AI方法能够有效捕捉这种周期性规律,模型预测很好地还原了臭氧的日变化振幅与相位。}
  \label{fig:online-o3-metrics}
\end{figure}

图\ref{fig:online-o3-metrics}展示了臭氧(\ozone)的预报性能。臭氧浓度具有显著的日周期变化特征,这种规律性的周期变化恰好是数据驱动方法的优势所在:AI模型能够从历史数据中准确学习这种周期性模式,而传统数值模型因光化学反应机理的复杂性往往难以精确刻画。

值得注意的是,从预报难度角度,臭氧对数据驱动模型与传统数值模型呈现截然相反的特性。对于本文采用的AI模型,臭氧的可预报性反而优于PM$_{2.5}$,其根本原因在于臭氧浓度具有高度规律的日周期变化模式——白天光化学生成、夜间滴定消耗的昼夜节律极为稳定,而深度学习模型擅长从历史数据中捕捉并外推这种周期性规律。

然而,对于CMAQ、WRF-Chem等传统数值模型,臭氧预报却是公认的技术难点。如第\ref{subsec:traditional_models}节所述,臭氧是典型的二次污染物,其生成涉及NO$_x$与VOC在紫外辐射下的复杂光化学反应链。数值模型为刻画这一过程需耦合包含上百个化学方程的气相机理(如CB06、RADM2),计算复杂度极高。更关键的是,臭氧预报精度高度依赖前驱物排放清单的准确性:VOC与NO$_x$的排放量、时空分布及配比的任何偏差都会通过非线性化学反应被放大。此外,臭氧生成存在\cqt{VOC控制区}与\cqt{NO$_x$控制区}的非线性响应特征,使排放误差影响更难预判。相比之下,数据驱动方法通过端到端学习直接建立输入特征与臭氧浓度的映射,有效规避了显式求解光化学方程组的需求,在保持精度的同时降低了对排放清单的敏感性。

\subsection{上海进博会保障案例}
\label{subsec:ciie_case}

在中国国际进口博览会(CIIE)这一国家级重大活动中,本系统被用于提供高精度的空气质量预报,以辅助保障决策。

\subsubsection{模型定量评估}

在前期针对长三角地区的严格测试中,本AI模型相较于多款业务化运行的传统数值模型(\CMAQ、WRF-Chem、NAQPMS),在\PM、\ozone、PM$_{10}$等关键污染物的预报上展现出显著优势。

\begin{table}[htbp]
\centering
\caption{KnowAir模型与传统数值模型的预报性能对比}
\caption*{在长三角地区测试中,KnowAir模型在\PM、\ozone、PM$_{10}$三种污染物预报任务上均取得最优性能。表中RMSE为均方根误差(越小越好),R为相关系数(越大越好)。}
\label{tab:model_vs_numerical}
\begin{tabular}{@{}llcc@{}}
\toprule
\textbf{污染物} & \textbf{模型} & \textbf{RMSE} $\downarrow$ & \textbf{R} $\uparrow$ \\
\midrule
\multirow{5}{*}{\PM} & \CMAQ (run 1) & 18.3 & 0.395 \\
& \CMAQ (run 2) & 14.5 & 0.425 \\
& WRF-Chem & 25.1 & 0.641 \\
& NAQPMS & 21.4 & 0.588 \\
& \textbf{KnowAir} & \textbf{5.7} & \textbf{0.820} \\
\midrule
\multirow{5}{*}{\ozone} & \CMAQ (run 1) & 48.3 & 0.548 \\
& \CMAQ (run 2) & 38.2 & 0.563 \\
& WRF-Chem & 79.5 & 0.644 \\
& NAQPMS & 96.6 & 0.585 \\
& \textbf{KnowAir} & \textbf{38.4} & \textbf{0.756} \\
\midrule
\multirow{5}{*}{PM$_{10}$} & \CMAQ (run 1) & 15.4 & 0.457 \\
& \CMAQ (run 2) & 14.8 & 0.519 \\
& WRF-Chem & 21.3 & 0.601 \\
& NAQPMS & 15.9 & 0.650 \\
& \textbf{KnowAir} & \textbf{7.3} & \textbf{0.856} \\
\bottomrule
\end{tabular}
\end{table}

如表\ref{tab:model_vs_numerical}所示,本模型在三种污染物的预报上均取得最优性能:相较传统数值模型,\PM 的RMSE大幅降低,相关系数显著提升;PM$_{10}$的预报改进同样明显。这表明AI模型的预报结果与实测值更为接近,且对污染变化趋势的把握更为准确。

\subsubsection{实战预报表现}

在进博会举办期间,本AI模型的每日预报结果与传统专家会商模式进行了直接对比。结果显示,AI模型展现出三方面显著优势。

\textbf{精确度优势}:AI模型能够提供具体的单值预测,而专家会商则提供较宽泛的区间预测,这种精确性对于精细化管控决策具有重要参考价值。

\textbf{准确性优势}:AI模型的单值预测结果持续优于专家会商的区间预测。在多个预报日次中,AI模型预测与实测值的偏差显著小于专家会商区间的中值偏差,部分情况下专家会商给出的预测区间甚至完全偏离了实测范围。

\textbf{中长期预报优势}:这种优势在中长期预报中尤为明显。在提前多日的预报中,AI模型仍能基本把握污染等级,而专家会商给出的区间预测参考价值相对有限。


\subsection{粤港澳模型比对测试}
\label{subsec:gba_case}

为在更广范围、更长时间尺度上与国内外顶尖模型进行对标,本研究参加了由中国环境监测总站等权威机构组织的\cqt{粤港澳空气质量预报比对测试}。该测试\footnote{\url{http://124.128.14.106:10086/noticeDetail/66bef8dc65cfab60187f6887}}对所有参比模型在未来多日\PM 和\ozone 的小时浓度预报能力上,进行了长达数月的持续评估。评估体系极为严格,综合考察了NMB(归一化平均偏差)、NME(归一化平均误差)、R(相关系数)等统计指标以及APR(准确率)、CSI(临界成功指数)等污染过程预报指标\citep{evaluation}。

\subsubsection{参赛表现与结果}

在官方公布的结果中,KnowAir模型(PCDCNet)在与包括各大高校、科研院所以及业务单位在内的众多模型的激烈竞争中,取得了\textbf{综合评分中位数第一名、均值第二名}的优异成绩。官方发布的评估结果直观展示了KnowAir模型的得分分布,其箱体和中位线均位于所有模型的最高区间,表现出极强的稳定性和准确性。

\subsubsection{结果分析与讨论}

本次比对中取得均值第一名的模型来自华南理工大学,其技术路线为先采用\CMAQ 数值模式并结合了最新的区域排放清单进行模拟,再利用AI模型进行后处理订正。该方法的成功证明了传统物理模型结合精细化输入在机理表达上的重要价值\citep{cn_zhuyun2023,zhuyun2024}。

然而,该技术路线对计算资源和人力投入要求极高:需要运行完整的气象驱动模型和化学传输模型,并依赖高精度、持续更新的区域排放清单。与之形成鲜明对比的是,KnowAir模型作为一套端到端的AI系统,仅需普通配置的云服务器,即可在数分钟内完成全国范围的预报,展现了显著的计算效率和经济性优势。

这一结果充分证明,本文提出的\textbf{物理引导深度学习范式},在保持SOTA(State-of-the-Art)级别准确性的同时,极大降低了预报系统的部署和运行成本,为空气质量预报服务的普及化提供了技术基础。


\subsection{商业化落地}
\label{subsec:commercial}

本研究的最终价值在于其成功的商业化落地。KnowAir模型作为核心技术引擎,已深度整合进彩云科技的C端和B端业务线,服务于数千万用户和众多头部企业。

\subsubsection{面向公众的应用}
\label{subsec:b2c}

在\cqt{彩云天气}APP中,KnowAir模型为数千万用户提供直观、及时的空气质量预报服务。主要功能涵盖三个方面:(1)\textbf{高分辨率污染物空间分布图}——基于SPIN模型的空间推断能力,生成覆盖全国的高分辨率污染物浓度场,用户可直观查看所在区域及周边的空气质量状况;(2)\textbf{精细化AQI和污染物浓度预报}——提供未来多日的逐小时污染物浓度预报,帮助用户合理安排户外活动;(3)\textbf{重污染天气追踪与预警}——对即将来临的重污染过程提前发出预警,涵盖污染过程的起止时间、峰值浓度和影响范围。

\begin{figure}[htbp]
  \centering
  \includegraphics[width=\linewidth]{figures/chap06_beijing_pollution.png}
  \caption{北京重污染事件的提前预报示例}
  \caption*{左图展示污染峰值期间的空气质量状况,右图展示污染消散后的状况。系统成功提前数日预报了该污染过程的发生与消散,预报结果与实际观测高度一致,充分验证了模型在真实业务场景中的可靠性。}
  \label{fig:beijing_pollution_case}
\end{figure}

图\ref{fig:beijing_pollution_case}展示了一次典型重污染事件的预报案例。从图中可以看出,系统成功提前预报了污染过程的发生、峰值和消散全过程。

\begin{figure}[htbp]
  \centering
  \includegraphics[width=\linewidth]{figures/chap06_lianghui.png}
  \caption{全国两会期间北京AQI监测与预报对比}
  \caption*{图中各曲线代表不同预报时效的预测值,灰色曲线(AQI\_OBS)为实测值。图中最后几天出现的AQI突升对应沙尘暴事件,属于预报难点,但模型仍能把握其趋势特征。}
  \label{fig:lianghui_monitoring}
\end{figure}

图\ref{fig:lianghui_monitoring}进一步展示了重大活动期间更长时间序列的预报性能。总体而言,各预报时效的预测曲线与实测值保持良好一致。特别值得指出的是,图中最后几天出现的AQI突升对应一次沙尘暴过境事件。沙尘暴属于突发性、高频信号的极端天气事件,其预报难度远高于常规污染过程。尽管如此,模型仍能较好地捕捉到污染物浓度快速上升的趋势,体现了一定的极端事件响应能力。


\subsubsection{面向企业的服务}
\label{subsec:b2b}

模型产生的高精度预报数据被封装成标准的数据API服务,为众多行业的头部企业提供支持。服务覆盖多个行业领域:科技公司将其用于智能家居空气净化器联动和手机天气应用;智能汽车企业将其集成于车载空调系统的智能空气质量管理;物流企业借此实现户外作业人员健康保护和路线优化;金融机构用于健康保险风险评估;零售企业则用于空气净化产品的智能推荐。这种广泛的商业采纳是对模型数据准确性、稳定性及其商业价值的最有力印证。


\subsubsection{可视化监控与运维平台}
\label{subsec:visualization}

\begin{figure}[htbp]
  \centering
  \includegraphics[width=\linewidth]{figures/chap06_caiyun_platform.png}
  \caption{彩云科技空气质量可视化平台界面}
  \caption*{该平台提供全国范围的实时空气质量分布图,支持多污染物切换、时间动画播放和站点详情查询,已成为公众了解空气质量状况的重要渠道。}
  \label{fig:caiyun_web}
\end{figure}

图\ref{fig:caiyun_web}展示了面向公众的空气质量可视化平台界面\footnote{\url{https://caiyunapp.com/map/}}。该平台基于KnowAir模型的输出,提供全国范围的实时空气质量分布图,支持\PM、\ozone、PM$_{10}$等多污染物切换,以及未来多日的时间动画播放功能。

\begin{figure}[htbp]
  \centering
  \includegraphics[width=\linewidth]{figures/chap06_monitoring_panel.png}
  \caption{系统运维监控面板}
  \caption*{该面板集成了多城市、多指标的实时监测与预报评估功能,包括各城市平均AQI分布(左上)、近期污染物浓度趋势(左中)、各时效预报误差时序(左下和右上)以及全国城市\PM MAE分布排序(右下)。}
  \label{fig:monitoring_accuracy}
\end{figure}

图\ref{fig:monitoring_accuracy}展示了面向运维人员的系统监控面板。该面板集成了多维度的性能评估指标,涵盖:各城市平均AQI分布用于把握全国空气质量概况;\PM 预报时序曲线用于评估不同预报时效的表现;NMB和NME时序用于监测系统性偏差;全国城市MAE排序用于识别预报薄弱区域。这套监控体系确保了系统的持续稳定运行和预报质量的持续改进。


% ------------------------------------------------------------
% 6.4 本章小结
% ------------------------------------------------------------
\section{本章小结}
\label{sec:deploy_summary}

本章详细阐述了将本文提出的深度学习模型从理论研究推向真实世界应用的全过程,实现了从\cqt{科学建模}到\cqt{工程服务}的完整跨越。主要贡献总结如下:

\textbf{(1)设计并实现了云原生的低成本自动化部署系统}。采用微服务架构,将数据接入、模型训练、在线推理和API服务解耦为独立组件,通过Kubernetes容器编排实现弹性扩展,通过GitOps流程实现从代码提交到生产部署的全程自动化。相较传统\CMAQ+WRF系统需要专业团队驻场运维,本系统仅需普通配置云服务器即可运行,运维成本降低90\%以上。

\textbf{(2)在多个高标准实战场景中验证了模型性能}。通过上海进博会空气质量保障、粤港澳官方模型比对测试等应用,充分证明了本模型系统相较于传统数值模式和其他AI模型的综合优势。在进博会保障中,AI模型展现出显著优于专家会商的预报精度;在粤港澳比对测试中,KnowAir取得综合评分中位数第一名。

\textbf{(3)成功将技术成果转化为商业产品}。通过彩云天气APP服务数千万公众用户,通过标准化API服务众多头部企业。这种广泛的商业采纳是对模型准确性、稳定性和实用价值的最有力印证。

综上所述,本章的工作标志着本研究完整地实现了从理论创新到社会与经济价值创造的闭环,为数据驱动的环境科学研究提供了可复制、可推广的工程化范例。同时也表明,物理引导的深度学习方法不仅在科学研究中具有先进性,在工程实践中同样具备高效率、高精度和高稳定性的特点,成功架起了前沿科研与产业应用之间的桥梁。  % 系统部署与案例分析
% ============================================================
% 第七章 总结与展望
% 基于复杂系统数据驱动建模的大气污染研究
% ============================================================

\chapter{总结与展望}
\label{chap:conclusion}


% ------------------------------------------------------------
% 7.1 研究工作总结
% ------------------------------------------------------------
\section{研究工作总结}
\label{sec:summary}

本论文围绕\cqt{大气污染复杂系统的数据驱动建模}这一核心主题,针对空气质量预测、推断与模拟三大核心任务,构建了\cqt{理论$\rightarrow$数据$\rightarrow$建模$\rightarrow$应用$\rightarrow$部署}的完整研究体系。

大气污染作为典型的开放复杂巨系统,具有多要素强耦合、高度非线性、复杂网络拓扑与多源异构数据等本质特征。化学传输模型虽具备物理可解释性,但面临计算成本高昂、对排放清单高度依赖等瓶颈;现有数据驱动方法虽然计算高效,却普遍存在物理一致性弱、难以支撑情景模拟等不足。本研究的核心思想是将大气科学领域知识融入表示学习全过程,构建\cqt{观测空间$\leftrightarrow$隐空间$\leftrightarrow$观测空间}的完整建模框架,从根本上弥合机理建模与数据驱动之间的鸿沟。各章工作具体总结如下。

物理启发的时空图神经网络建模范式(第\ref{chap:methodology}章)。系统阐述了本文的方法论基础,包括时空数据的图表示、图神经网络基础、物理启发融合方法三大核心内容,为后续应用章节提供统一的理论框架。

物理启发时空图神经网络的大气污染预测(第\ref{chap:prediction}章)。针对大气污染时空依赖复杂、物理一致性不足等问题,首先提出PM$_{2.5}$-GNN模型,验证了将风场信息编码为有向图边权的有效性,在多个城市群的\PM 预测中建立了基线;在此基础上进一步提出PCDCNet预测框架,在隐空间中设计过程解耦的三模块结构(LID--STD--TAD),分别对应对流--扩散方程中的化学反应与排放项、平流与扩散项、沉降与累积项,通过物理一致性约束将质量守恒嵌入表示学习过程。在京津冀与长三角区域的\PM 与\ozone 预测中,相较现有最优方法RMSE降低约13--23\%,实现了高精度的72小时协同预报。

基于多源数据融合的大气污染空间推断(第\ref{chap:inference}章)。针对地面监测站点稀疏、遥感AOD数据缺失严重等问题,提出SPIN推断框架。模型采用扩散--平流双图并行传播机制刻画物理传输过程,设计AOD空间梯度约束的掩码机制规避缺测影响;采用动态节点掩码训练策略赋予模型归纳式泛化能力。在30\%站点缺测条件下MAE达到9.5 $\mu$g/m$^3$,较基线方法误差降低约25\%。

未来污染情景模拟(第\ref{chap:simulation}章)。面向碳达峰碳中和战略目标,提出IGNN模型,首次将排放清单作为可控变量纳入深度学习框架,实现2025--2050年多情景空气质量预测,揭示了\PM 与\ozone 的反向演变趋势及气候惩罚效应;同时发现不同城市呈现差异化响应特征,例如北京、太原等VOC控制区的\ozone 浓度呈微弱下降趋势,而多数城市\ozone 持续上升,体现了排放结构与光化学机制的区域异质性。在与传统物理化学模型(CMAQ、WRF-Chem)的对比中,IGNN在典型污染事件模拟上展现出显著优势(\PM:相关系数从0.36--0.38提升至0.84;\ozone:相关系数从0.47--0.57提升至0.90),同时将单情景推理从数小时压缩至秒级,为大规模情景探索提供了可行路径。

系统部署与案例分析(第\ref{chap:deployment}章)。基于云原生架构构建KnowAir智能预报平台(核心为第\ref{chap:prediction}章提出的PCDCNet模型),完成从科研原型到业务系统的工程化部署。系统仅需普通配置云服务器即可在3分钟内完成全国72小时预报。在上海进博会保障等实战应用中,KnowAir与\CMAQ、WRF-Chem、NAQPMS等数值模式进行了直接对比,在\PM、\ozone 等关键污染物预报上展现出显著优势;在粤港澳官方模型比对测试中取得综合评分中位数第一名,并成功转化为商业产品服务数千万用户。


% ------------------------------------------------------------
% 7.2 主要创新点与贡献
% ------------------------------------------------------------
\section{主要创新点与贡献}
\label{sec:contributions}

本论文从方法论角度提炼出\cqt{物理启发的数据驱动建模}核心范式,主要创新点总结如下:

创新点一:物理启发的时空图神经网络预测框架(第\ref{chap:prediction}章)。设计LID--STD--TAD三模块架构,分别对应公式(\ref{eq:advection_diffusion})中的化学反应与排放项、平流与扩散项、沉降与累积项,实现过程解耦与联合建模;利用风速向量投影定义有向边权重,突破传统GNN各向同性图结构的局限,使模型表征\cqt{上风向影响下风向}的定向传输规律;提出领域一致性约束(DIC),将质量守恒嵌入训练目标,提升极端情景下的物理合理性。

创新点二:多源数据融合与归纳式空间推断方法(第\ref{chap:inference}章)。提出\cqt{以AOD空间梯度为约束而非输入}的融合策略,利用AOD与地面\PM 浓度之间的物理关联性——AOD反映大气柱气溶胶光学厚度,其空间梯度蕴含污染物水平分布的物理先验——将该先验以损失函数约束的形式融入模型训练,通过掩码机制规避遥感数据大面积缺测的影响,实现全天候连续制图;采用扩散--平流双图并行传播机制刻画物理传输过程;引入动态节点掩码训练,赋予模型对未见站点的归纳式泛化能力,突破转导式学习仅能处理固定图拓扑的局限。

创新点三:排放响应的深度学习情景模拟框架(第\ref{chap:simulation}章)。首次将排放清单作为可控变量纳入深度学习框架,通过IGNN模型构建从排放源到污染物浓度的端到端映射:利用融合图结构将多尺度排放数据(MEIC历史清单与DPEC未来情景)与气象场、站点观测进行联合编码,使模型学习排放强度变化对大气污染浓度场的响应关系,实现碳中和路径下2025--2050年的长期情景预测,揭示\PM 与\ozone 反向演变趋势与气候惩罚效应,为协同减排决策提供科学依据。

从学科贡献来看,本研究推动大气污染模型从\cqt{能预测}走向\cqt{能推断、能模拟、能决策}。更一般地,本文提炼出一套面向复杂系统的数据驱动建模框架:\cqt{领域知识$\rightarrow$图结构构建$\rightarrow$物理启发表示学习$\rightarrow$任务导向解码},其中图网络作为核心建模工具,具备天然适配空间拓扑结构、支持异构多源数据融合、能够显式编码物理传输机制(如风场驱动的有向传播、扩散--平流双通道)等优势。该框架不局限于大气污染领域,可迁移至气象预报、水文模拟、交通流预测等涉及时空演化与网络拓扑耦合的复杂系统问题。


% ------------------------------------------------------------
% 7.3 研究局限性
% ------------------------------------------------------------
\section{研究局限性}
\label{sec:limitations}

尽管本研究取得了重要进展,但仍存在以下局限性。

物理--化学机理的显式性有限。当前模型虽引入了守恒与一致性约束,但主要以软约束形式作用于隐空间表示。化学反应速率、传输方程等核心机理尚未能显式可学习表达,在极端或罕见工况下的外推能力有待提升。

不确定性刻画能力不足。现有模型主要输出点估计结果,对预测区间、置信度与极端事件风险的量化能力有限。多源输入数据本身存在的观测误差与系统偏差如何传递至预测结果,尚未在模型中得到充分刻画。

泛化范围需进一步扩展。本研究主要验证于中国典型区域,对于不同气候带、地形条件与排放结构的区域,模型的迁移能力尚需系统评估。

跨尺度因果机制的理论认识不足。本研究的Embed--隐空间动力学--Readout架构实现了信息压缩,但该架构是否真正捕捉了涌现的宏观因果规律,目前仍缺乏理论验证;编码器设计也未以最大化因果效应为优化目标。


% ------------------------------------------------------------
% 7.4 未来研究展望
% ------------------------------------------------------------
\section{未来研究展望}
\label{sec:outlook}

基于上述分析,本研究提出以下未来研究方向。

因果涌现视角下的宏微观协同建模。因果涌现(Causal Emergence)是指复杂系统经过粗粒化后,宏观尺度的因果效应可以超越微观尺度的现象,即宏观描述比微观描述具有更强的因果决定性\citep{hoel2013quantifying}。未来可引入因果涌现理论框架\citep{hoel2013quantifying,zhang2022neural},将有效信息(EI)作为隐空间表示学习的优化目标,使编码器自动发现因果效应最强的宏观变量;构建宏微观双向因果架构,探索多尺度因果涌现的相变规律。

物理--化学机制的可学习表达。探索神经偏微分方程(Neural PDEs)、神经算子与物理符号回归等技术,将大气扩散方程、化学反应动力学方程嵌入神经网络架构;利用符号回归从数据中发现简洁的解析表达式,自动提取化学反应速率常数等可解释参数。

多源数据同化与不确定性估计。引入贝叶斯深度学习框架估计隐空间表示及预测结果的后验分布;采用流匹配与扩散模型学习污染场的概率分布,生成多条可能演化轨迹;结合变分数据同化与深度学习构建可在线订正的实时预测系统。

生成式建模与极端情景推演。利用生成扩散模型对历史极端污染事件进行建模,增强模型对罕见高污染事件的模拟与预警能力;通过条件生成实现不同政策干预下的空气质量响应评估。

跨区域迁移与全球服务化推广。基于领域自适应与元学习技术开发可快速适配新区域的模型迁移框架;结合Copernicus大气监测服务、全球排放清单等开放数据源,将系统扩展至全球主要城市群。

智能决策与数字孪生地球。构建面向城市空气质量治理的数字孪生系统,基于\cqt{前向模拟+情景评估}框架实现排放--浓度--健康--政策的全链条闭环分析,结合强化学习探索最优减排路径。  % 总结与展望


% 其他部分
\backmatter

% 参考文献
\bibliography{ref/refs}  % 参考文献使用 BibTeX 编译
% \printbibliography       % 参考文献使用 BibLaTeX 编译

% 附录
% 本科生需要将附录放到声明之后,个人简历之前
%\appendix
%\input{data/appendix}


% 个人简历、在学期间完成的相关学术成果
% 本科生可以附个人简历,也可以不附个人简历
% !TeX root = ../wangshuo_phdthesis.tex

\begin{resume}

  \section*{学术论文}

  \begin{achievements}
    \item \textbf{Shuo Wang}, Yanran Li, Jiang Zhang, Qingye Meng, Lingwei Meng, Fei Gao. PM$_{2.5}$-GNN: A Domain Knowledge Enhanced Graph Neural Network For PM2.5 Forecasting[C]. ACM SIGSPATIAL International Conference on Advances in Geographic Information Systems, 2020.
    \item \textbf{Shuo Wang}, Yun Cheng, Qingye Meng, Olga Saukh, Jiang Zhang, Jingfang Fan, Yuanting Zhang, Xingyuan Yuan, Lothar Thiele. PCDCNet: A Surrogate Model for Air Quality Forecasting with Physical-Chemical Dynamics and Constraints[J]. under review, 2025.
    \item \textbf{Shuo Wang}, Mengfan Teng, Yun Cheng, Lothar Thiele, Olga Saukh, Shuangshuang He, Yuanting Zhang, Jiang Zhang, Gangfeng Zhang, Xingyuan Yuan, Jingfang Fan. Physics-Guided Inductive Spatiotemporal Kriging for PM2.5 with Satellite Gradient Constraints[J]. under review, 2025.
    \item Gangfeng Zhang, \textbf{Shuo Wang} (共同第一作者), Jing Xu, Tim R. McVicar, Yun Cheng, Jiang Zhang, Cesar Azorin-Molina, Lorenzo Minola, Peijun Shi. A deep learning approach predicts O$_3$ increase and PM$_{2.5}$ declines under future high carbon emission scenario across the Northern China Plain[J]. Urban Climate, 2026.
    \item Ziqi Lin, \textbf{Shuo Wang}, Jing Xu, Peijun Shi, Yaoyao Ma, Yiwen Wang, Gangfeng Zhang. A Graph Neural Networks approach predicted spatiotemporal changes of Ozone Concentrations in the Yangtze River Delta (China)[J]. Environmental Research Communications, 2025.
    \item Jing Xu, \textbf{Shuo Wang}, Na Ying, Xiao Xiao, Jiang Zhang, Zhiling Jin, Yun Cheng, Gangfeng Zhang. Dynamic Graph Neural Network with Adaptive Edge Attributes for Air Quality Prediction: A Case Study in China[J]. Heliyon, 2023.
    \item Xiao Xiao, Zhiling Jin, \textbf{Shuo Wang}, Jing Xu, Ziyan Peng, Rui Wang, Wei Shao, Yilong Hui. A Dual-Path Dynamic Directed Graph Convolutional Network for Air Quality Prediction[J]. Science of The Total Environment, 2022.
    \item 林子琪, \textbf{王硕}, 许菁, 史培军, 马瑶瑶, 王怡雯, 张钢锋. 基于图神经网络的长三角臭氧浓度时空格局模拟[J]. 地理学报, 2025.
  \end{achievements}

  \section*{专利}

  \begin{achievements}
    \item 许菁, \textbf{王硕}, 营娜, 金志凌, 程云, 张江. 一种基于自适应动态图神经网络的空气质量预测方法:中国, 202210625446.8[P]. 2022-06-02.
  \end{achievements}

\end{resume}


% 致谢
% !TeX root = ../wangshuo_phdthesis.tex

\begin{acknowledgements}
  衷心感谢导师张江教授。
\end{acknowledgements}


\end{document}
