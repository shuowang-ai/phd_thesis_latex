\documentclass[tikz,border=5pt]{standalone}
\usepackage{ctex}
\usepackage{tikz}
\usetikzlibrary{shapes.geometric, arrows.meta, positioning, fit, backgrounds, calc, decorations.pathreplacing}
\usepackage{xcolor}
\usepackage{amsmath}

% 定义颜色
\definecolor{inputcolor}{RGB}{66, 133, 244}      % Google Blue
\definecolor{encodercolor}{RGB}{52, 168, 83}     % Google Green
\definecolor{latentcolor}{RGB}{251, 188, 5}      % Google Yellow
\definecolor{decodercolor}{RGB}{234, 67, 53}     % Google Red
\definecolor{losscolor}{RGB}{156, 39, 176}       % Purple
\definecolor{physicscolor}{RGB}{0, 150, 136}     % Teal
\definecolor{bglight}{RGB}{245, 245, 245}

\begin{document}
\begin{tikzpicture}[
    node distance=1.2cm and 1.5cm,
    >={Stealth[length=3mm]},
    box/.style={rectangle, draw, rounded corners=3pt, minimum width=2.2cm, minimum height=0.9cm, align=center, font=\small},
    smallbox/.style={rectangle, draw, rounded corners=2pt, minimum width=1.6cm, minimum height=0.6cm, align=center, font=\scriptsize},
    inputbox/.style={box, fill=inputcolor!20, draw=inputcolor},
    encoderbox/.style={box, fill=encodercolor!20, draw=encodercolor},
    latentbox/.style={box, fill=latentcolor!20, draw=latentcolor},
    decoderbox/.style={box, fill=decodercolor!20, draw=decodercolor},
    lossbox/.style={box, fill=losscolor!20, draw=losscolor},
    outputbox/.style={box, fill=inputcolor!20, draw=inputcolor},
    physicsbox/.style={smallbox, fill=physicscolor!15, draw=physicscolor},
    arrow/.style={->, thick, color=black!70},
    dashedarrow/.style={->, thick, dashed, color=losscolor},
    brace/.style={decorate, decoration={brace, amplitude=8pt, raise=3pt}},
]

% ========== 输入层 ==========
\node[inputbox] (X) at (0, 0) {污染物浓度\\$\mathbf{X}$};
\node[inputbox, right=0.8cm of X] (M) {气象变量\\$\mathbf{M}$};
\node[inputbox, right=0.8cm of M] (E) {排放数据\\$\mathbf{E}$};

% 输入标签
\node[above=0.3cm of M, font=\small\bfseries, color=inputcolor] {多源输入数据};

% ========== 编码层 ==========
\node[encoderbox, below=1.5cm of M, minimum width=6cm] (encoder) {
    \textbf{编码层 (Encoder)}\\
    特征提取 $\cdot$ 异构数据融合 $\cdot$ 图结构嵌入
};

% 编码层物理约束标注
\node[physicsbox, right=0.5cm of encoder] (enc_physics) {图结构设计\\(风场有向图)};

% ========== 隐空间动力学 ==========
\node[latentbox, below=1.2cm of encoder, minimum width=6cm, minimum height=2.2cm] (latent) {};

% 隐空间内部模块
\node[smallbox, fill=latentcolor!30, draw=latentcolor!80] at ($(latent.center)+(-1.5, 0.4)$) (spatial) {空间模块\\GCN/ChebNet};
\node[smallbox, fill=latentcolor!30, draw=latentcolor!80] at ($(latent.center)+(1.5, 0.4)$) (temporal) {时间模块\\TCN/GRU};

% 隐空间标题
\node[above=0.1cm of latent.north, font=\small\bfseries, color=latentcolor!80!black] {\textbf{隐空间动力学 (Latent Dynamics)}};

% 隐空间底部说明
\node at ($(latent.center)+(0, -0.6)$) [font=\scriptsize] {时空交替建模: $[\text{S-Conv} \rightarrow \text{T-Conv}]^L$};

% 隐空间物理约束标注
\node[physicsbox, right=0.5cm of latent] (lat_physics) {架构设计\\(扩散核/平流核)};

% ========== 解码层 ==========
\node[decoderbox, below=1.2cm of latent, minimum width=6cm] (decoder) {
    \textbf{解码层 (Decoder)}\\
    隐表示 $\rightarrow$ 目标空间映射
};

% 解码层物理约束标注
\node[physicsbox, right=0.5cm of decoder] (dec_physics) {损失函数\\(质量守恒)};

% ========== 输出层 ==========
\node[outputbox, below=1.2cm of decoder] (output) {预测/推断/模拟结果 $\hat{\mathbf{X}}$};

% ========== 真实标签 ==========
\node[inputbox, left=3cm of output] (groundtruth) {观测数据\\(Ground Truth)};

% ========== 损失函数 ==========
\node[lossbox, below=1cm of output, minimum width=5cm] (loss) {
    \textbf{总损失函数}\\
    $\mathcal{L} = \mathcal{L}_{\text{data}} + \lambda \mathcal{L}_{\text{physics}}$
};

% 损失函数分解
\node[smallbox, fill=losscolor!15, draw=losscolor, below left=0.8cm and -0.5cm of loss] (loss_data) {数据拟合\\$\mathcal{L}_{\text{data}}$};
\node[smallbox, fill=physicscolor!15, draw=physicscolor, below right=0.8cm and -0.5cm of loss] (loss_physics) {物理约束\\$\mathcal{L}_{\text{physics}}$};

% ========== 参数更新 ==========
\node[box, fill=gray!15, draw=gray, right=2.5cm of loss, minimum width=2cm] (params) {模型参数\\$\Theta$};

% ========== 连接箭头 ==========
% 输入到编码层
\draw[arrow] (X.south) -- ++(0, -0.3) -| ($(encoder.north)+(-1.5, 0)$);
\draw[arrow] (M.south) -- ($(encoder.north)+(0, 0)$);
\draw[arrow] (E.south) -- ++(0, -0.3) -| ($(encoder.north)+(1.5, 0)$);

% 编码层到隐空间
\draw[arrow] (encoder.south) -- (latent.north);

% 隐空间内部
\draw[arrow, <->] (spatial) -- (temporal);

% 隐空间到解码层
\draw[arrow] (latent.south) -- (decoder.north);

% 解码层到输出
\draw[arrow] (decoder.south) -- (output.north);

% 输出到损失函数
\draw[arrow] (output.south) -- (loss.north);

% 真实标签到损失函数
\draw[arrow] (groundtruth.south) |- (loss.west);

% 损失函数到分解
\draw[arrow] (loss.south) -- ++(0, -0.3) -| (loss_data.north);
\draw[arrow] (loss.south) -- ++(0, -0.3) -| (loss_physics.north);

% 损失函数到参数
\draw[arrow] (loss.east) -- (params.west);

% 反向传播(虚线)
\draw[dashedarrow, rounded corners=10pt] (params.north) -- ++(0, 1) -| ($(encoder.east)+(0.3, 0)$) -- (encoder.east);
\node[font=\scriptsize, color=losscolor, rotate=90] at ($(params.north)+(0.5, 1.5)$) {反向传播};

% 物理约束连接
\draw[->, thick, color=physicscolor, dashed] (enc_physics.west) -- (encoder.east);
\draw[->, thick, color=physicscolor, dashed] (lat_physics.west) -- (latent.east);
\draw[->, thick, color=physicscolor, dashed] (dec_physics.west) -- (decoder.east);

% ========== 三类任务标注 ==========
\node[below=0.3cm of loss_data, font=\scriptsize, align=center] {MSE/MAE\\监督学习};
\node[below=0.3cm of loss_physics, font=\scriptsize, align=center] {DIC/AOD梯度\\物理一致性};

% ========== 底部任务分类 ==========
\node[below=2.8cm of loss, font=\small] (tasks) {
    \begin{tabular}{ccc}
    \textbf{预测任务} & \textbf{推断任务} & \textbf{模拟任务} \\
    (第3章) & (第4章) & (第5章) \\
    PM$_{2.5}$-GNN, PCDCNet & SPIN & IGNN
    \end{tabular}
};

% 任务连接
\draw[brace] ($(tasks.north west)+(0.5, 0.2)$) -- ($(tasks.north east)+(-0.5, 0.2)$)
    node[midway, above=0.4cm, font=\small] {统一框架支撑三类核心任务};

% ========== 与数值模型对比标注 ==========
\node[draw=gray, dashed, rounded corners=5pt, fill=gray!5, align=left, font=\scriptsize]
    at ($(loss.south)+(-4, -1.5)$) (compare) {
    \textbf{区别于数值模型:}\\
    $\bullet$ 数据驱动的参数学习\\
    $\bullet$ 端到端梯度优化\\
    $\bullet$ 自适应拟合观测数据
};

\end{tikzpicture}
\end{document}
